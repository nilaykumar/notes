\documentclass{../../mathnotes}

\usepackage{enumerate}
\usepackage{todonotes}
\usepackage{tikz-cd}

\title{Commutative Algebra: Problem Set 2}
\author{Nilay Kumar}
\date{Last updated: \today}


\begin{document}

\maketitle

\subsection*{Problem 5}

Let $k$ be a field. Then it's clear that $k[ [ t] ]$, the formal power series ring over $k$, is local, as we can write it as
a (disjoint) union $(x)\sqcup (\text{units})$. To see this, note that any power series with a constant term can be inverted
by a geometric series trick; given $x=c+\sum^\infty_{i=1} a_it^i, c\in k$,
\begin{align*}
    \frac{1}{x}=\frac{1}{c}\cdot\frac{1}{1+c^{-1}(\sum^\infty_{i=1} a_it^i)}=1+\left(c^{-1}\sum^\infty_{i=1} a_it^i\right)+\left(c^{-1}\sum^\infty_{i=1} a_it^i\right)^2+\ldots,
\end{align*}
which clearly yields another power series. Now, if a power series does not have a constant term, it must belong to the ideal $(t)$. This ideal is
maximal, as adding an additional element would mean adding a unit. Then, by the theorem proved in class, we see that $(k[ [t] ],(t))$ is a local ring.

\subsection*{Problem 6}

Consider again the formal power series ring $A[ [t] ]$, where $A$ is an domain. Hence $A[ [t] ]$ is an domain as well, and thus has
$(0)$ as a prime ideal. Additionally, since $(t)$ is clearly a minimal (and maximal, by locality) prime of the ring, we see that $A[ [t]]$ has
exactly two prime ideals.

Finding a local ring with three primes is a little more difficult; consider the ring $\C[x,y]/(xy)$. In the last problem set, we computed the primes
to be $(x),(y),(x,y),(x-\lambda, y), (x,y-\mu)$ for $\lambda,\mu \C^\times$. Localizing this ring about the prime $(x,y)$ eliminates the last two
(as they are not contained in $(x,y)$). Hence we are left with the three primes $(x),(y),(x,y)$.

\subsection*{Problem 7}

Let $R=k[ [t]]$ where $k$ is a field. We wish to find an example of a module $M$ over $R$ such that $M=tM$. This does not contradict Nakayama's
lemma, as the lemma applies only to $M$ a finite $R$-module. Consider the module $M=R[ [s]]/(ts-1)\cong k[ [ t,t^{-1}]]$.
Elements of this ring are of the form $\sum_{i=-\infty}^{\infty} a_i t^i$, and clearly multiplying by $t$ gives us back an element of $M$.
Furthermore, $M\subset tM$, as every element $x=\sum_{i=-\infty}^{\infty} a_i t^i$ of $M$ can be written as an element of $tM$, i.e. $t\sum_{i=-\infty}^{\infty} a_{i-1} t^i$.
Hence, $M=tM$ and $M\neq 0$, as desired.

\subsection*{Problem 8}

Let $R=\C[x]$ be the polynomial ring over the complex numbers. Let $\fr m_n$ for $n=1,2,3,\ldots$ be an infinite sequence of pairwise distinct maximal
ideals of $R$. Consider the product space $S=\prod_i^\infty R/\fr m_i$. Suppose there exists a surjection $R\overset{\phi}{\twoheadrightarrow}S$.
It's clear that the $\phi$ sends $\C\subset\C[x]$ to $\C\cdot (1,\ldots)\subset S$.
Note that there must exist an $a\in\C[x]$ such that $\phi(a)=(1,0,0,\ldots)$; since the maximal ideals under consideration are of the form $(x-\lambda)$ for
$\lambda\in\C$, this implies that $a$ is in $\fr m_i$ for $i>1$, i.e. $a$ must have an infinite number of roots. Of course, this implies that $a=0$,
which is a contradiction, as $\phi(0)=(0,\ldots)$. Hence there exists no such surjection.

\subsection*{Problem 9}

Let $k$ be a field. We wish to find the minimal prime ideals of $A=k[x,y,z]/(xy,xz,yz)$. The minimal primes of $A$ are in correspondence to the
smallest primes of $k[x,y,z]$ containing $(xy, xz, yz)$. Note first of all that whether or not $k$ is algebraically closed is irrelevant here as
we are looking only for minimal primes. None of the ideals $(x),(y),(z)$ contain $(xy,xz,yz)$ as $(x)$ cannot generate $yz$, for example. Furthermore,
consider principal ideals generated by polynomials of higher order (or of the form $x-\lambda$) will not work either as they will not generate
the needed elements (or will not be prime). Next one considers ideals generated by two elements. Right away we see that $(x,y),(x,z),(y,z)$ are prime
and that they contain $(xy,xz,yz)$. Let us check that these are minimal. First off, it's clear that any ideal generated by more than 2 elements will not
necessarily be contained in these, and hence will not be minimal. Furthermore, placing any higher-order polynomials in the place of $x,y,$ or $z$ will
either not be prime or will not generate $xy,xz,$ or $yz$. Hence we conclude that $(x,y),(x,z),$ and $(y,z)$ are indeed the minimal prime ideals of
$k[x,y,z]/(xy,xz,yz)$.
(Moreover we can see this geometrically by visualizing the coordinate axes, as discussed in class.)

\subsection*{Problem 10}

\end{document}
