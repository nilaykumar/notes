\documentclass{../../mathnotes}

\usepackage{enumerate}
\usepackage{todonotes}
\usepackage{tikz-cd}

\title{Commutative Algebra: Problem Set 3}
\author{Nilay Kumar}
\date{Last updated: \today}


\begin{document}

\maketitle

\subsection*{Problem 7}

Define the ring $A=k[f_1,f_2,\ldots]/(f_1^2,f_2^2,\ldots)$ and consider the ideal $I=A$. Consider the ideal
$I^n$ for some $n>0$. Let $g=f_1+f_2+\ldots+f_n$ in $I$. Then $g^n$ is an element of $I^n$, but note that
$g^n$ clearly contains the term $n!(f_1f_2\cdots f_n)\neq0$ and hence $I^n$ cannot be nilpotent.

\subsection*{Problem 8}

Since $B$ is a finite-type $A$ algebra ($A$,$B$ domains), we can write $B=A[x_1,\ldots,x_n]/I$ where $I$ is an ideal of $A[x_1,\ldots,x_n]$.
By finiteness of the extension $K\subset L$ (and since obviously $A\subset K,B\subset L$), $x_i\in L$ is algebraic over $K$.
Hence each $x_i$ solves a polynomial of degree $n_i$ with coefficients in $K$. By clearing denominators, we can obtain an
associated set of polynomials with coefficients in $A$; let us denote by $a_i$ the leading coefficient of the polynomial for $x_i$
and let $\alpha=a_1\cdots a_n$ be the product. We can now localize both $A$ and $B$ about the multiplicative set generated by $\alpha$.
We now have that $x_i\in B_\alpha$ are integral over $A_\alpha$. Now (by lemma 10.33.5 of the Stacks project, Tag 02JJ) we see that
the map $A_\alpha\to B_\alpha$ is finite, and by exactness of localization, injective.
This yields the following diagram:
\begin{equation*}
\begin{tikzcd}
    \Spec B\arrow{d} & \Spec B_\alpha\arrow[hookrightarrow]{l}\arrow[twoheadrightarrow]{d}\\
    \Spec A & \Spec A_\alpha\arrow[hookrightarrow]{l}
\end{tikzcd}
\end{equation*}
It follows from the properties of primes of localization that the image of the map $\Spec A_\alpha\hookrightarrow\Spec A$ is
$D(\alpha)$. But since $\Spec B_\alpha\to\Spec A_\alpha$ is a surjection and $\Spec B_\alpha\hookrightarrow\Spec B$, $\Spec A_\alpha\hookrightarrow\Spec A$
are injections, it follows that the image of $\Spec B\to\Spec A$ must also contain $D(\alpha)$. Since $D(\alpha)$ is open, we are done.

\subsection*{Problem 9}

Consider the evaluation homomorphism $k[x,y]\overset{\phi}{\to} k[t]$ that takes $P(x,y)\mapsto P(f(t),g(t))$ for the given $f(t),g(t)$.
Let us suppose for the sake of contradiction that there does not exist a $P\in k[x,y]$ such that $\phi(P)=0$. In other words, $\phi$ is injective.
Suppose $f(t)=a_nt^n+\ldots+a_0$. We can show that $t\in k[t]$ is in fact integral over $k[x,y]$, as it solves the equation
\begin{align*}
    \phi(x)&=f(t)\\
    0&=t^n+\frac{a_{n-1}}{a_n}t^{n-1}+\ldots+\frac{a_0-\phi(x)}{a_n}.
\end{align*}
Hence $\phi$ must be finite. This implies that $\Spec\phi$ is surjective, which implies that $\dim k[t]=\dim k[x,y]$. This is a contradiction;
thus $\phi$ cannot be injective, and there must exist a $P(x,y)\in k[x,y]$ such that $\phi(P)=0$.


\subsection*{Problem 10}

Let $A=\Z/4\Z$. It's clear that $A$ is Artinian as the ring is finite. Additionally, $A$ is clearly not an
algebra of finite-type over a field.

\subsection*{Problem 11}



\end{document}
