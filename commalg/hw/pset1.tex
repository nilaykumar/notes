\documentclass{../../mathnotes}

\usepackage{enumerate}
\usepackage{todonotes}
\usepackage{tikz-cd}

\title{Commutative Algebra: Problem Set 1}
\author{Nilay Kumar}
\date{Last updated: \today}


\begin{document}

\maketitle

\subsection*{Problem 5}

Let $k$ be a field and $A=k[x,y]/(x^2y^4-x^4y^2+1)$. We wish to construct a finite injective map as in the Noether normalization lemma. As an algebra, $A$ is generated by $\left\{ x,y \right\}$.
Using the trick from Noether normalization, we define $\tilde x=x-y^2$ in order to produce a monic polynomial. Inserting this into the polynomial $F$ generating the ideal,
we find that $1+y^8-y^{10}+2 y^6 \tilde{x}-4 y^8 \tilde{x}+y^4 \tilde{x}^2-6 y^6 \tilde{x}^2-4 y^4 \tilde{x}^3-y^2 \tilde{x}^4$. In $A$, $F=0$, and thus $y$ is integral over
$k[\tilde x]$. This implies that the map $\phi:k[\tilde x]\to A$ is finite (by lemma 3 in class). Injectivity of this map follows easily from what we did in class.

\subsection*{Problem 6}

The prime ideals of $\C[x,y]/(xy)$ are in one-to-one correspondence with the prime ideals of $\C[x,y]$ that contain $(xy)$.
Since $\C[x]$ is a principal ideal domain, we may use the following lemma from undergraduate algebra:
\begin{lem}
    Let $R$ be a principal ideal domain. The prime ideals of $R[y]$ are precisely those of the following form:
    \begin{itemize}
        \item $(0)$
        \item $(f(y))$ where $f$ is an irreducible polynomial,
        \item $(p,f(y))$ where $p\in R$ is prime and $f(y)$ is irreducible in $(R/p)[y]$
    \end{itemize}
\end{lem}
It follows straightforwardly, then, that such prime ideals that contain $(xy)$ are precisely $(x),(y),(x,y-\lambda),(x-\mu,y)$.
A more geometric approach, of course would be to think of the $x$ and $y$-axes, which would give us precisely this set.


\subsection*{Problem 7}

Let $k$ be a field. We wish to prove that $k[x,y]$ is not isomorphic to $k[x,y,z]$. Suppose for the sake of contradiction
that there exists a (module) isomorphism $\phi: k[x,y]\to k[x,y,z]$. Then $\phi$ is clearly (module-)finite, i.e. there exists a
set of generators $M\subset k[x,y,z]$ that $k[x,y]$-spans $k[x,y,z]$. It's clear that $1=z^0\in M$. The $k[x,y]$-span of 1
does not contain $z$, however, and hence we must have $z\in M$. By the same argument, we must have $z^n$ for arbitrarily high
powers $n\in\Z$. This is a contradiction - $k[x,y]$ cannot be isomorphic to $k[x,y,z]$.

\subsection*{Problem 8}

Let $k$ be a field. Suppose $A$ is a $k$-algebra and $f$ is a nonzerodivisor of $A$ such that $k[x,y]$ is isomorphic to $A_f$
as a $k$-algebra. Then, since the invertible elements of $k[x,y]$ are the field elements $k$, we must have that $f\in k$. Note
that since $k\to A$ is injective, we must have that $f\in k\subset A$ and hence $A=A_f=k[x,y]$ since $f$ is already invertible.

\subsection*{Problem 9}

Let $A$ be a ring and let $f$ be an element of $A$. We wish to show that $A_f=\{1,f,f^2,\ldots\}^{-1}A$ is isomorphic as an
$A$-algebra to $A[x]/(fx-1)$. Consider the following diagram:
\begin{equation*}
    \begin{tikzcd}
        {} & A\arrow[swap]{ld}{\phi}\arrow{rd}{g} &  \\
        A_f\arrow[dashed]{rr}{\psi} &  & A[x]/(xf-1)
    \end{tikzcd}
\end{equation*}
Using the universal property of localization (see Atiyah-MacDonald pp. 37-38), since $g(f^k)$ is a unit in $A[x]/(xf-1)$,
$\ker g=0$, and because every element of $A[x]/(xf-1)$ is of the form $g(a)g(s)^{-1}$, there exists a unique isomorphism
$\psi:A_f\to A[x]/(xf-1)$.

\end{document}
