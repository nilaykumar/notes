\documentclass{../../mathnotes}

\usepackage{enumerate}
\usepackage{todonotes}
\usepackage{tikz-cd}

\title{Commutative Algebra: Problem Set 6}
\author{Nilay Kumar}
\date{Last updated: \today}


\begin{document}

\maketitle

\subsection*{Problem 2}

This solution is based off of Rankeya Datta's solution, available on de Jong's REU webpage.

Let us assume a third axiom: there are four distinct points, where no three are incident to any line.
We proceed to prove this statement in cases. The first case is for when either the number of lines or points is infinite,
and the second case is when both of these numbers is finite. Throughout, we will denote the set of lines in our projective space $\Proj$
as $\mathcal{L}$ and the set of points as $\mathcal{P}$.

Now suppose that $\mathcal{L}$ is infinite. Then by the axioms, $\mathcal{P}$ must be infinite as well. Conversely, suppose that $\mathcal{P}$ is
infinite and $\mathcal{L}$ is finite. Every point must lie on some line and so there must be at least one line $\ell\in\mathcal{L}$ incident to
an infinite number of points. By the additional axiom we find that there must exist a point $p$ not incident to $\ell$, which induces
an infinite number of lines incident to $p$ and the infinite (unique) points on $\ell$. Thus we see that if either of $\mathcal{P},\mathcal{L}$ is infinite
then both must be. Let us treat the infinite case first.

Denote by $\Delta_\mathcal{P},\Delta_{\mathcal{L}}$ the diagonals of $\mathcal{P}\times\mathcal{P},\mathcal{L}\times\mathcal{L}$ respectively. As we have an infinite number
of points and lines, we see that
\begin{align*}
    |\mathcal{L}|&=|\mathcal{L}\times\mathcal{L}|=|\mathcal{L}\times\mathcal{L}-\Delta_\mathcal{L}|\\
    |\mathcal{P}|&=|\mathcal{P}\times\mathcal{P}|=|\mathcal{P}\times\mathcal{P}-\Delta_\mathcal{P}|.
\end{align*}
Next note that the axioms yield the injective maps
\begin{align*}
    \pi_1:\mathcal{L}\times\mathcal{L}-\Delta_\mathcal{L}\to\mathcal{P} &\hspace{10mm} (l_1,l_2)\mapsto l_1\cap l_2\\
    \pi_2:\mathcal{P}\times\mathcal{P}-\Delta_\mathcal{P}\to\mathcal{L} &\hspace{10mm} (p,q)\mapsto \overline{pq}
\end{align*}
where the intersection in $\pi_1$ is incidence and $\overline{pq}$ in $\pi_2$ is the line incident with $p$ and $q$.
It now suffices to show that $\pi_1,\pi_2$ are surjective, which is quite easy. The surjectivity of $\pi_2$ follows from noting
that any line $l\in\mathcal{L}$ must contain at least two distinct points $p$ and $q$, which yield $\pi_2(p,q)=\overline{pq}=l$.
Similarly, to show the surjectivity of $\pi_1$ take some $p\in \mathcal{P}$ and two distinct points $q$ and $r$ such that $p,q,r$ are not collinear
(again by the axioms). Then, since $\overline{pq}\neq\overline{pr}$, we see that $\pi_1(\overline{pq},\overline{pr})=p$. This concludes the infinite case.

Now suppose that both $\mathcal{P},\mathcal{L}$ are finite. We first prove two auxillary lemmas and then use them to prove the main claim.

\begin{lem}
    Let $p\in\mathcal{P}$. If $\mathcal{L}_p$ denotes the set of all lines passing through $p$, then $|\mathcal{L}_p|$ is indpendent
    of the choice of $p$.
\end{lem}
\begin{proof}
    Choose two distinct points $p,q\in\mathcal{P}$. It suffices to show that $|\mathcal{L}_p|=|\mathcal{L}_q|$. Note that there exists a
    unique line $\overline{pq}$ incident with $p$ and $q$ and a point $r$ on $\overline{pq}$ distinct from $p$ and $q$. Let $l\in\mathcal{L}_p-\{\overline{pq}\}$
    and $m\in\mathcal{L}_r-\{\overline{pq}\}$ - note that these two lines are distinct. Hence $l\cap m$ is a point not on $\overline{pq}$. Let $\mathcal{P}_{\Proj-\{\overline{pq}\}}$
    denote the set of points of $\Proj$ not on $\overline{pq}$. Hence we get a map
    $\phi:\left( \mathcal{L}_p-\{\overline{pq}\} \right)\times\left( \mathcal{L}_r-\{\overline{pq}\} \right)\to\mathcal{P}_{\Proj-\{\overline{pq}\}}$ defined by
    $(l,m)\mapsto l\cap m$. Note that the codomain is not $\varnothing$ by the axiom above and that $\phi$ is a bijection with inverse given by $s\mapsto (\overline{ps},\overline{rs})$.
    Hence we see that $(|\mathcal{L}_p-1|)^2=|\mathcal{P}_{\Proj-\{\overline{pq}\}}|$. One can show the same for $\mathcal{L}_q$ and hence we find that $|\mathcal{L}_p|=|\mathcal{L}_q|$.
\end{proof}

Let us denote the number of lines through any point by $c$.

\begin{lem}
    Let $l\in\mathcal{L}$. Let $\mathcal{P}_l$ be the set of points in $\Proj$ incident with $l$. Then $|\mathcal{P}_l|$ is independent of $l$ and $|\mathcal{P}_l|=c$ for all $l\in\mathcal{L}$.
\end{lem}
\begin{proof}
    Let $p$ be a point not incident with $l$. Then, $l\notin \mathcal{L}_p$. Define the map $\chi:\mathcal{L}_p\to\mathcal{P}_l$ where $\chi(m)=l\cap m$.
    Then $\chi$ has an inverse $\chi^{-1}:\mathcal{P}_l\to\mathcal{L}_p$ given by $s\mapsto \overline{ps}$. Hence $|\mathcal{P}_l|=|\mathcal{L}_p|=c$,
    and we are done.
\end{proof}

Now, since $\mathcal{P}=\mathcal{P}_{\Proj-\{\overline{pq}\}}\cup \mathcal{P}_{\{\overline{pq}\}}$ (disjoint), we see that we have $|\mathcal{P}|=(c-1)(c-1)+c=c(c-1)+1$.
Precisely the same argument can be made for $\mathcal{L}$, and we are done.



\subsection*{Problem 3}

Let $P,Q,R$ be three pairwise distinct points in $\Proj^1$. We wish to find a matrix that sends $P,Q,R$ to
$(1:0),(0:1),(1:1)$. Let us solve for such a matrix $A$, imposing
\begin{align*}
    \begin{pmatrix}a&b\\c&d\end{pmatrix}\begin{pmatrix}p_0\\p_1\end{pmatrix}&=\begin{pmatrix}\lambda\\0\end{pmatrix}\\
    \begin{pmatrix}a&b\\c&d\end{pmatrix}\begin{pmatrix}q_0\\q_1\end{pmatrix}&=\begin{pmatrix}0\\\mu\end{pmatrix}
\end{align*}
for some constants $\lambda,\mu$ and solving to find
\begin{align*}
    A=\frac{\kappa}{p_0q_1-p_1q_0}\begin{pmatrix} q_1\lambda&-q_0\lambda\\-p_1\mu&p_0\mu\end{pmatrix}.
\end{align*}
with $\kappa$ some constant (as $A$ is defined up to scaling). Finally, we solve
\begin{align*}
    A
        \begin{pmatrix}
            r_0\\r_1
        \end{pmatrix}
        = \begin{pmatrix}
            \gamma\\\gamma
        \end{pmatrix}.
\end{align*}
to find that
\begin{align*}
    A=\gamma\begin{pmatrix}\frac{-q_1}{q_0r_1-q_1r_0}&\frac{q_0}{q_0r_1-q_1r_0}\\\frac{p_1}{p_1r_0-p_0r_1}&\frac{-p_0}{p_1r_0-p_0r_1}\end{pmatrix}.
\end{align*}
Note that none of the denominators cause problems, due to the points being distinct.

\subsection*{Problem 4}

We wish to find a field $K$ and a conic over $K$ with no points. Take $K=\R$ and the conic to be given by
\[F=x_0^2+x_1^2+x_2^2\]
We see that a root must have $x_0=x_1=x_2=0$. As $(0,0,0)\in\R^3$ is not contained in projective space, this conic has no points.

\subsection*{Problem 5}

A degree 2 morphism from $\Proj^1\to\Proj^2$ is given by a triple $G_0,G_1,G_2$ of relatively prime quadratics. It is clear that if the
morphism is into a line or a conic, it will be onto, as it is non-constant. Let us show that the morphism is into, i.e. there exists
a line or conic such that when evaluated at the triple, yields zero. We claim that the case of the line is included in that of the conic; to see
this note that the equation for a line in projective space is given by $Ax_0+Bx_1+Cx_2=0$, which can simply be squared to get the associated
conic. For the morphism to be into a conic we must require that 
\[\sum_{i\leq j}\lambda_{ij}G_iG_j=0\]
where we are free to pick the six $\lambda_{ij}$. Note that each of the $G_i$s is a homogeneous polynomial of degree 2 in the coordinates $x_0,x_1$
of $\Proj^1$, and hence inserting $x_0,x_1$ into the above constraint, we find a homogeneous polynomial of degree 4 equal to zero. 
For this to happen we must fix $\lambda_{ij}$, by the relative primeness of the quadratics, i.e. we must solve 5 equations (this is the dimension
of the space of homogeneous polynomials of degree 4) with the six $\lambda_{ij}$. As the system is consistent (the augmented matrix clearly has the same
rank as the matrix of coefficients) it has a solution, and we are done.

\subsection*{Problem 6}

Consider $k[t]\subset k(t)\subset K$ where $k(t)\subset K$ is a finite extension. As $k[t]$ is a UFD, it is of course
integrally closed in $k(t)$, its field of fractions. As $K$ is finite over $k(t)$, there must be a finite number of elements
$\alpha_n\in K$ not contained in $k(t)$ that are algebraic over $k(t)$. Some subset of these will be integral over $k(t)$,
and hence integral over $k[t]\subset k(t)$. Thus it is clear that the integral closure of $k[t]$ in $K$ must be finite over
$k[t]$ (as we can simply take these elements as generators).

Next let us prove that for every discrete valuation $v$ on $k(t)$ there are finitely many discrete valuations $w_i$ on $K$ whose
restriction to $k(t)$ is $e_iv$ for some integer $e_i$. We suppose for now that there exists at least one discrete valuation that extends
(we prove this in the next part). Suppose we have pairwise distinct $w_1,\ldots,w_r$ on $K$ extending $v$. Pick (by a lemma
proved in class) a set of elements $\{f_i\}_{i=1}^n\in K$ (with $[K:k(t)]=n$) such that $w_i(f_i)>0,w_j(f_i)<0$ for $i\neq j$. If we can show that 
the $f_i^{-1}$ are linearly linearly independent to each other, we are done, as there can be no more than $n$ such $f_i$s in an extension of degree $n$
and hence no more than $n$ extensions $w$. Suppose $\sum_ia_if^{-1}=0$ for some $a_1,\ldots, a_n\in k(t)$.
Clearing denominators, we may assume that $a_i\in\mathcal{O}_v$ and that $a_{i_0}\notin\fr m_r$ for some $i_0$. Then
\[w_{i_0}\left(\sum_i a_if^{-1}_i\right)\geq\min w_{i_0}(a_if^{-1}_i)\geq\min e_{i_0}v(a_i)+w_{i_0}(f_i^{-1}).\]
But the first term is greater than or equal to zero (equal if $i=j$) and the second is always greater than equal to zero (equal if $i=i_0$).
Hence this could not have been zero. So $\sum_i a_if_i^{-1}\neq 0$, as desired.




\end{document}
