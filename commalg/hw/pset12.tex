\documentclass{../../mathnotes}

\usepackage{enumerate}
\usepackage{todonotes}
\usepackage{tikz-cd}

\title{Commutative Algebra: Problem Set 12}
\author{Nilay Kumar}
\date{Last updated: \today}


\begin{document}

\maketitle

\subsection*{Problem 4}

We prove the sheaf condition for the structure $\mathcal{B}$-pre-sheaf $\mathcal{O}$ of an affine scheme $X=\Spec A$ where $\mathcal{B}$ is the basis of standard opens.
We roughly follow Eisenbud and Harris (The Geometry of Schemes).
We know already that any $\mathcal{B}$-sheaf on $X$ extends uniquely to a sheaf on $X$, so it suffices to show that our $\mathcal{B}$-pre-sheaf is in fact
a $\mathcal{B}$-sheaf. This amounts to showing first that if any two sections $s_1$ and $s_2$ become locally equal, i.e. when restricted to each of the standard
opens $X_{f_i}$, then $s_1=s_2$, and second that if for each $i$ there exist $s_i\in \mathcal{O}(X_{f_i})$ such that for each pair $i,j$ $s_i|_{X_{f_if_j}}=s_j|_{X_{f_if_j}}$,
then there exists an $s\in \mathcal{O}(X)$ whose restriction to each $X_{f_i}$ is precisely $s_i$.

Let us prove the first condition. For $s_1,s_2$ to become equal in $X_{f_i}$ means that the difference $s_1-s_2$ must be annihilated by some power of each $f_i$.
Since the cover $X_{f_i}$ of $X$ is finite (see the quasicompactness problem below) $s_1-s_2$ is annihilated by the ideal generated by all the $f_i^N$ for some
$N$. As this ideal clearly contains a power of the ideal generated by all the $f_i$, which in turn generate the unit ideal (again, see below), $s_1$ must
equal $s_2$ in $A$ as the difference is annihilated by all elements of the ring.

Next consider the second condition. Each $s_i\in \mathcal{O}(X_{f_i})=A_{f_i}$ can of course be multiplied by a high enough power of $f_i$ to yield an element $h_i\in A$.
Again, by finiteness, one $N$ suffices for all $i$ and hence redefining $h_i=f_i^Ns_i$, we find that
\[f_j^Nh_i=f_j^Nf_i^Ns_i=f_j^Nf_i^Ns_j=f_i^Nh_j\]
where the second equality follows from the hypothesis that each of pair of sections agrees on the overlap. Just as above, the $f_i^N$ generate the unit ideal
and so we have
\[1=\sum_i e_if_i^N\]
for some $e_i\in R$ (geometrically, this is simply a partition of unity). Let us consider the section $s\in\mathcal{O}(X)$ defined by
\[s=\sum_i e_i h_i.\]
Let us show that this is the desired global section, i.e. that it restricts to each $X_{f_i}$ as $s_i$. Note that on $X_{f_i}$ we have that
\[f_i^Ns=\sum_j f_i^Ne_jh_j=\sum_je_jf_j^Nh_i=h_i=f_i^Ns_i.\]
But in this localization $f_i$ is a unit, and hence cancelling $f_i^N$ from both sides we find that $s|_{X_{f_i}}$, we do indeed get that $s=s_i$.


\subsection*{Problem 5}

Let $(X,\mathcal{O}_X)$ be the usual affine scheme $X=\Spec \Z$. Consider the quotient topology $Y$ where the points $(2)$ and $(3)$ are identified, and let
$\phi$ be the natural projection. Since $\phi$ is continuous, we may construct the direct image sheaf $\mathcal{O}_Y=\phi_*\mathcal{O}_X$. Hence we
obtain a ringed space $(Y,\mathcal{O}_Y)$. We claim that this is not a locally ringed space - to see this, we must compute the stalks of $\mathcal{O}_Y$.
Note first that the (basis of) opens in $Y$ containing $(\bar 2)\equiv \phi(2)=\phi(3)$ are of the form $D(x)$ where $x\notin (\bar 2)$. At the level of the sheaf
we see that:
\[\mathcal{O}_Y(D(x))=\mathcal{O}_X(\phi^{-1}(D(x)))=\mathcal{O}_X(D(\phi^{-1}(x)))=\mathcal{O}_X(D(x)),\]
where we have written by abuse of notation $\phi^{-1}(x)=x$ (as we are excluded the trouble points). 
Now the stalk is the disjoint union of these opens, together with their sections in $\mathcal{O}_Y$, modulo sections that become equal on some small enough opens.
Consider two opens $U=D(x_1),V=D(x_2)\subset Y$ containing $(\bar 2)$ with intersection $W=U\cap V=D(x_1x_2)$. Using the above properties of $\mathcal{O}_Y$
we see that $\mathcal{O}_Y(U)=\Z_{x_1},\mathcal{O}_Y(V)=\Z_{x_2},$ and $\mathcal{O}_Y(W)=\Z_{x_1x_2}$. For a section $f\in\mathcal{O}_Y(U)$ and a section
$g\in\mathcal{O}_Y(V)$ to restrict to the same section $h\in\mathcal{O}_Y(W)$ would require that $\psi(f)-\psi'(g)$ - with $\psi$ and $\psi'$ the localization maps -
be annihilated by an element of the multiplicative set generated by $x_1x_2$. As $\Z$ is an integral domain, 
we see that this is possible only if $\psi(f)=\psi'(g)$.
Now suppose $f=a/c$ and $g=b/d$ for $a,b\in\Z$ and $c,d$ in the respective multiplicative subsets; then, since $\psi,\psi'$ are injections, we must have that $ad=bc$.
We can assume that $a$ and $c$ are relatively prime as are $b$ and $d$ and hence we see that $a$ must divide $b$ and $b$ must divide $a$, and similarly for $d$ and $c$,
so $a=b$ and $c=d$. Hence $f=g$, i.e. the equivalence relation on the stalk becomes trivial and we are left with only a union $\cup_{\fr p}\Z[\fr p^{-1}]$ over all
the primes of $\Z$ except for $2$ and $3$ (due to the nature of the opens containing $(\bar 2)$). But this yields a ring with two maximal ideals $(2)$ and $(3)$.
Hence we see that the stalk $\mathcal{O}_{Y,(\bar 2)}$ is not local, and that $(Y,\mathcal{O}_Y)$ is a ringed space but not a locally ringed space.


\subsection*{Problem 6}

Let $(X,\mathcal{O}_X)$ be the locally ringed space given by $X=\Spec\Z_{(2)}$ with $\mathcal{O}_X$ the constant sheaf determined by $\Z_{(2)}$
and $(Y,\mathcal{O}_Y)$ be the locally ringed space given by $\Spec \C[ [t]]$ with $\mathcal{O}_Y$ the constant sheaf determined by $\C[ [t]]$.
Note that these are indeed locally ringed spaces as the stalks of these sheaves are local. We check this for $\Z_{(2)}$ and the result follows
similarly for $\C[ [t]]$. The spectrum of $\Z_{(2)}$ is given $\Spec \Z_{(2)}=\{(0),(2)\}$ where topologically the point $(0)$ is open along with
$\varnothing$ and $\Spec \Z_{(2)}$. The stalk $\mathcal{O}_{X,(2)}$ is hence just $\mathcal{O}_{X}(\Spec\Z_{(2)})=\Z_{(2)}$ (by connectedness).
For $\mathcal{O}_{X,(0)}$, note that the two open sets containing $(0)$ are $(0)$ and $\Spec\Z_{(2)}$, but the ring above each of these is $\Z_{(2)}$
and hence so is the stalk at $(0)$.

Now consider the map of rings $\psi:\Z_{(2)}\hookrightarrow\C[ [t]]$ and the induced map on spectra $\phi:\Spec\C[ [t]]\to\Spec\Z_{(2)}$.
As $\psi$ is simply the inclusion, $\phi$ takes $(0)\mapsto (0)$ and $(t)\mapsto (0)$, which is continuous.
Now consider the morphism $\Phi$ of ringed spaces from $(X,\mathcal{O}_X)$ to $(Y,\mathcal{O}_Y)$ given by the continuous map
$\phi$ defined as above and the morphism of sheaves $\phi^\#:\mathcal{O}_Y\to\phi_*\mathcal{O}_X$ taking $\mathcal{O}_Y(V)\mapsto \mathcal{O}_X(\phi^{-1}V)$
for every open $V$ in $Y$. Taking direct limits, we obtain the induced map on stalks (at a point $p$) $\phi_p:\Z_{(2)}\to\C[ [t]]$, which is clearly
not a local homomorphism, as the inverse image of the maximal $(t)$ yields $(0)$ instead of $(2)$.

\subsection*{Problem 7}

Recall that a space is quasicompact if every open cover has a finite subcover. Let us show that $\Spec R$ is
a quasi-compact topological space for any $R$. Let $\Spec R=\cup_iX_i$ be an open cover. As the basis for the topology of $\Spec R$
is given by opens of the form $D(f), f\in R$, it is clear that we may refine the cover $X_i$ to some union of standard opens
$\Spec R=\cup_\alpha D(f_\alpha)$ for some $f_\alpha\in R$. But $\cup_\alpha D(f_\alpha)$ covers $\Spec R$ if and only if no
prime contains all the $f_\alpha$ (right-to-left is obvious, left-to-right follows from the fact that the points of $D(f_\alpha)$ are in
correspondence with the primes of $R_{f_\alpha}$) which occurs if and only if the $f_\alpha$
span the unit ideal (one of seeing this is via Zorn's lemma).
Hence we see that the $D(f_\alpha)$
cover $\Spec R$ if and only the $f_\alpha$ span the unit ideal. But now we can simply choose the finite set of $f_i$ that generate the element
1 and construct a finite cover of $X$ given by the union of those specific $X_{f_i}$. Hence we see that $\Spec R$ is quasicompact.
Of course, as every standard open is of the form $\Spec R_f$, we find that every standard open is in fact quasicompact as well.


\end{document}
