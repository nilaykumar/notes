\documentclass{../../mathnotes}

\usepackage{enumerate}
\usepackage{todonotes}
\usepackage{tikz-cd}

\title{Commutative Algebra: Problem Set 4}
\author{Nilay Kumar}
\date{Last updated: \today}


\begin{document}

\maketitle


\subsection*{Problem 5}

Let $A=k[x,y,z]/(x^2y^2z^2, x^3y^2z)$. We wish to compute the dimension of $A$ at the maximal ideal $(x,y,z)$.
Since $x^2y^2z^2$ is a nonzerodivisor in $k[x,y,z]_{(x,y,z)}$, quotienting out by it yields a ring with dimension two (by lemma 46).
Next note that after quotienting out by $x^3y^2z$ we still have a chain of primes $(x)\subset (x,y)\subset (x,y,z)$ where the inclusions
are still proper (we can't get a $z$ in $(x,y)$ for example, because all we can do with $x^2y^2z^2$ is dock powers) and hence the dimension is two.


\subsection*{Problem 6}

Let $A=k[x,y,z]/(x^3-y^2,x^5-z^2,y^5-z^3)$. We wish to compute the dimension of $A$ at $(x,y,z)$. Consider first quotienting by $x^3-y^2$, which
is clearly prime (and a nonzerodivisor), and hence yields a domain, docking the dimension from 3 to 2. Next consider quotienting by $x^5-z^2$, which
is no longer prime, but is still a nonzerodivisor and hence yields a ring of dimension 1. Now if we quotient out the last generator, we obtain a ring in which


\subsection*{Problem 7}

Let $k$ be a field. Let $f\in k[x,y]$ be a polynomial and $a,b\in k$ be elements such that $f(a,b)=0$.
Let $\fr m=(x-a,y-b)$ be the corresponding maximal ideal in the ring $A=k[x,y]/(f)$. By construction, $A_\fr m$ is a local ring.
We wish to check that it is regular, i.e. that the maximal ideal has exactly $\dim A_\fr m=1$ generators. The number of generators is given
by $\dim_{A/\fr m}\fr m/\fr m^2=\dim_k\fr m/\fr m^2$. In this case it's clear that $\fr m$ is the ideal of all polynomials with root
at $(a,b)$, i.e. every element of $\fr m$ has a finite Taylor series about $(a,b)$ given by $\sum_{i+j\geq 1}c_{ij}(x-a)^i(y-b)^j$ for $c_{ij}\in k$.
The finiteness is obvious since we are simply dealing with polynomials.
It follows, then, that elements of $\fr m^2$ look like $\sum_{i+j\geq 2}d_{ij}(x-a)^i(y-b)^j$ for $d_{ij}\in k$, as we have squared away linear terms.
The quotient $\fr m/\fr m^2$ is then composed of elements of the form $c_{10}(x-a)+c_{01}(y-b)$, which is 2-dimensional as a vector space and isomorphic to $k^2$;
we must be careful to note, however that $(f)$
is zero in our original ring and hence its image in $\fr m/\fr m^2$, $(\partial_x f(a,b)(x-a)+\partial_y f(a,b)(y-b))$ must also be zero, which is a dimension one subspace. Hence we see that
the dimension of $\fr m/\fr m^2$ must be $2-1=1$, as desired.

%In this case,
%it's clear that $\dim A_\fr m$ is one (via theorems from class) and hence the ideal $(x-a,y-b)$ must collapse to a principal
%ideal in $A_\fr m$. In other words, there must be some way of solving $f$ to write $x-a$ in terms of $y-b$ or vice versa. Of
%course, this is formally possible via the implicit function theorem as long as one of $\partial f/\partial a,\partial f/\partial b$
%is non-zero, as desired.

\subsection*{Problem 9}

Consider $f=xy^2+x^2y=xy(y+x)$, which has zeros at $x=y=0$ and at $x=y$. We compute $\partial_xf=y^2+2xy$ and $\partial_yf=2yx+x^2$; the only singular point, then,
is $(0,0)$. Next consider $f=x^2-2x+y^3-3y^2+3y$; we compute $\partial_xf=2x-2$ and $\partial yf=3y^2-6y+3=3(y-1)^2$. The singular point is then $(1,1)$,
as $f(1,1)=0$. Finally, consider $f=x^n+y^n+1$, which has $\partial_xf=nx^{n-1}, \partial_yf=ny^{n-1}$. These derivatives are never zero except at $(0,0)$, which
is not a root of $f$, and hence $f$ has no singular points.

\subsection*{Problem 10}

Let $k$ an algebraically closed field. Let $f\in k[x,y]$ be a squarefree polynomial of degree 1. In other words, $f=ax+by+c$.
Clearly $f$ can only have singular points if it is a constant, which contradicts the degree being 1, and hence has no singular points.
Next consider degree 2: $f=ax^2+bxy+cy^2+dx+ey+f$. In this case, solving for the singular points involves simultaneously solving a linear
system, which yields a single point. We can assume that the system is non-degenerate because of the squarefree condition; if it were, it is
straightforward but tedious to show that $f$ can then be written as $(\sqrt{a}x\pm\sqrt{b}y+d/2)^2$, a square. Finally, we want our singular point
to land on the curve - this can always be done by adjusting $f$.

For degree 3, we solve two quadratics, and hence we expect 4 singular points, and again, not an infinite number due to the squarefree condition. In this way, we might
guess that in general for degree $d$ we have, at maximum, $(d-1)^2$ singular points.





\end{document}
