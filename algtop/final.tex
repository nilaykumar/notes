\documentclass{../mathnotes}

\usepackage{tikz-cd}
\usepackage{todonotes}

\title{Introduction to Algebraic Topology Final: Take-Home Portion}
\author{Nilay Kumar}
\date{Last updated: \today}


\begin{document}

\maketitle

\begin{prop}\hfill
    \begin{enumerate}[(a)]
        \item Let $X$ be a finite CW complex and $p:\tilde X\to X$ be an $n$-sheeted covering space.
            Prove that $\chi(\tilde X)=n\chi(X)$.
        \item Let $\Sigma_g$ denote the closed orientable surface of genus $g$. Give necessary and
            sufficient conditions for $\Sigma_h$ to cover $\Sigma_g$.
    \end{enumerate}
\end{prop}
\begin{proof}\hfill
    \begin{enumerate}[(a)]
        \item Given a cell structure on $X$, it suffices to show that $\tilde X$ can be given a cell structure with
            $n$ times the number of $i$-cells as $X$, for all $i$. We proceed by induction. For the base case,
            it's clear that we can find $nc_0(X)$ 0-cells on $\tilde X$ by applying $p^{-1}$ to the 0-cells on $X$.
            Now, as the induction
            step, we assume that $\tilde X$ has $nc_{i-1}(X)$ $(i-1)$-cells. Then, any $i$-cell in $X$ is attached via
            a map $\phi_\alpha: D^i_\alpha\to X$. Contractibility of the disk allows us to use the lifting criterion
            (Hatcher Proposition 1.33) and thus there exists a lift $\tilde\phi_\alpha:D^i_\alpha\to \tilde X$ for each
            starting point of the lift. Since this is an $n$-sheeted cover there are $n$ such starting points by the
            induction step, and hence there are $n$ such lifts. This yields the desired cell structure (note that the
            local homeomorphisms ensure that this cell structure covers the whole space). 
        \item The necessary condition follows immediately from the previous part:
            \begin{align*}
                2-2h &= n(2-2g)\\
                h &= n(g-1)+1.
            \end{align*}
            for some $n\in\N$, which is the number of sheets in the desired covering. Sufficiency is easy to see via
            Hatcher's example 1.41. Let $h=n(g-1)+1$. Deform $\Sigma_h$ into an $n$-spoked star shape with one of the
            holes at the center and the other split evenly along the spokes. Equipping $\Sigma_h$ with the action of
            $\Z_n$ (which is clearly a covering space action) and quotienting, we obtain $\Sigma_h/\Z_n=\Sigma_g$,
            and hence we have a covering $\Sigma_h\to \Sigma_g$, as desired.
    \end{enumerate}
\end{proof}

\newpage

\begin{prop}
    Let $K$ be a knot in $S^3$, i.e. $K$ is the image of a smooth embedding $S^1\to S^3$.
    \begin{enumerate}[(a)]
        \item Compute the homology of $S^3-K$.
        \item Compute $\pi_1(S^3-K_1)$ and $\pi_1(S^3-K_2)$ for $K_1$ the unknot and $K_2$
            the trefoil knot and conclude that they are distinct knots.
    \end{enumerate}
\end{prop}
\begin{proof}\hfill
    \begin{enumerate}[(a)]
        \item As $K$ is the image of a smooth embedding, we can continuously deform it to $S^1$. We then proceed by
            using Mayer-Vietoris on the 3-sphere $X=S^3$. Let $A$ be a toroidal neighborhood of $K$, and let $B=S^3-K$.
            Clearly $A$ is homotopic to $S^1$ and $A\cap B$ is homotopic to $T^2$. Hence we have nontrivial homologies
            $H_0(A)=\Z,H_1(A)=\Z$, $H_0(A\cap B)=\Z,H_1(A\cap B)=\Z^2,H_2(A\cap B)=\Z$, and $H_0(X)=\Z,H_3(X)=\Z$. As the interiors of $A$ and $B$ cover $X$, we obtain
            a long exact sequence
            \begin{equation*}
                \begin{tikzcd}
                    0\ar{r} & 0 \oplus H_3(S^3-K)\ar{r} & \Z\ar[swap]{dll}{\partial}\\
                    \Z\ar{r} & 0\oplus H_2(S^3-K)\ar{r} & 0\ar[swap]{dll}{\partial}\\
                    \Z^2\ar{r} & \Z\oplus H_1(S^3-K)\ar{r} & 0.
                \end{tikzcd}
            \end{equation*}
            Clearly $H_0(S^3-K)=\Z$ and $H_1(S^3-K)=\Z$. Next note that the map $\Z\to0\oplus H_2(S^3-K)$
            takes $x\mapsto (x,-x)$ and hence is zero (this can also be seen by noting that the 3-cell in $S^3-K$
            decomposes into the 3-cell of the toroidal neighborhood and the 3-cell of the complement, and hence
            the former is taken via the boundary map to the bounding torus). This implies that $H_2(S^3-K)=0$ by exactness
            and that the first $\partial$ map is the identity, which in turn forces $H_3(S^3-K)=0$.
            Higher homology groups are of course zero for dimensional reasons.
        \item Let us first compute $\pi_1(S^3-K_1)$: the complement of the unknot. We claim first that $\pi_1(S^3-K_1)\cong\pi_1(\R^3-K_1)$.
            This follows from van Kampen's theorem: write $S^3-K_1$ as the union of $\R^3-K_1$ and the open ball $B$
            formed by the complement in $S^3$ of a large closed ball in $\R^3$ containing $K_1$. As both $B$ and
            $B\cap(\R^3-K_1)$ are simply-connected, van Kampen's theorem yields the desired isomorphism. Finally
            note that $\R^3-K_1$ deformation retracts onto $S^1\vee S^2$ as drawn -- this follows by homeomorphically
            thickening an $\epsilon$-neighborhood of $S^1\vee S^2$ until it becomes all of $\R^3-K_1$.
            Hence $\pi_1(S^3-K_1)=\pi_1(S_1)*\pi_1(S^2)=\Z$.

            Next consider $\pi_1(S^3-K_2)$: the complement of the trefoil knot. We claim, following Hatcher's Example 1.24,
            that $S^3-K_2$ deformation retracts onto a 2-dimensional complex $X=X_{2,3}$ homeomorphic to
            the quotient space of a cylinder $S^1\times I$ under the identifications $(z,0)\sim(e^{\pi i/2}z,0)$ and $(z,1)\sim(e^{2\pi i/3}z,1)$.
            Assuming this for the moment, we can apply van Kampen to the top and bottom halves of this cylinder
            (more precisely open neighborhoods deformation retracting to the halves), each of which is a mapping
            cylinder over $S^1$ (of $z\mapsto z^2,z\mapsto z^3$) having fundamental group $\Z$. Similarly the
            intersection has fundamental group $\Z$. Since a loop in the overlap of the two halves represents
            twice the generator of one end and thrice the generator of the other, we find via van Kampen that
            $\pi_1(S^3-K_2)=\langle a,b\mid a^2=b^3\rangle.$
            
            Let us now justify the assertion above. Consider the decomposition of $S^3$ into two solid tori
            $S^1\times D^2$ and $D^2\times S^1$, the result of regarding $S^3$. This is a result of the equality
            $\partial D^4=\partial(D^2\times D^2)=\partial D^2\times D^2\cup D^2\times\partial D^2$.
            As the trefoil knot can be placed on the torus $S^1\times S^1$ (as the torus knot $K_{2,3}$), we see that visually, the first solid
            torus can be viewed as filling said torus and the second is ``perpendicular'' to the first, representing the
            complement of the first solid torus in $S^3$. 
            We note that in the first solid torus, if we fix a meridian circle $\{x\}\times\partial D^2$, the
            trefoil knot intersects it in two antipodal points. Drawing a diameter through the circle perpendicular
            to the line connecting the two points, and following this line for every meridian circle, we obtain
            deformation retraction onto one half of the mapping cylinder described above.
            The second half is obtained similarly from the second torus, and the two deformation retracts can
            be made to agree on their overlap $S^1\times S^1-K_2$.
    \end{enumerate}
\end{proof}

\begin{prop}
    Show that if $f:\RP^{2n}\to X$ is a covering map of a CW complex $X$ then $f$ is a homeomorphism.
\end{prop}

\begin{proof}
    We assume $X$ is path-connected, as otherwise the statement does not necessarily hold.
    Equip $\RP^{2n}$ with the usual CW structure with a cell in every dimension from $0$ to $2n$.
    Since $\RP^{2n}$ is a finite CW complex, it must be compact
    (as it is comprised of closed disks) and hence $X$ (via surjectivity of $f$ due to path-connectedness)
    must be compact
    as well. Applying Hatcher's Proposition A.1, which states that a compact subspace of a CW complex
    is contained in a finite subcomplex, we find that $X$ is finite.\footnote{
        Alternatively, one can just apply the properties of the covering space map. There
        exists an open cover $X=\cup X_\alpha$ such that $f^{-1}(X_\alpha)$ is a disjoint union
        of open sets in $\RP^{2n}$, each of which is mapped by $f$ homeomorphically onto $X_\alpha$.
        Thus, since $\RP^{2n}$ is a finite CW complex, $X$ must be finite as well.
    }
    We can now apply
    the results of the first problem to deduce that $\chi(\RP^{2n})=k\chi(X)$ for some $k\in\N$.
    But recall that the Euler characteristic is given by the alternating sum of the number of cells
    in each dimension, whence $\chi(\RP^{2n})=1$. This forces $k=\chi(X)=1$, implying that $\RP^{2n}$ is a
    single-sheeted cover of $X$, i.e. $\RP^{2n}\cong X$.
\end{proof}




\end{document}
