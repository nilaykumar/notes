\documentclass{../../mathnotes}

\usepackage{tikz-cd}
\usepackage{todonotes}

\title{Introduction to Algebraic Topology PSET 6}
\author{Nilay Kumar}
\date{Last updated: \today}


\begin{document}

\maketitle

\begin{prop}
    Hatcher exercise 1.3.10
\end{prop}
\begin{proof}
    See drawing below.
\end{proof}

\begin{prop}
    Hatcher exercise 1.3.18
\end{prop}
\begin{proof}
    Normality of an abelian covering $p:\tilde X\to X$ implies that $G(\tilde X)=\pi_1(\tilde X,\tilde x_0)/H$,
    where $H=p_*(\pi_1(\tilde X,\tilde x_0))$. Then $H\supset[\pi_1(\tilde X,\tilde x_0),\pi_1(\tilde X,\tilde x_0)]$.
    Now, pick $\tilde X$ such that $H=[\pi_1(\tilde X,\tilde x_0),\pi_1(\tilde X,\tilde x_0)]$.
    By the Galois classification of covering spaces, then, $\tilde X$ covers all abelian covers of $X$, because any
    abelian cover will have associated subgroup containing the commutator subgroup, and $\tilde X$ is
    unique up to isomorphism. For $S^1\vee S^1$, the universal abelian covering space must have deck
    group $\Z\oplus\Z$, and hence we can take the corresponding lattice $\Z^2$ as the covering space
    where we project the vertices to the wedged point, and the horizontal and vertical directions
    to $a$ and $b$ respectively. Similarly for $S^1\vee S^1\vee S^1$ -- the universal abelian covering space is
    the lattice $\Z^3$.
\end{proof}

\begin{prop}
    Let $n\geq 3$ be a natural number. Prove that the free group on two generators has a subgroup that is isomorphic
    to the free group on $n$ generators.
\end{prop}
\begin{proof}
    Let $\tilde X$ the wedge sum of a circle to $n$ circles at $n$ distinct points. The symmetry group
    of $\tilde X$ is clearly $\Z_n$, which acts as a covering space action on $\tilde X$. Hence the quotient,
    $X=\tilde X/\Z_n$, which is simply $S^1\vee S^1$, is covered normally by $\tilde X$. Since $\tilde X$
    is homotopy equivalent to $\vee^n S^1$ (with the free product taken $n$ times), we find that
    $*^n\Z\hookrightarrow\Z*\Z$, as the fundamental group of the covering space must inject
    into the fundamental group of the base space.
\end{proof}

\begin{prop}
    Hatcher exercise 1.3.20
\end{prop}
\begin{proof}
    Consider the three-sheeted cover of the Klein bottle by the Klein bottle as depicted below. Recall that
    the Klein bottle has fundamental group $\langle a,b \mid abab^{-1}\rangle$. We see that the lift of $b$
    at $x$ is a loop, whereas at $y$ it is simply a path from $y$ to $z$. Hence $b\in p_*(\pi_1(X,x))$ but
    $b\notin  p_*(\pi_1(X,y))$ by Hatcher proposition 1.31, and thus the covering is not normal.

    Next consider the universal cover $\R^2$ of the Klein bottle. If we tile the plane by squares as in Hatcher
    example 1.42. Consider the subgroup of translations of the plane isomorphic to $\Z\oplus\Z$ generated by
    $\langle a^3, a^2b^2\rangle$ as depicted in the diagram below. It is clear that $\R^2/\Z\oplus\Z\cong T^2$,
    as a six-fold covering of the Klein bottle. Algebraically, we can confirm that this covering is not normal;
    consider conjugation by $b$. The relation $abab^{-1}$ implies that $bab^{-1}=a^{-1}$, and hence
    $ba^2b^2b^{-1}=a^{-2}b^2$, which is not an element of $\langle a^3,a^2b^2 \mid abab^{-1}\rangle$
    (this is the image of the fundamental group of the torus in the fundamental group of the Klein bottle).
    We can check this topologically by noting that the lift of $a^2b^2$ at the point $x$ is a loop, but
    the lift at the point $y$ is just a path to $z$ (which is not a loop on the torus).
\end{proof}

\begin{prop}
    Hatcher exercise 1.3.24
\end{prop}
\begin{proof}
    Note carefully that here we are not provided with the semilocally simply-connected condition (on $X/G$), and hence we
    cannot directly use the Galois correspondence provided by Hatcher theorem 1.38. Instead, we can only
    use the path-connectedness and local path-connectedness of $X/G$ (which follows from the same for $X$, due
    to the nature of the quotient topology).
    \begin{enumerate}[(a)]
        \item We have a sequence of injections
            \begin{equation*}
                \begin{tikzcd}
                    \pi_1(X)\arrow[hook]{r}&\pi_1(Y)\arrow[hook]{r}&\pi_1(X/G)
                \end{tikzcd}
            \end{equation*}
            and since $\pi_1(X)\lhd \pi_1(X/G)$, we find that $\pi_1(X)\lhd \pi_1(Y)$, i.e. $X\to Y$ is a normal covering.
            Denote by $H\leq G$ the deck group $G(X\to Y)$. Then, if we write out
            \begin{equation*}
                \begin{tikzcd}
                    X\arrow{r}{p_1}& Y\arrow{r}{p_2}& X/G\\
                    X\arrow{r}{q_1}& X/H\arrow{r}{q_2}& X/G
                \end{tikzcd}
            \end{equation*}
            with $p\equiv p_2\circ p_1=q_2\circ q_1\equiv q$ (by construction) we find that
            \[ H\cong \frac{\pi_1(Y)}{p_{1*}(\pi_1(X))}\cong \frac{p_{2*}(\pi_1(Y))}{p_*(\pi_1(X))} \]
            but also that
            \[ H\cong \frac{\pi_1(X/H)}{q_{1*}(\pi_1(X))}\cong\frac{q_{2*}(\pi_1(X/H))}{q_*(\pi_1(X))}. \]
            This implies that $p_{2*}(\pi_1(Y))\cong q_{2*}(\pi_1(X/H)),$ which completes the proof by Hatcher proposition 1.37.
            %Let $Y$ be a path-connected space between $X$ and $X/G$. Let $H$ be the group of deck transformations
            %$G(X\to Y)$.  Clearly $H\leqslant G$, as any deck transformation of $X$ over $Y$ can be extended to a
            %deck transformation of $X$ over $X/G$ via composition with $p_2$. Hence we have two sequences of coverings
            %\begin{equation*}
            %    \begin{tikzcd}
            %        X\arrow{r}{p_1}& Y\arrow{r}{p_2}& X/G\\
            %        X\arrow{r}{q_1}& X/H\arrow{r}{q_2}\arrow{u}{\phi}& X/G
            %    \end{tikzcd}
            %\end{equation*}
            %where $p_2\circ p_1=q_2\circ q_1$ by construction.
            %If we define $\phi$ as taking $\bar x\in X/H$ (the orbit containing $x\in X$) to $p_1(x)$, we obtain
            %a covering space isomorphism. It is clear that this map is well-defined and a homeomorphism (as $p_1,q_1$ share the same
            %deck transformation) and $p_2\circ\phi=q_2$ because $p_2(\phi(\bar x))=(p_2\circ p_1)(x)$ and
            %$q_2(\bar x)=(q_2\circ q_1)(x)$. 
            %Let $Y$ be a path-connected space between $X$ and $X/G$, i.e. $X\xrightarrow{p_1} Y\xrightarrow{p_2} X/G$,
            %where the maps are covering maps. Let $H$ be the group of deck transformations $G(X\xrightarrow{p_1}Y)$.
            %Clearly $H\leqslant G$, as any deck transformation of $X$ over $Y$ can be extended to a deck transformation
            %of $X$ over $X/G$ via composition with $p_2$. Hence we have two sequences of coverings
            %\begin{equation*}
            %    \begin{tikzcd}
            %        X\arrow{r}{p_1}& Y\arrow{r}{p_2}& X/G\\
            %        X\arrow{r}{q_1}& X/H\arrow{r}{q_2}& X/G
            %    \end{tikzcd}
            %\end{equation*}
            %that we wish to show are isomorphic. We now assume that $X/G$ is semilocally simply-connected (as written,
            %Hatcher exercise 1.3.6 breaks this exercise, and more generally, there would be no way to cite the Galois
            %correspondence\footnote{I've checked this with Professor Hom}). Then for $Y\cong X/H$ as coverings of $X/G$,
            %it suffices to show that $p_{2*}(\pi_1(Y))$ is conjugate to $q_{2*}(\pi_1(X/H))$ as subgroups of $\pi_1(X/G)$.
            %Note that it suffices to prove that $\pi_1(X/G)/p_{2*}(\pi_1(Y))\cong \pi_1(X/G)/q_{2*}(\pi_1(X/H))$ (because
            %such an isomorphism gives an automorphism $\phi:\pi_1(X/G\to \pi_1(X/G)$ with $\phi($<++>.
        \item
            %Consider the following diagram.
            %\begin{equation*}
            %    \begin{tikzcd}
            %        X\arrow{r}{p_1}&X/H_1\arrow{r}{p_2}\arrow{d}{\phi}&X/G\\
            %        X\arrow{r}{q_1}&X/H_2\arrow{r}{q_2}&X/G
            %    \end{tikzcd}
            %\end{equation*}
            %Let us first show that if there exists a covering space isomorphism $\phi$, then
            %$H_1$ and $H_2$ are conjugate in $G$. Pick an $x\in X$ along with a $y\in X$ such that
            %$q_1(y)=\phi(p(x))$. We claim that $g$ 
        \item 
    \end{enumerate}
\end{proof}

\end{document}
