\documentclass{../../mathnotes}

\usepackage{tikz-cd}
\usepackage{todonotes}

\title{Introduction to Algebraic Topology PSET 7}
\author{Nilay Kumar}
\date{Last updated: \today}


\begin{document}

\maketitle

\begin{prop}
    Hatcher exercise 2.1.1
\end{prop}
\begin{proof}
    The $\Delta$-complex obtained from the 2-simplex $[v_0,v_1,v_2]$ with edges
    $[v_0,v_1]$ and $[v_1,v_2]$ identified with order preserved is just a cone $C(S^1)$.
\end{proof}

\begin{prop}
    Hatcher exercise 2.1.5
\end{prop}
\begin{proof}
    Let us compute the simplicial homology groups of the Klein bottle using the $\Delta$-complex 
    structure $K$ described on page 102 of Hatcher. It's clear that there is one 0-simplex ($v$), three
    1-simplices ($a,b,c$), and two 2-simplices ($U,L$). We obtain the chain complex
    \begin{equation*}
        \begin{tikzcd}
            0\arrow{r}{\partial_3}&\Z^2\arrow{r}{\partial_2}&\Z^3\arrow{r}{\partial_1}&\Z\arrow{r}{\partial_0}&0
        \end{tikzcd}
    \end{equation*}
    with $\partial_0v=0$, $\partial_1a=\partial_1b=\partial_1c=0$, and $\partial_2U=a+b-c,\partial_2L=a-b+c$.
    The Klein bottle is path-connected and hence $H^\Delta_0(K)=\Z$. Next note that $\ker\partial_2=0$ as
    no linear combination of $U$ and $L$ will map to 0 under $\partial_2$, and thus $H^\Delta_2(K)=0$.
    Computing $H^\Delta_1(K)$ is a bit more tedious: $H_1^\Delta(K)=\ker \partial_1/\text{im }\partial_2
    =\langle a,b,c\rangle/\langle a+b-c,a-b+c\rangle$. We can simplify this as follows:
    \begin{align*}
        \frac{\langle a,b,c\rangle}{\langle a+b-c,a-b+c\rangle}&=\frac{\langle a-b+c,b,c\rangle}{\langle a+b-c,a-b+c\rangle}\\
        &=\frac{\langle b,c\rangle}{\langle 2b-2c\rangle}
        =\frac{\langle b-c,c\rangle}{\langle 2(b-c)\rangle}
        =\frac{\langle d,c\rangle}{\langle 2d\rangle}\\
        &=\Z\oplus\Z_2.
    \end{align*}
    Hence the non-trivial simplicial homology groups are $H^\Delta_0(K)=\Z$ and $H^\Delta_1(K)=\Z\oplus\Z_2.$
\end{proof}

\begin{prop}
    Find a $\Delta$-complex structure for the orientable surface of genus two and compute
    the simplicial homology groups of this $\Delta$-complex.
\end{prop}
\begin{proof}
    Consider the $\Delta$-complex $\Sigma_2$ obtained by taking an octagon (with sides identified
    appropriately) and drawing segments from a fixed vertex to the other vertices (see figure). This
    yields one 0-simplex ($v$), nine 1-simplices ($a,\ldots, i$), and six 2-simplices ($\alpha,\ldots,\zeta$)
    and the chain complex
    \begin{equation*}
        \begin{tikzcd}
            0\arrow{r}{\partial_3}&\Z^6\arrow{r}{\partial_2}&\Z^9\arrow{r}{\partial_1}&\Z\arrow{r}{\partial_0}&0.
        \end{tikzcd}
    \end{equation*}
    Clearly $H^\Delta_0(\Sigma_2)=\Z$. More tedious is $H_1^\Delta(\Sigma_2)$:
    \begin{align*}
        H^\Delta_1(\Sigma_2)&=\frac{\ker\partial_1}{\text{im }\partial_2}\\
        &=\frac{\langle a,b,c,d,e,f,g,h,i\rangle}{\langle d+c-e,f+d-e,g+c-f,g+b-h,h+a-i,a+b-i\rangle}\\
        &=\frac{\langle b,c,d,e,f,g,h,i\rangle}{\langle c+d-e,d-e+f,c-f+g,b+g-h,h-b\rangle}\\
        &=\frac{\langle b,c,d,e,f,g,i\rangle}{\langle c+d-e,d-e+f,c-f+g,g\rangle}
        =\frac{\langle b,c,d,e,f,i \rangle}{\langle c+d-e,d-e+f,c-f\rangle}\\
        &=\frac{\langle b,c,d,e,i\rangle}{\langle c+d-e,c+d-e\rangle}=\frac{\langle b,c,d,e,i\rangle}{\langle c+d-e\rangle}\\
        &=\Z^4.
    \end{align*}
    Finally, $H_2^\Delta(\Sigma_2)=\ker\partial_2/\text{im }\partial_3=\ker\partial_2$. It is straightforward to check that
    $\ker\partial_2=\langle\alpha-\beta-\gamma+\delta+\varepsilon-\zeta\rangle$ (one can read this off of the figure via the
    the orientations of the 2-simplicies). Hence the non-trivial simplicial homology groups are $H^\Delta_0(\Sigma_2)=\Z,
    H^\Delta_1(\Sigma_2)=\Z^4,H^\Delta_2(\Sigma_2)=\Z$.
\end{proof}


\begin{prop}
    Find a $\Delta$-complex structure for $S^3$ and compute the simplicial homology groups
    of this $\Delta$-complex.
\end{prop}
\begin{proof}
    Consider the $\Delta$-complex of $S^3$ given by identifying (in the obvious way) all the faces of two 3-simplices $T_1$ and $T_2$.
    We obtain the chain complex
    \begin{equation*}
        \begin{tikzcd}
            0\arrow{r}{\partial_4}&\Z^2\arrow{r}{\partial_3}&\Z^4\arrow{r}{\partial_2}&\Z^6\arrow{r}{\partial_1}&\Z^4\arrow{r}{\partial_0}&0.
        \end{tikzcd}
    \end{equation*}
    Clearly $H^\Delta_0(S^3)=H^\Delta_1(S^3)=\Z$. Next, note that
    \[\partial_1a=y-x,\partial_1b=z-y,\partial_1c=z-x,\partial_1d=w-x,\partial_1e=w-y,\partial_1f=w-z,\]
    and
    \[\partial_2\alpha=a-d+e,\partial_2\beta=b-e+f,\partial_2\gamma=c-d+f,\partial_2\delta=a+b-c,\]
    and
    \[\partial_3T_1=\partial_3T_2=\alpha+\beta-\gamma-\delta.\]
    We can check that $\ker\partial_1$ is generated by $-c+d-f,-b+e-f,-a+d-e,a+b-c$, and that these are precisely generate
    $\text{im }\partial_2$. Hence $H^\Delta_1(S^3)=0$. Next, note that $\ker\partial_2$ is generated by $\alpha+\beta-\gamma-\delta$,
    which is precisely $\text{im }\partial_3$, and thus $H^\Delta_2(S^3)=0$ as well.
\end{proof}


\begin{prop}
    Hatcher exercise 2.1.8
\end{prop}
\begin{proof}
    Denote by $L$ the given $\Delta$-complex. Clearly $H^\Delta_0(L)=\Z$. To find the first homology group, note that
    there are two 1-simplices in $L$ that are always present, independent of $n$, call them $b$ and $c$. There are $n$
    other 1-simplices, denote them by $a_1,\ldots,a_n$. It is easy to see that since there are only two unique vertices,
    we must have
    \[\ker\partial_1=\langle a_i-a_{i+1},b,c\rangle\]
    for all $i$ except for $n$ (as otherwise we would obtain a linear dependence). Then, applying $\partial_2$
    to the 2-simplicies we obtain
    \[\text{im }\partial_2=\langle a_i-a_{i+1}+b, a_{i+1}-a_i+c,nb\rangle=\langle a_i-a_{i+1}+b,b+c,nb\rangle \]
    where again $i$ excludes $n$ due to the dependence (which we have explicitly written here as $nb=0$).
    If we now change basis of our free $\Z$-module $\ker\partial_1$ appropriately, we find that
    \[H_1^\Delta(L)=\frac{\langle b,c\rangle}{\langle b+c,nb\rangle}=\Z_n.\]
    Now, $H_2^\Delta=\ker\partial_2/\text{im }\partial_3$, and it is easy to see that both numerator and denominator
    are precisely $V_{i+1}-V_i+H_i-H_{i+1}$ where the $V$'s are the vertical 2-simplices, and the $H$'s are the
    the slanted 2-simplices. Hence $H_2^\Delta=0$. Finally, note that $H^\Delta_3(L)=\ker\partial_3$, which is of course 
    just the appropriate alternating sum of the 3-simplices, and hence $H^\Delta_3(L)=\Z$.
\end{proof}

\end{document}
