\documentclass{../../mathnotes}

\usepackage{tikz-cd}
\usepackage{todonotes}

\title{Introduction to Algebraic Topology PSET 5}
\author{Nilay Kumar}
\date{Last updated: \today}


\begin{document}

\maketitle

\begin{prop}
    Hatcher exercise 1.2.6
\end{prop}
\begin{proof}
    We follow the proof of Hatcher proposition 1.26. Suppose $Y$ is a space obtained
    from a path-connected subspace $X$ by attaching $n$-cells for a fixed $n\geq 3$.
    Note that each $n$-cell $e^n_\alpha$ is attached to $X$ via attaching maps
    $\phi_\alpha:S^{n-1}\to X$. If we choose in each $e^n_\alpha$ a point $y_\alpha$,
    and let $A=Y-\cup_\alpha\{y_\alpha\}$ and let $B=Y-X$. Note that $\pi_1(B)=0$, and
    hence by van Kampen's theorem applied to the cover $\{A,B\}$, we find that $\pi_1(Y)$
    is isomorphic to the quotient of $\pi_1(A)$ by the normal subgroup generated by the
    image of the map $\pi_1(A\cap B)\to\pi_1(A)$. Note, however, that $\pi_1(A\cap B)$
    can be computed by van Kampen's theorem to be the free product of $\pi_1(A_\alpha)$
    where $A_\alpha=A\cap B-\cup_{\beta\neq\alpha}e^n_\beta$. But $A_\alpha$ is simply
    a punctured $n$-cell, which is homotopy equivalent to $S^{n-1}$. Hence $\pi_1(A\cap B)=0$,
    and we see that the map $\pi_1(A\cap B)\to\pi_1(A)$ is the trivial map, and thus
    $\pi_1(Y)=\pi_1(A)=\pi_1(X)$, since $A$ deformation retracts to $X$ (in particular, the
    $n$-cells retract to the $S^{n-1}$ by which they are attached to $X$). This proves the claim.

    Consider now a discrete subspace $X\subset \R^n$. We claim that $Y=\R^n\setminus X$ is
    simply-connected. It is clear that $Y$ is path-connected; to show that the fundamental
    group is trivial, we construct open $n$-balls $B_i$ around each of the points of $X$
    (containing only one point of $X$). Then $Y$ deformation retracts onto $Y-\cup_i B_i$
    and since we can attach to $Y-\cup_i B_i$ $n$-balls to produce $\R^n$, the theorem above
    implies that for $n\geq 3$, $\pi_1(Y)=\pi_1(\R^n)=0$.
\end{proof}

\begin{prop}
    Problem 2
\end{prop}
\begin{proof}
    \hfill
    \begin{enumerate}[(a)]
        \item Recall that the stereographic projection gives a homeomorphism between
            the punctured sphere and $\R^2$. Hence, if we take an open cover of $S^2$
            to be two punctured spheres with holes at the north and south poles, van
            Kampen's theorem implies a surjection $\pi_1(\R^2)*\pi_1(\R^2)\twoheadrightarrow\pi_1(S^2)$.
            But $\R^2$ is contractible, and hence $\pi_1(\R^2)=0$. Such a surjection
            is then only possible if $\pi_1(S^2)=0$ as well. This generalizes to higher
            dimensions in exactly the same way, as the stereographic projection holds
            for $n$-spheres and $n$-planes, and $\pi_1(\R^n)=0$ for all $n\in\N$.
        \item The cell decomposition for $S^n$ is a 0-cell with an attached $n$-cell.
            As the 0-cell is a point, it is clearly path-connected, and hence by the
            previous exercise, we find that $\pi_1(S^n)=0$ for $n\geq 3$.
    \end{enumerate}
\end{proof}

\begin{prop}
    Hatcher exercise 1.3.2
\end{prop}
\begin{proof}
    This is clear. Take $\{U_\alpha\}$ and $\{V_\beta\}$ to be the chosen open covers of
    $X_1$ and $X_2$, respectively. Then $\{U_\alpha\times V_\beta\}$ furnishes an open cover of
    $X_1\times X_2$. If we let $\rho:\tilde X_1\times\tilde X_2\to X_1\times X_2$ be the
    natural projection, then $\rho^{-1}(U_\alpha\times V_\beta)=p_1^{-1}(U_\alpha)\times p_2^{-1}(V_\beta)$.
    This is, of course a disjoint union of subspaces that map homeomorphically down to
    $U_\alpha\times V_\beta$, by the fact that $\tilde X_1\to X_1$ and $\tilde X_2\to X_2$
    are covering spaces.
\end{proof}

\begin{prop}
    Hatcher exercise 1.3.4
\end{prop}
\begin{proof}
    Consider first $X=S^2\cup D$, where $D$ is a diameter. It is not hard to see that $X\simeq S^2\vee S^1$.
    Hence the construction of the universal cover proceeds almost exactly as in the case of $S^1$:
    we take $\R$ and attach to every integer point an $S^2$ (by identifying say, the point with the
    north pole of the sphere). This is simply-connected as it path-connected and is homotopy equivalent
    to the infinite wedge sum of spheres (which by van Kampen has trivial fundamental group). It
    is quite clear that this is a covering space for $X$.

    If we now have a circle intersected the sphere at two points denoted by $X$, it is clear
    that $X$ is homotopy equivalent to $S^2\vee S^1\vee S^1$. Hence the covering space
    is constructed in analogy to that of $S^1\vee S^1$, where one can take the Cayley
    complex discussed in class (and shown on p. 59 in Hatcher) and attach a sphere to every
    intersection point. This is of course simply connected, as it is again the wedge sum of
    spheres, and clearly projects down to $S^2\vee S^1\vee S^1$ as a covering map.
\end{proof}

\begin{prop}
    Hatcher exercise 1.3.9
\end{prop}
\begin{proof}
    Let $X$ be a path-connected, locally path-connected space with finite fundamental
    group. Consider a map $f:X\to S^1$. Note that the induced map $f_*:\pi_1(X)\to\Z$
    is a morphism from a finite group to $\Z$ and hence must be trivial (otherwise
    the image would form a finite subgroup, of which there are none in $\Z$), showing
    that $f_*(\pi_1(X,x_0))\subset p_*(\pi_1(\R,0))=0$. By Hatcher proposition 1.33,
    we find that $f$ lifts to a map $\tilde f:(X,x_0)\to(\R,0)$, but by contractibility
    of $\R$, this map can be homotoped to a constant map. Projecting this
    homotopy down to $S^1$ via $p$, we find that $f$ is in fact homotopic to 
    a constant map on $S^1$, as desired.
\end{proof}

\end{document}
