\documentclass{../../mathnotes}

\usepackage{tikz-cd}
\usepackage{todonotes}

\title{Introduction to Algebraic Topology PSET 8}
\author{Nilay Kumar}
\date{Last updated: \today}


\begin{document}

\maketitle

\begin{prop}
    Hatcher exercise 2.1.11 
\end{prop}
\begin{proof}
    Let $\iota:A\to X$ be the inclusion of $A$ into $X$, and $r:X\to X$ be the retract of $X$ onto $A$.
    The composition $r\circ \iota:A\to A$ yields the identity $\id_A:A\to A$. The induced maps on the homology are
    $(r\circ\iota)_*=r_*\circ\iota_*=\id:H_n(A)\to H_n(A)$. This map is of course injective, which implies that
    $\iota_*:H_n(A)\to H_n(X)$ must be injective as well.
\end{proof}

\begin{prop}
    Hatcher exercise 2.1.12
\end{prop}
\begin{proof}
    Let us show that the relation of chain homotopy between chain maps is an equivalence relation.
    Consider $f_\#,g_\#,h_\#:C_n(A)\to C_{n+1}(B)$. The relation is clearly reflexive, as $f_\#\sim f_\#$
    by the zero morphism $0:C_n(A)\to C_{n+1}(B)$. Symmetry holds as follows: if $f_\#\sim g_\#$ via a
    chain homotopy $h$, then $g_\#\sim f_\#$ via the chain homotopy $-h$, because then 
    \begin{align*}
        f_\#-g_\#&=\partial h+h\partial\\
        g_\#-f_\#&=-(\partial h+h\partial)\\
        &=\partial (-h)+(-h)\partial.
    \end{align*}
    Finally, the relation is transitive, because given $f_\#\sim g_\#$ via $H_1$ and $g_\#\sim h_\#$
    via $H_2$, we can add the two commutation relations to obtain that
    \begin{align*}
        f_\#-h_\#&=\partial H_1+H_1\partial+\partial H_2+H_2\partial\\
        &=\partial(H_1+H_2)+(H_1+H_2)\partial,
    \end{align*}
    as desired.
\end{proof}

\begin{prop}
    Hatcher exercise 2.1.14
\end{prop}
\begin{proof}
    
\end{proof}

\begin{prop}
    Hatcher exercise 2.1.15
\end{prop}
\begin{proof}
    Consider the exact sequence
    \begin{equation*}
        \begin{tikzcd}
            A\ar{r}{\alpha}&B\ar{r}{\beta}&C\ar{r}{\gamma}&D\ar{r}{\delta}&E.
        \end{tikzcd}
    \end{equation*}

    Exactness at $B$ requires $\ker\beta=\text{im }\alpha$, and hence $\alpha$ is surjective if and only
    if $\ker\beta=B$. Exactness at $D$ requires $\ker\delta=\text{im }\gamma$, and hence $\delta$ is injective
    if and only if $\text{im }\gamma=0$.
    Hence if (and only if) $\alpha$ is surjective and $\delta$ is injective then $\gamma=0$ and $\beta=0$ and the exactness at $C$
    (requiring that $\ker\gamma=\text{im }\beta$) forces $C=0$.

    Hence for a good pair $(X,A)$, we find that $H_n(X,A)=0$ if and only if the inclusion $A\to X$ induces
    isomorphisms on all homology groups, as the long exact sequence of theorem 2.13 splits into sequences
    \begin{equation*}
        \begin{tikzcd}
            0\ar{r}&\tilde H_n(A)\ar{r}{\iota_*}&\tilde H_n(X)\ar{r}&0
        \end{tikzcd}
    \end{equation*}
    for all $n$.
\end{proof}

\begin{prop}
    Let $A$ and $B$ be chain complexes. A chain map $f:A\to B$ is a \textnormal{chain homotopy equivalence}
    if there exists a chain map $g:B\to A$ such that $f\circ g\sim\id_B$ and $g\circ f\sim\id_A$ in the
    sense of chain homotopies.
    \begin{enumerate}[(a)]
        \item Prove that if $f:A\to B$ is a chain homotopy equivalence, then $f$ induces an isomorphism on homology.
        \item Give an example of chain complexes $A$ and $B$ with isomorphic homology but no chain homotopy equivalence
            between them. (Hint: let $A$ be $\Z$ in two consecutive gradings and zero everywhere else.)
    \end{enumerate}
\end{prop}
\begin{proof}\hfill
    \begin{enumerate}[(a)]
        \item Recall that chain-homotopic maps induce the same homorphism on homology. Hence $(f\circ g)_*=f_*\circ g_*=(\id_B)_*=\id_{H_n(B)}$
            and $(g\circ f)_*=g_*\circ f_*=(\id_A)_*=\id_{H_n(A)}$. As $\id_{H_n(B)}$ is injective, $f_*:H_n(A)\to H_n(B)$ must be as well, and 
            as $\id_{H_n(A)}$ is surjective, $f_*$ must be as well. Hence $f_*$ is an isomorphism.
        \item Consider the map of chain complexes $f:A_\bullet\to B_\bullet$ given by
            \begin{equation*}
                \begin{tikzcd}
                    \cdots\ar{r}&0\ar{r}\ar{d}{0}&\Z\ar{r}{\id_\Z}\ar{d}{0}&\Z\ar{r}\ar{d}{0}&0\ar{r}\ar{d}{0}&\cdots\\
                    \cdots\ar{r}&0\ar{r}&0\ar{r}&0\ar{r}&0\ar{r}&\cdots
                \end{tikzcd}
            \end{equation*}
            where each square clearly commutes. The homology groups of the two sequences are $H_\bullet(A)=0$ and $H_\bullet(B)=0$.
            However, there does not exist a chain homotopy equivalence between $A_\bullet$ and $B_\bullet$, as we now show.
            If there did exist one, there would exist a chain map $g:B_\bullet\to A_\bullet$ such that the appropriate compositions
            of $f$ and $g$ would be chain homotopic to $\id_B$ and $\id_A$ via some chain homotopy $h$. Of course, the only possible $g$
            is the zero morphism, and hence $g\circ f:A_\bullet\to A_\bullet$ is the zero map. Drawing the diagram
            \begin{equation*}
                \begin{tikzcd}
                    \Z\ar{r}{\id_\Z}\ar{d}{0}&\Z\ar{d}{0}\ar[swap]{dl}{h}\\
                    \Z\ar{r}{\id_\Z}&\Z
                \end{tikzcd}
            \end{equation*}
            we find that $h$ must be an automorphism of $\Z$ such that $\id_\Z\circ h+h\circ\id_\Z=-\id_\Z$.
            As the automorphisms of $\Z$ are $\id_\Z,-\id_\Z$, this equality cannot be satisfied and hence there does not exist
            a chain homotopy.
    \end{enumerate}
\end{proof}


\end{document}
