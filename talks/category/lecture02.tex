\documentclass{../../mathnotes}

\usepackage{tikz-cd}
\usepackage{amsmath}
\usepackage{todonotes}
\usetikzlibrary{decorations.pathmorphing}


\title{Lecture 2: Categories, functors, and natural transformations I\footnote{This talk follows [1] I.1-4 very closely.}}
\author{Nilay Kumar}
\date{June 4, 2014}


\begin{document}

\maketitle

\subsection*{(Meta)categories}

We begin, for the moment, with rather loose definitions, free from the technicalities of set theory.

\begin{defn}
    A \textbf{metagraph} consists of objects $a,b,c,\ldots$, \textbf{arrows} $f,g,h,\ldots$, and two
    operations, as follows. The first is the \textbf{domain}, which assigns to each arrow $f$ an object $a=\text{dom }f$,
    and the second is the \textbf{codomain}, which assigns to each arrow $f$ an object $b=\text{cod }f$.
    This is visually indicated by $f:a\to b$.
\end{defn}

\begin{defn}
    A \textbf{metacategory} is a metagraph with two additional operations. The first is the \textbf{identity},
    which assigns to each object $a$ an arrow $\id_a=1_a:a\to a$. The second is the \textbf{composition},
    which assigns to each pair $g,f$ of arrows with $\text{dom }g=\text{cod }f$ an arrow $g\circ f$ called
    their \textbf{composition}, with $g\circ f:\text{dom }f\to\text{cod }g$. This operation may be pictured as
    \begin{equation*}
        \begin{tikzcd}
            \;&b\ar{rd}{g}&\\
            a\ar{ru}{f}\ar[swap]{rr}{g\circ f}&&c
        \end{tikzcd}
    \end{equation*}
    We require further that: composition is associative,
    \[k\circ(g\circ f)=(k\circ g)\circ f,\]
    (whenever this composition makese sense) or diagrammatically that the diagram
    \begin{equation*}
        \begin{tikzcd}
            a\ar{rr}{k\circ(g\circ f)=(k\circ g)\circ f}\ar[swap]{d}{f}\ar[swap]{drr}{g\circ f} && d\\
            b\ar{urr}{k\circ g}\ar[swap]{rr}{g} && c\ar[swap]{u}{k}
        \end{tikzcd}
    \end{equation*}
    commutes, and that for all arrows $f:a\to b$ and $g:b\to c$, we have
    \[1_b\circ f=f \text{ and }g\circ 1_b=g,\]
    or diagrammatically that the diagram
    \begin{equation*}
        \begin{tikzcd}
            a\ar{r}{f}\ar{dr}{f} & b\ar{d}{1_b}\ar{rd}{g} &\\
            & b\ar{r}{g} & c
        \end{tikzcd}
    \end{equation*}
    commutes.
\end{defn}

Recall that a diagram is \textbf{commutative} when, for each pair of vertices $c$ and $c'$, any
two paths formed from direct edges leading from $c$ to $c'$ yield, by composition of labels, equal arrows
from $c$ to $c'$.

\begin{exmp}
    An example of a metacategory is that which has all sets as objects and all set-functions as arrows,
    with composition defined in the usual way.  It is easy to check that this does indeed satisfy the axioms above.
\end{exmp}

\begin{rem}
    Note that we could get rid of objects entirely and deal only with arrows, as objects correspond
    exactly to identity arrows.
\end{rem}

\begin{defn}
    A \textbf{directed graph} is a set $O$ of objects, a set $A$ of arrows (or morphisms), and two functions $\text{dom}:A\to O$,
    $\text{cod}:A\to O$. We define the set of \textbf{composable pairs} as
    \[A\times_OA=\{(g,f)\mid g,f\in A\text{ and }\text{dom }g=\text{cod }f\}.\]
\end{defn}

\begin{defn}
    A \textbf{category} is a directed graph with two additional functions, $\id:O\to A$ taking an object
    $c$ to an arrow $\id_c$ and $\circ:A\times_OA\to A$ taking a composable pair of arrows $(g,f)$ to
    an arrow $g\circ f$, such that $\text{dom }\id_a=a=\text{cod }\id_a$, $\text{dom }(g\circ f)=\text{dom }f$,
    and $\text{cod }(g\circ f)=\text{cod }g$. Moreover, we require that the associativity and identity law axioms
    for the case of metacategories also hold.
\end{defn}

\begin{rem}
    We tend to write ``$c\in C$'' and ``$f$ in $C$'' instead of $c\in O$ and $f\in A$. Furthermore, we write
    \[\text{hom}(b,c)=\{f\mid f\text{ in }C, \text{dom }f=b,\text{cod }f=c\}.\]
\end{rem}

\begin{exmp}
    Let us consider some basic examples of categories. The first, denoted \textbf{0}, is the empty category
    with objects and arrows both empty sets. The next simplest category is \textbf{1}, the category with one object
    and one arrow (the identity arrow). Less trivially, we can define \textbf{2}, the category with two objects $a,b$
    and just one arrow $a\to b$ not the identity. We could also define the category \textbf{3}, but you get the point.
\end{exmp}

\begin{exmp}
    A \textbf{monoid} is a category with one object. Each monoid is thus determined by the set of all its arrows, by
    the identity arrow, and by the rule for the composition of arrows. Since any two arrows have a composition,
    a monoid may be described as a set $M$ with a binary operation $M\times M\to M$ which is associative and has
    an identity. Hence a monoid is precisely a semigroup with identity (e.g. a group without inverses).

    In this sense, we see that a \textbf{group} is a category with one object in which every arrow has a
    (two-sided) inverse under composition.
\end{exmp}

In many fields of mathematics, the objects of study are, at the end of the day, sets with extra structure.
We would thus like to think about a category of sets. Unfortunately, as Russell's paradox shows, thinking about
the set of all sets very quickly becomes troublesome. Of course, we can think about the metacategory of sets,
but we can in fact do better.

Suppose that there is some big enough set $U$, call it the ``universe''. Describe a set as \textbf{small} if it
is a member of the universe. Now we can define the category \textbf{Set} with objects the set of small sets and arrows
all functions between them. Note that here we do not run into set-theoretic problems, and this construction
suffices for most purposes, as we can always choose our universe to be big enough. This definition leads to the definition
of a number of familiar categories, where the objects are all taken to be small.

\begin{exmp}
    We can consider categories of familiar algebraic objects, such as \textbf{Mon}, \textbf{Grp}, \textbf{Ab},
    \textbf{Rng}, and \textbf{CRng} with objects monoids, groups, abelian groups, rings, and commutative rings,
    respectively and morphisms the appropriate homomorphisms.

    Similarly, we can define categories of geometric spaces such as \textbf{Top}, \textbf{Toph}, \textbf{Man$^\text{p}$},
    and \textbf{Var} with objects topological spaces, topological spaces, manifolds of class $C^p$,
    and algebraic varieties respectively and morphisms continuous maps, homotopy classes of continuous maps, $C^p$ maps,
    and regular maps, respectively.
\end{exmp}

\subsection*{Functors and natural transformations}

A functor is a morphism of categories, more or less.

\begin{defn}
    Let $C$ and $B$ be categories. A \textbf{functor} $T:C\to B$ with domain $C$ and codomain $B$ consists of two functions:
    an object function which assigns to each object $c\in C$ an object $Tc\in B$ and an arrow function which assigns
    to each morphism $f:c\to c'$ of $C$ an arrow $Tf:Tc\to Tc'$ of $B$ in such a way that $T(\id_c)=\id_{Tc}$ and
    $T(g\circ f)=Tg\circ Tf$ whenever the composition $g\circ f$ is defined in $C$.
\end{defn}

\begin{exmp}
    A simple example is the power set functor $\mathcal{P}$ from \textbf{Set} to \textbf{Set}, which takes each set $X$
    to its power set $\mathcal{P}X$ and each map $f:X\to Y$ to a map $\mathcal{P}f:\mathcal{P}X\to\mathcal{P}Y$ which
    sends each $S\subset X$ to its image $fS\subset Y$. 
\end{exmp}

\begin{exmp}
    Functors arise frequently in algebra. Recall, for example, the commutator subgroup $[G,G]\subset G$ of a group,
    which captures the non-abelian behavior of $G$ as it is the smallest normal subgroup of $G$ such that the quotient of $G$
    by the subgroup yields an abelian group. Note that given a morphism of groups $f:G\to H$, $f[G,G]\leqslant [H,H]$, and
    so we can define a functor $G\mapsto [G,G]$ from \textbf{Grp} to \textbf{Grp}. Moreover, we have a functor from \textbf{Grp}
    to \textbf{Ab} taking $G\mapsto G/[G,G]$.

    As another example, take any two commutative rings $K$ and $K'$. We can consider the set of all non-singular
    $n\times n$ matrices with entries in $K$ as the general linear group $GL_n(K)$. In fact, given any morphism
    $f:K\to K'$, we obtain a morphism of groups $GL_nf:GL_n(K)\to GL_n(K')$, and hence we can think of $GL_n$ as a functor
    from \textbf{CRng} to \textbf{Grp}.

    A functor which ``forgets'' some or all of the structure of an algebraic object is commonly called a
    \textbf{forgetful functor}. Thus the forgetful functor from \textbf{Grp} to \textbf{Set} assigns to each
    group $G$ the set of its elements and assigns to each morphism $f:G\to G'$ the same function $f$, regarded just as
    a function between sets. Similarly the forgetful functor from \textbf{Rng} to \textbf{Ab} assigns to each
    ring $R$ the additive abelian group of $R$ and to each morphism $f:R\to R'$ of rings the same function,
    regarded just as a morphism of addition.
\end{exmp}

\begin{rem}
    Functors may be composed in the obvious manner. This composition is associative, and there is clearly an
    identity functor. This allows us to make precise the manner in which functors are morphisms between categories:
    we may consider the metacategory of all categories or the category \textbf{Cat} of all small categories.
\end{rem}

The following definitions will be useful later.

\begin{defn}
    An \textbf{isomorphism} of categories is a functor which has a two-sided inverse, or equivalently,
    is a bijection both on objects and arrows.
\end{defn}

\begin{rem}
    As we will see, the concept of an isomorphism of categories is usually stronger than we need.
    Much more useful will be the weaker concept of an equivalence of categories.
\end{rem}

\begin{defn}
    A functor $T:C\to B$ is \textbf{full} when to every pait $c,c'\in C$ and to every arrow $g:Tc\to Tc'$ of $B$,
    there is an arrow $f:c\to c'$ with $g=Tf$. One can check that the composition of two full functors is a full functor.
    
    A functor $T:C\to B$ is \textbf{faithful} when to every pair $c,c'$ of objects of $C$ and to every pair $f_1,f_2:c\to c'$
    of arrows of $C$, the equality $Tf_1=Tf_2:Tc\to Tc'$ implies $f_1=f_2$. One can check that the composition of
    faithful functors is faithful.
\end{defn}

These two properties may be visualized in terms of hom-sets; given a pair of objects $c,c'\in C$, the arrow function
of $T:C\to B$ assigns to each $f:c\to c'$ an arrow $Tf:Tc\to Tc'$ and so defines a function
\[T_{c,c'}:\text{hom}(c,c')\to\text{hom}(Tc,Tc')\]
taking $f\mapsto Tf$. $T$ is full when every such function is surjective and faithful when every such unction is injective.

\begin{exmp}
    The forgetful functor \textbf{Grp}$\to$\textbf{Set} is faithful but not full.
\end{exmp}

\begin{defn}
    A \textbf{subcategory} $S$ of a category $C$ is a collection of some of the objects and some of the arrows of
    $C$, which includes with each arrow $f$ both the domain and the codomain and with each object, its identity arrow
    and with each pair of composable arrows, their composition. These conditions ensure that these collections of
    objects and arrows themselves consitute a category $S$. The inclusion functor $S\to C$ is automatically
    faithful. We say that $S$ is a \textbf{full subcategory} of $C$ when the inclusion functor is full.
\end{defn}

\begin{rem}
    A full subcategory, given $C$, is thus determined by giving just the set of its objects, since the arrows between
    any two of these objects $s,s'$ are all morphisms $s\to s'$ in $C$. For example, the category \textbf{Set$_f$} of
    all finite sets is a full subcategory of \textbf{Set}.
\end{rem}

\begin{defn}
    Given two functors $S,T:C\to B$, a \textbf{natural transformation} $\tau:S\to T$ is a function which assigns
    to each object $c$ of $C$ an arrow $\tau_C=\tau c:Sc\to Tc$ of $B$ in such a way that every arrow $f:c\to c'$
    in $C$ yields a diagram
    \begin{equation*}
        \begin{tikzcd}
            Sc\ar{r}{\tau c}\ar[swap]{d}{Sf} & Tc\ar{d}{Tf}\\
            Sc'\ar{r}{\tau c'} & Tc'
        \end{tikzcd}
    \end{equation*}
    which is commutative. When this holds, we also say that $\tau_c:Sc\to Tc$ is \textbf{natural} in $c$
    For any objects $a,b,c,\ldots$, we call $\tau a,\tau b,\tau c$ the \textbf{components} of the natural transformation
    $\tau$. A natural transformation with every component $\tau c$ invertible in $B$ is called a
    \textbf{natural isomorphism} $\tau:S\cong T$.
\end{defn}

\begin{defn}
    We say that two categories $C$ and $D$ are \textbf{equivalent} (or that there is an \textbf{equivalence of categories}
    between $C$ and $D$) if there exists a pair of functors $S:C\to D$, $T:D\to C$ together with
    natural isomorphisms $I_C\cong T\circ S$, $I_D\cong S\circ T$.
\end{defn}

\begin{exmp}
    For each group $G$ the projection $p_G:G\to G/[G,G]$ to the abelianization defines a transformation $p$
    from the identity functor on \textbf{Grp} to the functor \textbf{Grp}$\to$\textbf{Ab}$\to$\textbf{Grp}.
    Moreover, $p$ is natural because each group homomorphism $f:G\to H$ defines the evident homomorphism $f'$
    for which the following diagram commutes:
    \begin{equation*}
        \begin{tikzcd}
            G\ar{r}{p_G}\ar[swap]{d}{f} & G/[G,G]\ar{d}{f'}\\
            H\ar{r}{p_H} & H/[H,H]
        \end{tikzcd}
    \end{equation*}
\end{exmp}

\begin{exmp}
    Given a group $G$ one can construct the so-called ``opposite group'' $G^{op}$ where the underlying set is the same but
    the group operation $a\cdot b$ is changed to $a\cdot^{op}b=b\cdot a$. It is clear that $G^{op}$ should be $G$
    in some obvious way. We claim that the functor $^{op}$ (check that this is a functor) from \textbf{Grp} to \textbf{Grp} is naturally isomorphic to the
    identity functor $\id_\text{\textbf{Grp}}$ from \textbf{Grp} to \textbf{Grp}.

    To prove this we need isomorphisms $\tau_G:G\to G^{op}$ for each group $G$ such that the diagram
    \begin{equation*}
        \begin{tikzcd}
            G\ar{r}{\tau_G}\ar[swap]{d}{f} & G^{op}\ar{d}{f^{op}}\\
            H\ar{r}{\tau_H} & H^{op}
        \end{tikzcd}
    \end{equation*}
    commutes for each $f:G\to H$. Define $\tau_G(g)=g^{-1}$, the inversion, which is clearly a group isomorphism
    as it reverses the order of operation and is its own inverse. Let us check that the diagram commutes. Note first
    that the induced homomorphism $f^{op}$ is simply $f$. We need to show that $\tau_H\circ f=f^{op}\circ\tau_G$.
    But this is true by definition of group homomorphisms: $f(g)^{-1}=f(g^{-1})$!

    This makes precise the statement that any group is ``naturally'' or ``obviously'' isomorphic to its opposite group.
\end{exmp}

\begin{exmp}
    Every finite-dimensional vector space $V$ is isomorphic to its dual space $V^*$, but the isomorphism requires a choice of basis.
    It turns out that there is no natural isomorphism between $V$ and $V^*$ in the sense of a natural transformation between
    the identity functor and the dual functor. There is, however, a natural isomorphism between the identity functor and the double-dual
    functor, and hence $V\cong V^{**}$ is a ``natural'' isomorphism.
\end{exmp}

\begin{thebibliography}{9}
    \bibitem{lamport94}
        Saunders MacLane. \textit{Categories for the Working Mathematician}.
        Graduate Texts in Mathematics, Springer, 1971.
\end{thebibliography}


\end{document}
