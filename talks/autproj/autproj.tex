\documentclass{../../mathnotes}

\usepackage{tikz-cd}
\usepackage{amsmath}
\usepackage{todonotes}


\title{A Note on Automorphisms of $\Proj^n_\A$}
\author{Nilay Kumar}
\date{February 7, 2014}


\begin{document}

\maketitle

%\setcounter{section}{-1}

Throughout this note, $\A=k$ denotes a field and $\A^n$ denotes the affine $n$-space over $k$.

\subsection*{Projective spaces}

Let us start by recalling the definition of projective space $\Proj^n_\A$ from last week.
\begin{defn}
    We denote by $\Proj_\A^n$ the space of lines passing through the origin of $\A^{n+1}$.
    More precisely, let $\A^*$ act on $\A^{n+1}-\{0\}$ by scaling as $\lambda\cdot(x_0,\cdots,x_n)=(\lambda x_0,\cdots, \lambda x_n)$
    and define \textbf{projective $n$-space} to be the quotient $\Proj_\A^n=\A^{n+1}-\{0\}/\A^*$ by this action.
\end{defn}

Thinking of projective space as parametrizing a set of lines can be confusing at times, as it somewhat
obscures the fact that $\Proj_\A^n$ is simply $\A^n$ with extra ``stuff'' added at infinity. Let us thus think
of projective space in terms of its coordinates.

\begin{defn}
    Consider a point $p\in\Proj_\A^n$. Treating $\Proj_\A^n$ as a quotient space, we can think of $p$ as the equivalence class
    of points on a line through $\A^{n+1}$. Suppose that this line passes through the point $\vec x=(x_0,\ldots, x_n)\in\A^{n+1}$.
    We write \textbf{homogeneous coordinates} for $p$ as
    \[p=[x_0:\cdots:x_n]=[\lambda x_0:\cdots:\lambda x_n],\]
    for any $\lambda\in\A^*$. Note that these coordinates respect the action of $\A^*$ defining $\Proj_\A^n$ and hence are well-defined.
    Moreover, not all $x_i$ can be zero.
\end{defn}

With this coordinate system in hand, let us try to build an intuitive picture of projective space
in the next few examples.

\begin{exmp}[Real projective line]
    Let $\A=\R$ and consider $\Proj^1_{\R}$, the space of lines through the origin of $\R^2$. It is clear that we can
    parametrize this space by the slopes of the lines everywhere except for the vertical line. This yields an $\R$'s worth
    of points, giving us the real line. If we now take the slope of the vertical line to be ``infinity,'' we obtain
    the whole projective space $\Proj^1_\R$. This is formalized by the homogeneous coordinates defined on $\Proj^1_\R$.

    Consider a point $[x:y]\in\Proj^1_\R$. Suppose $y\neq0$. Then, letting $\lambda=y^{-1}$,
    \[ [x:y]=y^{-1}\cdot[x,y]=[x/y:1]. \]
    If we now note that $x$ is free to range over all values in $\R$, we recover a copy of $\R$ embedding in $\Proj_\R^1$.
    We can think of this as the set of lines with finite slopes. If $y=0$, on the other hand, we can write
    \[ [x:0]=x^{-1}\cdot [x:0]=[1:0]. \]
    This implies that there is only one point in $\Proj^1_\R$ with $y=0$, which can be thought of as the ``point
    at infinity.'' It is important to note that the point at infinity is fundamentally no ``different'' than the other points
    in $\Proj_\R^1$ -- a concept that will become clear when we show that automorphisms of projective space can swap points
    at infinity with, say, the origin (in this case $[0:1]$). In this sense, we can think of the projective line as a disjoint union 
    or a ``compactification''
    $\Proj^1_\R=\R^1\sqcup\{\infty\}$
    (with the topology extended to the quotient topology of $\R^2/\R^*$).
\end{exmp}

\begin{exmp}[Real projective plane]
    Let $\A=\R$ and consider $\Proj^2_\R$, the space of lines through the origin of $\R^3$. Using homogeneous coordinates,
    consider a point $[x:y:z]\in\Proj^2_\R$. If $z\neq 0$, we can scale points to be of the form $[x:y:1]$, with $x,y$ free
    to vary in $\R^2$. This gives us a copy of $\R^2\subset\Proj^2_\R$. Now suppose $z=0$ and $y\neq 0$. In this
    subset, we can write points as $[x:1:0]$, which gives us a copy of $\R\subset\Proj^2_\R$. This can be interpreted as
    a ``line at infinity.'' Finally, if $z=0$ and $y=0$, we obtain a single point $[1:0:0]$, interpreted as a ``point at
    infinity.'' As this exhausts all possibilities, we may write the projective plane as a disjoint union $\Proj^2_\R=\R^2\sqcup\R^1\sqcup\{\infty\}.$ 
\end{exmp}

\begin{exmp}[Riemann sphere]
    Let $\A=\C$ and consider $\Proj^1_\C$, the space of ``complex lines'' through the origin of $\C^2$. Also known as the
    complex projective line, we can analyze $\Proj^1_\C$ in much the same way we dealt with the real projective line.
    Given a point $[z_1:z_2]\in\Proj^1_\C$, we can consider two cases: either $z_2=0$, in which case we have a single point
    $[1:0]$, or $z_2\neq0$, in which case we get a complex line (or plane, depending on your terminology) of points $[z_1:1]$.
    In analogy to the real case, we can think of $\Proj^1_\C$ as the usual complex plane along with a point at infinity.

    One way to visualize the complex projective line is through what is known as the Riemann sphere. Consider a sphere $S^2$
    sitting tangent to the origin above the complex plane. It is easy to check that lines passing through both the north
    pole of the sphere and the complex plane give a bijection between $S^2\setminus\{N\}$ and $\C$ -- this is known as the
    stereographic projection. This bijection can be extended to a bijection between the whole sphere $S^2$ and $\Proj^1_\C$
    by treating the north pole as the point at infinity. This yields a nice visualization of the complex projective line.
\end{exmp}

\begin{rem}
The above examples hint at a pattern for real projective spaces: $\Proj^n_\R=\R^n\sqcup\cdots\sqcup\R^1\sqcup\{\infty\}.$
Indeed, the reader familiar with topology will note that this is precisely the usual cell decomposition for $\Proj_\R^n$.
Similar statements apply for the complex case.
\end{rem}

\subsection*{Automorphisms of projective space}

Projective spaces will turn out to be an excellent backdrop for the classical algebraic geometry we will be discussing,
and so, among other things, it will be useful to understand the automorphisms of projective spaces.

When discussing spaces such as $\R^n$ or $\C^n$ we typically treat them as finite-dimensional vector spaces, which places us firmly
in the realm of linear algebra: objects are vector spaces and morphisms between them are linear maps. In this category,
automorphisms are simply invertible linear maps from a vector space to itself. This set of automorphisms in fact forms a group,
known as the general linear group $GL_n$, i.e. the group of $n\times n$ matrices with non-zero determinant.
%Indeed, given an automorphism of $\R^2$, say, $T:\R^2\to\R^2$, we can evaluate it on 

Unfortunately,
projective spaces $\Proj_\A^n$ do not live in the category of vector spaces, as there is no canonical way to add points. 
There is, however, a well-defined action of $GL_{n+1}(\A)$ on $\Proj_\A^n$ given -- just as in the case of vector spaces -- by matrix multiplication.

\begin{defn}
    We define the action of $GL_{n+1}(\A)$ on $\Proj_\A^n$ by matrix multiplication,
    \begin{align*}
        \begin{pmatrix}
            a_{00}&\hdots & a_{0n}\\
            \vdots &\ddots & \vdots\\
            a_{n0}&\hdots & a_{nn}
        \end{pmatrix}
        \begin{bmatrix}
            x_0\\\vdots\\x_n
        \end{bmatrix}
        =
        \begin{bmatrix}
            a_{00}x_0+\cdots+a_{0n}x_n\\
            \vdots\\
            a_{n0}x_0+\cdots+a_{nn}x_n
        \end{bmatrix},
    \end{align*}
    for some $[x_0:\cdots:x_n]\in\Proj^n_\A$.
\end{defn}

\begin{exmp}
    Consider the action of $GL_2(\R)$ on $\Proj^1_\R$ given by
    \[\begin{pmatrix}
        0&1\\
        1&0
    \end{pmatrix}
    \begin{bmatrix}
        x\\y
    \end{bmatrix}
    =
    \begin{bmatrix}
        y\\x
    \end{bmatrix}.\]
    Geometrically, this automorphism ``inverts'' the projective line. It swaps the origin $[0:1]$ and infinity $[1:0]$ and
    sends points on the real line $[x:1]$ to $[1:x]=[x^{-1}:1]$.
\end{exmp}

\begin{exmp}
    Consider again the action of $GL_2(\R)$ on $\Proj^1_\R$ but this time given by 
    \[\begin{pmatrix}
        2&0\\
        0&2
    \end{pmatrix}
    \begin{bmatrix}
        x\\y
    \end{bmatrix}
    =
    \begin{bmatrix}
        2x\\2y
    \end{bmatrix}
    =
    \begin{bmatrix}
        x\\y    
    \end{bmatrix}.\] More generally, any scalar matrix $\lambda\cdot\id$ (for $\lambda\in\A^*$) will act on $\Proj^n_\A$
    as the identity, due to the scaling laws of projective space.
\end{exmp}

The previous example highlights a redundancy in the action of $GL_{n+1}(\A)$ on $\Proj^n_\A$: there are non-trivial elements of the group
that act trivially on the space. In the interests of finding the group of self-maps that most accurately ``represent'' projective space, then,
we should consider elements of $GL_{n+1}(\A)$ only up to scalar multiples. This motivates the following definition.

\begin{defn}
    Let $\lambda\cdot \id$ be the (normal) subgroup of scalar matrices in $GL_{n}(\A)$. Define the \textbf{projective general linear} group
    of degree $n$, $PGL_n(\A)$, by the quotient
    \[PGL_{n}(\A)=GL_n(\A)/\lambda\cdot\id.\]
\end{defn}

It is worth mentioning at this point that the notion of an automorphism of projective space has not yet been defined. Indeed, we have not
even decided what kind of objects we should treat projective spaces as! In the context of curves in projective space however,
it is useful to treat $\Proj^n_\A$ as a projective algebraic set. Recall that a projective algebraic set is a set of points in projective
space cut out by the vanishing of a set of homogeneous polynomials\footnote{Convince yourself that it makes sense to think about the
    \textit{vanishing} of homogeneous polynomials in projective coordinates, but that the \textit{value} of an arbitrary polynomial is not well-defined.};
in particular, a projective algebraic curve is the set of points in
$\Proj^2_\A$ satisfying $f(x_0,x_1,x_2)=0$, for some homogeneous $f$. Then it is clear that choosing $f=0$ yields the entire projective plane,
from which it follows that projective spaces are projective algebraic sets.

When one introduces objects of a category (in this case, projective algebraic sets), one also introduces maps, or morphisms, between
them. Unfortunately, we have not yet built up enough machinery to define morphisms between projective algebraic sets.
It should be a welcome surprise, then, that in the case of $\Proj^n_\A$ we do not \textit{need} to understand such morphisms,
as the following theorem shows.

\begin{thm}
    Let $\Aut(\Proj^n_\A)$ be the group of automorphisms of $\Proj^n_\A$. Then
    \[\Aut(\Proj^n_\A)=PGL_{n+1}(\A),\]
    where the action of $PGL_{n+1}(\A)$ on $\Proj^n_\A$ is by matrix multiplication as described above.
\end{thm}

Hence we will stay away from the machinery of morphisms and instead view automorphisms of projective space simply as actions of $PGL_{n+1}(\A)$,
which are similar to the more familiar linear transformations of vector spaces. We will not prove the above theorem, as it requires a surprising
amount of algebraic geometry\footnote{For a proof in the scheme-theoretic case, see Hartshorne ch. II ex. 7.1.1.}; instead, we will prove it for
the case $n=1$ in the Appendix.

\subsection*{Basic examples}

Let us now turn to some examples, properties, and uses of automorphisms of projective space.

\begin{thm}\todo{finish}
    Automorphisms preserve certain properties
\end{thm}

\begin{exmp}[Intersections of curves]
\end{exmp}

\begin{exmp}[Circle inversions]
    Consider the complex projective line $\Proj^1_\C$ with homogeneous coordinates $[z_0:z_1]$ and the automorphism
    that takes $[z_0:z_1]\mapsto[z_1:z_0]$. For $z_1,z_2\neq0$, we find that
    \[ [z_0:z_1]=[z_0/z_1:1]\mapsto [z_1:z_0]=[z_1/z_0:1]. \]
    For $z_2=0$, $[1:0]\mapsto[0:1]$, i.e. the origin and the point at infinity are swapped.

    In terms of the geometry of the complex plane, this is a ``geometric inversion'' about the unit circle followed by complex conjugation.
    Those familiar with Euclidean plane geometry will recall that geometric inversion is a trick that allows one to
    invert the plane about a given circle. This inversion preserves the measure of angles (but not the direction), and hence
    is often used to make difficult problems in geometry more tractable. The complex inversion we have defined above,
    however, is geometric inversion followed by complex conjugation, but still preserves the non-trivial properties of
    geometric inversion.
\end{exmp}

In the category of vector spaces, automorphisms are uniquely determined by the image of a basis. Things are not as straightforward
for projective space, as asking the same question is analogous to asking how many one-dimensional subspaces of $\A^{n+1}$ determine
an automorphism of $\Proj^n_\A$. Instead, one is forced to go back to the vector space $\A^{n+1}$ underlying $\Proj^n_\A$, find
the unique linear transformation in $GL_{n+1}(\A)$, and descend to an element of $PGL_{n+1}(\A)$.
Before we do this, however, let us make the following definition.

\begin{defn}
    Let $x_0,\ldots, x_{n+1}$ be $n+2$ points in $\Proj^n_\A$. We say that these points are in \textbf{general position}
    if each subset of $n+1$ points has representative vectors that are linearly independent in $\A^{n+1}$.
\end{defn}

\begin{exmp}
    Any two distinct points in $\Proj^1_\A$ are represented by linearly independent vectors; thus, any three distinct points
    in $\Proj^1_\A$ are in general position.
\end{exmp}

This leads us to the following important theorem.

\begin{thm}[First fundamental theorem of projective geometry]
    Let $x_0,\ldots, x_{n+1}$ and $y_0,\ldots, y_{n+1}$ be in general position in $\Proj^n_\A$. Then there exists a unique
    automorphism $\Gamma:\Proj^n_\A\to\Proj^n_\A$ such that $\Gamma(x_i)=y_i$.
\end{thm}

\begin{proof}
    We start with existence.
    Fix representatives $v_0,\ldots,v_{n+1}$ for $x_0,\ldots,x_{n+1}$. By construction, the first $n+1$ are independent (in $\A^{n+1}$),
    and hence we can write
    \[v_{n+1}=\sum_{i=0}^{n}\lambda_i v_i,\]
    for some $\lambda_i\in\A$. Furthermore, note that $\lambda_i\neq0$ for all $i$, as otherwise, we would find a linear dependence between
    $n+1$ of the vectors, contradicting general position. Hence if we rescale each representative $v_i$ (except for $v_{n+1}$) by $\lambda_i^{-1}$,
    we find that
    \[v_{n+1}=\sum_{i=0}^nv_i.\]
    Furthermore, these $v_i$ are unique by linear independence -- if $\sum_{i=0}^nv_i=\sum_{i=0}^n\mu_i v_i,$ then $\mu_i=1$.
    We can do the same for $y_0,\ldots,y_{n+1}$ by choosing representatives $w_i$ such that
    \[w_{n+1}=\sum_{i=0}^nw_i,\]
    uniquely.
    Now, since $v_0,\ldots, v_n$ are linearly independent, they form a basis for $\A^{n+1}$, and hence
    determine a (unique) linear transformation $T:\A^{n+1}\to\A^{n+1}$ satisfying $T(v_i)=w_i$. By independence
    of $w_0,\cdots, w_n$, we see that $T$ is in fact invertible, i.e. $T\in GL_{n+1}(\A)$.
    We denote by $\Gamma$ the image of $T$ under the quotient map $GL_{n+1}(\A)\twoheadrightarrow PGL_{n+1}(\A)$.
    It is clear that $\Gamma$ is the desired automorphism.

    Uniqueness is straightforward: suppose $T':\A^{n+1}\to\A^{n+1}$ provides another automorphism $\Gamma':\Proj^n_\A\to\Proj^n_\A$
    such that $\Gamma'(x_i)=y_i$. Then $T'(v_i)=\mu_iw_i$ for some scalar multiples $w_i\in\A$, and we find that
    \[\mu_{n+1}w_{n+1}=T'(v_{n+1})=\sum_{i=0}^nT'(v_i)=\sum_{i=0}^n\mu_iw_i.\]
    By uniqueness of the expression $w_n=\sum_{i=0}^nw_i$, forcing $\mu_i/\mu_{n+1}=1$, i.e. $T'=\mu_i/\mu_{n+1}T$,
    and hence $T$ and $T'$ must have the same image in $PGL_{n+1}$: $\Gamma'=\Gamma$.
\end{proof}

\begin{rem}
    A simple corollary of this theorem is that three points in $\Proj^1_\A$ determine an automorphism. The usual choice of three points is $[0:1],[1:1],$ and $[1:0]$.
    In other words, there is a unique automorphism of $\Proj^1_\A$ sending three distinct points to $0,1,$ and $\infty$. Note, for example, that in the case of $\A=\F_2$,
    the finite field with two elements, these are the only three points in $\Proj^1_{F_2}$, and hence $PGL_{2}(\F_2)\cong S_3$, the symmetric group on three letters.
\end{rem}

\begin{exc}
    Show that $n+2$ points in $\Proj^n_\A$ are in general position if and only if they are the image under the quotient
    map $\A^{n+1}\twoheadrightarrow\Proj^n_\A$ of $\{e_0,\ldots,e_n,e_0+\cdots+e_n\}$, where $\{e_i\}$ is a basis for $\A^{n+1}$
    (hint: see proof of theorem).
\end{exc}



\subsection*{Appendix}\todo{do this at some point}

\end{document}
