\section{Applications of Laurent series: classification of singularities}

Suppose $f$ has a (potential) isolated singularity at $z_0$. In other words, $f$ is analytic in a deleted neighborhood of $z_0$, $D(Z_0;r)/\{z_0\}$, for some $r > 0$.
We can write a Laurent expansion in $0<|z-z_0|<r$:
\begin{align*}
    f(z)&=\sum_{k=-\infty}^\infty a_k (z-z_0)^k\\
    \nm{with} a_k&=\frac{1}{2\pi i}\int_{\partial D(z_0;R)} \frac{f(z)}{(z-z_0)^k} dz
\end{align*}

Let us see how we can use this to classify singularities. We have 3 cases:
\begin{enumerate}
    \item No negative powers of $(z-z_0)$ appear in the series expansion: $a_k=0\; \forall k<0$. Then we call $z_0$ a \textbf{removable singularity}.
    \item There is an $n>0$ such that $a_{-n}\neq0$ and $a_k=0 \fa k<-n$. Then we call $z_0$ a \textbf{pole} of $f(z)$.
    \item There are infinitely many negative powers of $(z-z_0)$ in the expansion. Then we call $z_0$ an \textbf{essential singularity}.
\end{enumerate}

\subsection{Removable singularities}
Let us first study removable singularities:
\begin{align*}
    f(z)=\sum_{k=0}^\infty a_k(z-z_0)^k\; \nm{for} 0<|z-z_0|<r
\end{align*}
Clearly $f$ is also analytic at $z_0$ if we simply define $f(z_0)=a_0$. In this case we have
\begin{align*}
    f(z)=\sum_{k=0}^\infty a_k(z-z_0)^k\; \nm{for} 0\leq|z-z_0|<r.
\end{align*}
Let us examine the example of $f(z)=\frac{\sin z}{z}$ for $0<|z|<\infty$
\begin{align*}
    \sin z &= z - \frac{z^3}{3!}+\frac{z^5}{5!}-\dots\\
    f(z)&=1-\frac{z^2}{3!}+\frac{z^4}{5!}-\dots
\end{align*}
Note that the series expansion for $f(z)$ can be defined for all $z$ and thus 0 is a removable singularity of $f$.

\begin{remark}
    If $f$ has a removable singularity at $z_0$, then $f$ is bounded around $z_0$. In other words, $\exists M>0$ with $|f(z)|<M \fa 0<|z-z_0|<r$. The converse is also true.
\end{remark}

\begin{thm}[Riemann's theorem of removable singularities]
    Let $z_0$ be a potential isolated singularity of $f$. If
    \[\lim_{z\ra z_0}f(z)(z-z_0)=0\]
    then $f$ has a removable singularity at $z_0$.
\end{thm}
\begin{remark}
    Note also that if $|f(z)|\leq 1/|z-z_0|^{1-\varepsilon}$ for some positive $\varepsilon$, then $f$ has a removable singularity at $z_0$.
    Consequently, this is a much stronger bound than that provided by the previous remark.
\end{remark}
\begin{proof}
    We want to show that $a_k=0\;\forall k<0$ in the Laurent expansion. Take $k<0$. Then, $k+1\leq0$.
    As $\lim_{z\ra z_0}f(z)(z-z_0)=0$, we can find $\delta>0$ such that $|f(z)(z-z_0)|<\varepsilon\;\fa z \nm{in} 0<|z-z_0|<\delta<r$.
    From the formula for the coefficients and the M-L formula, we now have that
    \begin{align*}
        |a_k|&=\big|\frac{1}{2\pi i}\int_{\partial D(z_0;\delta)}\frac{f(z)(z-z_0)}{(z-z_0)^{k+2}}dz\big|\\
        &\leq \frac{1}{2\pi}\frac{\varepsilon}{\delta^{k+2}}\cdot \length\partial D(z_0; \delta)=\varepsilon\delta^{-(k+1)}\\
        &\leq \varepsilon\delta^0=\varepsilon
    \end{align*}
    Thus we have that $|a_k|<\varepsilon, \fa\varepsilon>0$, which yields that the coefficients must be identically zero for $k<0$.
\end{proof}

\subsection{Poles}
Let us now examine poles. From the series expansion, there is an $n>0$ such that $a_{-n}\neq0$ but $a_k=0$ for all $k<-n$. In this case, $z_0$ is called a pole of order $n$ of $f$.
\begin{align*}
    f(z)&=\sum_{k=-n}^\infty a_k(z-z_0)^k\\
    =\frac{a_{-n}}{(z-z_0)^n}+\frac{a_{-n+1}}{(z-z_0)^{n-1}}&+\dots+\frac{a_{-1}}{z-z_0}+a_0+a_1(z-z_0)+\dots
\end{align*}
\begin{definition}
    The sum of negative powers
    \begin{align*}
        P(z)=\sum_{k=-n}^{-1} a_k(z-z_0)^k
    \end{align*}
    is called the \textbf{principal part} of $f$ at the pole $z_0$.
\end{definition}
\begin{remark}
Note that for our series, if we take $f(z)-P(z)$ we obtain the analytic function $\sum_{k=0}^na_k(z-z_0)^k$.
Incidentally, if $n=1$, we call the pole \textbf{simple} and if $n=2$, we call the pole \textbf{double}.
\end{remark}

\begin{lemma}
    Suppose $f$ is analytic in a region $\Omega$ and has a zero at a point $z_0\in\Omega$ with $f\not\equiv0$ in $\Omega$. Then there exists in a neighborhood $U\in\Omega$ of $z_0$, a non-vanishing
    analytic function $g$ on $U$ and a unique positive integer $n$ such that $f(z)=(z-z_0)^ng(z)$ for all $z\in U$. Note that the number $n$ is called the \textbf{order of the zero} $z_0$
    of the function $f$ (also called the multiplicty).
\end{lemma}
\begin{proof}
    In a small neighborhood $D(z_0; R)$ of $z_0$, $f$  cannot be identically 0 by the uniqueness theorem. In this neighborhood let us expand:
    \begin{align*}
        f(z)&=\sum_{k=0}^\infty a_k(z-z_0)^k\\
        f(z_0)&=0 \implies a_0=0
    \end{align*}
    Therefore there must be an $n\geq 1$ such that $a_n\neq 0$, otherwise $f$ would vanish identically. Thus we write:
    \begin{align*}
        f(z)&=a_n(z-z_0)^n+a_{n+1}(z-z_0)^{n+1}+\dots\\
        &=(z-z_0)^n\left[ a_n+a_{n+1}(z-z_0)+\dots \right]
    \end{align*}
    Let $g(z)=a_n+a_{n+1}(z-z_0)+\dots$. Clearly it is analytic in D. Additionally, since $\lim_{z\ra z_0} g(z)=a_n\neq0$, we have
        $|g(z)|\geq|a_n|/2$
    whenever $|z-z_0|<r$. Thus let us take $U=D$.

    Let us now show the uniqueness of $n$. Suppose $\exists n<m$ such that $f(z)=(z-z_0)^ng(z)=(z-z_0)^mh(z)$ where $g$ and $h$ are analytic in a neighborhood $V$ of $z_0$, with
    $g, h\neq 0$ in $V$. When $z\neq z_0$ with $z\in V$, $g(z)=(z-z_0)^{m-n} h(z)$. Thus, $g(z_0)=\lim_{z\ra z_0}g(z)=\lim_{z\ra z_0}(z-z_0)^{m-n} h(z)\equiv 0$. This is a contradiction,
    and thus $n$ must be unique.
\end{proof}

\begin{thm}[Characterization of a pole]
   Let $z_0$ be an isolated singularity of $f$. Then $z_0$ is a pole of $f(z)$ of order $n$:
   \begin{enumerate}
       \item iff $f(z)=g(z)/(z-z_0)^n$ where $g$ is analytic and non-zero at $z_0$.
       \item iff 
           \begin{equation*}
               h(z)=
               \begin{cases}
                   1/f(z) & \nm{if} z\neq z_0\\
                   0 & \nm{if} z=z_0
               \end{cases}
           \end{equation*}
           is analytic at $z_0$ and has a zero of order $n$ at $z_0$.
       \item iff $|f(z)|\ra \infty$ when $z\ra z_0$
   \end{enumerate}
\end{thm}

\begin{proof}
    Let us attack the first claim. Suppose $f$ has a pole of order $n$ at $z_0$. Then, using the expansion for poles above, 
    \begin{align*}
        f(z)=\frac{a_{-n}+a_{-n+1}(z-z_0)+\dots+a_{-1}(z-z_0)^{n-1}+a_0(z-z_0)^n+\dots}{(z-z_0)^n}\equiv \frac{g(z)}{(z-z_0)^n}
    \end{align*}
    It is clear that $g$ is analtyic at $z_0$ and does not vanish, and so we reach the result. Let us now prove the converse. If $g$ is analytic at $z_0$, and does not vanish, let us expand it
    into a power series about $z_0$:
    \begin{align*}
        g(z)=b_0+b_1(z-z_0)+\dots \nm{with} g(z_0)=b_0\neq0
    \end{align*}
    Hence,
    \begin{align*}
        f(z)&=\frac{g(z)}{(z-z_0)^n}=\frac{b_0+b_1(z-z_0)+\dots}{(z-z_0)^n}\\
        &=\frac{b_0}{(z-z_0)^n} + \frac{b_1}{(z-z_0)^{n-1}} + \dots
    \end{align*}
    Thus $f$ has a pole of order $n$ at $z_0$, and the first point is proved.

    Let us now prove the second claim.
    Suppose $f$ has a pole of order $n$ at $z_0$. Then by the first point, we can represent
    \begin{align*}
        f(z)=\frac{g(z)}{(z-z_0)^n}
    \end{align*}
    where $g$ is analytic and non-zero at $z_0$. Take the function $M(z)=1/g(z)$, which is also analytic at $z_0$. Then,
    \begin{align*}
        \frac{1}{f(z)}=\frac{(z-z_0)^n}{g(z)}=(z-z_0)M(z)
    \end{align*}
    is analytic at $z_0$ and has a zero of order $n$ at $z_0$. Conversely, if $1/f(z)$ has a zero of order $n$ at $z_0$, by the above theorem,
    \begin{align*}
        \frac{1}{f(z)}=(z-z_0)^n K(z)
    \end{align*}
    for some $K$ analytic at $z_0$ and $K(z_0)\neq0$. Hence,
    \begin{align*}
        f(z)=\frac{1/K(z)}{(z-z_0)^n}\equiv\frac{g(z)}{(z-z_0)^n}
    \end{align*}
    Here $g(z)=1/K(z)$ is analytic at $z_0$ and $g(z_0)\neq0$. Then, by the first point $f$ has  a pole of order $n$ at $z_0$.

    Let us now prove the third point. If $f$ has a pole of order $n$ at $z_0$ then by the first point, $f(z)=g(z)/(z-z_0)^n$ for some $g$ analytic and nonvanishing at $z_0$.
    Thus,
    \begin{align*}
        |f(z)|=\frac{|g(z)|}{|z-z_0|^n}\ra \infty.
    \end{align*}
    Conversely, suppose that $f(z)\ra\infty$ as $z\ra z_0$. Obviously, $f(z)\neq 0$ for $z$ near $z_0$.  Thus we can define $h(z)=1/f(z)$, which is analytic around $z_0$ and
    $h(z)\ra 0$ as $z\ra z_0$. Hence, by the Riemann's theorem proved above, $h(z)$ must have a removable singularity at $z_0$ and $h(z_0)=0$. Since $h\neq 0$ in the neighborhood
    of $z_0$, by the above theorem on zeroes, we can write $h(z)=(z-z_0)^ng(z)$ for some $n>0$ and $g$ analytic in a neighborhood $V$ of $z_0$ and $g\neq 0$ in $V$. Now,
    \begin{align*}
        f(z)=\frac{1}{h(z)}=\frac{1/g(z)}{(z-z_0)^n}
    \end{align*}
    By the first point, $f$ has a pole of order $n$ at $z_0$, as $1/g(z)$ is analytic and non-zero in $V$.
\end{proof}

We shall cover the topic of essential singularities next lecture.
