\documentclass{../../mathnotes}

\usepackage{tikz-cd}


\title{Notes on Modules}
\author{Nilay Kumar}
\date{Last updated: \today}


\begin{document}

\maketitle

\setcounter{section}{-1}

\section{Modules}

Let $A$ be a ring. An \textbf{$A$-module} is an Abelian group $M$ with a multiplication map $A\times M\to M$, written $(f,m)\mapsto fm$
that satisfies
\begin{enumerate}[(i)]
    \item $f(m\pm n)=fm\pm fn$;
    \item $(f+g)m=fm+gm$;
    \item $(fg)m=f(gm)$;
    \item $1_Am=m$
\end{enumerate}
for all $f,g\in A$ and $m,n\in M$. We call a subset $N\subset M$ a \textbf{submodule} if $fm+gn\in N$ for all $f,g\in A$ and $m,n\in N$.
A \textbf{homomorphism} is a map $t:M\to N$ of $A$-modules that is $A$-linear, in the obvious sense that $t(fm+gn)=ft(m)+gt(n)$ for all $f,g\in A$
and $m,n\in M$.


\section{Exact Sequences}








\bibliography{notes}
\bibliographystyle{alpha}

\end{document}
