\documentclass{../mathnotes}

\title{RTG Notes}
\author{Nilay Kumar}
\date{Last updated: \today}


\begin{document}

\maketitle

\setcounter{section}{-1}

\section{Introduction}

\subsection{Adam}

What is algebraic geometry? It is the study of solutions to systems of polynomial equations.

\begin{defn}
    An \textbf{algebraic variety} $V(f_1,\ldots,f_s$ is the set of points $p$ such that $f_1(p)=f_2(p)=\cdots=0$
\end{defn}

One nice thing is that we can work over different fields and get different results. In fact, we can often get
hard number theoretic problems. Take for example $V(x^n+y^n-z^n)$ over $\Q$ -- analyzing this necessitates Fermat's
last theorem!

\begin{exmp}
    Let us, for example, find rational solutions to $x^2+y^2=z^2$. We can rewrite this as $X^2+Y^2$ by a
    substitution. It's clear that we have a circle which goes through $(0,1)$. If we assume there exists a
    rational point $(p,q)$ also on the circle, it's clear that the line going from $(0,1)$ and $(p,q)$ has rational
    slope. Conversely, any line with rational slope will also hit some rational point on the circle. Thus we can
    parametrize the solutions to our problem by rational slopes $m$.
\end{exmp}

In light of this example, how do we work backwards from parametrizations (toric varieties) to varieties?

\begin{exmp}
    Let $C=\left\{ (t,t^2,t^3)\in\R^3|t\in\R \right\}$. $C$ is, in fact, an algebraic variety:
    \[C=V(y-x^2,z-x^3).\]
    Are there other choices? Well we could try to start with the parametrizations and try to eliminate the parameter.
    Consider the homomorphism $\phi:\R[x,y,z]\to\R[t]$ given by $\phi(x)=t,\phi(y)=t^2,\phi(z)=t^3$. Clearly the set
    we want is $\ker\phi$, i.e. the set of polynomials in $x,y,z$ that vanish when $x=t,y=t^2,z=t^3$. We've reduced 
    the problem to determining the kernel of some ring homomorphism.
\end{exmp}

The general question is, given a homomorphism $\phi:\R[x_1,\ldots,x_s]\to\R[r,s,t]$, how do we describe the kernel?
Let's look at another example.

\begin{exmp}
    Consider the parametrizations
    \[\phi(x_1)=s^4,\phi(x_2)=s^3t,\phi(x_3)=st^3,\phi(x_4)=t^4.\]
    One equation is $x_1x_4-x_3x_2$. However, it's clear that these problems can quickly get rather difficult.
\end{exmp}

These are the types of questions we will generally be talking about.

\subsection{Pablo}

\subsection{Dan}



\end{document}
