\documentclass{../mathnotes}

\title{Representation Theory Notes}
\author{Nilay Kumar}
\date{Last updated: \today}


\begin{document}

\maketitle

\setcounter{section}{-1}

\begin{defn}
    A \textbf{representation} of a finite group $G$ on a finite-dimensional complex vector space $V$
    is a homomorphism $\rho: G\to GL(V)$ of $G$ to the group of automorphisms of $V$. A map $\phi$ between
    two representations $V$ and $W$ of $G$ is a vector space map $\phi:V\to W$ such that $g\cdot\phi=\phi\cdot g$
    and is called a \textbf{morphism} of representations.

    A \textbf{subrepresentation} of a representation of $V$ is a vector subspace $W$ of $V$ which is invariant
    under $G$. A representation $V$ is called irreducible if there is no proper nonzero invariant subspace $W$ of $V$.

    If $V$ and $W$ are representations, the direct sum $V\oplus W$ and the tensor product $V\otimes W$ are also representations.
    These are given by $g(v,w)=(gv,gw)$ and $g(v\otimes w)=gv\otimes gw$.
\end{defn}

\end{document}
