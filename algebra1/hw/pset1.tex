\documentclass{../../mathnotes}

\usepackage{enumerate}

\title{Modern Algebra I: Problem Set 1}
\author{Nilay Kumar}
\date{Last updated: \today}


\begin{document}

\maketitle

\subsection*{Problem 1}

We wish to show that $A\cap (B\cup C)=(A\cap B)\cup(A\cap C)$. Pick an element $x\in A\cap(B\cup C)$. By definition,
$x\in A$ and either $x\in B$ or $x\in C$. It is clear, that one of $(A\cap B)$ or $(A\cap C)$ must contain $x$.
Thus, $A\cap(B\cup C)\subseteq (A\cap B)\cup(A\cap C)$. Conversely, pick an element $x\in(A\cap B)\cup(A\cap C)$.
By definition, $x$ is either in $(A\cap B)$ or $(A\cap C)$, i.e. $x$ is either in $A$ and $B$ or in $A$ and $C$.
Hence $x\in A\cap(B\cup C)$, implying that $(A\cap B)\cup(A\cap C)\subseteq A\cap(B\cup C)$, and we are done.

\subsection*{Problem 2}

\begin{enumerate}[(a)]
    \item $P(\left\{ 0,1 \right\})=\left\{\varnothing, \left\{ 0 \right\},\left\{ 1 \right\},\left\{ 0,1 \right\} \right\}$;
    \item $P(\varnothing)=\{\varnothing\}$;
    \item One can compute the cardinality of the power set of $S$ by counting the number of ways one can choose subsets of $S$
        (of all possible sizes):
        \begin{equation*}
            |P(S)|=\sum_{i=0}^n \binom{i}{n}=2^n
        \end{equation*}
        by the binomial theorem (for $x=y=1$).
\end{enumerate}

\subsection*{Problem 3}

Let $S=\left\{ 1,2,3 \right\}$. Define $f:S\to S$ as the transposition $(12)$, and $g:S\to S$ as the transposition $(23)$.
It's clear that $g(f(1))=3$ and $f(g(1))=2$, and hence $f\circ g\neq g\circ f$. For the case of $\R$, consider $f:\R\to\R$
by $x\mapsto x^2$ and $g:\R\to\R$ given by $x\mapsto x+1$. Note that $f(g(1))=4$ while $g(f(1))=2$.

\subsection*{Problem 4}

\begin{enumerate}[(a)]
    \item Suppose $g\circ f$ is injective; assume for the sake of contradiction that $f$ is not injective. Then there
        exist $a,b$ distinct such that $f(a)=f(b)$. Then $g(f(a))=g(f(b))$, contradicting that $g\circ f$ is injective.
    \item Consider 
\end{enumerate}



\end{document}
