\documentclass{article}
\usepackage[utf8]{inputenc}

%\usepackage[margin=1in]{geometry}
\usepackage{amsmath}
\usepackage{amssymb}
\usepackage{amsthm}
\usepackage{enumerate}
\usepackage{tikz-cd}

\newcommand{\nm}[1]{\;\textnormal{#1}\;}
\newcommand{\ra}[0]{\rightarrow}
\newcommand{\fa}[0]{\;\forall}
\newcommand{\R}{\mathbb{R}}
\newcommand{\Q}{\mathbb{Q}}
\newcommand{\Z}{\mathbb{Z}}
\newcommand{\F}{\mathbb{F}}
\newcommand{\C}{\mathbb{C}}
\newcommand{\CP}{\mathbb{C}\mathbb{P}}
\newcommand{\RP}{\mathbb{R}\mathbb{P}}
\newcommand{\Proj}{\mathbb{P}}
\newcommand{\N}{\mathbb{N}}
\newcommand{\p}{\partial}
\newcommand{\fr}{\mathfrak}
\newcommand{\OO}{\mathcal{O}}
\newcommand{\A}{\mathbb{A}}


\DeclareMathOperator{\Ker}{Ker}
\DeclareMathOperator{\codim}{codim}
\DeclareMathOperator{\Tr}{Tr}
\DeclareMathOperator{\Res}{Res}
\DeclareMathOperator{\ord}{ord}
\DeclareMathOperator{\Hom}{Hom}
\DeclareMathOperator{\length}{length}
\DeclareMathOperator{\res}{Res}
\DeclareMathOperator{\Int}{Int}
\DeclareMathOperator{\Ext}{Ext}
\DeclareMathOperator{\Aut}{Aut}
\DeclareMathOperator{\Gal}{Gal}
\DeclareMathOperator{\Sym}{Sym}
\DeclareMathOperator{\Lie}{Lie}
\DeclareMathOperator{\Pro}{Proj}
\DeclareMathOperator{\id}{Id}
\DeclareMathOperator{\tr}{tr}
\DeclareMathOperator{\irr}{irr}
\DeclareMathOperator{\supp}{supp}
\DeclareMathOperator{\trdeg}{trdeg}
\DeclareMathOperator{\Spec}{Spec}
\DeclareMathOperator{\Nm}{Nm}
\DeclareMathOperator{\hgt}{ht}
\theoremstyle{plain}
\newtheorem{thm}{Theorem}
\newtheorem*{thm*}{Theorem}
\newtheorem{lem}[thm]{Lemma}
\newtheorem*{lem*}{Lemma}
\newtheorem{cor}[thm]{Corollary}
\newtheorem*{cor*}{Corollary}
\newtheorem{prop}[thm]{Proposition}
\newtheorem*{prop*}{Proposition}
\newtheorem{exc}{Exercise}

\theoremstyle{definition}
\newtheorem{defn}{Definition}
\newtheorem{exmp}{Example}

\theoremstyle{remark}
\newtheorem*{rem}{Remark}

\title{Commutative Algebra \`a la M.~Thaddeus}
\author{Notes by Nilay Kumar}
\date{}

\begin{document}

\maketitle

The following result will be quoted extensively.

\begin{lem}[Nakayama, Azumaya, Krull]
     Let $A$ be a ring, $M$ a finitely generated $A$-module, and $I$ an ideal of $A$.
     Suppose that $IM=M$. Then there exists an element $a\in A$ of the form $a=1+x$,
     $x\in I$ such that $aM=0$. If, moreover, if $I$ is contained in the Jacobson radical
     $\mathcal{J}(A),$ then $M=0$.
\end{lem}
\begin{proof}
    We follow Matsumura.
    Let $M=Aw_1+\cdots+Aw_s$. We induct on $s$. Let $M'=M/Aw_s$. By the induction
    hypothesis there exist $x\in I$ such that $(1+x)M'=0$, i.e. $(1+x)M\subset Aw_s$.
    Note that for the base case $s=1$, the module is trivial, so we can take $x=0$.
    Since $M=IM$, we have that
    $(1+x)M=I(1+x)M\subset I(Aw_s)=Iw_s$, and hence we can write $(1+x)w_s=yw_s$ for some
    $y\in I$. Then $(1+x-y)(1+x)M=0$, and $(1+x-y)(1+x)\equiv 1\mod I$ proving the
    first assertion. If in particular $I\subset\mathcal{J}(A)$ then $1+x$ is not contained
    in any maximal, and thus a unit. This implies that $M=0$.
\end{proof}

The following corollary is often useful.
\begin{cor}
    Let $A$ be a ring, $M$ an $A$-module, $N$ and $N'$ submodules of $M$, and $I$ an
    ideal of $A$. Suppose that $M=N+IN'$, and that either (a) $I$ is nilpotent or (b)
    $I\subset\mathcal{J}(A)$ and $N'$ is finite. Then $M=N$.
\end{cor}
\begin{proof}
    In case (a) we have $M/N=I(M/N)=I^2(M/N)=\cdots=0.$ In case (b), apply Nakayama's
    Lemma to $M/N$.
\end{proof}

\subsection*{November 11, 2014}

Recall the following theorem from last time.
\begin{thm}
    A ring $R$ is Artinian if and only if $R$ is Noetherian of Krull dimension 0.
\end{thm}

\begin{exmp}
    An algebra $R$ over $k$ a field is Artinian if and only if $R$ is a finite-dimensional
    vector space, e.g.
    \[R=k[x,y]/(x^m,y^n,f(x,y),g(x,y)).\]
    Note, incidentally, if we have $f,g$ monomials (i.e.~a monomial ideal), we can draw a lattice to represent
    this ring, and we obtain Young diagrams.
\end{exmp}

\begin{exc}
    A ring $R$ is Artinian if and only if $\Spec R$ is finite and discrete.
\end{exc}

\begin{exc}
    $\Spec R=\Spec R_1\sqcup \Spec R_2$ as a ringed space if and only if $R\cong R_1\times R_2$.
\end{exc}

\begin{thm}[Structure theorem for Artinian rings]
    Any Artinian ring $R$ is a product of finitely many Artinian local rings.
\end{thm}
\begin{proof}
    C.f. Atiyah-MacDonald. However, there may be a proof through thinking geometrically
    via the above exercises.
\end{proof}

\begin{cor}
    If $(R,\fr m)$ is Artinian local then $\fr m$ is nilpotent.    
\end{cor}
\begin{proof}
    Apply Nakayama's lemma to the chain of $R$-modules
    \[\fr m\supset \fr m^2\supset \fr m^3\supset \cdots\]
    which stabilizes.
\end{proof}

\begin{cor}
    For $R$ Noetherian, $\fr p$ is a prime minimal among those containing some ideal $\mathcal{I}\leq R$,
    if and only if $R_{\fr p}/\mathcal{I}_{\fr p}$ is Artinian.
\end{cor}
\begin{proof}
    The ring $R_{\fr p}/\mathcal{I}_{\fr p}$ is Noetherian, and the primes in this ring correspond
    to the primes $\mathcal{I}\subset\fr q\subset\fr p$. So $\fr p$ is the only one if and only if $R_{\fr p}/\mathcal{I}_{\fr p}$
    is Artinian.
\end{proof}

\begin{prop}
    Suppose $R$ is Noetherian and $\fr p$ is a prime containing a nonzerodivisor $x$. Then $\hgt\fr p\geq 1$.
\end{prop}
\begin{proof}
    The nilradical $\sqrt{0}\leq R$ is, in general, the intersection of all primes in $R$. Consider the
    chain
    \[\sqrt{0}=\bigcap_{\fr q\in R}\fr q\subset \bigcap_{\substack{\fr q\in R\\\fr q\neq\fr q_1}}\fr q
        \subset \bigcap_{\substack{\fr q\in R\\\fr q\neq\fr q_1,\fr q_2}}\fr q\subset\cdots\]
    for fixed primes $\fr q_i$. Since $R$ is Noetherian, this chain must stabilize (in this case, to some prime), and hence $\sqrt{0}$
    can be written as the intersection of finitely many primes in $R$.
    Thus
    \[\fr p\supset \sqrt{0}=\fr p_1\cap\cdots\cap\fr p_n\]
    for some primes $\fr p_i$. Now, $\fr p$ must contain some
    $\fr p_i$. Suppose it didn't: taking $x_i\in \fr p_i\setminus \fr p$, we find that $x_1\cdots x_n\in\sqrt{0}\setminus\fr p$,
    a contradiction. To establish that the height is one, now, it suffices to show that each of these minimal primes
    consist entirely of zerodivisors, so as not to be equal to $\fr p_i$. So let $\fr q$ be any minimal prime.
    Suppose $\fr q$ contains a nonzerodivisor $x$. Since $R_{\fr q}$ is Artinian, for $x=x/1\in R_{\fr q}$,
    the chain
    \[(x)\supset (x)^2\supset (x)^3\supset\cdots\]
    stabilises, i.e. $(x)^{n+1}=(x)^n$ and $x^n=yx^{n+1}$ for $y\in R_{\fr q}$.
    Then $x^n(1-xy)=0$ which implies that $1-xy=0$ and hence $x$ is a unit, i.e. $x\notin \fr q_{\fr q}$. This
    shows that $x\notin\fr q$.\footnote{We have used the following easy fact: if $x$ is a
        nonzerodivisor in $R$ then $x$ is a nonzerodivisor in $R_{\fr q}$. Show this!}
\end{proof}

Now we prove an estimate in the opposite direction. But first we introduce the following notion.

\begin{defn}
    The $n^\text{th}$ \textbf{symbolic power} of $\fr p\subset R$ is defined to be
    \[\fr p^{(n)}=\{x\in R\mid sx\in\fr p^n\text{ for some }s\in R\setminus\fr p\},\]
    i.e. $\fr p^{(n)}$ is the inverse image of $\fr p_{\fr p}^n\leq R_{\fr p}$.
    In particular $(\fr p^{(n)})_{\fr p}=(\fr p_{\fr p})^{n}$.
\end{defn}

\begin{rem}
    The geometric inuition behind this definition is unclear, though the algebraic value lies in
    the following results.
\end{rem}

\begin{lem}
    If $xy\in \fr p^{(n)}$ and $x\notin\fr p$ then $y\in\fr p^{(n)}$.
\end{lem}
\begin{proof}
    If $xy\in\fr p^{(n)}$ then $(xy)/1\in(\fr p_{\fr p})^n$, which implies that $y/1\in(\fr p_{\fr p})^n$
    since $x/1$ is a unit. This shows that $y\in\fr p^{(n)}$.
\end{proof}

\begin{thm}[Krull's Hauptidealsatz]
    If $R$ is Noetherian and $\fr p$ is minimal among primes containing an element $x\in R$ then
    $\hgt\fr p\leq 1$.
\end{thm}

\begin{rem}
    Consider the example of a variety. If we consider the subvariety cut out
    by an irreducible element $f$ in the coordinate ring, the ideal $(f)$ is obviously minimal
    containing $f$. This, together with the two estimates above, force the codimension of this
    subvariety to be exactly 1.
\end{rem}

\begin{proof}
    Fix a prime $\fr q\subsetneq\fr p$. By construction, $x\notin\fr q$. It is sufficient to
    show, now, that $\hgt\fr q=0$, which is the same as showing that $\dim R_{\fr q}=0$.
    Replacing $R$ with $R_{\fr p}$, we assume without loss of generality that $(R,\fr p)$ is local.\footnote{In
    other words, a counterexample remains a counterexample after localization.}
    Since $\fr p$ is minimal among primes containing $(x)$, we find, by
    the above Corollary, that $R/(x)$ is Artinian. We have a descending chain $\fr q^{(n)}$ in $R$
    whose image in $R/(x)$ is a descending chain of ideals in $R/(x)$.
    This chain stabilizes, and hence the corresponding chain $(x)+\fr q^{(n)}$ in $R$ must stabilize:
    \[(x)+\fr q^{(n+1)}=(x)+\fr q^{(n)}\supset\fr q^{(n)}.\]
    Thus, for all $f\in\fr q^{(n)}$, \[f=rx+g,\] for $r\in R,g\in\fr q^{(n+1)}$.
    In particular, $rx\in\fr q^{(n)}$, $x\notin\fr q$, so by the above Lemma, $r\in\fr q^{(n)}$.
    Now $\fr q^{(n)}=(x)\fr q^{(n)}+\fr q^{(n+1)}$, and $\fr q^{(n)}/\fr q^{(n+1)}=(x)\fr q^{(n)}/\fr q^{(n+1)}$,
    so by Nakayama's lemma (everything in sight is finitely generated) we find that $\fr q^{(n)}=\fr q^{(n+1)}$
    since $(x)$ is in the Jacobson radical of $R$. Hence
    \[(\fr q_{\fr q})^n=(\fr q^{(n)})_{\fr q}=(\fr q^{(n+1)})_{\fr q}=(\fr q_{\fr q})^{n+1}\]
    and applying Nakayama's lemma again, we find that $(\fr q_{\fr q})^n=0$.
    So if $0\subset\mathcal{I}\subset\fr q$, for $\mathcal{I}$ prime, then 
    \[\fr q_{\fr q}\subset \sqrt{0_{\fr q}}\subset\sqrt{\mathcal{I}_{\fr q}}\subset \sqrt{\fr q_{\fr q}}.\]
    It follows that $\mathcal{I}_{\fr q}=\fr q_{\fr q}$, and hence $\mathcal{I}=\fr q$ and $\dim R_{\fr q}=0$.
\end{proof}

\begin{cor}
    If $R$ is Noetherian and $\fr p$ is a minimal prime over a nonzerodivisor $x$ then $\hgt\fr p=1$.
\end{cor}

\subsection*{November 13, 2014}

\begin{thm}[Generalized Hauptidealsatz]
    If $R$ is Noetherian and we have $x_1,\ldots,x_k\in R$ with $\fr p$ a minimal
    prime over $(x_1,\ldots,x_k)$ then $\hgt\fr p\leq k$.
\end{thm}

\begin{rem}
    Geometrically, we think of the following. If $V\subset\A^n$ (or more generally any affine
    variety) is one irreducible component of the zero set of $k$ equations, then
    $\codim V\leq k$.
\end{rem}

\begin{proof}
    We proceed by induction on $k$. The previous Hauptidealsatz proves $k=1$.
    Again, we can assume, without loss of generality, that $R$ is local and $\fr p$ is maximal.
    Consider $\fr q$ such that $\fr q\subsetneq \fr p$ and is maximal among primes satisfying
    this condition (there exists a maximal such prime due to the Noetherian hypothesis).
    Since $\fr p$ is minimal containing $(x_1,\ldots,x_k)$, we assume that $x_1\notin\fr q$.
    It suffices to show that $\fr q$ is the minimal prime over $k-1$ elements, by induction.
    Now since $x_1\notin\fr q$, $\fr p$ is minimal over $(x_1)+\fr q$. By two Corollaries above,
    we find that $\fr p$ is nilpotent mod $(x_1)+\fr q$ and hence
    \[x_i^n=a_ix_1+y_i,\]
    for $a_i\in R$ and $y_i\in\fr q$. We claim that $\fr q$ is minimal over $(y_2,\ldots,y_k)$.
    Indeed, $\fr p$ is minimal (among primes) over $(x_1,y_2,\ldots,y_k)$ so if $(x_1,y_2,\ldots,y_k)\subset\ell$
    for a prime $\ell$, we find that $x_i^n\in\fr l$ and hence $x_i\in\ell$. Now, by the
    ordinary Hauptidealsatz, we note that
    \[\frac{\fr p}{(y_2,\ldots,y_k)}\leq \frac{R}{(y_2,\ldots,y_k)}\]
    has height less than or equal to 1. Moreover,
    \[\frac{\fr q}{(y_2,\ldots,y_k)}\subsetneq\frac{\fr p}{(y_2,\ldots,y_k)}\]
    must have height 0, and hence $\fr q$ is minimal over $(y_2,\ldots,y_k)$.
\end{proof}

\begin{cor}
    For $k=\bar k$, $\dim k[x_1,\ldots,x_n]=n$.
\end{cor}
\begin{proof}
    A maximal increasing chain of primes ideals must end with a maximal ideal, which
    by the Nullstellensatz, is the ideal of a point in $\A^n$, call it, up to a change of
    coordinates, $\fr m=(x_1,\ldots,x_n)$. By the generalized Hauptidealsatz, we find that
    $\hgt\fr m\leq n$. But $\hgt\fr m\geq n$ due to the presence of the chain
    \[(0)\subset (x_1)\subset \cdots\subset (x_1,\ldots,x_n).\]
\end{proof}

\begin{rem}
    This holds true even for $k\neq\bar k$, but we will prove this later.
\end{rem}

\begin{defn}
    We define the \textbf{dimension} of an affine algebraic set $X$ to be the Krull dimension
    of the coordinate ring $\mathcal{O}_X(X)$.
\end{defn}

\begin{rem}
    It is clear that the dimension of an affine variety is the maximal length $m$ of
    chains of irreducible subvarieties
    \[V_0\subset V_1\subset\cdots\subset V_m\subset V.\]
\end{rem}

\begin{exc}
    \begin{enumerate}
        \item Show using the Hauptidealsatz that for a hypersurface $V(f)\subset\A^n$,
            we have that $\dim V(f)=n-1$ if $f$ is not constant.
        \item Let $f,g\in k[x_1,\ldots,x_n]$ be irreducible, with $f\neq \lambda g$,
            and let $V$ be any irreducible component of $V(f)\cap V(g)$. Then
            $\fr p_V\leq k[x_1,\ldots,x_n]/(f)$ is minimal over $(g)$ and hence $\hgt\fr p_V=1$
            in $\mathcal{O}_{V(f)}$. Then $\hgt \pi^{-1}\fr p_V=2$ in $k[x_1,\ldots,x_n]$
            and thus $\dim V=n-2$.
        \item Find hypotheses on $f,g,h$ such that each irreducible component of $V(f)\cap V(g)\cap V(h)$
            has codimension 3.
    \end{enumerate}
\end{exc}

\begin{rem}
    The following result is, in some sense, a converse to the generalized Hauptidealsatz.
\end{rem}

\begin{thm}
    If $R$ is Noetherian, then every prime of height $n$ is minimal over some ideal generated
    by $n$ elements.
\end{thm}

\begin{rem}
    Geometrically, we see that every irreducible variety of $\codim n$ is an irreducible component
    of an intersection of $n$ hypersurfaces. In general, however, it need not be precisely
    the intersection of $n$ hypersurfaces.
\end{rem}

\begin{lem}[Prime avoidance]
    Let $I$ be an ideal in $R$ such that $I\not\subset\fr p_1,\ldots,\fr p_n$ prime. Then
    $I\not\subset\fr p_1\cup\cdots\cup\fr p_n$.
\end{lem}
\begin{proof}
    Suppose $I\subset \fr p_1\cup\cdots\cup\fr p_n$ but $I$ is not contained in any $\fr p_i$.
    Discarding some of the $\fr p_i$, we may assume that $I$ is not contained in any smaller union.
    Then there exist $x_i\in I\setminus (\fr{p_1}\cup\cdots\cup\hat{\fr{p_i}}\cup\cdots\cup\fr{p_n})$.
    In particular, $x_i\in\fr p_i$. Then $x_1+x_2x_3\cdots x_n\in I\setminus(\fr p_1\cup\cdots\fr p_n)$,
    which is a contradiction.
\end{proof}

\begin{rem}
    What is the geometric intuition here?
\end{rem}

\begin{proof}[Proof of theorem]
    We proceed by induction on $m$. The case of $n=0$ is trivial (consider the ideal $(0)$).
    For $n=1$, $\sqrt{0}=\fr p_1\cap\cdots\cap\fr p_n$ by the Noetherian condition, where $\fr p_i$
    are all the primes of height 1.
    If $\hgt\fr p=1$,
    $\fr p\not\subset\fr p_i$ and hence by prime avoidance, $\not\subset \fr p_1\cup\cdots\cup\fr p_n$.
    Then there exists an $x\in\fr p\setminus(\fr p_1\cup\cdots\cup\fr p_n)$ with $\fr p$ minimal
    over $(x)$.

    Now consider $n>1$. There is a decreasing chain
    \[\fr p=\fr p_n\supsetneq\fr p_{n-1}\supsetneq\cdots\supsetneq \fr p_1\]
    with $\hgt\fr p_1=1$. The $n=1$ case tells us that $\fr p_1$ is minimal over
    $x_1$, where $x_1$ is not in any prime of height 0. Now we claim that in $R/(x_1)$
    $\hgt\bar{\fr p}=n-1$. Clearly $n-1\leq\hgt\bar{\fr p}\leq n$, but if 
    \[\bar{\fr p}=\bar{\fr q}_n\supsetneq\bar{\fr q}_{n-1}\supset\cdots\supset\bar{\fr q}_0\leq R/(x_1),\]
    then $\fr q_0\leq R$ is a prime of height 0 containing $x_1$, which is a contradiction.
    The induction hypothesis tells us that $\bar{\fr p}$ is minimal over $(\bar{x_2},\ldots,\bar{x_n})\leq R/(x_1)$,
    which shows that $\fr p$ is minimal over $(x_1,\ldots, x_n)$ for any lifts $x_i$.
\end{proof}



\subsection*{November 17, 2014}

Recall our wishlist for dimension:
\begin{enumerate}
    \item $\dim \A^n=n$;
    \item if $X\subset Y$ then $\dim X=\dim Y$;
    \item $\dim X\times Y=\dim X+\dim Y$;
    \item $X\subset Y$ closed implies $\dim X\leq\dim Y$ with equality if and only if $X=Y$;
    \item For $f\neq0\in k[x]$, $V(f)\neq\varnothing\subset X$, $\dim V(f)=\dim X-1$;
    \item if $X\cong Y$ then $\dim X=\dim Y$.
\end{enumerate}
Recall, we have shown $1,4,5,6$ for Krull dimension.

We return, for now, to integrality and localization. The following proposition
shows that localization commutes with integral closure.

\begin{prop}
    $A\subset B$ rings with $C$ the integral closure of $A$ in $B$, and with $S\subset A$ multiplicatively
    closed. Then $S^{-1}C$ is the integral closure of $S^{-1}A$ in $S^{-1}B$.
\end{prop}

\begin{proof}
    Take $s\in S,x\in C$, i.e.
    \[x^n+a_1x^{n-1}+\cdots+a_n=0\]
    for $a_i\in A$.  Dividing through by $s^{n}$, we find that $x/s$ lives in the integral
    closure of $S^{-1}A$.

    Conversely, if $x/s$ lives in the integral closure, i.e.
    \[\left(\frac{x}{s}\right)^n+\frac{\alpha_1}{s_1}\left( \frac{x}{s} \right)^{n-1}+\cdots+\frac{\alpha_n}{s_n}=0\]
    for $\alpha_i\in A$. Let $t=s_1\cdots s_n$ and multiply by $(st)^n$. Then we must have that $tx\in C$,
    i.e. $x=tx/t\in S^{-1}C$.
\end{proof}

\begin{cor}
    For $R$ a domain, the following are equivalent:
    \begin{enumerate}[(a)]
        \item $R$ is normal;
        \item $R_{\fr p}$ is normal for all prime $\fr p\subset R$
        \item $R_{\fr m}$ is normal for all maximals $\fr m\subset R$
    \end{enumerate}
    \begin{proof}
        (a) implies (b) via the above proposition, and (b) trivially implies (c).
        Moreover, (c) implies (a): let $C$ be the integral closure of $R$ in its
        field of fractions $F$. Then $C_{\fr m}$ is the integral closure of $R_{\fr m}$
        by the previous proposition, and $R_m\hookrightarrow C_{\fr m}$ is surjective
        by hypothesis. This implies that $R\hookrightarrow C$ is surjective.\footnote{This was shown earlier.}
    \end{proof}
\end{cor}

\begin{lem}
    If $S\subset R$ are normal domains, with $R$ integral over $S$, $R$ is
    a field if and only if $S$ is a field.
\end{lem}
\begin{proof}
    If $R$ is a field, then every $0\neq s\in S$ is a unit in $R$, and hence
    every $s\neq 0$ is a unit in $S$\footnote{See midterm.} and thus $S$ is a field.
    Conversely, if $S$ is a field, it suffices to find an inverse for $0\neq x\in R$.
    By integrality, we have
    \[x^n+s_1x^{n-1}+\cdots+s_n=0,\]
    for $n$ minimal (i.e. $s_n\neq 0$, as we could cancel $x$). If $n=1$, we are done, so we assume $n>1$.
    Multiplying through by $1/s_n$, we find that 
    \[1=0x(x^{n-1}+\cdots+s_{n-1}),\]
    which completes the proof.
\end{proof}

\begin{lem}
    Let $S\subset R$ be rings with $R$ integral over $S$. Let $I\subset R$ be an ideal
    and $J=I\cap S\subset S$ be the ideal intersected with $X$. Then:
    \begin{enumerate}[(a)]
        \item $S/J$ is a subring of $R/I$;
        \item $R/I$ is integral over $S/J$;
        \item If $I$ is prime, then $J$ is prime;
        \item If $I$ is prime, then $I$ is maximal if and only if $J$ is maximal.
    \end{enumerate}
\end{lem}
\begin{proof}
    (a), (b), and (c) are trivial. (d) follows from the previous lemma and (b).
\end{proof}

\begin{thm}
    Let $S\subset R$ be rings with $R$ integral over $S$. Let $\fr p\subset S$ be a prime.
    Then there exists a $\fr P\subset R$ prime such that $\fr P\cap S=\fr p$.
\end{thm} 

\begin{exmp}
    Consider $R=k[x]$ and $S=k[x^2]$. This yields a double-sheeted cover $\A^1\to\A^1$ sending $x\mapsto x^2$,
    i.e. $\fr p=(x^2-1)\subset S$ and $\fr P=(x-1)$ or $\fr P=(x+1)$. Thus the lift of a prime
    via the above theorem is not necessarily unique.
\end{exmp}

\begin{proof}
    Consider the localizations $S_{\fr p}$ and $R_{\fr p}=(S\setminus\fr p)^{-1}R$.
    Note that $R_{\fr p}$ need not be local in general, but by the earlier proposition
    is integral over $S_{\fr p}$. Let $\fr M\subset R_{\fr p}$ be a maximal ideal.\footnote{Use Zorn's lemma.}
    Then $\fr m=\fr M\cap S_{\fr p}$ is maximal by (d) of the lemma above. Of course, $\fr m$
    is the unique maximal of $S_{\fr p}$, i.e.
    \[\fr m=\{\frac{a}{b}\mid a\in\fr p,b\in S\setminus\fr p\}.\]
    Take the inverse image of $\fr M$ under $R\to R_{\fr p}$,
    \[\fr P\equiv\{x\in R\mid \frac{x}{1}\in\fr M\},\]
    which need not be maximal, but is certainly prime. We claim that $\fr P\cap S=\fr p$.
    First, if $x\in\fr p$, then $x\in S$ and $x/1\in\fr m\subset\fr M$ so $x\in\fr P$.
    Conversely, if $x\in \fr P\cap S$, then $x/1\in\fr M\cap S_{\fr p}=\fr m$ and hence
    $(bx-a)u=0$ in $S$, for some $b,u\in S\setminus\fr P$. But then $bxu=au\in\fr P$ so
    $x\in\fr P$ (since $b,u\in S\setminus\fr P$).
\end{proof}

\begin{lem}
    If $S\subset R$ are rings with $R$ integral over $S$, and $I\subset R$ is an ideal
    with $S\cap I=0$, then all $x\in I$ are zerodivisors.
\end{lem}
\begin{proof}
    For $0\neq x\in I$, let
    \[x^n+s_1x^{n-1}+\cdots + s_n=0,\]
    with $s_i\in S$, $n$ minimal. Then
    \[s_n=-x(x^{n-1}+s_1x^{n-2}+\cdots+s_{n-1}),\]
    but the right hand side is in $I$ and hence $s_n=0$. But minimality requires that
    the second term cannot be zero, and hence $x$ must be a zerodivisor.
\end{proof}

\begin{prop}
    If $S\subset R$ are rings and $R$ is integral over $S$, with $\fr P\subset Q\subset R$
    are ideals with $\fr P$ prime, if $\fr P\cap S=Q\cap S$, then $\fr P=\fr Q$.
\end{prop}
\begin{proof}
    Let $\fr p=\fr P\cap S$. Then $S/\fr p\subset R/\fr P$ and indeed, $R/\fr P$ is integral
    over $S/\fr p$ by a lemma above. By the ideal correspondence, we have an ideal
    $Q/\fr P\subset R/\fr p$ and since $\fr P\cap S=Q\cap S$, we find that $Q/\fr P\cap S/\fr p=0$.
    By the previous lemma, we find that all elements of $Q/\fr P$ are zerodivisors. Since
    $R/\fr P$ is an integral domain, $Q/\fr P=0$ and hence $Q=\fr P$.
\end{proof}

\section*{January 20, 2015}

The following is a course outline, not necessarily in any order:
\begin{itemize}
    \item Smooth curves via valuations (Hartshorne I.6)
    \item Properness \& the valuative criterion
    \item Graded rings, homogeneous ideals, projective varieties, saturation
    \item Rudiments of intersection theory (Hartshorne I.7, Kempf, Shafarevich)
    \item Smoothness criteria \& Bertini's theorem (Kempf)
    \item Normality \& Zariski's Main Theorem (Mumford)
    \item Differentials (Kempf)
    \item Coherent sheaves, divisors, line bundles, Riemann-Roch (Kempf, Hartshorne IV), cohomology
\end{itemize}

\subsection{Smooth \& normal varieties}

Henceforth we assume that all algebraic varieties are irreducible over an
algebraically closed field $k$.

Recall that a variety is a separated ringed space covered by finitely
many affine varieties. ``Schemes may occasionally be invoked, but mostly
only for comic relief.''

\begin{defn}
    For $x\in X$, the local ring $\mathcal{O}_{X,x}$ is the ring of 
    germs of rational functions on $X$ which are regular at $X$.
    In fact, $\mathcal{O}_{X,x}\cong k[U]_{\fr m}$, $U$ any affine neighborhood
    of $x$, $\fr m$ maximal ideal of $x$.
\end{defn}

\begin{exmp}
    The Zariski cotangent space $\fr m/\fr m^2$ is the same (naturally isomorphic)
    in $U$ as in $\mathcal{O}_{X,x}$. We know that $\dim X=\dim \mathcal{O}_{X,x}$
    (coordinate rings of affine varieties are catenary).
\end{exmp}

\begin{defn}
    A variety $X$ is \textbf{smooth} at $x\in X$ if $\mathcal{O}_{X,x}$ is
    \textbf{regular}, i.e. $\dim_k\fr m/\fr m^2=\dim X$.
\end{defn}

\begin{prop}
    For $X$ closed in $\A^n$, $X$ is smooth if and only if the Jacobian
    criterion is satisfied, i.e. there exist polynomials$f_1,\ldots,f_{n-\dim X}\in I(X)$
    whose derivatives are linearly independent at $x$.
\end{prop}
\begin{proof}
    Without loss of generality, assume we are working at the origin. Let $I=I(X)$ and
    $(x_i)=\fr m\subset k[x_i]/I$ the ideal of $x$. Then
    \[\fr m=\frac{(x_i)}{I} \text{ for }(x_i)\subset k[x_i]\]
    so
    \[\frac{\fr m}{\fr m^2}=\frac{(x_i)}{I+(x_ix_j)}=\frac{k\langle x_i\rangle}{\text{linear terms of elts of }I},\]
    where the angle brackets represent linear span. If $X$ is smooth of $\dim d$,
    there exist $n-d$ independent linear terms. Conversely, if there exist $n-d$ independent linear terms,
    $\dim X\leq \dim\fr m/\fr m^2\leq d$.
\end{proof}

\begin{rem}
    If the field is not algebraically closed of positive characteristic,
    a closed point of a (affine) scheme does not satisfy the Jacobian condition
    but local ring may be regular. Suppose, for example,
    we have $a\in k\setminus k^p$, i.e. $x^p-a$ has no root (e.g. $x\in\F_p(x)$).
    Then consider $f(x,y)=y^2-x^p+a$ and $V(f)\subset\A^2=\Spec k[x,y]$ (note $\A^2\neq k^2$).
    One checks that $f$ is an irreducible polynomial and so $V(f)$ is indeed a variety.
    But note that $\partial f/\partial x=0$ and $\partial f/\partial y=0$. So at
    the maximal ideal $(y,x^p-a)=(y)$ the Jacobian criterion is not satisfied.
    But if $\fr m=(y)$ then $\dim\fr m/\fr m^2=1$. Hence the concepts of smoothness
    and regular are not equivalent in general for schemes.
\end{rem}


\end{document}

