\documentclass{../mathnotes}


\title{QM for Mathematicians: PSET 7}
\author{Nilay Kumar}
\date{Last updated: \today}


\begin{document}

\maketitle

\subsection*{Problem 1}

We wish to show that elementary antisymmetric matrices $L_{jk}$ satisfy the same commutation relations as the quadratic elements
$\gamma_j\gamma_k/2$ of the Clifford algebra ${\rm Cliff}(n,\mathbb{R})$. First note that $(L_{jk})_{\alpha\beta}=-\delta_{j\alpha}\delta_{k\beta}
+\delta_{j\beta}\delta_{k\alpha}$, and so we have
\begin{align*}
    [L_{jk}, L_{mn}]_{\alpha\beta}=&(L_{jk}L_{mn})_{\alpha\beta}-(L_{mn}L_{jk})_{\alpha\beta}\\
    =&(L_{jk})_{\alpha\gamma}(L_{mn})_{\gamma\beta}-(L_{mn})_{\alpha\gamma}(L_{jk})_{\gamma\beta}\\
    =&\sum_\gamma\left( -\delta_{j\alpha}\delta_{k\gamma}+\delta_{j\gamma}\delta_{k\alpha} \right) \left( -\delta_{m\gamma}\delta_{n\beta}+\delta_{m\beta}\delta_{n\gamma} \right)\\
    &-\sum_\gamma\left( -\delta_{m\alpha}\delta_{n\gamma}+\delta_{m\gamma}\delta_{n\alpha} \right) \left( -\delta_{j\gamma}\delta_{k\beta}+\delta_{j\beta}\delta_{k\gamma} \right)\\
    =&\sum_\gamma\delta_{m\gamma}\delta_{j\gamma}L_{kn}+\delta_{n\gamma}\delta_{j\gamma}L_{mk}+\delta_{k\gamma}\delta_{n\gamma}L_{jm}+\delta_{k\gamma}\delta_{m\gamma}L_{nj}\\
    =&\delta_{mj}L_{kn}+\delta_{nj}L_{mk}+\delta_{kn}L_{jm}+\delta_{km}L_{nj},
\end{align*}
where we have expanded out the matrix multiplication and then, in the last step, collapsed the sum over $\gamma$ via the Kronecker deltas.

Now, for the Clifford algebra, using the anticommutation relation $[\gamma_j,\gamma_k]_+=2\delta_{jk}$,
\begin{align*}
    \frac{1}{4}[\gamma_j\gamma_k,\gamma_m\gamma_n]&=\frac{1}{4}\left(\gamma_j\gamma_k\gamma_m\gamma_n-\gamma_m\gamma_n\gamma_j\gamma_k\right)\\
    &=\frac{1}{4}\left(\gamma_j\gamma_k\gamma_m\gamma_n+\gamma_m\gamma_j\gamma_n\gamma_k+2\delta_{jn}\gamma_m\gamma_k\right)\\
    &=\frac{1}{4}\left(\gamma_j\gamma_k\gamma_m\gamma_n-\gamma_j\gamma_m\gamma_n\gamma_k+2\delta_{jn}\gamma_m\gamma_k-2\delta_{jm}\gamma_n\gamma_k\right)\\
    &=\;\cdots\\
    &=\frac{1}{2}\delta_{jm}\gamma_k\gamma_n+\frac{1}{2}\delta_{nj}\gamma_m\gamma_k+\frac{1}{2}\delta_{kn}\gamma_j\gamma_m+\frac{1}{2}\delta_{km}\gamma_n\gamma_j,
\end{align*}
recovering the same commutation relations as above. Thus, the Lie algebras of ${\rm Spin}(n)$ and ${\rm SO(n)}$ are the same.

\subsection*{Problem 2}

To show that conjugation by an exponential of the quadratic Clifford algebra $\frac{1}{2}\gamma_j\gamma_k$ yields a rotation,
we first note that the square of this quadratic element is simply $-\frac{1}{4}$. This allows us to write, 
\begin{align*}
    e^{\frac{\theta}{2}\gamma_j\gamma_k}=\cos\left( \frac{\theta}{2} \right)+\gamma_j\gamma_k\sin\left( \frac{\theta}{2} \right).
\end{align*}
We then have, using the anti-commutation relation $[\gamma_j,\gamma_k]_+=2\delta_{jk}$,
\begin{align*}
    e^{-\frac{\theta}{2}\gamma_j\gamma_k}&(v_j\gamma_j+v_k\gamma_k)e^{\frac{\theta}{2}\gamma_j\gamma_k}=\left( \cos\frac{\theta}{2}-\gamma_j\gamma_k\sin\frac{\theta}{2} \right)
    \left(v_j\gamma_j+v_k\gamma_k  \right)\left( \cos\frac{\theta}{2}+\gamma_j\gamma_k\sin\frac{\theta}{2} \right)\\
    =&\cos^2\frac{\theta}{2}(v_j\gamma_j+v_k\gamma_k)+\cos\frac{\theta}{2}\sin\frac{\theta}{2}(v_j\gamma_j+v_k\gamma_k)\gamma_j\gamma_k\\
    &-\cos\frac{\theta}{2}\sin\frac{\theta}{2}\gamma_j\gamma_k(v_j\gamma_j+v_k\gamma_k)-\sin^2\frac{\theta}{2}\gamma_j\gamma_k(v_j\gamma_j+v_k\gamma_k)\gamma_j\gamma_k\\
    =&\cos^2\frac{\theta}{2}(v_j\gamma_j+v_k\gamma_k)-\sin\theta\gamma_j\gamma_k(v_j\gamma_j+v_k\gamma_k)+\delta_{jk}\sin\theta(v_j\gamma_j+v_k\gamma_k)\\
    &-\sin^2\frac{\theta}{2}(v_j\gamma_j+v_k\gamma_k+2v_j\gamma_j\gamma_k\gamma_j\delta_{jk}+2v_k\gamma_j\gamma_k\gamma_k\delta_{jk})
\end{align*}
If we assume that $j\neq k$, then this simplifies via a double angle identity to:
\begin{align*}
    &\cos\theta(v_j\gamma_j+v_k\gamma_k)-\gamma_j\gamma_k\sin\theta(v_j\gamma_j+v_k\gamma_k)\\
    =&\cos\theta(v_j\gamma_j+v_k\gamma_k)-\sin\theta(-v_j\gamma_k+v_k\gamma_j)\\
    =&(v_j\cos\theta-v_k\sin\theta)\gamma_j+(v_j\sin\theta+v_k\cos\theta)\gamma_k,
\end{align*}
and so we are left with a rotation in the $j-k$ plane, as desired.

\subsection*{Problem 3}

If we consider the fermionic oscillator with 2 degrees of freedom, the Hamiltonian becomes
\begin{align*}
    H&=a_1^\dagger a_1+a_2^\dagger a_2 -\mathbf{1}\\
    &=N_1+N_2-\mathbf{1},
\end{align*}
and since spinors are eigenstates of the Hamiltonian, they are necessarily eigenstates of $\hat U=\exp(iH\theta)$. Consequently,
we can write
\begin{align*}
    \hat U|0,0\rangle&=e^{-i\theta}|0,0\rangle\\
    \hat U|1,0\rangle&=|1,0\rangle\\
    \hat U|0,1\rangle&=|0,1\rangle\\
    \hat U|1,1\rangle&=e^{i\theta}|1,1\rangle.
\end{align*}
The Hamiltonian thus generates a $U(1)$ action on the spinors as
\begin{align*}
    \hat U \iff
    \left(
    \begin{array}[]{cccc}
        e^{-i\theta} &  &  & \\
         & 1 &  & \\
         &  & 1 & \\
         &  &  & e^{i\theta}
    \end{array}
    \right)
\end{align*}
is clearly a unitary matrix. This is easily extended to three degrees of freedom:
\begin{align*}
    H&=a_1^\dagger a_1+a_2^\dagger a_2 -a_3^\dagger a_3 -\frac{3}{2}\\
    &=N_1+N_2+N_3-\frac{3}{2},
\end{align*}
and
\begin{align*}
    \hat U|0,0,0\rangle&=e^{-3i\theta/2}|0,0,0\rangle\\
    \hat U|1,0,0\rangle&=e^{-i\theta/2}|1,0,0\rangle\\
    \hat U|0,1,0\rangle&=e^{-i\theta/2}|0,1,0\rangle\\
    \hat U|0,0,1\rangle&=e^{-i\theta/2}|0,0,1\rangle\\
    \hat U|1,1,0\rangle&=e^{i\theta/2}|1,1,0\rangle\\
    \hat U|1,0,1\rangle&=e^{i\theta/2}|1,0,1\rangle\\
    \hat U|0,1,1\rangle&=e^{i\theta/2}|0,1,1\rangle\\
    \hat U|1,1,1\rangle&=e^{3i\theta/2}|1,1,1\rangle\\
\end{align*}
with the unitary matrix representation
\begin{align*}
    \hat U \iff
    \left(
    \begin{array}[]{cccccccc}
        e^{-3i\theta/2} &  &  &  &  &  &  & \\
         & e^{-i\theta/2} &  &  &  &  &  & \\
         &  & e^{-i\theta/2} &  &  &  &  & \\
         &  &  & e^{-i\theta/2} &  &  &  & \\
         &  &  &  & e^{i\theta/2} &  &  & \\
         &  &  &  &  & e^{i\theta/2} &  & \\
         &  &  &  &  &  & e^{i\theta/2} & \\
         &  &  &  &  &  &  & e^{3i\theta/2}\\
    \end{array}
    \right).
\end{align*}

If we consider the Clifford algebra in 4 dimensions, i.e. the fermionic oscillator with 2 degrees of freedom, we have
\begin{align*}
    \gamma_1&=a_1+a_1^\dagger\\
    \gamma_2&=-i(a_1-a_1^\dagger)\\
    \gamma_3&=a_2+a_2^\dagger\\
    \gamma_4&=-i(a_2-a_2^\dagger).
\end{align*}
The Hamiltonian, in this setup is given by
\begin{align*}
    H&=a_1^\dagger a_1 + a_2^\dagger a_2 + -\mathbf{1}\\
    &=\frac{i}{4}\left( \gamma_1\gamma_2-\gamma_2\gamma_1+\gamma_3\gamma_4-\gamma_4\gamma_3 \right)\\
    &=\frac{i}{2}\left( \gamma_1\gamma_2+\gamma_3\gamma_4 \right),
\end{align*}
using the anticommutation rules.
It should be fairly obvious how to extend $\gamma_i$ and the Hamiltonian to an arbitrary number of even dimension of the algebra.

Now, if we consider the action of the Hamiltonian on vectors in the Clifford algebra, we have in $n=2m$ dimensions,
\begin{align*}
    e^{-iH\theta}\left(\sum_jv_j\gamma_j\right)e^{iH\theta}=e^{-\frac{\theta}{2}\sum_k^m \gamma_{2k-1}\gamma_{2k}}\left(\sum_jv_j\gamma_j\right)e^{-\frac{\theta}{2}\sum_l^m \gamma_{2l-1}\gamma_{2l}}
\end{align*}
We can take advantage of the fact that $[\gamma_j\gamma_k,\gamma_m\gamma_n]=0$ (due to anticommutativity) for the two sets of $\gamma$'s from distinct copies of $\mathbb{C}$,
and use the Baker-Campbell-Hausdorff formula to factor the exponentials:
\begin{align*}
   \prod_k^m(e^{-\frac{\theta}{2}\gamma_{2k-1}\gamma_{2k}})\left(\sum_jv_j\gamma_j\right)\prod_l^m(e^{-\frac{\theta}{2}\gamma_{2l-1}\gamma_{2l}})
\end{align*}
Thankfully, many of these terms vanish -- in particular due to the commutation rules for $\gamma$'s (and their products, and the exponentials of their products),
the exponentials of quadratics when acting on terms in the sum that are not in the same copy of $\mathbb{C}$ commute through and annihilate with the conjugates
on the right-hand side. Thus, the terms that do not vanish simply give rotations of the form of those in problem 2, with one for $m$:
\begin{align*}
    &\sum_m^n e^{-\frac{\theta}{2}\gamma_{2m-1}\gamma_{2m}}(v_{2m-1}\gamma_{2m-1}+v_{2m}\gamma_{2m})e^{-\frac{\theta}{2}\gamma_{2m-1}\gamma_{2m}}\\
    =&\sum_m^n (v_{2m-1}\cos\theta-v_{2m}\sin\theta)\gamma_{2m-1}+(v_{2m-1}\sin\theta+v_{2m}\cos\theta)\gamma_{2m}
\end{align*}
The matrix representation of this rotation is of course just a block diagonal matrix with each block being a rotation in that copy of $\mathbb{C}$.

Next, let us consider the rotation by an angle $\theta$ in the $j-k$ plane in 4 or 6 dimensions. The element of the spin group that perform such a
rotation, we saw earlier, are of the form $\exp(\theta\gamma_j\gamma_k/2)$. If we act these on spinors of the fermionic oscillator space, we
must consider the case where the two $\gamma$'s are from the same copy of $\mathbb{C}$ as well as the case in which they are from different
copies. The first case is fairly simple:
\begin{align*}
    e^{\frac{\theta}{2}\gamma_j\gamma_k}&=\cos\frac{\theta}{2}+\gamma_j\gamma_k\sin\frac{\theta}{2}\\
    &=\cos\frac{\theta}{2}+i\sin\frac{\theta}{2}-2i\sin\frac{\theta}{2}N_1\\
    &=e^{i\frac{\theta}{2}}-2i\sin\frac{\theta}{2}N_1
\end{align*}
where we have used the fact that
\begin{align*}
    \gamma_j\gamma_k=(a_1+a_1^\dagger)\cdot -i(a_1-a_1^\dagger)=-2ia_1^\dagger a_1+i.
\end{align*}
If we denote such rotations by $\hat R$, we can write for 4 dimensions,
\begin{align*}
    \hat R|0,0\rangle&=e^{i\theta/2}|0,0\rangle\\
    \hat R|1,0\rangle&=e^{-i\theta/2}|1,0\rangle\\
    \hat R|0,1\rangle&=e^{i\theta/2}|0,1\rangle\\
    \hat R|1,1\rangle&=e^{-i\theta/2}|1,1\rangle.
\end{align*}
so we can represent
\begin{align*}
    \hat R \iff
    \left(
    \begin{array}[]{cccc}
        e^{i\theta/2} &  &  & \\
         & e^{-i\theta/2} &  & \\
         &  & e^{i\theta/2} & \\
         &  &  & e^{-i\theta/2}
    \end{array}
    \right)
\end{align*}
For 6 dimensions,
\begin{align*}
    \hat R|0,0,0\rangle&=e^{i\theta/2}|0,0,0\rangle\\
    \hat R|1,0,0\rangle&=e^{-i\theta/2}|1,0,0\rangle\\
    \hat R|0,1,0\rangle&=e^{i\theta/2}|0,1,0\rangle\\
    \hat R|0,0,1\rangle&=e^{i\theta/2}|0,0,1\rangle\\
    \hat R|1,1,0\rangle&=e^{-i\theta/2}|1,1,0\rangle\\
    \hat R|1,0,1\rangle&=e^{-i\theta/2}|1,0,1\rangle\\
    \hat R|0,1,1\rangle&=e^{i\theta/2}|0,1,1\rangle\\
    \hat R|1,1,1\rangle&=e^{-i\theta/2}|1,1,1\rangle\\
\end{align*}
with the unitary matrix representation
\begin{align*}
    \hat R \iff
    \left(
    \begin{array}[]{cccccccc}
        e^{i\theta/2} &  &  &  &  &  &  & \\
         & e^{-i\theta/2} &  &  &  &  &  & \\
         &  & e^{i\theta/2} &  &  &  &  & \\
         &  &  & e^{i\theta/2} &  &  &  & \\
         &  &  &  & e^{-i\theta/2} &  &  & \\
         &  &  &  &  & e^{-i\theta/2} &  & \\
         &  &  &  &  &  & e^{i\theta/2} & \\
         &  &  &  &  &  &  & e^{-i\theta/2}\\
    \end{array}
    \right).
\end{align*}

Things get more complicated when we consider rotations that are not generated by elements in the same copy of $\mathbb{C}$.
\begin{align*}
    e^{\frac{\theta}{2}\gamma_j\gamma_n}&=\cos\frac{\theta}{2}+\gamma_j\gamma_n\sin\frac{\theta}{2}\\
    &=\cos\frac{\theta}{2}-i\sin\frac{\theta}{2}\left( a_1a_2-a_1^\dagger a_2^\dagger+a_1^\dagger a_2 -a_1a_2^\dagger \right).
\end{align*}
Again, we write for 4 dimensions
\begin{align*}
    \hat R|0,0\rangle&=\cos\frac{\theta}{2}|0,0\rangle + i\sin\frac{\theta}{2}|1,1\rangle\\
    \hat R|0,1\rangle&=\cos\frac{\theta}{2}|0,1\rangle - i\sin\frac{\theta}{2}|1,0\rangle\\
    \hat R|1,0\rangle&=\cos\frac{\theta}{2}|1,0\rangle + i\sin\frac{\theta}{2}|0,1\rangle\\
    \hat R|1,1\rangle&=\cos\frac{\theta}{2}|1,1\rangle - i\sin\frac{\theta}{2}|0,0\rangle\\
\end{align*}
so we can represent
\begin{align*}
    \hat R \iff
    \left(
    \begin{array}[]{cccc}
        \cos\frac{\theta}{2} &  &  &i\sin\frac{\theta}{2} \\
        & \cos\frac{\theta}{2} & -i\sin\frac{\theta}{2} & \\
        & i\sin\frac{\theta}{2} & \cos\frac{\theta}{2} & \\
        -i\sin\frac{\theta}{2}&  &  & \cos\frac{\theta}{2}
    \end{array}
    \right)
\end{align*}
For 6 dimensions,
\begin{align*}
    \hat R|0,0,0\rangle&=\cos\frac{\theta}{2}|0,0,0\rangle+i\sin\frac{\theta}{2}|1,1,0\rangle\\
    \hat R|1,0,0\rangle&=\cos\frac{\theta}{2}|1,0,0\rangle+i\sin\frac{\theta}{2}|0,1,0\rangle\\
    \hat R|0,1,0\rangle&=\cos\frac{\theta}{2}|0,1,0\rangle-i\sin\frac{\theta}{2}|1,0,0\rangle\\
    \hat R|0,0,1\rangle&=\cos\frac{\theta}{2}|0,0,1\rangle+i\sin\frac{\theta}{2}|1,1,1\rangle\\
    \hat R|1,1,0\rangle&=\cos\frac{\theta}{2}|1,1,0\rangle-i\sin\frac{\theta}{2}|0,0,0\rangle\\
    \hat R|1,0,1\rangle&=\cos\frac{\theta}{2}|1,0,1\rangle+i\sin\frac{\theta}{2}|0,1,1\rangle\\
    \hat R|0,1,1\rangle&=\cos\frac{\theta}{2}|0,1,1\rangle-i\sin\frac{\theta}{2}|1,0,1\rangle\\
    \hat R|1,1,1\rangle&=\cos\frac{\theta}{2}|1,1,1\rangle-i\sin\frac{\theta}{2}|0,0,1\rangle\\
\end{align*}
with the matrix representation
\begin{align*}
    \hat R \iff
    \left(
    \begin{array}[]{cccccccc}
        \cos\frac{\theta}{2} &  &  &  &i\sin\frac{\theta}{2}  &  &  & \\
        &\cos\frac{\theta}{2} &i\sin\frac{\theta}{2}  &  &  &  &  & \\
        & -i\sin\frac{\theta}{2} & \cos\frac{\theta}{2} &  &  &  &  & \\
        &  &  & \cos\frac{\theta}{2} &  &  &  & i\sin\frac{\theta}{2}\\
        -i\sin\frac{\theta}{2}&  &  &  & \cos\frac{\theta}{2} &  &  & \\
        &  &  &  &  & \cos\frac{\theta}{2} &\sin\frac{\theta}{2}  & \\
        &  &  &  &  & -i\sin\frac{\theta}{2} & \cos\frac{\theta}{2} & \\
        &  &  & -i\sin\frac{\theta}{2} &  &  &  & \cos\frac{\theta}{2}\\
    \end{array}
    \right).
\end{align*}

\subsection*{Problem 4}

Recall that for the classic analog of fermionic quantum mechanics, we define Grassman variables $\theta_i$ that are defined by
$\left\{ \theta_j,\theta_k \right\}_+=2\delta_{jk}$. As this implies that $\theta_i^2=1$, all polynomials in an arbitrary number
of Grassman variables are below degree 2 in each $\theta_i$. We defined $\psi_i=\theta_{2i-1}+i\theta_{2i}$ with 
\[\left\{ \frac{\partial}{\partial \psi_i},\psi_i \right\}_+=1,\]
and
\[\left\{ \frac{\partial}{\partial \psi_i},\frac{\partial}{\partial \psi_i} \right\}_+=0=\left\{ \psi_i,\psi_i \right\}_+,\]
where $\psi$ is constructed in analog to $a^\dagger$ and $\frac{\partial}{\partial \psi_i}$ is constructed in analog to $a$ (we require the dimension
of the Grassman algebra to be even, of course). This is
consistent with the idea that polynomials in this mechanics represent our states, and thus derivatives ``annihilate'' states, while
multiplying by $\psi$ ``raises'' states (degree-wise). In parallel to the inner product defined for the Bargmann-Fock
representation of the bosonic oscillator, we define the inner product via the Berezin integral:
\begin{align*}
    \langle f_1 | f_2 \rangle = \int \overline{f_1(\psi)}f_2(\psi)\prod_j^m e^{\bar\psi_j \psi_j}d\psi_j d\bar\psi_j,
\end{align*}
where $f_1,f_2$ are polynomials of $m$ Grassman variables (abbreviated as $\psi$) and the integrals are taken over all such variables.
To see that our $\psi_i$ and $\partial/\partial\psi_i$ are adjoint with respect to this inner product, we require
\begin{align*}
    \langle f_1 | \psi_i | f_2 \rangle =  \langle f_2 | \frac{\partial}{\partial \psi_i} | f_1 \rangle^*.
\end{align*}
Let us start with the right hand side:
\begin{align*}
    \langle f_2 | \frac{\partial}{\partial \psi_i} | f_1 \rangle^*&=\left(\int \overline{f_1(\psi)}\frac{\partial}{\partial \psi_i}f_2(\psi)\prod_j^me^{\bar\psi_j \psi_j} d\psi_j d\bar\psi_j\right)^*\\
    &=\int f_1(\psi)\frac{\partial}{\partial\bar\psi_i}\overline{f_2(\psi)}\prod_j^me^{\psi_j\bar\psi_j} d\bar\psi_j d\psi_j\\
    &=\int f_1(\psi)\overline{f_2(\psi)}\prod_j^m\frac{\partial}{\partial\bar\psi_i}e^{\bar\psi_j\psi_j} d\psi_j d\bar\psi_j\\
    &=\int f_1(\psi)\overline{f_2(\psi)}\psi_i\prod_j^me^{\bar\psi_j\psi_j} d\psi_j d\bar\psi_j\\
    &=\int f_1(\psi)\psi_i\overline{f_2(\psi)}\prod_j^me^{\bar\psi_j\psi_j} d\psi_j d\bar\psi_j
\end{align*}
where in the third equality we have integrated by parts (as legitimized in class) and have used the fact that $[\psi_j,\bar\psi_j]=0$ 
to reorder terms. Additionally, we have used the fact that the derivative of the exponential behaves as usual (the exponential is just a linear polynomial
as all higher terms vanish). But this is just the left hand side of the requirement for adjointness, and so we are done.

\end{document}
