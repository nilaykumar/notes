\documentclass{../mathnotes}


\title{QM for Mathematicians: PSET 9}
\author{Nilay Kumar}
\date{Last updated: \today}


\begin{document}

\maketitle

\subsection*{Problem 1}

We wish to compute commutators of the form $[K_l,P_m]$ where $K_l$ generate boosts and $P_m$ generate
translations. We can do this by following the derivation in the notes:
\begin{align*}
    [X,Y]&=\frac{d}{dt}\left( e^{tX}Ye^{-tX} \right)|_{t=0},
\end{align*}
where the term in parentheses, using the property of semi-direct products becomes, for a translation $(a,1)\in\mathcal{P}$
and a Lorentz transformation $(0,\Lambda)\in\mathcal{P}$,
\begin{align*}
    (0,\Lambda)(a,1)(0,\Lambda)^{-1}&=(0,\Lambda)(a,1)(0,\Lambda^{-1})\\
    &=(0,\Lambda)(a+\Lambda^{-1}0,\Lambda^{-1})\\
    &=(0,\Lambda)(a,\Lambda^{-1})\\
    &=(\Lambda a,1)
\end{align*}
If we take a $\Lambda=e^{tK_1}$, we find:
\begin{align*}
    \frac{d}{dt}\Lambda a|_{t=0}=\frac{d}{dt}
    \left( 
    \begin{array}[]{cccc}
        \cosh t & \sinh t & 0 & 0\\ 
        \sinh t & \cosh t & 0 & 0\\ 
        0 & 0 & 1 & 0\\ 
        0 & 0 & 0 & 1 
    \end{array}
    \right)
    \left( 
    \begin{array}[]{c}
        P_0\\
        P_1\\
        P_2\\
        P_3
    \end{array}
    \right)
    \bigg|_{t=0}=
    \left( 
    \begin{array}[]{c}
        P_1\\
        P_0\\
        0\\
        0
    \end{array}
    \right).
\end{align*}
For $\Lambda=e^{tK_2}$, we find:
\begin{align*}
    \frac{d}{dt}\Lambda a|_{t=0}=\frac{d}{dt}
    \left( 
    \begin{array}[]{cccc}
        \cosh t & 0 & \sinh t & 0\\ 
        0 & 1 & 0 & 0\\ 
        \sinh t & 0 & \cosh t & 0\\ 
        0 & 0 & 0 & 1 
    \end{array}
    \right)
    \left( 
    \begin{array}[]{c}
        P_0\\
        P_1\\
        P_2\\
        P_3
    \end{array}
    \right)
    \bigg|_{t=0}=
    \left( 
    \begin{array}[]{c}
        P_2\\
        0\\
        P_0\\
        0
    \end{array}
    \right).
\end{align*}
For $\Lambda=e^{tK_3}$, we find:
\begin{align*}
    \frac{d}{dt}\Lambda a|_{t=0}=\frac{d}{dt}
    \left( 
    \begin{array}[]{cccc}
        \cosh t & 0 & 0 & \sinh t\\ 
        0 & 1 & 0 & 0\\ 
        0 & 0 & 1 & 0\\ 
        \sinh t & 0 & 0 & \cosh t 
    \end{array}
    \right)
    \left( 
    \begin{array}[]{c}
        P_0\\
        P_1\\
        P_2\\
        P_3
    \end{array}
    \right)
    \bigg|_{t=0}=
    \left( 
    \begin{array}[]{c}
        P_3\\
        0\\
        0\\
        P_0 
    \end{array}
    \right).
\end{align*}
Thus we see that in general we have $[K_j,P_0]=P_j$, $[K_j,P_j]=P_0$, and $[K_j,P_k]=0$ if $j\neq k,k\neq 0$ (otherwise).

\subsection*{Problem 2}

For the real scalar quantum field theory, we can write the momentum operator as:
\begin{align*}
    \hat P&=-i\int d^3x \dot\phi(x) \left( -i\nabla \right)\phi(x)=-\int d^3x \dot\phi(x)\nabla\phi(x)
\end{align*}
using $\pi=\dot\phi$. We can now use the expansion:
\begin{align*}
    \phi(x)&=\int \frac{d^3k}{(2\pi)^{3/2}\sqrt{2\omega_k}}\left( a_ke^{-ikx}+a_k^\dagger e^{ikx} \right)\\
    \dot\phi(x)&=\int \frac{d^3k}{(2\pi)^{3/2}\sqrt{2\omega_k}}i\omega_k\left( a_ke^{-ikx}-a_k^\dagger e^{ikx} \right)\\
    \nabla\phi(x)&=\int \frac{d^3k}{(2\pi)^{3/2}\sqrt{2\omega_k}}ik\left( -a_ke^{-ikx}+a_k^\dagger e^{ikx} \right)\\
\end{align*}
Inserting these above yields:
\begin{align*}
    \hat P&=-\int d^3x\frac{d^3k d^3k'}{(2\pi)^3 \sqrt{4\omega_k\omega_{k'}}}k'\omega_{k}\left( a_ke^{-ikx}-a_k^\dagger e^{ikx} \right)\left( -a_{k'}e^{-ik'x}+a_{k'}^\dagger e^{ik'x} \right)\\
    &=-\int d^3x\frac{d^3k d^3k'}{(2\pi)^3 \sqrt{4\omega_k\omega_{k'}}}k'\omega_{k}\left( -a_ka_{k'}e^{-ix(k+k')}-a_k^\dagger a_{k'}^\dagger e^{ix(k+k')}+a_k^\dagger a_{k'}e^{ix(k-k')}+a_ka_{k'}^\dagger e^{ix(k-k')} \right)\\
    &=-\int\frac{d^3k d^3k'}{(2\pi)^3 \sqrt{4\omega_k\omega_{k'}}}k'\omega_{k}\left( -a_ka_{k'}-a_k^\dagger a_{k'}^\dagger\right)\delta(k+k')+k'\omega_k\left(a_k^\dagger a_{k'}+a_ka_{k'}^\dagger\right)\delta(k-k')\\
    &=-\int\frac{d^3k}{(2\pi)^3 2\omega_k}k\omega_k\left( -a_ka_{-k}-a_k^\dagger a_{-k}^\dagger \right)+k\omega_k\left(a_k^\dagger a_k+a_ka_k^\dagger\right)
\end{align*}
Note that the first term vanishes as it is odd, and we are left with
\begin{align*}
    \hat P=&\int \frac{d^3k}{(2\pi)^3 2\omega_k}k\omega_k \left( a_k^\dagger a_k+a_ka_k^\dagger \right)\\
    &=\int \frac{d^3k}{(2\pi)^3} k\; a_k^\dagger a_k
\end{align*}

On the other hand, for the complex quantum field theory, we have
\begin{align*}
    \phi(x)&=\int \frac{d^3k}{(2\pi)^{3/2}\sqrt{2\omega_k}}\left( a_ke^{ikx}+b_k^\dagger e^{-ikx} \right)\\
    \dot\phi(x)&=\int \frac{d^3k}{(2\pi)^{3/2}\sqrt{2\omega_k}}i\omega_k\left( -a_ke^{ikx}+b_k^\dagger e^{-ikx} \right)\\
    \nabla\phi(x)&=\int \frac{d^3k}{(2\pi)^{3/2}\sqrt{2\omega_k}}ik\left( a_ke^{ikx}-b_k^\dagger e^{-ikx} \right)\\
    \phi^\dagger(x)&=\int \frac{d^3k}{(2\pi)^{3/2}\sqrt{2\omega_k}}\left( a_k^\dagger e^{-ikx}+b_k e^{ikx} \right)\\
    \dot\phi^\dagger(x)&=\int \frac{d^3k}{(2\pi)^{3/2}\sqrt{2\omega_k}}i\omega_k\left( a_k^\dagger e^{-ikx}-b_ke^{ikx} \right)\\
    \nabla\phi^\dagger(x)&=\int \frac{d^3k}{(2\pi)^{3/2}\sqrt{2\omega_k}}ik\left(-a_k^\dagger e^{-ikx}+b_ke^{ikx} \right).
\end{align*}
The momentum density is given by:
\begin{align*}
    \mathcal{T}^{0i}=-\frac{\p\mathcal{L}}{\p\dot\phi}\nabla\phi-\nabla\phi^\dagger\frac{\p\mathcal{L}}{\p\dot\phi^\dagger}
\end{align*}
and so the momentum is found by integrating over space:
\begin{align*}
    \hat P=-\int d^3x \left( \dot\phi^\dagger\nabla\phi+\nabla\phi^\dagger\dot\phi \right)
\end{align*}
Let us look at the first term:
\begin{align*}
    \hat P_1&=\int \frac{d^3xd^3kd^3l}{(2\pi)^3\sqrt{4\omega_k\omega_l}}\omega_kl\left( a_k^\dagger e^{-ikx}-b_ke^{ikx} \right)
    \left( a_le^{ilx}-b_l^\dagger e^{-ilx} \right)\\
    &=\int \frac{d^3xd^3kd^3l}{(2\pi)^3\sqrt{4\omega_k\omega_l}}\omega_kl\left( a_k^\dagger a_le^{-i(k-l)x}+b_kb_l^\dagger e^{i(k-l)x}
    -a_k^\dagger b_l^\dagger e^{-i(k+l)x}-b_ka_l e^{i(k+l)x}\right)\\
    &=\int \frac{d^3kd^3l}{(2\pi)^3\sqrt{4\omega_k\omega_l}}\omega_kl\left( a_k^\dagger a_l\delta(k-l)+b_kb_l^\dagger \delta(k-l)
    -a_k^\dagger b_l^\dagger \delta(k+l)-b_ka_l \delta(k+l)\right)\\
    &=\int\frac{d^3k}{(2\pi)^32} k\left(a_k^\dagger a_k+b_k^\dagger b_k\right)
\end{align*}
where in the last step we have discarded the odd term and in the remaining term made use of commutation relations. The
term $\hat P_2$ contributes precisely the same term and thus the total momentum operator is given by:
\[\hat P=\int\frac{d^3k}{(2\pi)^3} k\left(a_k^\dagger a_k+b_k^\dagger b_k\right)\]

\subsection*{Problem 3}

We have seen that we can construct a free theory of two real scalar fields that has an $SO(2)$ internal symmetry.
We can instead consider a free theory of two complex scalar fields with Lagrangian:
\[\mathcal{L}=\p_\mu\psi^\dagger\p^\mu\psi+\p_\mu\phi^\dagger\p^\mu\phi-m^2\psi^\dagger\psi-m^2\phi^\dagger\phi.\]
Note that we can take some unitary two-by-two matrix $U$ and define
\begin{align*}
    \left(
    \begin{array}[]{c}
        \psi' \\
        \phi'
    \end{array}
    \right)
    =
    U
    \left(
    \begin{array}[]{c}
        \psi \\
        \phi
    \end{array}
    \right)
\end{align*}
then we have that
\begin{align*}
    \psi'^\dagger\psi'+\phi'^\dagger\phi'
    =
    \left(\begin{array}[]{cc}
        \psi' & \phi'
    \end{array}\right)
    \left( 
    \begin{array}[]{c}
        \psi'\\
        \phi'
    \end{array}
    \right)
    =
    \left(\begin{array}[]{cc}
        \psi & \phi
    \end{array}\right)
    U U^\dagger
    \left( 
    \begin{array}[]{c}
        \psi\\
        \phi
    \end{array}
    \right)
    =\psi^\dagger\psi+\phi^\dagger\phi.
\end{align*}
Hence, since the Lagrangian is dependent on the fields in this way (the derivatives can be ignored as $U$ is constant in spacetime),
it clearly has a $U(2)$ symmetry. To find the operators that give the Lie algebra action for this symmetry on the state space,
let us first determine what the Lie algebra of $U(2)$ looks like. We have already seen in class that the elements of the algebra
are two-by-two skew-Hermitian matrices, i.e. $X$ such that $X^\dagger=-X$. It is fairly clear that any such matrix can be written as
\begin{align*}
    X=
    \left(
    \begin{array}[]{cc}
        ai & be^{-i\gamma} \\
        -be^{i\gamma} & di
    \end{array}\right)
\end{align*}
for any real $a,b,d,\gamma$. We can split this up to find a basis for $\mathfrak{u}(2)$:
\begin{align*}
    X&=a
    \left( 
    \begin{array}[]{cc}
        i & 0\\
        0 & 0
    \end{array}
    \right)
    +d
    \left( 
    \begin{array}[]{cc}
        0 & 0 \\
        0 & i
    \end{array}
    \right)
    +b
    \left( 
    \begin{array}[]{cc}
        0 & e^{-i\gamma} \\
        -e^{i\gamma} & 0
    \end{array}
    \right)\\
    &=a
    \left( 
    \begin{array}[]{cc}
        i & 0\\
        0 & 0
    \end{array}
    \right)
    +d
    \left( 
    \begin{array}[]{cc}
        0 & 0 \\
        0 & i
    \end{array}
    \right)
    +b\cos\gamma
    \left( 
    \begin{array}[]{cc}
        0 & 1 \\
        -1 & 0
    \end{array}
    \right)
    -b\sin\gamma
    \left( 
    \begin{array}[]{cc}
        0 & i \\
        i & 0
    \end{array}
    \right).
\end{align*}
Going back to the group by exponentiating, we see that the first and second terms represents multiplying $\psi$ and $\phi$
respectively by a phase, leaving the Lagrangian invariant. Exponentiating the third term yields a ``rotation'' of the fields
into each other (it's the familiar generator for rotations), while the final term yields something that looks like a rotation
but with a $+i$ and a $-i$ in front of the sine terms; a compex rotation if you will.



\end{document}
