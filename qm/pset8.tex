\documentclass{../mathnotes}


\title{QM for Mathematicians: PSET 8}
\author{Nilay Kumar}
\date{Last updated: \today}


\begin{document}

\maketitle

\subsection*{Problem 1}

We wish to compute the propagator $\langle 0 | \hat\psi(x',t')\hat\psi(x,t)|0\rangle$. Let us first work with the
Fourier transform
\begin{align*}
    \tilde G(k',t',k,t)&=\langle 0 | \hat\psi(k',t')\hat\psi(k,t)|0\rangle\\
    &=\langle 0|a_{k'}^\dagger(t')a_k(t)|0\rangle\\
    &=\langle k',t'|k,t\rangle\\
    &=\langle k',t_0|e^{ik'^2t'/2m}e^{-ik^2t/2m}|k,t_0\rangle\\
    &=e^{-ik^2(t-t')/2m}\delta(k-k').
\end{align*}
We can now Fourier transform back,
\begin{align*}
    G(x',t',x,t)&=\int \frac{dk dk'}{2\pi}e^{i(kx-k'x')}e^{-ik^2(t-t')/2m}\delta(k-k')\\
    &=\int \frac{dk}{2\pi}e^{ik(x-x')}e^{-ik^2(t-t')/2m}\\
    &=\sqrt{\frac{m}{2\pi i(t-t')}}e^{2mi(x-x')^2/(t-t')}.
\end{align*}
Note that the integral of all space over this expression is 1, and when $x=x'$, we are left with the square root, which blows up as
$t\to t'$, as expected for a delta function of $x-x'$. Furthermore, if $x\neq x'$, and we let $t\to t'$, the exponential oscillates
extremely fast and thus, under an integral, takes any function multiplying it to zero. Consequently, we can write
\[\lim_{t\to t'}G(x',t',x,t)=\delta(x'-x).\]

\subsection*{Problem 2}

For non-relativistic free quantum field theory with a background potential, we can write
\begin{align*}
    \hat N_V&=\hat N=\int dx'\; \hat\psi^\dagger(x')\hat\psi(x')\\
    \hat H_V&=\int dx\; \hat\psi^\dagger(x)\left( -\frac{1}{2m}\frac{\p^2}{\p x^2}+V(x) \right)\psi(x)
\end{align*}
Computing the commutator, we have
\begin{align*}
    [\hat H_V, \hat N_V]=\int dx dx'\;&\hat\psi^\dagger(x)\left( -\frac{1}{2m}\frac{\p^2}{\p x^2}+V(x) \right)\hat\psi(x)\hat\psi^\dagger(x')\hat\psi(x')\\
    -&\hat\psi^\dagger(x')\hat\psi(x')\hat\psi^\dagger(x)\left( -\frac{1}{2m}\frac{\p^2}{\p x^2}+V(x) \right)\hat\psi(x).
\end{align*}
The terms involving the potential appear as
\begin{align*}
    \int dx dx'\;V(x)\left(\hat\psi^\dagger(x)\hat\psi(x)\hat\psi^\dagger(x')\hat\psi(x')-\hat\psi^\dagger(x')\hat\psi(x')\hat\psi^\dagger(x)\hat\psi(x)\right).
\end{align*}
This term vanishes because when $x\neq x'$, the field operators commute, allowing us to reorder them, and when $x=x'$, the first and second terms are exactly the same.
Consequently, we are left with the commutator in the case of no background potential:
\begin{align*}
    [H_V,N_V]=[H,N]=0.
\end{align*}
This is zero because we know that in the absence of a background potential, a non-relativistic free field theory conserves particle number
(this is seen directly by working in momentum space -- then the derivatives become order-insensitive factors of $k$ and we can again
use the field operator commutation relations as we did for the potential term).

Furthermore, the particle number operator $\hat N$ generates a $U(1)$ symmetry -- to see this, take the canonical basis for the Fock space.
Since each basis state has a well-defined particle number $n\in\N$ (simply because of how we defined the basis in class), the action of $\hat N$ on such a basis
state will return the same state, multiplied by $n$. Thus, if we act on a basis state with
\begin{align*}
    e^{i\hat N}=1+i\hat N-\frac{\hat N^2}{2!}-\cdots,
\end{align*}
we simply get back the state multiplied by $e^{in}$. Thus, all of the eigenvalues of $e^{i\hat N}$ are phases, and thus it is a unitary action.

\subsection*{Problem 3}

Let us define the classical angular momentum functions as
\begin{align*}
    L_3=\int d^3x \left( x\psi^*(\vec x)\frac{\p \psi}{\p y} - y\psi^*(\vec x)\frac{\p \psi}{\p x} \right)
\end{align*}
(and analogously for $L_1,L_2$). We can check, rather tediously, that these Poisson-commute with the Hamiltonian defined as
\begin{align*}
    H=\int d^3x \frac{1}{2m}|\nabla\psi|^2,
\end{align*}
as follows (I might have dropped some constants and neglected $x$ vs. $x'$ here and there, but we get zero regardless):
\begin{align*}
    \{L_3,H\}&=\frac{1}{2m}\int d^3xd^3x'\;x\left\{\psi^*(\vec x)\frac{\p \psi(\vec x)}{\p y},|\nabla\psi(\vec x')|^2\right\}
    -y\left\{\psi^*(\vec x)\frac{\p \psi(\vec x}{\p x},|\nabla\psi(\vec x')|^2\right\}\\
\end{align*}
If we focus on the first term, we can use the product properties of Possion brackets, pull derivatives out, and use the
Poisson-commutation relations to simplify to
\begin{align*}
    \{L_3,H\}=\frac{i}{2m}\int d^3xd^3x'\;&\left( \frac{\p \psi^*}{\p x}\frac{\p}{\p x}\delta^{(3)}(\vec x-\vec x')
    +\frac{\p \psi^*}{\p y}\frac{\p}{\p y}\delta^{(3)}(\vec x-\vec x') +\frac{\p \psi^*}{\p z}\frac{\p}{\p z}\delta^{(3)}(\vec x-\vec x') \right)\\
    &\left( x\frac{\p \psi}{\p y}-y\frac{\p \psi}{\p x} \right)
\end{align*}
Integrating the delta functions by parts and collapsing the $\vec x'$ integral, we find
\begin{align*}
    \{L_3,H\}&=\int d^3x \nabla^2\psi^*(\vec x)\left( x\frac{\p \psi}{\p y}-y\frac{\p \psi}{\p x} \right)+\frac{\p \psi^*}{\p x}\left( \frac{\p \psi}{\p y}+x\frac{\p^2\psi}{\p x\p y}-y\frac{\p^2\psi}{\p x^2} \right)+\cdots
\end{align*}
where the dots represent analogous terms for $y$ and $z$. Note that we can now integrate the Laplacian out front by parts, and term-by-term,
everything cancels. Thus, this angular momentum operator Poisson-commutes with the Hamiltonian. The same holds for $L_1,L_2$.

We define the corresponding quantized operators as
\begin{align*}
    \hat H=\int d^3x \frac{|\nabla\hat\psi|^2}{2m}=\frac{1}{2m}\int d^3x \frac{\p\hat\psi^\dagger}{\p x}\frac{\p\hat\psi}{\p x}+\frac{\p\hat\psi^\dagger}{\p y}\frac{\p\hat\psi}{\p y}
    +\frac{\p\hat\psi^\dagger}{\p z}\frac{\p\hat\psi}{\p z}
\end{align*}
and
\begin{align*}
    \hat L_3=-i\hbar\int d^3x \left( x\hat\psi^\dagger(\vec x)\frac{\p \hat\psi}{\p y} - y\hat\psi^\dagger(\vec x)\frac{\p \hat\psi}{\p x} \right).
\end{align*}
We can now compute the commutator,
\begin{align*}
    [\hat L_3,\hat H]=-\frac{i\hbar}{2m}\int d^3x d^3x'\;x[\hat\psi^\dagger(\vec x)\frac{\p \hat\psi}{\p y},|\nabla\hat\psi(\vec x')|^2]-
    y[\hat\psi^\dagger(\vec x)\frac{\p \hat\psi}{\p x},|\nabla\hat\psi(\vec x')|^2].
\end{align*}
Note that this is almost exactly what we had before, except with operators and commutators instead of Poisson brackets.
Since we can perform the same operations with the commutator (such as the ``product rule'' and pulling out derivatives), the steps follow almost precisely
as above, and we find that the commutator vanishes, i.e. that the Hamiltonian commutes with the angular momentum operator.

Finally, we compute
\begin{align*}
    [\hat L_1,\hat L_2]&=
    [-i\hbar\int d^3x \left( y\hat\psi^\dagger(\vec x)\frac{\p \hat\psi}{\p z} - z\hat\psi^\dagger(\vec x)\frac{\p \hat\psi}{\p y} \right),
    -i\hbar\int d^3x \left( z\hat\psi^\dagger(\vec x)\frac{\p \hat\psi}{\p x} - x\hat\psi^\dagger(\vec x)\frac{\p \hat\psi}{\p z} \right)]\\
    &=-ih\int d^3x d^3x'\;yz'[\hat\psi^\dagger(\vec x)\frac{\p\hat\psi}{\p z},\hat\psi^\dagger(\vec x')\frac{\p \hat\psi(\vec x')}{\p x'}]+
    xz'[\hat\psi^\dagger(\vec x)\frac{\p\hat\psi}{\p y},\hat\psi^\dagger(\vec x')\frac{\p \hat\psi(\vec x')}{\p z'}]\\
    -&zz'[\hat\psi^\dagger(\vec x)\frac{\p\hat\psi}{\p y},\hat\psi^\dagger(\vec x')\frac{\p \hat\psi(\vec x')}{\p x'}]-
    x'y[\hat\psi^\dagger(\vec x)\frac{\p\hat\psi}{\p z},\hat\psi^\dagger(\vec x')\frac{\p \hat\psi(\vec x')}{\p z'}]
\end{align*}
Each of the commutators can be expanded by using the product rule repeatedly -- some of these terms vanish, due to the commutation relations,
while others will yield delta functions in certain terms.
When the smoke clears, however, the terms containing delta functions cancel and  we are left with precisely the third angular momentum
relation
\begin{align*}
    [\hat L_1,\hat L_2]&=-i\hbar\int d^3x \left( x\hat\psi^\dagger(\vec x)\frac{\p \hat\psi}{\p y} - y\hat\psi^\dagger(\vec x)\frac{\p \hat\psi}{\p x} \right)
    =L_3.
\end{align*}

\subsection*{Problem 4}

We wish to show that $\mathfrak{so}(4,\C)\cong \mathfrak{sl}(2,\C)\times\mathfrak{sl}(2,\C)$. First recall that in class we showed
that
\begin{align*}
    \mathfrak{so}(1,3)\otimes\C\cong \mathfrak{sl}(2,\C)\times\mathfrak{sl}(2,\C)
\end{align*}
by taking certain complex linear combinations of rotations and boosts. It turned out that there were two types of linear combinations,
and each acted as complexified unitary 2-by-2 matrices, which are simply elements of $\mathfrak{so}(2,\C)$ (since complexifying removes
the unitary condition, but retains tracelessness).

Note, however, complexifying $\mathfrak{so}(1,3)$ effectively
erases all distinctions between metrics; i.e if the elements are allowed to be complex, the metric signature that the matrices keep invariant
depend on whether certain components are real or imaginary. The complexified Lie algebra, in some sense, ``contains'' all of the metric
signatures as long as the matrix elements are complex, and so we can write $\mathfrak{so}(4,\C)\cong\mathfrak{so}(1,3)\otimes\C$.
It follows, then, that $\mathfrak{so}(4,\C)\cong \mathfrak{sl}(2,\C)\times\mathfrak{sl}(2,\C)$.

Next recall that, in class, we showed at the group level that $Spin(4)\cong SU(2)\times SU(2)$, $Spin(1,3)\cong SL(2,\C)$, and
$Spin(2,2)\cong SL(2,\R)\times SL(2,\R)$. From this, we can deduce that $\mathfrak{spin}(4)=\mathfrak{su}(2)\times\mathfrak{su}(2)$,
$\mathfrak{spin}(1,3)\cong\mathfrak{sl}(2,\C)$, and $\mathfrak{spin}(2,2)=\mathfrak{sl}(2,\R)\times\mathfrak{sl}(2,\R)$. By what we showed
above, then, since $\mathfrak{su}(2)\times\mathfrak{su}(2)$ is an algebra that is contained in $\mathfrak{sl}(2,\C)\times \mathfrak{sl}(2,\C)$,
it is clear that $\mathfrak{spin}(4)$ is a subalgebra of $\mathfrak{so}(4,\C)$. Similarly, $\mathfrak{spin}(1,3)\cong\mathfrak{sl}(2,\C)$ is an algebra that is contained
(trivially) in $\mathfrak{sl}(2,\C)\times\mathfrak{sl}(2,\C)$, and is thus a subalgebra of $\mathfrak{so}(4,\C)$. Finally, $\mathfrak{sl}(2,\R)\times\mathfrak{sl}(2,\R)$
is an algebra contained in $\mathfrak{sl}(2,\C)\times\mathfrak{sl}(2,\C)$ and thus $\mathfrak{spin}(2,2)$ is a subalgebra of $\mathfrak{so}(4,\C)$.


\end{document}
