\documentclass{../../mathnotes}

\usepackage{tikz-cd}
\usepackage{todonotes}
\usepackage{youngtab}

\title{Representation Theory: Final PSET}
\author{Nilay Kumar}
\date{Last updated: \today}


\begin{document}

\maketitle

\begin{exc}
    Describe the invariants of the Weyl group action of type $G_2$ on $\C^2$.
\end{exc}
\begin{proof}[Solution]
    Recall that the Weyl group $W_{G_2}$ of the group $G_2$ is the dihedral group of order 12,
    whose action on $\C^2$ is given by the usual plane rotations and reflections. By Chevalley-Shephard-Todd,
    we find that the invariant ring $A=\C[x,y]^{W_{G_2}}$ must be a polynomial ring generated by
    two fundamental invariants (as $W_{G_2}$ is a reflection group of rank 2) $f(x,y)$ and $g(x,y)$ of degrees
    $d_1$ and $d_2$ respectively. Recall that the product $d_1d_2$ must equal the order of the group,
    and the sum $d_1+d_2-2$ must equal the number of reflections in $W_{G_2}$. Hence we find that the
    generators of the invariant ring $A$ must be of degree 2 and 6. But then it is clear that the degree
    two generator must be $f=x^2+y^2$, as the Weyl group action preserves the norm of a vector. Furthermore,
    the degree six generator must be $g=x^2(\sqrt{3}x/2-y/2)^2(\sqrt{3}x/2+y/2)^2$, as this polynomial
    is invariant under any permutation of the vertices of a hexagon (inscribed in the unit circle) and
    hence invariant under the $W_{G_2}$ action. Hence $A=\C[x,y]^{W_{G_2}}=\C[f(x,y),g(x,y)]$.
\end{proof}

\begin{exc}
    Compute the Poincar\'e series of the algebra of invariants for $S_3$ acting
    on $\C^6=\C^3\otimes\C^2$ by permuting the coordinates in the first tensor factor.
\end{exc}
\begin{proof}[Solution]
    Recall Molien's formula for the Poincar\'e series of the algebra of invariants of $\C^6=\text{Specm }A$:
    \[p_{A^{S_3}}(t)=\frac{1}{|S_3|}\sum_{\sigma\in S_3}\frac{1}{\det\left( \id-t\sigma \right)}.\]
    Note first that the summand is a class function, as $\det(\id-t\tau\sigma\tau^{-1})=\det(\tau(\id-t\sigma)\tau^{-1})=\det(\id-t\sigma)$,
    and hence we can rewrite Molien's formula as a sum over conjugacy classes:
    \[p_{A^{S_3}}(t)=\frac{1}{|S_3|}\sum_{[\sigma]\in S_3}\frac{|[\sigma]|}{\det\left(\id-t[\sigma] \right)}.\]
    Fixing a basis for $\C^3\otimes\C^2$ as $\{e_1\otimes f_1,e_2\otimes f_1,e_3\otimes f_1,e_1\otimes f_2,e_2\otimes f_2,e_3\otimes f_3\}$,
    where $e_i,f_i$ are the standard bases for $\C^3,\C^2$ respectively, we can write explicitly representatives of each
    conjugacy class of $S_3$:
    \begin{align*}
        \det(\id-te) &=
        \begin{vmatrix}
            1-t\\ & 1-t\\ &&1-t\\&&&1-t\\&&&&1-t\\&&&&&1-t
        \end{vmatrix}
        =(1-t)^6,\\
        \det(\id-t(12)) &=
        \begin{vmatrix}
            1&-t\\ -t& 1\\ &&1-t\\&&&1&-t\\&&&-t&1\\&&&&&1-t
        \end{vmatrix}
        =(1-t)^2(1-t^2)^2,\\
        \det(\id-t(123)) &=
        \begin{vmatrix}
            1&&-t\\ -t& 1\\ &-t&1\\&&&1&&-t\\&&&-t&1\\&&&&-t&1
        \end{vmatrix}
        =(1-t^3)^2.
    \end{align*}
    This yields
    \begin{align*}
        p_{A^{S_3}}(t)=\frac{1}{6}\left( \frac{1}{(1-t)^6}+\frac{3}{(1-t)^2(1-t^2)^2}+\frac{2}{(1-t^3)^2} \right).
    \end{align*}
\end{proof}

\newpage

\begin{exc}
    Recall that the character of an irreducible $GL(n)$-module with highest weight $\lambda$
    is called a Schur function:
    \[s_\lambda(x_1,\ldots,x_n)=\frac{\det\left( x_i^{\lambda_j+n-j} \right)}{\prod_{i<j}(x_i-x_j)}.\]
    Expand $p_ks_\lambda$ in Schur functions, where $p_k=\sum x_i^k$.
\end{exc}
\begin{proof}[Solution]
    Denoting the denominator of $s_\lambda$ by $\Delta$, we write
    \begin{align*}
        \Delta p_ks_\lambda &= \sum_{l=1}^nx_l^k\det\left( x_i^{\lambda_j+n-j} \right)\\
        &=\sum_{l=1}^nx_l^k\sum_{\sigma\in S^n}\text{sgn}(\sigma)\prod_{i=1}^nx_i^{\lambda_{\sigma(i)}+n-\sigma(i)}\\
        &=\sum_{l=1}^nx_{\sigma^{-1}(l)}^k\sum_{\sigma\in S^n}\text{sgn}(\sigma)\prod_{i=1}^nx_i^{\lambda_{\sigma(i)}+n-\sigma(i)}\\
        &=\sum_{l=1}^n\sum_{\sigma\in S^n}\text{sgn}(\sigma)\prod_{i=1}^nx_i^{\lambda_{\sigma(i)}+k\delta_{i,\sigma^{-1}(l)}+n-\sigma(i)}\\
        &=\sum_{l=1}^n\sum_{\sigma\in S^n}\text{sgn}(\sigma)\prod_{i=1}^nx_i^{\lambda_{\sigma(i)}+k\delta_{\sigma(i),l}+n-\sigma(i)}\\
        &=\sum_{l=1}^n\det\left( x_i^{\lambda_j+k\delta_{j,l}+n-j} \right).
    \end{align*}
    In words, we find that multiplying $s_\lambda$ by $p_k$ yields a sum over $l$ of determinants similar
    to the original determinant but now with an extra $k$ added in the powers of the entries in the $l$th column.
    Unfortunately, these resulting determinants are not necessarily Schur functions, as the associated
    numbers $\lambda_j+k\delta_{j,l}$ may not form a partition (due to ordering issues). We fix this by
    rearranging the $\lambda_j+k\delta_{j,l}+n-j$ in decreasing order. Of course, if two terms in the 
    sequence are equal, the determinant vanishes, so we may assume that for some $p\leq l$ we have
    \[\mu_{p-1}+n-p+1>\mu_l+n-l+r>\mu_p+n-p\]
    and hence $\det\left( x_i^{\lambda_j+k\delta_{j,l}+n-j} \right)=(-1)^{l-p}\det\left( x_i^{\rho_j+n-j} \right)$, where
    $\rho$ is the partition
    \[\rho=(\lambda_1,\cdots,\lambda_{p-1},\lambda_l+p-l+k,\lambda_p+1,\cdots,\lambda_{l-1}+1,\lambda_l,\cdots,\lambda_n).\]
    It is now easy to check that $\theta=\rho-\lambda$ is a skew hook of length $k$. Now recall that
    the height $\text{ht}(\theta)$ is one less than the number of rows $\theta$ spans, and hence we can write
    finally that
    \[p_ks_\lambda=\sum_\rho(-1)^{\text{ht}(\rho-\lambda)}s_\rho,\]
    where the sum is over all partitions $\rho\supset\lambda$ such that $\rho-\lambda$ is a skew hook of length $l$.
\end{proof}

\begin{exc}
    Show that
    \[s_{(\lambda_1,\ldots,\lambda_n,0)}(x_1,\ldots,x_n,0)=s_\lambda(x_1,\ldots,x_n).\]
    This rule defines the Schur function in infinitely many variables, all but finitely many
    of which are zero.
\end{exc}
\begin{proof}[Solution]
    Note that
    \begin{align*}
        s_{(\lambda_1,\ldots,\lambda_n,0)}(x_1,\ldots,x_n,x_{n+1}) &=
        \frac{
        \begin{vmatrix}
            x_1^{\lambda_1+n} & \cdots & x_1^{\lambda_n+1} & 1\\
            \vdots & &\vdots & \vdots\\
            x_n^{\lambda_1+n} & \cdots & x_n^{\lambda_n+1} & 1\\
            x_{n+1}^{\lambda_1+n} & \cdots & x_{n+1}^{\lambda_n+1} & 1\\
        \end{vmatrix}
        }{\prod_{i<j}^{n+1}(x_i-x_j)}
    \end{align*}
    and hence taking $x_{n+1}=0$, we obtain  
    \begin{align*}
        s_{(\lambda_1,\ldots,\lambda_n,0)}(x_1,\ldots,x_n,0) &= 
        \frac{
        \begin{vmatrix}
            x_1^{\lambda_1+n} & \cdots & x_1^{\lambda_n+1}\\
            \vdots & &\vdots\\
            x_n^{\lambda_1+n} & \cdots & x_n^{\lambda_n+1}\\
        \end{vmatrix}
    }{\prod_{i}^nx_i \prod_{i<j}^{n}(x_i-x_j)}\\
    &=s_\lambda(x_1,\ldots,x_n),
    \end{align*}
    where in the last step we have factored out an $x_i$ from the $i$th row of the determinant.
\end{proof}

\begin{exc}
    Let $\C^\infty$ have a basis $b_k$ for $k\in\Z$. Define the half-infinite wedge product $\Lambda^{\infty/2}\C^\infty$
    of $\C^\infty$ to be the span of the vectors
    \[v_S=b_{s_1}\wedge b_{s_2}\wedge \cdots\]
    where the set $S=\{s_1>s_2>\cdots\}$ contains finitely many positive integers and all but finitely
    many negative integers.

    Let the matrix units $E_{ij}\in\fr{gl}_\infty$ act on such monomials by the rules of linear algebra if
    $i\neq j$ and by
    \[\left( \sum a_i\pi(E_{ii}) \right)v_S=\left( \sum_{s\in S}a_s-\sum_{s\in\Z_{\leq 0}}a_s \right)v_S\]
    if $i=j$. Note that the right hand side yields a finite sum. Show that
    \[ [\pi(E_{ij}),\pi(E_{kl})]=\pi([E_{ij},E_{kl}])-(E_{ij},[J,E_{kl}])\]
    where $J=\sum_{k\leq 0}E_{kk}$ and $(E_{ij},E_{kl})=\delta_{jk}\delta_{il}$ is the invariant bilinear form.
\end{exc}
\begin{proof}[Solution]
    Let us first compute first the left-hand side. By the definition above, we see that
    \[\pi(E_{ij})=E_{ij}-\delta_{ij}\theta(i\leq 0)\id,\]
    and hence
    \begin{align*}
        [\pi(E_{ij}),\pi(E_{kl})] &= [E_{ij},E_{kl}]
    \end{align*}
    as the identity commutes with everything. For the first term on right-hand side, we notice that
    \[ [E_{ij},E_{kl}]=\delta_{kj}E_{il}-\delta_{il}E_{kj} \]
    and so
    \begin{align*}
        \pi([E_{ij},E_{kl}])&=\delta_{kj}E_{il}-\delta_{il}\delta_{kj}\theta(i\leq 0)\id-\delta_{il}E_{kj}+\delta_{il}\delta_{kj}\theta(k\leq 0)\id\\
        &=[E_{ij},E_{kl}]-\delta_{il}\delta_{kj}(\theta(i\leq 0)-\theta(k\leq 0))\id.
    \end{align*}
    Finally, if we compute the second term on the right-hand side, we have
    \begin{align*}
        (E_{ij},[J,E_{kl}]) &= \left(E_{ij},\sum_{m\leq 0}[E_{mm},E_{kl}]\right)\\
        &=\left(E_{ij},\sum_{m\leq 0}\delta_{km}E_{ml}-\delta_{ml}E_{km}\right)\\
        &=\left(E_{ij},(\theta(k\leq 0)-\theta(l\leq 0))E_{kl}\right)\\
        &=\delta_{jk}\delta_{il}\left( \theta(k\leq 0)-\theta(i\leq 0) \right).
    \end{align*}
    Comparing this to the second term in the expression for $\pi([E_{ij},E_{kl}])$ above, we obtain the
    required formula.
\end{proof}

\begin{exc}
    Let $v_\lambda$ be $v_S$ for $S=\{\lambda_i-i+1\}$, where $\lambda_1\geq\lambda_2\geq\cdots\geq 0$ as
    in Exercise 4. Show that the map that takes $v_\lambda$ to the Schur function $s_\lambda$ in infinitely
    many variables takes the operator
    \[\alpha_{-k}=\sum_{i\in\Z}E_{i,i-k}\]
    to multiplication by the power-sum function $p_k$ as in Exercise 3.
\end{exc}
\begin{proof}
    This is easy to see. Any given term $E_{j,j-k}$ in $\alpha_{-k}$ takes $b_{j-k}$ and turns
    it into a $b_{j}$, i.e. raises it by $k$. Under the action of $\alpha_{-k}$, we are thus left
    with a sum of finitely many terms (due to the ``dense'' negative basis vectors the action is zero
    on $b_{-n-k}$ and lower), each of which now may
    be ordered incorrectly due to the transformation of $b_{\lambda_j-j+1}$ to $b_{\lambda_j+n-j+1}$. Just
    as in exercise 3, however, the number of wedge flips needed to obtain the correct ordering is given
    by the height $\text{ht}(\rho-\lambda)$ as defined previously, and hence we obtain precisely the multiplication
    by $p_k$.
\end{proof}

\end{document}
