\documentclass{../mathnotes}

\usepackage{tikz-cd}
\usepackage{amsmath}
\usepackage{todonotes}


\title{Riemann Surfaces: Lecture Notes}
\author{Nilay Kumar}
\date{Last updated: \today}


\begin{document}

\maketitle

%\setcounter{section}{-1}

\section*{Class 10}

Recall that we are studying function theory on the torus $\C/\Lambda$, where $\Lambda=\left\{ m\omega_1+n\omega_2; m,n\in\Z \right\}$.
We had produced a candidate
\begin{align*}
    \sigma(z)=z\prod_{\omega\in\Lambda^\times}\left( 1+\frac{z}{w} \right)e^{-\frac{z}{\omega}+\frac{1}{2}\frac{z^2}{\omega^2}}.
\end{align*}
Let us check that this product converges by examining its logarithm:
\begin{align*}
    \log\left\{ \cdots \right\}&=\log(1+\frac{z}{\omega})-\frac{z}{\omega}+\frac{z^2}{2\omega^2}\\
    &=\left( \frac{z}{\omega}-\frac{1}{2}\frac{z^2}{\omega^2}+\ldots \right)-\frac{z}{\omega}+\frac{1}{2}\frac{z^2}{\omega^2},
\end{align*}
which clearly converges. Hence $\sigma(z)$ is holomorphic for $z\in\C$. Recall that $\sigma'(z)/\sigma(z)=\zeta(z)$ and so $\partial_z\log\sigma(z+\omega_a)=\zeta(z+\omega_a)$.
Thus we have (from before) that
\begin{align*}
    \eta_a=\zeta(z+\omega_a)-\zeta(z)=\partial_z\log\sigma(z+\omega_a)-\partial_z\log\sigma(z),
\end{align*}
which gives us periodicity information. Integrating and exponentiating, we see that
\begin{align*}
    \sigma(z+\omega_a)=\sigma(z)e^{\eta_az+c_a}
\end{align*}
where $c_a$ is the constant of integration, and taking $z=-\omega_a/2$, we find that
\begin{align*}
    \sigma(\omega_a/2)=\sigma(-\omega_a/2)e^{-\eta_a\frac{\omega_a}{2}+c_a}.
\end{align*}
It is easy to check, however, that $\sigma$ is odd, and hence we find that
\begin{align*}
    \sigma(z+\omega_a)=-\sigma(z)e^{\eta_a(z+\frac{\omega_a}{2})}.
\end{align*}

So we have found that $\sigma(z)$ is holomorphic on $\C$ and that $\sigma(z)=0$ if and only if $z=0\mod\Lambda$.
Now that we have constructed such a $\sigma$, let us give another proof of Abel's theorem. First recall our previous statement of
Abel's theorem.

\begin{thm}[Abel's theorem]
    Let $P_1,\ldots,P_M,Q_1,\ldots,Q_N$ be points in $\C$. Then there exists a meromorphic $f$ with zeroes at $P_i$ and poles at $Q_i$
    if and only if $M=N$ and $\sum_{i=1}^MA(P_i)=\sum_{i=1}^NA(Q_i)$.
\end{thm}
Recall that the Abel map takes $\C/\Lambda\ni p\mapsto A(p)=\int_{p_0}^p\omega$ where the value of the integral is taken modulo the lattice
generated by $\oint_A\omega,\oint_B\omega$. Take $p_0=0$ and $\omega=dz$, which is a well-defined form, and if we take $A$ to align with $\omega_2$
and $B$ to align with $\omega_1$, we see that $\oint_A\omega=\oint_A dz=\omega_1$ and similarly $\oint_B\omega=\omega_2$. Hence the map simply
takes $p$ to $\int_0^p dz \mod\Lambda=p$ where $p$ is viewed as a complex number.

Let us now restate Abel's theorem.
\begin{thm}[Abel's theorem, v.2]
    Let $P_1,\ldots,P_M,Q_1,\ldots,Q_N$ be points in $\C$. Then there exists a meromorphic $f$ with zeroes at $P_i$ and poles at $Q_i$
    if and only if $M=N$ and $\sum_{i=1}^M P_i=\sum_{i=1}^NQ_i\mod\Lambda$.
\end{thm}
\begin{proof}
    Consider the function
    \begin{align*}
        f(z)=\frac{\prod_{i=1}^M\sigma(z-P_i)}{\prod_{i=1}^N\sigma(z-Q_i)}.
    \end{align*}
    We should be a little careful to note that $\sigma$ is a function not on the torus $\C/\Lambda$, but a function on $\C$ (it transforms under a lattice
    translation!). Hence we must be cognizant of the fact that $P_i,Q_i$ here are some chosen representatives in $\C$ of the equivalence classes of the points $P_i,Q_i$.
    It should be clear that $f(z)$ is meromorphic with zeroes at every representative of each $P_i$s and poles at every representative of each $Q_i$.
    The natural question, now, is whether this function extends to a function on the torus. 
    To check this, let us see whether it is doubly periodic using what we know about $\sigma$:
    \begin{align*}
        f(z+\omega_a)&=f(z)\frac{\prod_{i=1}^Me^{\eta_a(z-P_i)}}{\prod_{i=1}^Ne^{\eta_a(z-Q_i)}}\\
        &=f(z)e^{-\eta_a\left( \sum_{i=1}^MP_i-\sum_{i=1}^NQ_i \right)}.
    \end{align*}
    Hence we wish to choose $P_i,Q_i$ representatives such that the exponential becomes unity. By hypothesis, this can be done (by shifting one, if necessary).
\end{proof}


Let us now return to Weierstrass theory. Given $\omega=dz$, we defined $\omega_0=\mathcal{P}(z)dz$ which has a double pole at 0 and
$\partial_z\log\sigma(z)=\zeta(z)$ and $\zeta'(z)=-\mathcal{P}(z)$. Now we can construct a form $\omega_{PQ}$ with residues $1,-1$ at $P,Q$ respectively,
by assigning $\omega_{PQ}(z)=\left( \zeta(z-P)-\zeta(z-Q) \right)\omega=\partial_z\log\frac{\sigma(z-P)}{\sigma(z-Q)}dz$. What Weierstrass theory
tells us that we can write everything in terms of $\sigma$, our analog of $z$.

\subsection*{Jacobi theory: $\theta$-functions}

Consider again the torus $\C/\Lambda$, where we now normalize the lattice as $\Lambda=\left\{ m+n\tau; m,n\in\Z\right\}$ with $\text{Im }\tau>0$ (by linear independence,
it cannot be real). This simply corresponds to picking $\omega_1=1,\omega_2/\omega_1=\tau$. Next define the \textbf{theta-function}
\begin{align*}
    \theta(z|\tau)=\sum_{n\in \Z}e^{\pi in^2\tau+2\pi inz},
\end{align*}
in which the structure of the lattice is explicitly clear (unlike in the Weierstrass theory). Let us examine its main properties.

First, note that $\theta(z|\tau)$ is holomorphic in $z\in\C$ because the series converges for all $z$; this is due to the term
\begin{align*}
    |e^{\pi in^2(\tau_1+i\tau_2)}|=|e^{\pi in^2\tau_1}e^{-\pi^2n^2\tau_2}|=e^{-\pi n^2\tau_2}
\end{align*}
for $\tau=\tau_1+i\tau_2$, whose decay dominates due to the $n^2$. Next, notice that
\begin{align*}
    \theta(z+1|\tau)&=\theta(z|\tau)\\
    \theta(z+\tau|\tau)&=e^{-\pi i\tau-2\pi iz}\theta(z|\tau),
\end{align*}
where the second is obtained by completing the square. Though $\theta$ is not invariant, its zeroes are.

Furthermore, we claim that $\theta(z|\tau)$ vanishes at exactly one point modulo lattice translates. It suffices to compute
the integral $\oint_C\frac{\theta'(z|\tau)}{\theta(z|\tau)}dz$, as it yields $2\pi i$ times the difference in the number of zeroes and poles in
a given region. We shall integrate over the curve $C$ where $C$ traverses the circumference of one lattice segment (i.e. the whole torus):
\begin{align*}
    \oint_C \frac{\theta'(z|\tau)}{\theta(z|\tau)}dz=\oint_B\left(-\frac{\theta'(z|\tau)}{\theta(z|\tau)}+\frac{\theta'(z+1|\tau)}{\theta(z+1|\tau}\right)
    +\oint_A\left(\frac{\theta'(z|\tau)}{\theta(z|\tau)}-\frac{\theta'(z+\tau|\tau)}{\theta(z+\tau|\tau)}\right).
\end{align*}
But these are just the shifts in the logarithmic derivative, and since $\partial_z\log\theta(z+\tau|\tau)=-2\pi i+\partial_z\log\theta(z|\tau)$ using
the transformation rules above, we see that our integral simplifies to
\begin{align*}
    \oint_C \frac{\theta'(z|\tau)}{\theta(z|\tau)}dz=2\pi i\oint_A dz=2\pi i.
\end{align*}
Of course, since $\theta$ is holomorphic, it has no poles, and hence we see that we have one zero. The zero, in fact, occurs in the center:
$\theta\left( (1+\tau)/2|\tau \right)=0$. To see this, consider the following function:
\begin{align*}
    \theta\left( z+\frac{1+\tau}{2}|\tau \right)&=\sum_{n\in\Z}\exp\left( \pi i n^2\tau+2\pi i n\left(z+\frac{1+\tau}{2}\right) \right)\\
    &=i\exp\left( -\pi i\frac{\tau}{4}-\pi iz \right)\sum_{n\in\Z}\exp\left( \pi i(n+\frac{1}{2})^2\tau+2\pi i(n+\frac{1}{2})(z+\frac{1}{2}) \right)\\
    &=i\exp\left( -\pi i\frac{\tau}{4}-\pi iz \right)\theta_1(z|\tau)
\end{align*}
where we have completed the square and defined the function $\theta_1$. We claim that $\theta_1$ is an odd function, which would imply that $\theta_1$ vanishes
at zero, which would prove the claim about the location of the zero. Hence let us verify that $\theta_1$ is odd; switching $z\mapsto-z$ yields in the exponent
\begin{align*}
    \log\theta_1(z|\tau)=\pi i\left( n+\frac{1}{2} \right)^2\tau+2\pi i\left( n+\frac{1}{2} \right)\left( -z+\frac{1}{2} \right).
\end{align*}
If we switch the indices $n\mapsto m$ such that $n+\frac{1}{2}=-\left( m+\frac{1}{2} \right)$, we find that the exponent is now
\begin{align*}
    \log\theta_1(z|\tau)=\pi i\left( m+\frac{1}{2} \right)^2\tau+2\pi i\left( m+\frac{1}{2} \right)\left(\left( z+\frac{1}{2} \right)-2\pi i\left( m+\frac{1}{2} \right)\right),
\end{align*}
and hence $\theta_1$ is odd. Now we see that the function we want is in fact $\theta_1(z|\tau)$ as it is odd, holomorphic, and has one zero.

We leave it as an exercise to show that
\begin{align*}
    \sigma(z)=\omega_1\exp\left( \eta_1\frac{z^2}{\omega_1}\right)\frac{\theta_1\left( \frac{z}{\omega_1}|\tau \right)}{\theta_1'(0|\tau)}
\end{align*}

\section*{Class 11}

We claim that the theta-function theory is more powerful than what we have been using so far - to see this, let us prove Abel's theorem.
Recall that the theorem states that there exists a meromorphic $f$ with zeroes at $P_i$ and poles and $Q_j$ if and only if $N=M$ and $\sum_i A(P_i)=\sum_j A(Q_j)$.
The idea is to express
\begin{align*}
    f(z)=\frac{\prod_{i=1}^N\theta_1(x-P_i)}{\prod_{i=1}^N\theta_1(z-Q_i)}
\end{align*}
and check double-periodicity. It is an exercise to check that
\begin{align*}
    \theta_1(z+1|\tau)&=-\theta_1(z|\tau)\\
    \theta_1(z+\tau|\tau)&=\exp\left( -\pi i\tau-2\pi i(z+1/2) \right)\theta_1(z|\tau),
\end{align*}
from which periodicity follows easily. Of course, we must be careful to note that the $P_i,Q_i$ used here are in fact chosen representatives.

Next let us define a meromorphic form
\begin{align*}
    \omega_{PQ}=\partial_z\log\frac{\theta_1(z-P)}{\theta_1(z-Q)}dz,
\end{align*}
which, it is easy to check, has poles at $P,Q$ with opposite residues. Additionally, one can check that this expression is well-defined on the lattice, 
i.e. invariant under a shift. We leave it as a simple exercise to show that
\[\omega_P(z)=\partial^2_z\log\frac{\theta_1(z-P|\tau)}{\theta_1'(0|\tau)}dz\]
is a meromorphic form with a double pole at $P$ and is well-defined on the lattice.

But in fact, we can go even farther with this theta-function. Indeed, one attractive feature is that there exists a product expansion for $\theta(z|\tau)$.
\begin{thm}
    We can expand
    \begin{align*}
        \theta(z|\tau)=\prod_{n=1}^\infty (1-q^{2n)}(1+q^{2n-1}e^{2\pi iz})(1+q^{2n-1}e^{-2\pi iz})
    \end{align*}
    where $q\equiv e^{\pi i\tau}$.
\end{thm}
\begin{proof}
    Define
    \begin{align*}
        T(z|\tau)=\prod_{n=1}^\infty (1-q^{2n)}(1+q^{2n-1}e^{2\pi iz})(1+q^{2n-1}e^{-2\pi iz}).
    \end{align*}
    We claim that $T(z|\tau)$ is equal to zero exactly when $z$ is $(1+\tau)/2\mod\Lambda$ and that the zeros are simple. This can be
    checked by some simple algebra. It's also easy to show that $T(z|\tau)$ is holomorphic in $\C$ and that $\tau(z+1|\tau)=T(z|\tau)$.
    Moreover
    \begin{align*}
        T(z+\tau|\tau)&=\prod_{n=1}^\infty (1-q^{2n})\prod\left( 1+q^{2n+1}e^{2\pi iz} \right)\left( 1+q^{2n-3}e^{-2\pi iz}q^{-2} \right)\\
        &=\prod_{n=1}^\infty (1-q^{2n})\frac{\prod_{n=1}^\infty\left( 1+q^{2n-1}e^{2\pi i z} \right)}{1+qe^{2\pi iz}}\prod_{n=1}^\infty\left( 1+q^{2n-1}e^{-2\pi iz} \right)
        &=\frac{1-q^{-1}e^{-2\pi iz}}{1-qe^{2\pi iz}}T(z|\tau)=q^{-1}e^{-2\pi iz}.
    \end{align*}
    Recall that $\theta$ followd a similar condition. This shows that $\theta(z|\tau)/T(z|\tau)=c$,
    where $c$ is a constant independent of $z$ that can depend on $\tau$. Next we claim that $c(\tau)=1$. For this we show that there exists a $c$ such that $c(\tau)=c(4\tau)=c(4^k\tau)$
    and $c(\tau)=\lim_{k\to\infty}c(4^k\tau)=1$, which shows the proof. Hence let us prove that $c(\tau)=c(4\tau)$ using $\theta(z|\tau)=C(\tau)T(z|\tau)$.

    Take $z=1/2$. Then $e^{2\pi iz=e^{\pi i}}\geq 1-$ and $\theta(1/2|\tau)=\sum_{n\in\Z}e^{\pi i n^2\tau}(-1)^n$ but $T(1/2|\tau)=\prod^{\infty}_{n=1}(1-q^{2n)(1-q^{2n-1}}(1-q^{2n-1})$.
    and hence $c(\tau)=\sun e^{\pi in\tau}(-1)^n/\prod(1-q^n)(1-q^{2n-1})$. Next take $z=1/4$
\end{proof}

\end{document}
