\documentclass{../mathnotes}

\usepackage{tikz-cd}
\usepackage{amsmath}
\usepackage{todonotes}


\title{Riemann Surfaces: Lecture Notes}
\author{Nilay Kumar}
\date{Last updated: \today}


\begin{document}

\maketitle

%\setcounter{section}{-1}

\section*{Class 10}

Recall that we are studying function theory on the torus $\C/\Lambda$, where $\Lambda=\left\{ m\omega_1+n\omega_2; m,n\in\Z \right\}$.
We had produced a candidate
\begin{align*}
    \sigma(z)=z\prod_{\omega\in\Lambda^\times}\left( 1+\frac{z}{w} \right)e^{-\frac{z}{\omega}+\frac{1}{2}\frac{z^2}{\omega^2}}.
\end{align*}
Let us check that this product converges by examining its logarithm:
\begin{align*}
    \log\left\{ \cdots \right\}&=\log(1+\frac{z}{\omega})-\frac{z}{\omega}+\frac{z^2}{2\omega^2}\\
    &=\left( \frac{z}{\omega}-\frac{1}{2}\frac{z^2}{\omega^2}+\ldots \right)-\frac{z}{\omega}+\frac{1}{2}\frac{z^2}{\omega^2},
\end{align*}
which clearly converges. Hence $\sigma(z)$ is holomorphic for $z\in\C$. Recall that $\sigma'(z)/\sigma(z)=\zeta(z)$ and so $\partial_z\log\sigma(z+\omega_a)=\zeta(z+\omega_a)$.
Thus we have (from before) that
\begin{align*}
    \eta_a=\zeta(z+\omega_a)-\zeta(z)=\partial_z\log\sigma(z+\omega_a)-\partial_z\log\sigma(z),
\end{align*}
which gives us periodicity information. Integrating and exponentiating, we see that
\begin{align*}
    \sigma(z+\omega_a)=\sigma(z)e^{\eta_az+c_a}
\end{align*}
where $c_a$ is the constant of integration, and taking $z=-\omega_a/2$, we find that
\begin{align*}
    \sigma(\omega_a/2)=\sigma(-\omega_a/2)e^{-\eta_a\frac{\omega_a}{2}+c_a}.
\end{align*}
It is easy to check, however, that $\sigma$ is odd, and hence we find that
\begin{align*}
    \sigma(z+\omega_a)=-\sigma(z)e^{\eta_a(z+\frac{\omega_a}{2})}.
\end{align*}

So we have found that $\sigma(z)$ is holomorphic on $\C$ and that $\sigma(z)=0$ if and only if $z=0\mod\Lambda$.
Now that we have constructed such a $\sigma$, let us give another proof of Abel's theorem. First recall our previous statement of
Abel's theorem.

\begin{thm}[Abel's theorem]
    Let $P_1,\ldots,P_M,Q_1,\ldots,Q_N$ be points in $\C$. Then there exists a meromorphic $f$ with zeroes at $P_i$ and poles at $Q_i$
    if and only if $M=N$ and $\sum_{i=1}^MA(P_i)=\sum_{i=1}^NA(Q_i)$.
\end{thm}
Recall that the Abel map takes $\C/\Lambda\ni p\mapsto A(p)=\int_{p_0}^p\omega$ where the value of the integral is taken modulo the lattice
generated by $\oint_A\omega,\oint_B\omega$. Take $p_0=0$ and $\omega=dz$, which is a well-defined form, and if we take $A$ to align with $\omega_2$
and $B$ to align with $\omega_1$, we see that $\oint_A\omega=\oint_A dz=\omega_1$ and similarly $\oint_B\omega=\omega_2$. Hence the map simply
takes $p$ to $\int_0^p dz \mod\Lambda=p$ where $p$ is viewed as a complex number.

Let us now restate Abel's theorem.
\begin{thm}[Abel's theorem, v.2]
    Let $P_1,\ldots,P_M,Q_1,\ldots,Q_N$ be points in $\C$. Then there exists a meromorphic $f$ with zeroes at $P_i$ and poles at $Q_i$
    if and only if $M=N$ and $\sum_{i=1}^M P_i=\sum_{i=1}^NQ_i\mod\Lambda$.
\end{thm}
\begin{proof}
    Consider the function
    \begin{align*}
        f(z)=\frac{\prod_{i=1}^M\sigma(z-P_i)}{\prod_{i=1}^N\sigma(z-Q_i)}.
    \end{align*}
    We should be a little careful to note that $\sigma$ is a function not on the torus $\C/\Lambda$, but a function on $\C$ (it transforms under a lattice
    translation!). Hence we must be cognizant of the fact that $P_i,Q_i$ here are some chosen representatives in $\C$ of the equivalence classes of the points $P_i,Q_i$.
    It should be clear that $f(z)$ is meromorphic with zeroes at every representative of each $P_i$s and poles at every representative of each $Q_i$.
    The natural question, now, is whether this function extends to a function on the torus. 
    To check this, let us see whether it is doubly periodic using what we know about $\sigma$:
    \begin{align*}
        f(z+\omega_a)&=f(z)\frac{\prod_{i=1}^Me^{\eta_a(z-P_i)}}{\prod_{i=1}^Ne^{\eta_a(z-Q_i)}}\\
        &=f(z)e^{-\eta_a\left( \sum_{i=1}^MP_i-\sum_{i=1}^NQ_i \right)}.
    \end{align*}
    Hence we wish to choose $P_i,Q_i$ representatives such that the exponential becomes unity. By hypothesis, this can be done (by shifting one, if necessary).
\end{proof}


Let us now return to Weierstrass theory. Given $\omega=dz$, we defined $\omega_0=\mathcal{P}(z)dz$ which has a double pole at 0 and
$\partial_z\log\sigma(z)=\zeta(z)$ and $\zeta'(z)=-\mathcal{P}(z)$. Now we can construct a form $\omega_{PQ}$ with residues $1,-1$ at $P,Q$ respectively,
by assigning $\omega_{PQ}(z)=\left( \zeta(z-P)-\zeta(z-Q) \right)\omega=\partial_z\log\frac{\sigma(z-P)}{\sigma(z-Q)}dz$. What Weierstrass theory
tells us that we can write everything in terms of $\sigma$, our analog of $z$.

\subsection*{Jacobi theory: $\theta$-functions}

Consider again the torus $\C/\Lambda$, where we now normalize the lattice as $\Lambda=\left\{ m+n\tau; m,n\in\Z\right\}$ with $\text{Im }\tau>0$ (by linear independence,
it cannot be real). This simply corresponds to picking $\omega_1=1,\omega_2/\omega_1=\tau$. Next define the \textbf{theta-function}
\begin{align*}
    \theta(z|\tau)=\sum_{n\in \Z}e^{\pi in^2\tau+2\pi inz},
\end{align*}
in which the structure of the lattice is explicitly clear (unlike in the Weierstrass theory). Let us examine its main properties.

First, note that $\theta(z|\tau)$ is holomorphic in $z\in\C$ because the series converges for all $z$; this is due to the term
\begin{align*}
    |e^{\pi in^2(\tau_1+i\tau_2)}|=|e^{\pi in^2\tau_1}e^{-\pi^2n^2\tau_2}|=e^{-\pi n^2\tau_2}
\end{align*}
for $\tau=\tau_1+i\tau_2$, whose decay dominates due to the $n^2$. Next, notice that
\begin{align*}
    \theta(z+1|\tau)&=\theta(z|\tau)\\
    \theta(z+\tau|\tau)&=e^{-\pi i\tau-2\pi iz}\theta(z|\tau),
\end{align*}
where the second is obtained by completing the square. Though $\theta$ is not invariant, its zeroes are.

Furthermore, we claim that $\theta(z|\tau)$ vanishes at exactly one point modulo lattice translates. It suffices to compute
the integral $\oint_C\frac{\theta'(z|\tau)}{\theta(z|\tau)}dz$, as it yields $2\pi i$ times the difference in the number of zeroes and poles in
a given region. We shall integrate over the curve $C$ where $C$ traverses the circumference of one lattice segment (i.e. the whole torus):
\begin{align*}
    \oint_C \frac{\theta'(z|\tau)}{\theta(z|\tau)}dz=\oint_B\left(-\frac{\theta'(z|\tau)}{\theta(z|\tau)}+\frac{\theta'(z+1|\tau)}{\theta(z+1|\tau}\right)
    +\oint_A\left(\frac{\theta'(z|\tau)}{\theta(z|\tau)}-\frac{\theta'(z+\tau|\tau)}{\theta(z+\tau|\tau)}\right).
\end{align*}
But these are just the shifts in the logarithmic derivative, and since $\partial_z\log\theta(z+\tau|\tau)=-2\pi i+\partial_z\log\theta(z|\tau)$ using
the transformation rules above, we see that our integral simplifies to
\begin{align*}
    \oint_C \frac{\theta'(z|\tau)}{\theta(z|\tau)}dz=2\pi i\oint_A dz=2\pi i.
\end{align*}
Of course, since $\theta$ is holomorphic, it has no poles, and hence we see that we have one zero. The zero, in fact, occurs in the center:
$\theta\left( (1+\tau)/2|\tau \right)=0$. To see this, consider the following function:
\begin{align*}
    \theta\left( z+\frac{1+\tau}{2}|\tau \right)&=\sum_{n\in\Z}\exp\left( \pi i n^2\tau+2\pi i n\left(z+\frac{1+\tau}{2}\right) \right)\\
    &=i\exp\left( -\pi i\frac{\tau}{4}-\pi iz \right)\sum_{n\in\Z}\exp\left( \pi i(n+\frac{1}{2})^2\tau+2\pi i(n+\frac{1}{2})(z+\frac{1}{2}) \right)\\
    &=i\exp\left( -\pi i\frac{\tau}{4}-\pi iz \right)\theta_1(z|\tau)
\end{align*}
where we have completed the square and defined the function $\theta_1$. We claim that $\theta_1$ is an odd function, which would imply that $\theta_1$ vanishes
at zero, which would prove the claim about the location of the zero. Hence let us verify that $\theta_1$ is odd; switching $z\mapsto-z$ yields in the exponent
\begin{align*}
    \log\theta_1(z|\tau)=\pi i\left( n+\frac{1}{2} \right)^2\tau+2\pi i\left( n+\frac{1}{2} \right)\left( -z+\frac{1}{2} \right).
\end{align*}
If we switch the indices $n\mapsto m$ such that $n+\frac{1}{2}=-\left( m+\frac{1}{2} \right)$, we find that the exponent is now
\begin{align*}
    \log\theta_1(z|\tau)=\pi i\left( m+\frac{1}{2} \right)^2\tau+2\pi i\left( m+\frac{1}{2} \right)\left(\left( z+\frac{1}{2} \right)-2\pi i\left( m+\frac{1}{2} \right)\right),
\end{align*}
and hence $\theta_1$ is odd. Now we see that the function we want is in fact $\theta_1(z|\tau)$ as it is odd, holomorphic, and has one zero.

We leave it as an exercise to show that
\begin{align*}
    \sigma(z)=\omega_1\exp\left( \eta_1\frac{z^2}{\omega_1}\right)\frac{\theta_1\left( \frac{z}{\omega_1}|\tau \right)}{\theta_1'(0|\tau)}
\end{align*}

\section*{Class 11}

We claim that the theta-function theory is more powerful than what we have been using so far - to see this, let us prove Abel's theorem.
Recall that the theorem states that there exists a meromorphic $f$ with zeroes at $P_i$ and poles and $Q_j$ if and only if $N=M$ and $\sum_i A(P_i)=\sum_j A(Q_j)$.
The idea is to express
\begin{align*}
    f(z)=\frac{\prod_{i=1}^N\theta_1(x-P_i)}{\prod_{i=1}^N\theta_1(z-Q_i)}
\end{align*}
and check double-periodicity. It is an exercise to check that
\begin{align*}
    \theta_1(z+1|\tau)&=-\theta_1(z|\tau)\\
    \theta_1(z+\tau|\tau)&=\exp\left( -\pi i\tau-2\pi i(z+1/2) \right)\theta_1(z|\tau),
\end{align*}
from which periodicity follows easily. Of course, we must be careful to note that the $P_i,Q_i$ used here are in fact chosen representatives.

Next let us define a meromorphic form
\begin{align*}
    \omega_{PQ}=\partial_z\log\frac{\theta_1(z-P)}{\theta_1(z-Q)}dz,
\end{align*}
which, it is easy to check, has poles at $P,Q$ with opposite residues. Additionally, one can check that this expression is well-defined on the lattice, 
i.e. invariant under a shift. We leave it as a simple exercise to show that
\[\omega_P(z)=\partial^2_z\log\frac{\theta_1(z-P|\tau)}{\theta_1'(0|\tau)}dz\]
is a meromorphic form with a double pole at $P$ and is well-defined on the lattice.

But in fact, we can go even farther with this theta-function. Indeed, one attractive feature is that there exists a product expansion for $\theta(z|\tau)$.
\begin{thm}
    We can expand
    \begin{align*}
        \theta(z|\tau)=\prod_{n=1}^\infty (1-q^{2n)}(1+q^{2n-1}e^{2\pi iz})(1+q^{2n-1}e^{-2\pi iz})
    \end{align*}
    where $q\equiv e^{\pi i\tau}$.
\end{thm}
\begin{proof}
    Define
    \begin{align*}
        T(z|\tau)=\prod_{n=1}^\infty (1-q^{2n)}(1+q^{2n-1}e^{2\pi iz})(1+q^{2n-1}e^{-2\pi iz}).
    \end{align*}
    We claim that $T(z|\tau)$ is equal to zero exactly when $z$ is $(1+\tau)/2\mod\Lambda$ and that the zeros are simple. This can be
    checked by some simple algebra. It's also easy to show that $T(z|\tau)$ is holomorphic in $\C$ and that $\tau(z+1|\tau)=T(z|\tau)$.
    Moreover
    \begin{align*}
        T(z+\tau|\tau)&=\prod_{n=1}^\infty (1-q^{2n})\prod\left( 1+q^{2n+1}e^{2\pi iz} \right)\left( 1+q^{2n-3}e^{-2\pi iz}q^{-2} \right)\\
        &=\prod_{n=1}^\infty (1-q^{2n})\frac{\prod_{n=1}^\infty\left( 1+q^{2n-1}e^{2\pi i z} \right)}{1+qe^{2\pi iz}}\prod_{n=1}^\infty\left( 1+q^{2n-1}e^{-2\pi iz} \right)
        &=\frac{1-q^{-1}e^{-2\pi iz}}{1-qe^{2\pi iz}}T(z|\tau)=q^{-1}e^{-2\pi iz}.
    \end{align*}
    Recall that $\theta$ followd a similar condition. This shows that $\theta(z|\tau)/T(z|\tau)=c$,
    where $c$ is a constant independent of $z$ that can depend on $\tau$. Next we claim that $c(\tau)=1$. For this we show that there exists a $c$ such that $c(\tau)=c(4\tau)=c(4^k\tau)$
    and $c(\tau)=\lim_{k\to\infty}c(4^k\tau)=1$, which shows the proof. Hence let us prove that $c(\tau)=c(4\tau)$ using $\theta(z|\tau)=C(\tau)T(z|\tau)$.

    Take $z=1/2$. Then $e^{2\pi iz=e^{\pi i}}\geq 1-$ and $\theta(1/2|\tau)=\sum_{n\in\Z}e^{\pi i n^2\tau}(-1)^n$ but $T(1/2|\tau)=\prod^{\infty}_{n=1}(1-q^{2n)(1-q^{2n-1}}(1-q^{2n-1})$.
    and hence $c(\tau)=\sum e^{\pi in\tau}(-1)^n/\prod(1-q^n)(1-q^{2n-1})$. Next take $z=1/4$
\end{proof}

\section{Semester 2}

This semester we will start by describing the $L^2$ estimates of Hormander. Later we will delve into its applications,
including the Kodaira embedding theorem, the lower bounds for the Bergman kernel, and the ideas of canonical metrics
and stability.

\begin{rem}
    Suppose we have a holomorphic line bundle $L\to X$. We may ask the following questions.
    \begin{enumerate}[(a)]
        \item $H^0(X,L)=0$?
        \item Take some $s\in H^0(X,L)$ with $||s||_{L^2}=1$. How big can $s(z_0)$ be at a given point $z_0$?
    \end{enumerate}
    We will be building machinery to address these questions, which depend sensitively on the geometry.
\end{rem}

\subsection{Review}

Let us recall some techniques from last semester. Let $X=\cup_{\mu}X_\mu$ be a complex $n$-manifold with $X_\mu$ coordinate charts.
Hence each $X_\mu$ is homeomorphic (and thus biholomorphic) to $\C^n$ and $\Phi_\mu\circ\Phi_\nu^{-1}$ is holomorphic with invertible differentials\todo{what?}.
Let $E\to X$ be a holomorphic vector bundle. Recall that a rank-$r$ vector bundle is completely characterized by its transition functions $t_{\mu\nu\beta}^\alpha(z)$ (matrix valued
in general) defined on $X_\mu\cap X_\nu$ with $1\leq\alpha,\beta\leq r$. Note that the transition functions satisfy the cocycle condition.
We denote by $\Gamma(X,E)$ the space of sections of $E$ (recall that this means that $\phi_\mu^\alpha(z_\mu)=t_{\mu\nu\beta}^\alpha(z)\phi_\nu^\beta(z_\nu)$.
For $E$ to be holomorphic, it must have holomorphic transition functions.

Given a section $\phi\in\Gamma(X,E)$, we obtain a section $\bar\partial\phi\in\Gamma(X,E\otimes\Lambda^{0,1})$ via \textbf{covariant differentiation}.
More explicitly, on $X_\mu$, we write naively
\[\bar\partial\phi^\alpha\equiv\left(\frac{\partial}{\partial \bar z^j_\mu}\phi_{\mu}^\alpha\right)(z_\mu).\]
Fortunately, this is indeed a section. To see this, we note that $\phi^\alpha_\mu(z_\mu)=t_{\mu\nu\beta}^\alpha(z)\phi_\nu^\beta(z_\nu)$ and hence
\begin{align*}
    \frac{\partial}{\partial \bar z^j_\mu}\phi^\alpha_\mu(z_\mu)&=t_{\mu\nu\beta}^\alpha(z)\frac{\partial}{\partial \bar z^j_\mu}\left( \phi_\nu^\beta(z_\nu) \right)\\
    &=t_{\mu\nu\beta}^\alpha(z)\overline{\frac{\partial z_\nu^k}{\partial z_\mu^j}}\frac{\partial \phi^\beta_\nu}{\partial\bar z_\nu^k}(z).
\end{align*}
Thus we define $\Lambda^{0,1}$ to be the (antiholomorphic) vector bundle with transition functions $\overline{\partial z_\nu^k/\partial z^j_\mu}$.

Next, recall the definition of a \textbf{Hermitian metric} $H=H_{\bar \alpha\beta}(z)$ on a vector bundle $E$: we have $(H_\mu)_{\bar\alpha\beta}(z_\mu)$ on $X_\mu$
such that it is a positive-definite matrix for each $z_\mu$ satisfying
\[|\phi|^2_H\equiv(H_\mu)_{\bar\alpha\beta}\overline{\phi_\mu^\alpha}\phi_\mu^\beta=(H_\nu)_{\bar\gamma\delta}\overline{\phi_\nu^\gamma}\phi_\nu^\delta\]
on $X_\mu\cap X_\nu$.
This quantity can be thought of as the length of the vector $\phi$ with respect to the metric $H$, which is by construction invariant of coordinate chart.
Using metrics, we can introduce covariant derivatives of sections on a holomorphic vector bundle $E$ with respect to a metric $H_{\bar\alpha\beta}$.
Take $\phi\in\Gamma(X,E)$ and define on $X_\mu$
\[(\nabla_j\phi)^\alpha\equiv H^{\alpha\bar\gamma}\partial_j(H_{\bar\gamma\beta}\phi_\mu^\beta),\]
where $H^{\alpha\bar\gamma}H_{\bar\gamma\beta}=\delta^{\alpha}_\beta$. It is easy to see that $\nabla\phi\in\Gamma(X,E\otimes\Lambda^{1,0})$,
whose transition functions are $\partial z_\nu^k/\partial z^j_\mu$.

In summary, we write
\begin{align*}
    \nabla_{\bar j}\phi^{\alpha}&=\partial_{\bar j}\phi^\alpha\\
    \nabla_j\phi^\alpha&=H^{\alpha\bar\gamma}\partial_j(H_{\bar\gamma\beta}\phi_\mu^\beta)\\
    &=\partial_j\phi^\alpha+(H^{\alpha\bar\gamma}\partial_jH_{\bar\gamma\beta})\phi^{\beta}\\
    &=\partial_j\phi^\alpha+A_{j\beta}^\alpha\phi^\beta,
\end{align*}
where, in matrix notation, we have the \textbf{connection} $A_j=H^{-1}\partial_j H$. It is a priori not obvious that these two derivatives
must commute. Indeed, we define the \textbf{curvature} $F$ of the metric $H_{\bar\alpha\beta}$ on $E\to X$ to be
\[ [\nabla_{\bar j},\nabla_k]\phi^\alpha=-F_{\bar jk\beta}^\alpha \phi^\beta.\]
We leave it as an exercise that $[\nabla_{\bar j},\nabla_{\bar k}]=0$ and $[\nabla_j,\nabla_k]=0$. More explicitly,
we can write
\begin{align*}
    [\nabla_{\bar j},\nabla_k]\phi^\alpha&=\nabla_{\bar j}\left( \nabla_k\phi^\alpha \right)-\nabla_k\left( \nabla_{\bar j}\phi^{\alpha} \right)\\
    &=(\partial_{\bar j}A_{k\beta}^\alpha)\phi^{\beta},
\end{align*}
and hence $F_{\bar jk\beta}^\alpha=-\partial_{\bar j}A_{k\beta}^\alpha$. In matrix notation, we can simply write 
\[F_{\bar jk}=-\partial_{\bar j}A_k=-\partial_{\bar j}\left( H^{-1}\partial_k H \right).\]
We define the corresponding \textbf{curvature form} to be
\[F=\frac{i}{2\pi}F_{\bar jk\beta}^\alpha(z)dz^k\wedge d\bar z^j\in\Gamma(X,E\otimes E^*\otimes\Lambda^{1,1})=\Gamma(X,\End(E)\otimes\Lambda^{1,1}),\]
i.e. a $\End(E)$-valued $(1,1)$-form.

\subsection{Bochner-Kodaira Formulas}

Let $X$ be a complex manifold, compact without boundary. Take $E\to X$ to be a holomorphic vector bundle of rank $r$ on $X$.
We consider the \textbf{$\bar\partial$ complex}:
\[\cdots\xrightarrow{\bar\partial}\Gamma(X,E\otimes\Lambda^{p,q})\xrightarrow{\bar\partial}\Gamma(X,E\otimes\Lambda^{p,q+1})\xrightarrow{\bar\partial}\cdots.\]

Let us be more precise. Consider some $\phi\in\Gamma(X,E\otimes\Lambda^{p,q})$. We can write explicitly:
\[\phi^\alpha=\frac{1}{p!q!}\sum\phi^\alpha_{\bar j_1,\ldots\bar j_q,i_1,\ldots,i_p}(z)dz^{i_p}\wedge\cdots\wedge dz^{i_1}\wedge d\bar z^{j_q}\wedge\cdots\wedge d\bar z^{j_i}.\]
Now what exactly do we mean by $\bar\partial$? We define
\begin{align*}
    \bar\partial\phi&\equiv\frac{1}{p!q!}\sum\left(\bar\partial\phi_{\bar j_1,\ldots\bar j_q,i_1,\ldots,i_p}\right)\wedge dz^{i_p}\wedge\cdots\wedge dz^{i_1}\wedge d\bar z^{j_q}\wedge\cdots\wedge d\bar z^{j_i}\\
    &=\frac{1}{p!q!}\sum\left(\partial_{\bar k}\phi_{\bar j_1,\ldots\bar j_q,i_1,\ldots,i_p}d\bar z^k\right)\wedge dz^{i_p}\wedge\cdots\wedge dz^{i_1}\wedge d\bar z^{j_q}\wedge\cdots\wedge     d\bar z^{j_i}.
\end{align*}
We leave it as an exercise for the reader to check that this is well-defined (follows as per the usual de Rham exterior derivative).

\begin{exmp}
    What is $\bar\partial$ on $\Gamma(X,E\otimes\Lambda^{0,0})=\Gamma(X,E)$? By definition, $\bar\partial\phi=\partial_{\bar k}\phi^\alpha d\bar z^k$.
\end{exmp}
\begin{exmp}
    What is $\bar\partial$ on $\Gamma(X,E\otimes\Lambda^{0,1})$? Given a section, we can write $\phi=\sum \phi^\alpha_{\bar j}d\bar z^j$.
    In this case,
    \begin{align*}
        \bar\partial\phi^\alpha&=\sum\left( \bar\partial \phi^\alpha_{\bar j} \right)\wedge d\bar z^j\\
        &=\sum\left( \partial_{\bar k}\phi^{\alpha}_{\bar j}d\bar z^k \right)\wedge d\bar z^j\\
        &=\frac{1}{2}\sum\left( \partial_{\bar k}\phi_{\bar j}^\alpha-\partial_{\bar j}\phi^{\alpha}_{\bar k} \right)d\bar z^k\wedge d\bar z^j.
    \end{align*}
    Hence one finds the coefficient $(\bar\partial \phi)_{\bar j\bar k}=\left(\partial_{\bar k}\partial_{\bar j}-\partial_{\bar j}\partial_{\bar k}\right)$
    with no factor of $1/2$ out front, because we now have a two-form.
\end{exmp}

Let us now introduce a metric $H_{\bar\alpha\beta}$ on $E$ and a metric $g_{\bar kj}$ on $T^{1,0}(X)$.
This allows us to compute scalar norms of sections as $|\phi|^2_H=H_{\bar\alpha\beta}\overline{\phi^\alpha}\phi^\beta$ for $\phi\in\Gamma(X,E)$.
We will work with the metric and hope that in the end, our results will be independent of the metric (where lengths are not involved).
There is an induced $L^2$ metric on $\Gamma(X,E\otimes \Lambda^{p,q})$: given $\phi,\psi\in\Gamma(X,E\otimes\Lambda^{p,q})$, we define, using
multi-index notation,
\begin{align*}
    \langle \phi,\psi\rangle = \frac{1}{p!q!}\sum\int \phi^\alpha_{\bar JI}\overline{\psi^\beta_{\bar KL}}H_{\bar\beta\alpha}g^{K\bar J}g^{I\bar L}\frac{\omega^n}{n!},
\end{align*}
where, by definition,
\[g^{K\bar J}=g^{k_1\bar j_1}\cdots g^{k_q\bar j_q}\]
if $K=(k_1,\ldots, k_q)$ and $J=(j_1,\ldots,j_q)$,
and
\[\omega\equiv \frac{i}{2}g_{\bar kj}dz^j\wedge d\bar z^k.\]
We now define the \textbf{formal adjoint of $\bar\partial$} by
\begin{align*}
    \langle \bar\partial \phi,\psi\rangle =\langle\phi,\bar\partial^\dagger\psi\rangle
\end{align*}
for all $\phi\in C^\infty(X,E\otimes\Gamma^{p,q})$ and $\psi\in C^\infty(X,E\otimes\Gamma^{p,q+1})$.
Hence we can draw:
\begin{equation*}
    \begin{tikzcd}
        \Gamma(X,E\otimes\Lambda^{p,q})\arrow[bend left]{r}{\bar\partial} & \Gamma(X,E\otimes\Lambda^{p,q+1})\arrow[bend left]{l}{\bar\partial^\dagger}
    \end{tikzcd}
\end{equation*}
Next define the \textbf{Laplacian} $\Box:\Gamma(X,E\otimes\Lambda^{p,q})\to\Gamma(X,E\otimes\Lambda^{p,q})$ by $\Box=\bar\partial^\dagger\bar\partial+\bar\partial\bar\partial^\dagger$.
This raises the question: when do we have that
\[\dim\ker\Box\bigg|_{\Gamma(X,E\otimes\Lambda^{p,q})}=0?\]
It will turn out that for $q=1$, if this is true, we will indeed be able to find sections on this bundle.

To approach this question, we use the Bochner-Kodaira formulas. We claim that
\begin{align*}
    (\Box\phi)^\alpha_{\bar JI}=-g^{k\bar l}\nabla_k\nabla_{\bar l}\phi^\alpha_{\bar JI}+\text{t.t} + \text{c.t.}
\end{align*}
where t.t. and c.t. stand for torsion and curvature forms respectively. This relates a kind of geometric laplacian (lhs) to a
more analytic laplacian (rhs) modulo certain correction terms. In practice, we will work with K\"ahler metrics, in which
the torsion terms disappear. In this sense, we can schematically write (after integrating by parts)
\[\langle \Box\phi,\phi\rangle=||\nabla_{\bar l}\phi_{\bar J I}^\alpha||^2+\langle\text{c.t.}\phi,\phi\rangle.\]
This immediately implies that if the curvature terms are positive (loosely speaking), then $\ker\Box=0$.

We begin by deriving the Bochner-Kodaira formula. We will simplify life by working with $(0,1)$-forms, but the results
will extend fairly easily. Let us first compute $\bar\partial^\dagger$ more explicitly. In this case, we look at
\begin{equation*}
    \begin{tikzcd}
        \Gamma(X,E\otimes\Lambda^{0,0})\arrow[bend left]{r}{\bar\partial}&\Gamma(X,E\otimes\Lambda^{0,1})\arrow[bend left]{r}{\bar\partial}\arrow[bend left]{l}{\bar\partial^\dagger} & \Gamma(X,E\otimes\Lambda^{0,2})\arrow[bend left]{l}{\bar\partial^\dagger}.
    \end{tikzcd}
\end{equation*}
Let us compute the second formal adjoint that appears in the diagram. Pick some $\phi\in\Gamma(X,E\otimes\Lambda^{0,1})$ such that $\phi=\sum\phi_{\bar j}^\alpha d\bar z^j$. Then $\bar\partial\phi^\alpha=\sum\partial_{\bar k}\phi_{\bar j}^\alpha d\bar z^k\wedge d\bar z^j$. Also, pick $\psi=\frac{1}{2}\sum\psi_{\bar j\bar k}d\bar z^k\wedge d\bar z^j$ in $\Gamma(X,E\otimes\Lambda^{0,2})$. We impose that
\begin{align*}
    \langle\bar\partial\phi,\psi\rangle&=\langle \phi,\bar\partial^\dagger\psi\rangle
\end{align*}
The left hand side appears to be
\[ \frac{1}{2}\left( \int(\partial_{\bar p}\phi_{\bar m}^\alpha-\partial_{\bar m}\phi^\alpha_{\bar p}) \overline{\psi_{\bar j\bar k}^\beta}H_{\bar \beta\alpha}g^{j\bar m}g^{k\bar p}\frac{\omega^n}{n!} \right)
    \]
But recall that
\begin{align*}
    \nabla_{\bar p}\phi^{\alpha}_{\bar m}&=\partial_{\bar p}\phi_{\bar m}^\alpha-\Gamma^{\bar l}_{\bar p\bar m}\phi_{\bar l}^\alpha
\end{align*}
where $\Gamma_{\bar p\bar m}^{\bar l}=g^{k\bar l}(\partial_{\bar p}g_{\bar mk}).$ Hence we may write
\[\partial_{\bar p}\phi^\alpha_{\bar m}-\partial_{\bar m}\phi_{\bar p}^\alpha=\nabla_{\bar p}\phi^\alpha_{\bar m}-\nabla_{\bar m}\phi_{\bar p}^{\alpha}+(\Gamma^{\bar l}_{\bar p\bar m}-\Gamma^{\bar l}_{\bar m\bar p})\phi_{\bar l}^\alpha.\]
We denote the term in the parentheses by $T^{\bar l}_{\bar p\bar m}$ and call it the \textbf{torsion of the covariant derivative $\nabla_{\bar p}$}. Let us now move to the K\"ahler case.
The metric $g_{\bar kj}$ is said to be \textbf{K\"ahler} if $\Gamma^{\bar l}_{\bar p\bar m}=\Gamma^{\bar l}_{\bar m\bar p}$. Note that this
condition is equivalent to $\partial_{\bar p}g_{\bar mk}=\partial_{\bar m}g_{\bar pk}$ or $\partial_pg_{\bar km}=\partial_m g_{\bar kp}$.
Henceforth we assume that $g_{\bar kj}$ is K\"ahler.

Now in the computation of the formal adjoint above, we can write
\begin{align*}
    \langle\bar\partial\phi,\psi\rangle&=\int (\nabla_{\bar p}\phi^\alpha_{\bar m})\overline{\psi^\beta_{\bar j\bar k}}H_{\bar\beta\alpha}g^{j\bar p}g^{k\bar m}\frac{\omega^n}{n!}\\
    &=\int\phi^\alpha_{\bar m}\overline{(-g^{p\bar j}\nabla_p\psi^{\beta}_{\bar j\bar k})}H_{\bar\beta\alpha}g^{k\bar m}\frac{\omega^n}{n!},
\end{align*}
where we have de-antisymmetrized the covariant derivatives and then integrated by parts. This yields immediately the expression for the formal adjoint:
\[(\bar\partial^\dagger\psi)_{\bar k}^\beta=-g^{p\bar j}\nabla_p\psi^{\beta}_{\bar j\bar k}.\]
To see this in more detail, see the next paragraph, where we will perform the computation explicitly for the other formal adjoint.
So much for the computation of the second formal adjoint in the diagram above.\todo{check this!}

Let us now compute the first formal adjoint. Of course,
\[\langle\bar\partial\phi,\psi\rangle=\langle\phi,\bar\partial^\dagger\psi\rangle\]
for $\phi\in C^\infty(X,E)$ and $\psi\in C^\infty(X,E\otimes\Lambda^{0,1})$. We can write $\bar\partial\phi=\partial_{\bar j}\phi^\alpha d\bar z^j$
and $\psi=\psi^{\alpha}_{\bar k}d\bar z^k$. The above equation requires
\begin{align*}
    \int \partial_{\bar j}\phi^\alpha\overline{\psi^\beta_{\bar k}}H_{\bar \beta\alpha}g^{k\bar j}\frac{\omega^n}{n!}=\int\phi^\alpha\overline{\left( \bar\partial^\dagger\psi \right)^\beta}H_{\bar\beta\alpha}\frac{\omega^n}{n!}.
\end{align*}
Let us for now define $W_\alpha^{\bar j}\equiv \overline{\psi^\beta_{\bar k}H_{\bar\alpha\beta}g^{j\bar k}}$. Observe now that
\begin{align*}
    \left( \partial_{\bar j}\phi^\alpha \right)W^{\bar j}_\alpha&\equiv\left( \nabla_{\bar j}\phi^\alpha \right)W^{\bar j}_\alpha\\
    &=\nabla_{\bar j}\left( \phi^\alpha W^{\bar j}_\alpha \right)-\phi^\alpha\left( \nabla_{\bar j} W^{\bar j}_\alpha\right).
\end{align*}
This will be useful when integrating by parts.
Now note that the $n$th wedge power of $\omega$ simplifies to yield
\begin{align*}
    \int\left( \partial_{\bar j}\phi^\alpha \right)W_\alpha^{\bar j}\left( \det g_{\bar qp} \right)&=\int\nabla_{\bar j}\left( \phi^\alpha W^{\bar j}_\alpha \right)\det g_{\bar qp}
    -\int \phi^\alpha\left( \nabla_{\bar j}W^{\bar j}_\alpha \right).
\end{align*}
We claim that if the metric $g_{\bar kj}$ is K\"ahler, then $\int \nabla_{\bar j}(\phi^\alpha W_\alpha^{\bar j})\det g_{\bar qp}=0$. To see this, first define
$V^{\bar j}=\phi^\alpha W_\alpha^{\bar j}$. Consider
\begin{align*}
    \left( \nabla_{\bar j}V^{\bar j} \right)\det g_{\bar qp}&=\left(\partial_{\bar j}V^{\bar j}+\Gamma^{\bar j}_{\bar j\bar k}V^{\bar k}\right)\det g_{\bar qp}\\
    &=\partial_{\bar j}\left( V^{\bar j}\det g_{\bar qp} \right)-V^{\bar j}\left( \partial_{\bar j}\det g_{\bar qp} \right)+\Gamma^{\bar j}_{\bar j\bar k}V^{\bar k}\det g_{\bar qp}.
\end{align*}
Now note that
\begin{align*}
    \partial_{\bar j}\det g_{\bar qp}=\left( \det g_{\bar qp}\right)g^{l\bar n}\partial_{\bar j}g_{\bar nl},
\end{align*}
which comes from the fact that
\begin{align*}
    \delta\log\left( \det A \right)&=\sum\frac{\delta \lambda_j}{\lambda_j}\\
    &=\tr\left( A^{-1}\delta A \right).
\end{align*}
But now recall that for $g_{\bar kj}$ is K\"ahler if and only if $\Gamma^{\bar j}_{\bar k\bar m}=\Gamma^{\bar j}_{\bar m\bar k}$,
and hence the we see the last two terms in  the expression above cancel. Hence the term picked up by integration by parts vanishes.
Hence we are left with the equality
\begin{align*}
    -\int\phi^\alpha\nabla_{\bar j}\overline{\left( \psi^{\beta}_{\bar k}H_{\bar\alpha\beta}g^{j\bar k} \right)}\frac{\omega^n}{n!}&=
    \int\phi^\alpha\overline{\left( -g^{j\bar k}\nabla_j\psi^\beta_{\bar k} \right)}H_{\bar \beta\alpha}\frac{\omega^n}{n!}.
\end{align*}
This yields the desired formula:
\begin{align*}
    ( \bar\partial^\dagger\psi )^{\beta}=-g^{j\bar k}\nabla_j\psi^\beta_{\bar k}.
\end{align*}

Now let us compute the Laplacian $\Box=\bar\partial\bar\partial^\dagger+\bar\partial^\dagger\bar\partial$. Set $\phi=\sum \phi^\alpha_{\bar j}d\bar z^j\in\Gamma(X,E\otimes\Lambda^{0,1})$.
First note that
\begin{align*}
    \left( \bar\partial\bar\partial^{\dagger}\phi \right)&=\bar\partial\left( -g^{j\bar k}\nabla_j\phi^{\alpha}_{\bar k} \right)\\
    &=\partial_{\bar l}\left( -g^{j\bar k}\nabla_j\phi_{\bar k}^\alpha \right)d\bar z^l.
\end{align*}
Hence, noting that the expression in parentheses is a section of a holomorphic bundle, the $\partial_j$ is simply a covariant derviative, which
commutes with the metric, and we can write
\[\left( \bar\partial\bar\partial^{\dagger}\phi \right)^\alpha_{\bar l}=-g^{j\bar k}\nabla_{\bar l}\nabla_{j}\phi^\alpha_{\bar k}.\]
Next note that
\begin{align*}
    \bar\partial\phi^\alpha=\frac{1}{2}\sum\left( \nabla_{\bar k}\phi^\alpha_{\bar j}-\nabla_{\bar j}\phi^\alpha_{\bar k} \right)d\bar z^k\wedge d\bar z^j
\end{align*}
for $g_{\bar kj}$ K\"ahler. Hence we can write
\begin{align*}
    \left( \bar\partial^\dagger\bar\partial\phi \right)^\beta_{\bar l}&=-g^{k\bar m}\nabla_{k}(\bar\partial\phi)^\beta_{\bar l\bar m}\\
    &=-g^{k\bar m}\nabla_k\left( \nabla_{\bar m}\phi^\alpha_{\bar l}-\nabla_{\bar l}\phi^\alpha_{\bar m} \right)\\
    &=-g^{k\bar m}\nabla_k\nabla_{\bar m}\phi^\alpha_{\bar l}+g^{k\bar m}\nabla_k\nabla_{\bar l}\phi^\alpha_{\bar m}.
\end{align*}
Summing the two terms of the Laplacian and switching appropriate dummy indices, we find that
\begin{align*}
    (\Box\phi)^\alpha_{\bar l}&=-g^{k\bar m}\nabla_k\nabla_{\bar m}\phi^\alpha_{\bar l}+g^{k\bar m}\nabla_k\nabla_{\bar l}\phi^\alpha_{\bar m}-g^{k\bar m}\nabla_{\bar l}\nabla_{\bar k}\phi^\alpha_{\bar m}\\
    &=-g^{k\bar m}\nabla_k\nabla_{\bar m}\phi^\alpha_{\bar l}+g^{k\bar m}[\nabla_k,\nabla_{\bar l}]\phi^\alpha_{\bar m}\\
    &=-g^{k\bar m}\nabla_k\nabla_{\bar m}\phi^\alpha_{\bar l}+g^{k\bar m}\left( F_{\bar lk\beta}^\alpha\phi^\beta_{\bar m}+R_{\bar lk\bar m}^{\bar p}\phi^\alpha_{\bar p} \right),
\end{align*}
where as usual $F_{\bar lk\beta}^\alpha=-\partial_{\bar l}\left( J^{\alpha\bar\gamma}\partial_k H_{\bar\gamma\beta} \right)$
and $R_{\bar lk\bar m}^{\bar p}=g^{\bar pq}R_{\bar lkq}^rg_{\bar mr}$ where $R_{\bar lkq}^r=-\partial_{\bar l}\left( g^{r\bar s}\partial_kg_{\bar sq} \right)$.
We can simplify this a little bit more, obtaining
\begin{align}
    (\Box\phi)^\alpha_{\bar l}=-g^{k\bar m}\nabla_k\nabla_{\bar m}\phi^\alpha_{\bar l}+F_{\bar l\beta}^{\bar m\alpha}\phi^\beta_{\bar m}+R_{\bar l}^{\bar p}\phi^\alpha_{\bar p},
\end{align}
where $R_{\bar l}^{\bar p}\equiv g^{k\bar m}R_{\bar lk\bar m}^{\bar p}$ is the \textbf{Ricci curvature}.

Consider now the inner product with $\phi$:
\begin{align*}
    \int (\Box\phi)^{\alpha}_{\bar l}\overline{\phi^\beta_{\bar m}}H_{\bar \beta\alpha}g^{m\bar l}\frac{\omega^n}{n!}&=-\int g^{j\bar k}\nabla_j\nabla_{\bar k}\phi^\alpha_{\bar l}\overline{\phi^\beta_{\bar m}}H_{\bar\beta\alpha}g^{m\bar l}\frac{\omega^n}{n!}+\int (F_{\bar l\gamma}^{\bar p\alpha}\phi^\gamma_{\bar p}+R_{\bar l}^{\bar p}\phi^\alpha_{\bar p})\overline{\phi^\beta_{\bar m}}H_{\bar\beta\alpha}g^{m\bar l}\frac{\omega^n}{n!}\\
    \langle\Box\phi,\phi\rangle&=\int \nabla_{\bar k}\phi^\alpha_{\bar l}\overline{\nabla_{\bar j}\phi^\beta_{\bar m}}g^{j\bar k}H_{\bar\beta\alpha}g^{m\bar l}\frac{\omega^n}{n!}+\int F^{m\bar p}_{\bar\beta\alpha}\phi_{\bar p}^\alpha\overline{\phi^\beta_{\bar m}}+(R^{m\bar p}H_{\bar\beta\alpha})\phi^\alpha_{\bar p}\overline{\phi^\beta_{\bar m}}\frac{\omega^n}{n!}\\
    \langle\Box\phi,\phi\rangle&=||\bar\nabla\phi||^2+\int (F^{m\bar p}_{\bar\beta\alpha}+R^{m\bar p}H_{\bar\beta\alpha})\phi^\alpha_{\bar p}\overline{\phi^\beta_{\bar m}}\frac{\omega^n}{n!}.
\end{align*}
This yields the following simple corollary.

\begin{cor}
If $F^{m\bar p}_{\bar\beta\alpha}+R^{m\bar p}_{\bar\beta\alpha}>0$, then $\ker\Box\bigg|_{\Gamma(X,E\otimes\Lambda^{0,1})}=0.$
\end{cor}

Further, we can prove the following result.

\begin{cor}
Let $L\to X$ be a \textbf{positive} holomorphic line bundle over a compact manifold $X$, i.e. there exists a metric $h$ on $L$
with $-\partial_{j}\partial_{\bar k}\log h>0$. Set $\omega=-\frac{i}{2}\partial\bar\partial\log h$ (as coefficients). Since $L$ is positive, $\omega$ is a metric, which
is automatically K\"ahler:
\[\partial_l g_{\bar kj}=-\partial_l\partial_j\partial_{\bar k}\log h=\partial_j g_{\bar kl}.\]
Equip $X$ with the K\"ahler metric $\omega$ and consider $\Box$ on $L^M\otimes\Lambda^{0,1}$.
Then for $M\geq 1$,
\[\ker\Box\bigg|_{\Gamma(X,L^m\otimes\Lambda^{0,1})}=0.\]
\end{cor}
\begin{proof}
    When $E=L$, we can write
    \begin{align*}
        F_{\bar kj}=-\partial_j\partial_{\bar k}\log h
    \end{align*}
    and hence in the case of line bundles, we have that
    \begin{align*}
        F^{m\bar p}\equiv F^{m\bar p}_{\bar\beta\alpha}=g^{m\bar k}g^{j\bar p}F_{\bar kj}h.
    \end{align*}
    Hence the Bochner-Kodaira formula simplifies to
    \begin{align*}
        \langle\Box\phi,\phi\rangle=||\bar\nabla\phi||^2+\int (F^{m\bar p}+R^{m\bar p})\phi_{\bar p}\overline{\phi_{\bar m}}h\frac{\omega^n}{n!}.
    \end{align*}
    Now if we take $L\mapsto L^M$, $R$ does not change, as it is the Ricci curvature of the metric on the base manifold. On the other hand,
    the curvature $F$ of $L$ is now multiplied by $M$, as the curvature is given by the logarithm above (and hence the power $M$ becomes
    multiplicative). For $L^M\to X$, then, the Bochner-Kodaira formula reads:
    \begin{align*}
        \langle\Box\phi,\phi\rangle=||\bar\nabla\phi||^2+\int (MF^{m\bar p}+R^{m\bar p})\phi_{\bar p}\overline{\phi_{\bar m}}h\frac{\omega^n}{n!}.
    \end{align*}
    In the case of positive line bundle $L$, we have a K\"ahler metric on $X$, which is precisely the curvature of the metric on $L$, $\omega$,
    which yields the fact that
    \[F^{m\bar p}=g^{m\bar p}.\]
    On $L^M\to X$, then, we can further simplify the Bochner-Kodaira formula to
    \begin{align*}
        \langle\Box\phi,\phi\rangle&=||\bar\nabla\phi||^2+\int (Mg^{m\bar p}+R^{m\bar p})\phi_{\bar p}\overline{\phi_{\bar m}}h\frac{\omega^n}{n!}.
    \end{align*}
    Since $g^{m\bar p}>0$, we can choose $M$ large enough such that
    \[Mg^{m\bar p}+R^{m\bar p}>0,\]
    which concludes the proof.
\end{proof}

\begin{exc}
    Derive the Bochner-Kodaira formula for the case of $\Lambda^{0,2}$. It will be good for your soul.
\end{exc}

\subsection{Other Bochner-Kodaira Formulae}

\begin{defn}
    Define the \textbf{Hodge} operator $\Lambda$ as follows.
    Let $\Phi\in\Gamma(X,E\otimes\Lambda^{p+1,q+1})$ for some
    bundle $E\to X$. Then $(\Lambda\Phi)_{\bar KJ}=g^{l\bar p}\Phi^\alpha_{p\bar l\bar KJ}\in\Gamma(X,E\otimes\Lambda^{p,q})$.
\end{defn}

\begin{exc}
    Show that $[\partial,\Lambda]=\bar\partial^\dagger$ and $[\bar\partial,\Lambda]=-\partial^\dagger$.
\end{exc}

Now consider $\bar\Box=\partial\partial^\dagger+\partial^\dagger\partial$ on $\Gamma(X,E\otimes\Lambda^{p,q})$.
\begin{thm}[Kodaira-Akizuki-Nakano]
    Let $L\to X$ be a line bundle. Then
    \[\Box=\bar\Box+[F,\Lambda].\]
\end{thm}
\begin{proof}
    We sketch the proof: 
    \begin{align*}
        \Box-\bar\Box&=[\partial,\Lambda]\bar\partial+\bar\partial[\partial,\Lambda]+[\bar\partial,\Lambda]\partial+\partial[\bar\partial,\Lambda]\\
        &=(\bar\partial\partial+\partial\bar\partial)\Lambda-\Lambda(\bar\partial\partial+\partial\bar\partial)\\
        &=[\bar\partial\partial+\partial\bar\partial,\Lambda]\\
        &=[F,\Lambda]
    \end{align*}
\end{proof}

In practice, we use the following lemma when we apply the KAN theorem, which gives us a handle on positivity.
\begin{lem}
    Let $\omega$ and $F$ be simultaneously diagonalized, i.e. if $\zeta^a=\zeta^a_j dz^j$ is a basis of forms, then
    $\omega=\frac{i}{2}\sum_a\zeta^a\wedge\overline{\zeta^a}$ and $F=\frac{i}{2}\sum_a\lambda_a\zeta^a\wedge\overline{\zeta^a}$.
    Then
    \begin{align*}
        \langle[F,\Lambda]\Psi,\Psi\rangle&=\frac{1}{2}\sum_{KJ}\left(\sum_{a\in J}\lambda_a+\sum_{b\in K}\lambda_b-\sum_{c=1}^n\lambda_c\right)|\Psi_{\bar KJ}|^2.
    \end{align*}
\end{lem}
Positivity can be tested directly from the term in parentheses. In particular, on positive line bundles, the sums are particularly simple: $p+q-n$.
Hence we find via this lemma that for a positive line bundle $L$,
\[\ker\Box\bigg|_{L\otimes\Lambda^{p,q}}=0\]
when $p+q-n>0$.

This concludes the easy part of this theory. Now we will turn to the cohomology of the $\bar\partial$ complexes.

\subsection*{Cohomology}

Let $E\to X$ be a holomorphic line bundle, with $H_{\bar\alpha\beta}$ a metric on $E$ and $g_{\bar kj}$ a (K\"ahler) metric on $X$.
Recall the $\bar\partial$ complex from above. We define the \textbf{cohomology} of the $\bar\partial$ complex as
\begin{align*}
    H^{p,q}_{\bar\partial}(X,E\otimes\Lambda^{p,q})=\ker\bar\partial|_{\Gamma(X,E\otimes\Lambda^{p,q})}/\text{Im }\bar\partial|_{\Gamma(X,E\otimes\Lambda^{p,q-1})}.
\end{align*}
This makes sense because the complex is exact by construction.
This is the analogue of de Rham cohomology, where the \textbf{de Rham cohomology}
\begin{align*}
    H^p_{dR}(X)=\ker d|_{\Gamma(x,\Lambda^p)}/\text{Im }d|_{\Gamma(X,\Lambda^{p-1})}
\end{align*}
is defined for smooth compact manifolds $X$, and carries some sort of topological data of $X$. In our case, we are dealing with complex manifolds
and hence the cohomology of the $\bar\partial$ complex depends sensitively on the complex structure of the manifold.
There are two approaches to computing $H^p(X)$ (or at least, to determine when it is 0), namely Hodge theory and the $L^2$ estimates
of H\"ormander. 
In Hodge theory, the key statment is that $H^{p}_{dR}(X)$ is isomorphic to $\ker\Delta$, where $\Delta$ is the Laplacian $\Delta=dd^\dagger+d^\dagger d$
(having introduced a metric). What is of course remarkable is that the left-hand side of the equality is metric-independent. Then one can use certain
vanishing theorems to show that this kernel, and hence the cohomology group, is zero.
On the other hand, the $L^2$ estimates approach gives conditions under which the equation $d=v$ for $v\in L^2(X,\Lambda^{p+1})$ and $dv=0$, admits solutions.
This approach has intrinsic interest also, however, because this equation involves only one derivative.

The key step in the Hodge theory approach is the construction of an operator $G$ such that
\[G\Delta=\Delta G=I-\Pi\]
where $\Pi:L^2(X,\Lambda^p)\to\ker\Delta$ is the orthogonal projection. Furthermore, $dG=Gd$ and $d^\dagger G=Gd^\dagger$.
Assume the existence of $G$. Then
\begin{align*}
    u=(\Delta g)u+\Pi u=d(d^\dagger Gu)+d^\dagger(dGu)+\Pi u
\end{align*}
for any smooth $u$. If $u\in\ker d$ then $du=0$ and we are left with
\begin{align*}
    u&=d(d^\dagger Gu)+\Pi u\\
    [u]&=[\Pi u].
\end{align*}
But by definition, $\Pi u\in\ker\Delta$ and we are done.
Moreover, suppose we want to solve the $d$ equation $du=v$. Just take $u=d^\dagger Gv$. Then
\[du=dd^\dagger Gv=(\Delta-d^\dagger d)Gv=\Delta Gv=v-\Pi v.\]
Hence if $v$ is of class zero, then $v$ is given by this equation. Otherwise, the equation cannot be solved.

Now how does one prove the existence of $G$? This follows (with more work, of course) from the following estimate
\begin{align*}
    ||u||_{W^{k+2,2}}\leq c(||\Delta u||^2_{W^{k,2}}+||u||^2_{W^{k,2}})
\end{align*}
for all $u\in C^\infty(X,\Lambda^p)$ where we have the Sobolev norm
\begin{align*}
    ||u||^2_{W^{k,p}}\equiv \sum_{|\alpha|\leq k}||D^\alpha u||^2_{L^p}.
\end{align*}
for $p>1$. These kinds of estimates hold for elliptic operators (in this case the Laplacian).

\subsection*{The $L^2$ estimates approach}

Let $L\to X$ be a holomorphic line bundle over $X$ a compact complex manifold and let $h$ be a metric on $L$
and $g_{\bar kj}$ to be K\"ahler. We will in fact work with distributions instead of functions. Consider
$L^1_{loc}(\R^n)$. Let $f\in L^1_{loc}(\R^n).$ The derivative $\partial f/\partial x_1$ in the sense of
distributions is the functional
\begin{align*}
    C^\infty_0(\R^n)\ni\phi\mapsto -\int f\frac{\partial\phi}{\partial x_1}
\end{align*}
If $f\in C^1_{loc}(\R^n)$ we can integrate by parts and the functional becomes
\[C^\infty_0(\R^n)\ni\phi\mapsto \int\frac{\partial f}{\partial x_1}\phi.\]
This gives us a generalized way of thinking about the derivative. Hence we define the domain of $\bar\partial$
to be \[\text{Dom }\bar\partial_{p,q}=\left\{\phi\in L^2(X,L\otimes\Lambda^{p,q})\mid\exists\psi\in L^2(X,L\otimes\Lambda^{p,q+1})\text{ with }\bar\partial\phi=\psi\right\},\]
where the equality is taken in the sense of distributions. Similarly, we must define the domain of $\bar\partial^\dagger$:
\[\text{Dom }\bar\partial^\dagger=\left\{ \psi\in L^2(X,L\otimes\Lambda^{p,q+1})\mid \exists \phi\in L^2(X,L\otimes\Lambda^{p,q})\text{ with }\bar\partial^\dagger\psi=\phi \right\}\]
again where the equality is taken in the sense of distributions as well as $\langle\bar\partial\lambda,\psi\rangle=\langle\lambda,\phi\rangle$ for all $\lambda\in\text{Dom }\bar\partial$.
Restricting this extra equation to $\lambda\in C^\infty_0$, we obtain the weaker condition that $\bar\partial^\dagger\psi=\phi$ in the sense of distributions.

Our main goal is to solve $\bar\partial u=f$, where $f\in L^2(X,L\otimes\Lambda^{0,1})$.
The key lemma is as follows.
\begin{lem}
 Let $g_{\bar kj}$ be a K\"ahler metric on $X$ and $h$ a metric on $L$. Assume that the following
inequality holds:
\[||\bar\partial u||^2_{L^2}+||\bar\partial^\dagger u||^2_{L^2}\geq\int_X \langle Au,u\rangle\]
for all $u\in\text{Dom }\bar\partial_1\cap\text{Dom }\bar\partial^\dagger_0$ and $A$ is a positive definite matrix.
Then, for all $f\in L^2(X,L\otimes \Lambda^{0,1})$, with $\bar\partial f=0$ in the sense of distributions, and there exists $v\in L^2(X,L)$
satisfying $\bar\partial v=f$ and $\int_x|v|^2\leq\int_X\langle A^{-1}f,f\rangle$. 
\end{lem}
Here we are distinguishing the two different $\bar\partial$ operators
that are present. More explicitly, we write $u=\sum u_{\bar j}d\bar z^j$ with $u_{\bar j}$ a section of $L$ and
\[\langle Au,u\rangle=A^{l\bar j}u_{\bar j}\overline{u_{\bar l}}h.\]
Then
\[\int_X \langle Au,u\rangle=\int_X A^{l\bar j}u_{\bar j}\overline{u_{\bar l}}h\frac{\omega^n}{n!}\]
Furthermore, $v\in L^2(X,L\otimes \Lambda^{0,0})$ and hence \[\int|v|^2=\int |v|^2h\frac{\omega^n}{n^!}\]
and
\[\int\langle A^{-1}f,f\rangle=\int (A^{-1})^{l\bar j}f_{\bar j}\overline{f_{\bar l}}h\frac{\omega^n}{n!}.\]
Hence we can write the restriction on $v$ more explicitly as
\[\int |v|^2h\frac{\omega^n}{n!}\leq \int (A^{-1})^{l\bar j}f_{\bar j}\overline{f_{\bar l}}h\frac{\omega^n}{n!}.\]

We use the following easy fact from functional analysis.
\begin{lem}
    Let $\phi\in\text{Dom }\bar\partial_0^\dagger$, and let $\phi=\phi_1+\phi_2$ where $\phi_1\in\ker\bar\partial_1$
    and $\phi_2\in(\ker\bar\partial_1)^\perp$. Then $\phi_1,\phi_2\in\text{Dom }\bar\partial_0^\dagger$.
\end{lem}
Note that the decomposition in the lemma makes sense because $\ker\bar\partial_1$ is a closed subspace of $L^2$.
\begin{proof}
    Since $\bar\partial_1\bar\partial_0=0$, it follows that the range of $\bar\partial_0$ is contained in $\ker\bar\partial_1$.
    This implies that if $\phi_2\perp\ker\bar\partial_1$ then $\phi_2\perp(\text{Range }\bar\partial_0)^\perp$, i.e. perpendicular
    to the space $0=\langle\bar\partial_0\psi,\phi\rangle$ for $\psi\in\text{Dom }\bar\partial_0$. This in turn means that
    $\langle\bar\partial_0\psi,\phi_2\rangle=\langle\psi,0\rangle$ for all $\psi\in\text{Dom }\bar\partial$, and thus that
    $\phi_2\in\text{Dom }\bar\partial_0^\dagger$ and $\bar\partial_0^\dagger\phi_2=0$.
\end{proof}

\begin{proof}[Proof of earlier lemma]
    Consider the functional $T$ defined by: $\bar\partial_0^\dagger\phi\mapsto\langle\phi,f\rangle$ for $\phi\in\text{Dom }\bar\partial_0^\dagger$.
    Note this is defined only on a subspace.
    It is not \textit{a priori} obvious that this is well defined - there might be multiple such $\phi$. Decompose $\phi=\phi_1+\phi_2$
    with $\phi_2\in\ker\bar\partial_1,\phi_2\perp\ker\bar\partial_1$. We can write 
    \begin{align*}
        \langle\phi,f\rangle&=\langle\phi_1,f\rangle+\langle\phi_2,f\rangle\\
        &=\langle\phi_1,f\rangle.
    \end{align*}
    By the lemma just proved, we see that $\phi_1\in\text{Dom }\bar\partial_0^\dagger\cap\text{Dom }\bar\partial_1$.
    Using Cauchy-Schwarz with respect to the $L^2$ norm determined by $A$ (it is positive definite), we find
    \begin{align*}
        |\langle\phi_1,f\rangle|^2&\leq\left( \int\langle A\phi_1,\phi_1\rangle \right)\left( \int\langle A^{-1}f,f\rangle\right)\\
        &\leq\left( ||\bar\partial\phi_1||^2+||\bar\partial^\dagger\phi_1||^2 \right)\left( \int\langle A^{-1}f,f\rangle\right)\\
        =||\bar\partial^\dagger\phi_1||^2\int\langle A^{-1}f,f\rangle.
    \end{align*}
    This shows that the functional defined above does indeed make sense, once in the context of the estimates of the hypothesis.
    Now, by the Hahn-Banach theorem stated below, applied to this functional, we obtain a functional $\tilde T:L^2\to \C$
    extending $T$ such that $||\tilde T||=||T||$. This implies that there exists a $u\in L^2$ such that $\tilde T(\psi)=\langle\psi,u\rangle$
    and $||u||=||\tilde T||$ for any $\psi\in L^2$. In particular, take $\psi=\bar\partial_0^\dagger\phi$ with $\phi\in C_0^\infty$.
    Then $\langle\bar\partial_0^\dagger\phi_1,u\rangle=T(\bar\partial_0^\dagger\phi)=\langle\phi,f\rangle$, which is equivalent
    to the fact that $\bar\partial u=f$ in the sense of distributions.
\end{proof}

\begin{thm}[Hahn-Banach]
    Let $V\subset\mathcal{B}$ a Banach space and $T$ be a linear functional $V\ni v\mapsto T(v)$
    with $|T(v)|\leq A||v||$. Then there exists an extension $\tilde T$ of $T$ to the whole of $\mathcal{B}$,
    such that $\tilde T$ satisfies $|T(v)|\leq A||v||$ for all $v\in\mathcal{B}$.
\end{thm}
\begin{proof}
    Omitted.
\end{proof}

When do the estimates in the hypothesis of the lemma hold? Recall the Bochner-Kodaira formula on $C^\infty(X,L\otimes\Lambda^{0,1})$,
\begin{align*}
    \langle\Box\phi,\phi\rangle=||\bar\partial \phi||^2+||\bar\partial^\dagger \phi||^2
\end{align*}
where $\Box=\bar\partial\bar\partial^\dagger+\bar\partial^\dagger\bar\partial$. Acting on smooth forms, we found
\begin{align*}
    \langle\Box\phi,\phi\rangle=||\nabla_{\bar k}\phi_{\bar j}||^2_{L^2}+\int(F_{\bar kj}+R_{\bar kj})\phi^j\overline{\phi^k}h\frac{\omega^n}{n!}.
\end{align*}
Recall that $F_{\bar kj}=-\partial_j\partial_{\bar k}\log h$ and $R_{\bar kj}=R_{\bar kj}P_p=R_{\bar kp}P_j$.
Note that the above lemma will hold if $F_{\bar kj}+R_{\bar kj}>0$ and the inequality required by the lemma extends
from $C_0^\infty$ to $\text{Dom }\bar\partial_0\cap\text{Dom }\bar\partial_1^\dagger$.

Note first that on a compact manifold $C^\infty_0$ is dense in $\text{Dom }\bar\partial_0\cap\text{Dom }\bar\partial_1^\dagger$ with respect
to the norm $||u||_{L^2}+||\bar\partial u||_{L^2}+||\bar\partial^\dagger u||_{L^2}$.

\begin{thm}
    Let $L\to X,h,g_{\bar kj}$ be as above. Writing $h=e^{-\phi}$, assume that
    \begin{align*}
        -\partial_j\partial_k\phi+ R_{\bar kj}\geq\epsilon g_{\bar kj}
    \end{align*}
    for some $\epsilon>0$. Then for any $f\in L^2(X,L\otimes\Lambda^{0,1})$ with $\bar\partial f=0$, there exists
    $u\in L^2(X,L)$ solving $\bar\partial u=f$ and
    \[\int |u|^2e^{-\phi}\frac{\omega^n}{n!}\leq\frac{1}{\epsilon}\int g^{k\bar j}f_{\bar j}\overline{f_k}e^{-\phi}\frac{\omega^n}{n!}.\]
\end{thm}

Note that a useful variation of this setup is to solve the equation $\bar\partial u=f$ for $f\in L^2(X,L\otimes\Lambda^{n,1})$. Why?
Observe that if $u\in L^2(X,L\otimes\Lambda^{n,0})$, then $\int|u|^2_h\equiv\int u\bar ue^{-\phi}$. Note that here there is no volume form
needed, as the integrand is an $n,n$ form. Hence there is no need for a metric, and the estimates tend to be better in this case.
Note also that $L\otimes\Lambda^{n,1}=(L\otimes\Lambda^{n,0})\otimes\Lambda^{0,1}$. But $L'=L\otimes\Lambda^{n,0}$ is simply
another holomorphic line bundle. Hence we can apply our previous theorem with $L$ replaced by $L'$. Thus we impose
$\varepsilon g_{\bar kj}\leq F'_{\bar k}j+R_{\bar kj}=(F_{\bar kj}-R_{\bar kj})+R_{\bar kj}=F_{\bar kj}$ (the curvature of $\Lambda^{n,0}$ is negative
the Ricci curvature). Hence we obtain the theorem above but for this case of $f\in L^2(X,L\otimes\Lambda^{n,1})$.

\begin{exc}
    Write this theorem down carefully and supply the missing steps.
\end{exc}

Observe that this in fact extends to $X$ not compact, but complete as a metric space, i.e.
the estimate in the hypothesis still extends from $C^\infty_0$ to $\text{Dom }\bar\partial_0^\dagger\cap\text{Dom }\bar\partial_1$.
Even further, if $(X,g_{\bar kj})$ is not necessarily complete, but instead there exists a metric $g'_{\bar kj}$ that is K\"ahler
and complete, then one applies the theorem to $(X,g^\delta_{\bar kj}=g_{\bar kj}+\delta g'_{\bar kj}).$ Applying the theorem
now, we obtain a sequence $u_\delta$, uniformly bounded, which weakly converges to a solution $u$ (c.f. J.P. Demailly's online
book).

\todo[inline]{fill in missing lectures}

\subsection{Kodaira embedding}

\begin{thm}[Kodaira embedding]
    Let $L\to X$ be a positive, holomorphic line bundle, over a compact, K\"ahler  manifold $(X,\omega)$. 
    Let $H^0(X,L^k)$ denote the space of holomorphic sections of $L^k$, with $N_{k+1}\equiv\dim H^0(X,L^k)$.
    Let $\left\{ s_\alpha(z) \right\}_{\alpha=0}^{N_{k+1}}$ be a basis for $H^0(X,L^k)$. The Kodaira map is
    defined
    \[\iota_k: X\ni z \mapsto [s_0(z):\cdots:s_{N_k}(z)]\in\CP^{N_k}.\]
    Then there exists a $k_0$ such that for $k\geq k_0$, the map $\iota_k$ is an embedding of $X$
    into $\CP^{N_k}$.
\end{thm}

Let us first sketch the idea of the proof. We claim that this theorem can be reformulated
in terms of sheaves and \v{C}ech cohomology, which we will introduce shortly. Indeed, given $N$ points $z_1,\ldots, z_N\in X$,
with $N$ integers $k_1,\ldots, k_N$, we can define a sheaf $\mathcal{I}_{k_1z_1+\cdots+k_nz_N}$ of functions
vanishing at $z_i$ of order $k_i$, respectively.
Kodaira embedding follows if we can prove that for all $k_i$, there exists a $k_0$ such that the maps
\[\check{H}^0(X,L^k)\to\check{H}^0(X,L^k\otimes\mathcal{O}/\mathcal{I}_{k_1z_1+\cdots+k_Nz_N})\]
are surjective, where $\mathcal{O}$ is the local ring of holomorphic germs. To do this, we will use the fact
that this map sits inside the exact sequence induced by
\begin{equation*}
    \begin{tikzcd}
        0\arrow{r}&L^k\otimes\mathcal{I}\arrow{r}&L^k\otimes\mathcal{O}\arrow{r}&L^k\otimes\mathcal{O}/\mathcal{I}\arrow{r}&0,
    \end{tikzcd}
\end{equation*}
which is a long exact sequence
\begin{equation*}
    \begin{tikzcd}
        0\arrow{r}&\check H^0(X,L^k\otimes\mathcal{I})\arrow{r}&\check H^0(X,L^k\otimes \mathcal{O})\arrow{r}&\check H^0(X,L^k\otimes\mathcal{O}/\mathcal{I})\arrow{d}&\\
        &&&\check H^1(X,L^k\otimes \mathcal{I})\arrow{r}&\cdots
    \end{tikzcd}
\end{equation*}
Hence it suffices to show, by exactness, that
\[\check H^1(X,L^k\otimes\mathcal{I}_{k_1z_1+\cdots+k_Nz_N})=0.\]

We shall show that there exists a $\phi$ plurisubharmonic such that for any $k_1,\ldots,k_N$, there exists a $k$ depending
only on $k_1,\ldots,k_N$, such that $\mathcal{J}_{\phi}\subset\mathcal{I}_{k_1z_1+\cdots+k_Nz_N}$ and $e^{-\phi}$ is a singular
metric on $L^k\otimes\Lambda^{n,0}$ with $i\partial\bar\partial\phi\geq\varepsilon\omega$.
Note that $\mathcal{J}_\phi$ is the sheaf of holomorphic functions $f$ near $z$ such that $\int_{U_z}|f|^2e^{-\phi}<\infty$.
This implies that
\[H^1_{\bar\partial}(X,L^k\otimes\Lambda^{n,0}\otimes\mathcal{J}_\phi)=0,\]
where this cohomology group is defined to be $\ker \bar\partial|_{L^k\otimes\Lambda^{n,1}\otimes\mathcal{J}_\phi}/\text{im }\bar\partial|_{L^k\otimes\Lambda^{n,0}\otimes\mathcal{J}_\phi}$.
But this is essentially just H\"ormander's theorem. To complete the proof, we will use the fact that
\[H^1_{\bar\partial}(X,L^k\otimes \Lambda^{n,0}\otimes\mathcal{J}_\phi)=\check H^1(X,L^k\otimes\Lambda^{n,0}\otimes\mathcal{J}_\phi),\]
which one might think of as a version of the de Rham theorem.

Let us now consider the basics of \v{C}ech cohomology. Let $X$ be a topological space with an open cover $\underline{U}=\{U_\alpha\}$.
A \text{sheaf} $\mathcal{E}$ on $X$ is an assignment of a group $\Gamma(U_\alpha,\mathcal{E})$ to each $U_\alpha$, with restriction
maps $\rho_{VU}:\Gamma(V,\mathcal{E})\to\Gamma(U,\mathcal{E})$ for $U\subset V$, with the obvious consistency conditions, as well as the
extension property that for $s_1\in \Gamma(U_1,\xi),s_2\in\Gamma(U_2,\xi)$ with $\rho_{U_1,U_1\cap U_2}s_1=\rho_{U_2,U_1\cap U_2}s_2$
then there exists a unique $t\in\Gamma(U_1\cap U_2,\xi)$ with $\rho_{(U_1\cup U_2),U_1}s_1,\rho_{U_1\cup U_2,U_2}s_2$.

Let us look at some examples of sheaves. The simplest is the constant sheaf, say for example $\mathcal{S}=\underline{\Z}$,
where $\Gamma(U,\mathcal{S})$ is the group of integer-valued continuous (constant) functions on $U$, or $\mathcal{S}=\underline{R}$
where $\Gamma(U,\mathcal{R})$ is the group of smooth real-valued functions on $U$. A more sophisticated example is as follows.
Take a subvariety $V\subset X$. One can define a sheaf $\mathcal{V}$ where $\Gamma(U,\mathcal{V})$ is the group of holomorphic
functions on $U$ vanishing on $V$. Another example is $\mathcal{J}_\phi$, where $\Gamma(U,\mathcal{J}_\phi)$ is the group
of holomorphic functions $f$ on $U$ such that $\int_U |f|^2e^{-\phi}<\infty$.

Let us now turn to cohomology groups. Given a sheaf $\mathcal{E}$, we define the space of $p$-cycles as
\[\mathcal{C}^p=\left\{ \sigma\mid U_{\alpha_0},\ldots,U_{\alpha_p}\to \sigma_{\alpha_0,\ldots,\alpha_p}\in\Gamma(U_{\alpha_0}\cap\cdots\cap U_{\alpha_p},\mathcal{E}) \right\}\]
and the $\delta$ operator $\delta:\mathcal{C}^p\to\mathcal{C}^{p+1}$ such that
\[(\delta\sigma)_{\alpha_0,\ldots,\alpha_{p+1}}=\sum(-1)^j\sigma_{\alpha_0,\ldots,\hat\alpha_j,\alpha_{p+1}}\in\Gamma(U_{\alpha_0}\cap\cdots\cap U_{\alpha_p},\mathcal{E})\]
It is easy to check that $\delta^2=0$. Hence we define
\[\check H^p(\underline{U},\mathcal{E})\equiv \ker\delta|_{\mathcal{}^p}/\text{im }\delta|_{\mathcal{C}^{p+1}}.\]
Now let us see how these quantities depend on the choice of open covering. Let $\underline{U}=\{U_\alpha\}$ and $\underline{U}'=\{U_\beta'\}$ be
two coverings of $X$. We say that $\underline{U}$ is a refinement of $\underline{U}'$ if there exists a $\phi:\{\alpha\}\to\{\beta\}$ with
$U_\alpha\subset U'_{\phi(\alpha)}$. Then we have a map on cocycles
\[(\phi\sigma)_{\alpha_0,\ldots,\alpha_p}\equiv\sigma_{\phi(\alpha_0),\ldots,\phi(\alpha_p)}\in\Gamma(\underline{U}',\mathcal{E}),\]
and thus a map $\Gamma(\underline{U}',\mathcal{E})\ni\sigma\to \phi(\sigma)\in\Gamma(\underline{U},\mathcal{E})$. This map in turn
induces a map on cohomology groups $\check H^p(\underline{U}',\mathcal{E})\to\check H^p(\underline{U},\mathcal{E})$.
We can now define
\[\check H^p(X,\mathcal{E})=\lim_{j\to\infty}\check H^p(\underline{U}_j,\mathcal{E})=\cup_j\check H^p(\underline{U}_j,\mathcal{E})/\sim\]
where $\underline{U}_{j+1}\subset\underline{U}_j$ is a sequence of refinements, and the equivalence relation is given as $\sigma\in \check H^p(\underline{U}_j,\mathcal{E})$
and $\tau\in \check H^p(\underline{U}_k,\mathcal{E})$ are equivalent if there exists an $m>j,k$ with $\phi_{jm}\sigma=\phi_{km}\tau$.

\begin{lem}[Leray]
    If $\underline{U}$ is a covering with $\check H^p(U_\alpha\cap\cdots\cap U_{\alpha_p},\mathcal{E})=0$ for all $p\geq 1$,
    then $\check H^p(X,\mathcal{E})=\check H^p(\underline{U},\mathcal{E})$.
\end{lem}

This practical lemma allows us to bypass working with the complicated inductive limit defined above, and instead work
with such coverings $\underline{U}$.

Let us now turn to some examples of cohomology groups.
\begin{exmp}
    By definition, $\check H^0(X,\mathcal{E})=\{\sigma\in\mathcal{C}^0, \delta\sigma=0\}$. What does this condition mean?
    Well clearly $\sigma:U_\alpha\to \Gamma(U_\alpha,\mathcal{E})$. Meanwhile,
    \[\mathcal{C}^1\ni (\delta\sigma)_{\alpha\beta}=\sigma_\beta-\sigma_\alpha,\]
    and hence for this to be zero means that $\sigma_\beta=\sigma_\alpha$ on $U_\alpha\cap U_\beta$.
    Of course, by the sheaf axiom, this implies that there exists a $\sigma\in\Gamma(U_\alpha\cup U_\beta,\mathcal{E})$
    that restricts to the $\sigma_\alpha,\sigma_\beta$ appropriately. Hence we find that
    \[\check H^0(X,\mathcal{E})=\Gamma(X,\mathcal{E})\].

    As another example, consider the Cousin problem. Fix a domain $\Sigma$ in some Riemann surface $X$.  Given a sequence
    of points $\{p_j\}\subset\Omega$, does there exist a meromorphic function with a pole of order $n_j$
    at $p_j$? This is certainly doable on $U_j$, where $U_j$ are small neigbhorhoods of the $p_j$, by the charts
    that transport $U_j$ to a disk in the complex plane. Hence this problem is trivially locally, but the
    problem of matching these $U_j$ is not so easy. It is not surprising that this problem can be rephrased
    as a question of cohomology. Consider $\sigma\in\mathcal{C}^1(X,\mathcal{O})$, where $\mathcal{O}$ is the
    sheaf of holomorphic functions. Indeed $\sigma_{jk}\equiv f_j-f_k$ on $U_j\cap U_k$ is now holomorphic
    as the poles have cancelled. Observe now that if $\check H^1(X,\mathcal{O})=0$, then a global function exists
    with the given properties because this implies that $\sigma_{jk}=(\delta\tau)$ for $\tau\in\mathcal{C}^0(X,\mathcal{O})$.
    But this means that $\sigma_{jk}=\tau_j-\tau_k$, i.e. $f_j-f_k=\tau_j-\tau_k$ and hence $f_j-\tau_j=f_k-\tau_k$,
    which gives us the global meromorphic function, as these match.

    Consider now the second Cousin problem. Given the same setup as before, does there exist a holomorphic function
    with the given orders of vanishing at the $p_j$? We leave it as an exercise to show that this is possible
    if $\check H^1(X,\mathcal{O}^*)=0$, where $\Gamma(U,\mathcal{O}^*)$ are the non-vanishing holomorphic functions
    on $U$. As another exercise, we claim that an element $\sigma\in\mathcal{C}^1(X,\mathcal{O}^*)$ gives rise
    to a line bundle $L\to X$.\footnote{The way to see this is to note that $\sigma$ associates to $U_\alpha\cap U_\beta$
        a section $\sigma_{\alpha\beta}\in\Gamma(U_\alpha\cap U_\beta,\mathcal{O}^*)$, which is multiplicative and hence
        $0=\delta\sigma\iff\sigma_{\alpha\beta}=\sigma_{\alpha\gamma}\sigma_{\gamma\beta}$: the cocycle condition.
        This implies that $\check H^1(X,\mathcal{O}^*)$ is the space of line bundles $L$ with transition functions $\sigma_{\alpha\beta}$
        modulo $\sigma_{\alpha\beta}\sim\sigma_{\alpha\beta}'$ if $\sigma_{\alpha\beta=\sigma_{\alpha\beta}'}f_\alpha f_\beta^-1$.
        From the point of view of line bundles, this makes sense because on one line bundle $L$, $\phi_\alpha=\sigma_{\alpha\beta}\phi_\beta$,
        while on the other $\phi_\alpha'=\sigma_{\alpha\beta}'\phi'_\beta$, between which there is a natural correspondence because
        we can write $(f_\alpha^{-1}\phi_\alpha)=\sigma'_{\alpha\beta}(f_\beta^{-1}\phi_\beta)$, which gives us a mapping
        $\Gamma(X,L)\ni\phi\mapsto\phi'\equiv f_\alpha^{-1}\phi_\alpha\in\Gamma(X,L')$. This one-to-one mapping allows
        us to think of the line bundles as equivalent, as $f_\alpha$ are holomorphic and non-vanishing.}
\end{exmp}
The moral of these examples is that global problems can be rephrased in terms of cohomology.

Now let us turn to short exact sequences of sheaves. Suppose we have sheaves $\mathcal{E},\mathcal{F},\mathcal{G}$
and maps
\begin{equation*}
    \begin{tikzcd}
        0\arrow{r}&\mathcal{E}\arrow{r}{\Phi}&\mathcal{F}\arrow{r}{\Psi}&\mathcal{G}\arrow{r}&0
    \end{tikzcd}
\end{equation*}
which are exact as maps of sheaves. By this we mean that for each $z$,
the sequence
\begin{equation*}
    \begin{tikzcd}
        0\arrow{r}&\mathcal{E}_z\arrow{r}&\mathcal{F}_z\arrow{r}&\mathcal{G}_z\arrow{r}&0
    \end{tikzcd}
\end{equation*}
is exact as a sequence of groups. Then we obtain the following long exact sequence.
\begin{equation*}
    \begin{tikzcd}
        0\arrow{r}&\check H^0(X,\mathcal{E})\arrow{r}&\check H^0(X,\mathcal{F})\arrow{r}&\check H^0(X,\mathcal{G})\arrow{d}{\delta*}&\\
        &&&\check H^1(X,\mathcal{E})\arrow{r}&\cdots
    \end{tikzcd}
\end{equation*}

Let us now try to construct $\delta^*$, the \textbf{coboundary operator}.
Consider first some $\sigma\in\mathcal{C}^p(X,\mathcal{G})$, i.e. $\delta\sigma=0$.
There exists, for each stalk, a $\tau\in \mathcal{C}^p(X,\mathcal{F})$ with $\Psi(\tau)=\sigma$,
by surjectivity of $\Psi$. Note now that $\delta\tau=\Phi(\lambda)$ with $\lambda\in\mathcal{C}^{p+1}(X,\mathcal{E})$,
because $\Psi(\delta\tau)=\delta\Psi(\tau)=\delta\sigma=0$, and hence belongs to the kernel of $\Psi$ and
hence lies in the image of $\Phi$. Indeed, $\lambda$ is closed, as $\Phi(\delta\lambda)=\delta\Phi(\lambda)=\delta^2\tau=0$,
which implies that $\delta\lambda=0$, as $\Phi$ is injective. Hence we obtain a map
\[\mathcal{C}^p(X,\mathcal{G})\ni\sigma\mapsto\lambda\in\mathcal{C}^{p+1}(X,\mathcal{E}),\]
which yields the $\delta^*$ operator.

We want to equate the de Rham/Dolbeaut cohomology groups with \v{C}ech cohomology groups, as the former
can be computed with methods discussed in previous lectures. Note the following important fact:
if $\mathcal{E}$ is a \textbf{fine} sheaf, then $\check H^p(X,\mathcal{E})=0$ for $p\geq 1$. In particular,
a fine sheaf is a sheaf such that if $\sum\chi_\alpha=1$ is a partition of unity (with $\supp\chi_\alpha\subset U_\alpha$),
then there is a map
\[\Gamma(U_\alpha,\mathcal{E})\ni s\mapsto \chi_\alpha s\in\Gamma(U_\alpha,\mathcal{E}).\]
In other words, fineness means that there is a sensical way of using partitions of unity,
compatible with the sheaf. As an example, consider the sheaf of smooth $p$-forms -- this is clearly fine
as we can multiply by smooth bump functions. As a non-example, simply consider something valued in the integers,
for which one can obviously not do the same. Similarly for sheafs of holomorphic functions, as
bump functions will destroy holomorphicity (recall bump functions are smooth but not analytic).
The above observation can be proved fairly simply: let $\sigma\in\mathcal{C}^p(X,\mathcal{E})$, i.e. $\delta\sigma=0$.
Then $\sigma=\delta\tau$, and $\tau_{\alpha_0,\ldots,\alpha_{p-1}}=\sum_\gamma\chi_\gamma\sigma_{\gamma_0,\ldots,\gamma_p-1}$
for $\tau\in\mathcal{C}^{p-1}$. We leave the explicit details as an exercise.
As it happens, the sheaves that we will be interested in will not be fine, but we will place them in exact sequences
with other sheaves that are fine.

Let us illustrate explicitly why fine sheafs have trivial higher cohomology. Let $\sigma\in\mathcal{C}^q$ (i.e. $\delta\sigma=0$.
We want to show that $\sigma=\delta\tau$. But one can simply take
\[\tau_{\alpha_0,\ldots,\alpha_{q-1}}\equiv\sum\chi_\gamma\sigma_{\gamma;\alpha_0,\ldots,\alpha_{q-1}}.\]
To see why this works, let us just look at the example where $q=1$: take $\mathcal{C}^1\ni\sigma=(\sigma_{\alpha\beta}).$
Then $\tau_\alpha=\sum\chi_\gamma\sigma_{\gamma\alpha}$. Then
\[(\delta\tau)_{\alpha\beta}=\tau_\beta-\tau_\alpha=\sum\chi_\gamma(\sigma_{\gamma\beta}-\sigma_{\gamma\alpha})=\sum\chi_\gamma\sigma_{\alpha\beta}=\sigma_{\alpha\beta},\]
since $\sigma_{\gamma\beta}=\sigma_{\gamma\alpha}+\sigma_{\alpha\beta}$. This proves the claim, and for higher $q$ there
are more indices, but the idea is the same.

\begin{thm}[de Rham]
    Let $X$ be a smooth, connected manifold. Then
    \[\check H^q(X,\R)=H_{\rm dR}^q(X)\equiv\ker d|_{\Lambda^q}/d\Lambda^{q-1}.\]
\end{thm}

\begin{proof}
    Consider the following exact sequence of sheaves:
    \begin{equation}
        \label{eq:les}
        \begin{tikzcd}
            0\arrow{r}&\R\arrow{r}&\Lambda^0\arrow{r}&\cdots\arrow{r}&\Lambda^{q-1}\arrow{r}{d}&\Lambda^q\arrow{r}{d}&\Lambda^{q+1}\arrow{r}&\cdots,
        \end{tikzcd}
    \end{equation}
    where $\R$ is the constant sheaf (we are treating $\Lambda_z^q$ as the germs of $q$-forms at $z$, as the $d$ map descends naturally
    to the stalks and $\R$ as the constant sheaf). Observe that exactness means that for $\sigma\in\Lambda_z^q$ and $d\sigma=0$
    implies that there exists a $\tau\in\Lambda_z^{q-1}$ with $\sigma=d\tau$. This is of course a local condition, and hence by
    the Poincar\'e Lemma, this sequence makes sense as locally we can always write a form as being exact.

    Note now that
    all of the $\Lambda^q$ sheaves are fine (though the constant sheaf $\R$ is not). It is useful to rewrite this in terms
    of short exact sequences, because we can then talk about the induced long exact sequence (as mentioned earlier).
    Introduce the sheaf $\mathcal{Z}^q$ where $\mathcal{Z}^q_z$ is the space of germs of $q$-forms $\sigma$ with $d\sigma=0$.
    Observe that $\mathcal{Z}^q$ is not fine, as $d(\chi\sigma)\neq 0$. We claim that Eq.~(\ref{eq:les}) is equivalent to
    \begin{equation}
        \begin{tikzcd}
            0\arrow{r}&\R\arrow{r}&\Lambda^0\arrow{r}{d}&\mathcal{Z}^1\arrow{r}&0\\
            0\arrow{r}&\mathcal{Z}^q\arrow{r}&\Lambda^q\arrow{r}{d}&\mathcal{Z}^{q+1}\arrow{r}&0
        \end{tikzcd}
        \label{eq:2ses}
    \end{equation}
    We can apply the de Rham theorem to get the following exact sequences:
    \begin{equation}
        \begin{tikzcd}
            0\arrow{r}&\check H^0(X,\R)\arrow{r}&\check H^0(X,\Lambda^0)\arrow{r}&\check H^0(X,\mathcal{Z}^1)\arrow{d}\\
            &&&\check H^1(X,\R)\arrow{r}&\check H^1(X,\Lambda^1)\arrow{r}&\cdots
        \end{tikzcd}
        \label{eq:les2}
    \end{equation}
    But since $\Lambda^i$ are fine, we see that all but the first higher cohomologies vanish,
    and hence we find that
    \[\check H^1(X,\R)=\check H^0(X,\mathcal{Z}^1)/d\check H^0(X,\Lambda^0)=H^1_{\rm dR}(X).\]
    This proves the theorem for $q=1$.
    
    It is straightforward to see that similarly,
    \[\check H^q(X,\mathcal{Z}^1)=\check H^{q+1}(x,\R).\]
    Next we use the second short exact sequence from Eq.~(\ref{eq:2ses}) to get
    \begin{equation*}
        \begin{tikzcd}
            0\arrow{r}&\check H^0(X,\mathcal{Z}^q)\arrow{r}&\check H^0(X,\Lambda^q)\arrow{r}&\check H^0(X,\mathcal{Z}^{q+1})\arrow{d}{\delta^*}\\
            &&&\check H^1(X,\mathcal{Z}^q)\arrow{r}&\check H^1(X,\Lambda^q)\arrow{r}&\cdots
        \end{tikzcd}
    \end{equation*}
    which yields that
    \[\check H^1(X,\mathcal{Z}^q)=\check H^0(X,\mathcal{Z}^{q+1})/d\check H^0(X,\Lambda^q)=H^{q+1}_{\rm dR}(X).\]
    Finally, we wish to show that $\check H^1(X,\mathcal{Z}^q)=\check H^q(X,\mathcal{Z}^1)$.
    By looking at a later piece of this second long exact sequence, we see
    \begin{equation*}
        \begin{tikzcd}
            \cdots\arrow{r}&\check H^p(X,\Lambda^q)\arrow{r}&\check H^p(X,\mathcal{Z}^{q+1})\arrow{d}\\
            &&\check H^{p+1}(X,\mathcal{Z}^q)\arrow{r}&\check H^{p+1}(X,\Lambda^p)\arrow{r}&\cdots
        \end{tikzcd}
    \end{equation*}
    which implies that $\check H^p(X,\mathcal{Z}^{q+1})=\check H^{p+q}(X,\mathcal{Z}^q)$. This proves the theorem.
\end{proof}

\begin{thm}[Dolbeault]
    Let $X$ be a complex, connected manifold. Then
    \[\check H^q(X,\Omega^p)=H_{\bar\partial}^{p,q}(X),\]
    where $H_{\bar\partial}^{p,q}=\ker \bar\partial|_{\Lambda^{p,q}}/\bar\partial\Lambda^{p,q-1}$ is the 
    $(p,q)$ Dolbeault cohomology group, and $\Omega^p$ is the sheaf with stalks $\Omega^p_z$ germs of holomorphic
    $(p,0)$ forms near $z$.
\end{thm}
\begin{proof}
    The proof follows exactly as in the de Rham case, starting with the complex
    \begin{equation*}
        \begin{tikzcd}
            0\arrow{r}&\Omega^p\arrow{r}&\Lambda^{p,0}\arrow{r}&\Lambda^{p,1}\arrow{r}&\cdots\arrow{r}&\Lambda^{p,q-1}\arrow{r}&\Lambda^{p,q}\arrow{r}{\bar\partial}&\Lambda^{p,q+1}\arrow{r}&\cdots
        \end{tikzcd}
    \end{equation*}
    which makes sense because of the $\bar\partial$-Poincar\'e lemma. We immediately find via methods of the last proof
    that $\check H^q(X,\Omega^p)$ is equal to the cohomology group
    of the global sections, which is $H^{p,q}_{\bar\partial}(X)$.
\end{proof}

Let us now extend these ideas to multiplier ideal sheaves. Let $L\to X$ be a holomorphic line bundle over $(X,\omega)$ K\"ahler
and let $h\equiv e^{-\phi}$ be a metric on $L$. Define now the \textbf{multiplier ideal sheaf} $\mathcal{I}_{\phi}$ whose stalks $\mathcal{I}_\phi(z)$ are
the germs of holomorphic functions near $z$ satisfying $\int_U|f(z)|^2e^{-\phi}<\infty$ (recall that the weight $h$ that we have
in mind is singular, and hence this is a nontrivial condition).
Define $\mathcal{L}$ to the sheaf whose stalks $\mathcal{L}_z$ are the germs of sections $\alpha$ of $L\otimes \Lambda^{n,q}$
satisfying $\int_U|\alpha|^2e^{-\phi}<\infty$ and $\bar\partial\alpha$ (in the sense of distributions) is a function satisfying
$\int_U|\bar\partial\alpha|^2e^{-\phi}<\infty$. Note that here $n$ is fixed for techinical reasons that make it easier in respect
to H\"ormander's theorem.
Consider now the following sequence of sheaves (it is non-trivial to see that it is exact, as this requires H\"ormander's theorem).
\begin{equation}
    \begin{tikzcd}
        0\arrow{r}&L\otimes\Lambda^{n,0}\otimes\mathcal{I}_\phi\arrow{r}&\mathcal{L}^0\arrow{r}&\cdots\arrow{r}&\mathcal{L}^{q-1}\arrow{r}{\bar\partial}&\mathcal{L}^{q}\arrow{r}{\bar\partial}&\mathcal{L}^{q+1}\arrow{r}&\cdots
    \end{tikzcd}
    \label{les3}
\end{equation}
We now apply the previous machinery, which tells us that a resolution by fine sheaves yields that the \v{C}ech cohomology
is equal to the cohomology of global sections, i.e.
\begin{equation}
    \check H^q(X,L\otimes\Lambda^{n,0}\otimes\mathcal{I}_\phi)=\ker\bar\partial|_{\mathcal{L}^q}/d\mathcal{L}^{q-1}.
    \label{eq:nadel}
\end{equation}

\begin{thm}[Nadel vanishing (1988)]
    Let $L\to X$, $(X,\omega)$ K\"ahler as before, and assume that $h$ satisfies $i\partial\bar\partial\phi\geq\varepsilon\omega$,
    for some $\varepsilon>0$. Then $\check H^1(X,L\otimes \Lambda^{n,0}\otimes\mathcal{I}_\phi)=0$.
\end{thm}
\begin{proof}
    Apply H\"ormander's theorem twice: first to small opens in $X$ to show that the sequence above is exact
    (it plays the role of the Poincar\'e lemma), and then again to show that the right-hand side of Eq.~(\ref{eq:nadel})
    is zero.
\end{proof}

We now have enough machinery to prove the imbedding theorem.\todo{restate it here}

\begin{proof}[Proof of Kodaira imbedding]
     It suffices to show that for all $N$, there exist $k$ and constants $a,b$ such that
     \[H^0(X,L^k)\to H^0(X,L^k\otimes\mathcal{O}/\mathcal{I}_{k_1z_1,\cdots,k_Nz_N})\]
     is onto for any $z_1,\ldots,z_N\in X$, as long as 
     \[k\geq a\sum_{i=1}^Nk_i+b.\]

     Let us review a bit. Note that the stalk at $w$, $\mathcal{I}_{z_1}(w)$, is the space of $f$ holomorphic in a neigbhorhood of $w$, with $f(z_1)=0$.
     Meanwhile, the stalk $\mathcal{O}(w)$ is just the holomorphic functions near $w$. Thus the stalks at the quotient sheaf not at
     $z_1$ are trivial (as $\mathcal{I}_{z_1}$ and $\mathcal{O}$ match up no at $z_1$), while the stalk at $z_1$ is simply 
     the values of $f$ at $z_1$, i.e. $\C$, as we can write $f(z)=f(z_1)+(z-z_1)g(z)$. Indeed, $\mathcal{I}_{z_1}/\mathcal{O}$
     is thus known as the skyscraper sheaf. Now, what does it mean for
     \[H^0(X,L^k)\to H^0(X,L^k\otimes\mathcal{O}/\mathcal{I}_{z_1})\]
     to be surjective? This simply means for any $\tau\in L^k_{z_1}$, there exists a section $s\in H^0(X,L^k)$ with
     $s(z_1)=\tau$. More generally, if one has that
     \[H^0(X,L^k)\to H^0(X,L^k\otimes \mathcal{O}/\mathcal{I}_{z_1,\cdots,z_N}),\]
     is surjective, where the quotient sheaf is simply a multiple-skyscraper sheaf, then for all $\tau_1,\ldots,\tau_N$,
     for $\tau_j\in L^k_{z_j}$, there exists a section $s\in H^0(X,L^k)$ such that $s(z_j)=t_j$.

     So why does the imbedding follow from the claim above?
     Note first that for any $z\in X$, not all $s_\alpha(z)$ can vanish. Otherwise, for all $s\in H^0(X,L^k)$,
     $s(z)=\sum_{\alpha=0}^Nc_\alpha s_\alpha(z)$ would imply that $s(z)=0$. But this contradicts that
     $H^0(X,L^k)\to H^0(X,L^k\otimes\mathcal{O}/\mathcal{I}_{z_1})$ is onto.
     Furthermore, the map is injective: let $z_1,z_2\in X$ with $\iota_k(z_1)=\iota_k(z_2)$, i.e.
     $s_\alpha(z_1)s_\beta(z_2)=s_\alpha(z_2)s_\beta(z_1)$ for any $\alpha,\beta$. Now if we take
     $s(z)=\sum c_\alpha s_\alpha(z)$ and $t(z)=\sum d_\beta s_\beta(z)$, we find that
     $s(z_1)s_\beta(z_2)=s(z_2)s_\beta(z_1)$, which implies that $s(z_1)t(z_2)=s(z_2)t(z_2)$.
     This contradicts the fact that $H^0(X,L^k)\to H^0(X,L^k\otimes\mathcal{O}/\mathcal{I}_{z_1,z_2})$
     is surjective (as we can choose sections such that the constraint above is not satisfied).
     Let us now check that the differential is one-to-one. Note that
     \[\delta z\mapsto \left( \sum_{j=1}^n\frac{\partial s_0}{\partial z_j}\delta z^j,\cdots,\sum_{j=1}^n\frac{\partial s_N}{\partial z_j}\partial z^j \right),\]
     but in projective space we have to quotient by the span of the vector $(s_0(z),\ldots,s_N(z))$.
     But the above variation gives zero after this quotient at some point $z_1$ if
     \[\sum_{j=1}^n\frac{\partial s_\alpha(z_1)}{\partial z^j}\delta z^j=\lambda s_\alpha(z_1).\]
     But this implies that for any $s\in H^0(X,L^k)$,
     \[\sum_{j=1}^n\frac{\partial s(z_1)}{\partial z^j}\delta z^j=\lambda s(z_1),\]
     which contradicts that $H^0(X,L^k)\to H^0(X,L^k\otimes \mathcal{O}/\mathcal{I}_{2z_1})$ is surjective.


     To prove this statement, we use Nadel's vanishing theorem above, which tells us that
     \[\check H^1(X,K_X\otimes\tilde L\otimes \mathcal{I}_{\phi})\]
     where $\tilde L\to(X,\omega)$ is a line bundle admitting a possibly singular metric
     such that $-\partial_j\partial_{\bar k}\log h\geq g_{\bar kj}$, and $\mathcal{I}_\phi(w)$
     is the space of $f$ holomorphic in a neighborhood of $w$ with $\int_U|f|^2e^{-\phi}<\infty$,
     the stalk of the multiplier ideal sheaf.
     Note that we have the exact sequences
     \begin{equation*}
         \begin{tikzcd}
             0\arrow{r}&\mathcal{I}_\phi\arrow{r}&\mathcal{O}\arrow{r}&\mathcal{O}/\mathcal{I}_{\phi}\arrow{r}&0\\
             0\arrow{r}&L^k\otimes\mathcal{I}_\phi\arrow{r}&L^k\otimes\mathcal{O}\arrow{r}&L^k\otimes\mathcal{O}/\mathcal{I}_{\phi}\arrow{r}&0\\
         \end{tikzcd}
     \end{equation*}
     This implies the long exact sequence
     \begin{equation*}
         \begin{tikzcd}
             H^0(X,L^k\otimes\mathcal{O})\arrow{r}&H^0(X,L^k\otimes\mathcal{O}/\mathcal{I}_\phi\arrow{d}&&\\
             &H^1(X,L^k\otimes \mathcal{I}_\phi)\arrow{r}&\cdots
         \end{tikzcd}
     \end{equation*}
     We want to apply Nadel's theorem, with $K_X\otimes\tilde L=L^k$ with $\tilde L=K^{-1}_X\otimes L^k$
     to show surjectivity. Hence, set $\tilde L=K_X^{-1}\otimes L^k$ -- we shall construct a metric $\tilde h$
     on $\tilde L$ such that $-i\partial\bar\partial\log\tilde h\geq\varepsilon\omega$ and $\mathcal{I}_\phi$
     is a subsheaf of $\mathcal{I}_{k_1z_1,\ldots,k_Nz_N}$. If this is possible, then we find that
     $H^0(X,L^k)\to H^0(X,L^k\otimes\mathcal{O}/\mathcal{I}_\phi)$ is surjective and that
     $\mathcal{O}/\mathcal{I}_\phi\to\mathcal{O}/\mathcal{I}_{k_1z_1,\cdots,k_Nz_N}$.

     By hypothesis, $L$ is a positive line bundle, and hence there exists a smooth metric $h_0$ on $L$, whose
     curvature $\omega_0=-i/2\partial\bar\partial\log h_0>0$. Equip $X$ with the K\"ahler metric $\omega_0$,
     i.e. a norm on $T^{1,0}(X)$. Such a metric induces a metric on $K_X^{-1}$, which is just the maximal wedge
     product of $T^{1,0}(X)$. Introduce a metric $h$ on $L^k$ by $h\equiv h_0^ke^{-\phi}$, where we will specify $\phi$.
     For simplicity, let $N=1$, as the general case is similar. Let $\chi(z)$ be a cutoff function
     near $z_1$, and then let
     \[\phi=\gamma\chi(z)\log|z-z_1|^2,\]
     where $\gamma$ is some large constant. We can compute the curvature of $h$ to be
     \begin{align*}
         -i\partial\bar\partial\log h&=-ik\partial\bar\partial\log h_0+i\partial\bar\partial\left( \gamma\chi(z)\log|z-z_1|^2 \right)\\
         &=-ik\partial\bar\partial\log_h+\gamma\chi(z)\left( i\partial\bar\partial\log|z-z_1|^2 \right)\\
         &\geq k\omega_0/2
     \end{align*}
     where the derivatives of $\chi$ vanish, as it is identically 1 on a neighborhood, and we have noted
     that the second term in the second equation is bounded, and hence we can pick $k$ large enough to dominate.
     Now Let $\tilde h$ be the metric on $K_X^{-1}\otimes L^k$ induced by the metric $K^{-1}_X$ and the metric
     $h$ on $L^k$. Hence we find that
     \begin{align*}
        -i\partial\bar\partial\log\tilde h&=(-\partial\bar\partial\log\omega_0^n)-i\partial\bar\partial\log h\\
        &\geq k\omega_0/4
     \end{align*}
     for $k\gg 1$. Finally, recall from earlier work that for $\gamma$ large enough, $\mathcal{I}_\phi$ is a subsheaf
     of $\mathcal{I}_{k_1z_1}$. This completes the proof.
\end{proof}

\subsection{Canonical metrics and connections on vector bundles}

As far as we know today, the laws of physics are, at the fundamental level, very geometric in nature.
On the other hand, in geometry, there is an important problem of finding geometric structures on manifolds
that are canonical in some sense. The following specific problems that we will focus on bring these
two problems together:
\begin{enumerate}[(a)]
    \item Given a vector bundle $E\to X$ a smooth vector bundle over $(X,g_{ij})$ smooth, and a functional
        \[I(A)=\int_X |F(A)|^2\sqrt{g}dx\]
        on the (curvature of the) connections on $E$. One may attempt to, in analogy to the principal of least action,
        attempt to find the critical points of $I(A)$. Of particular interest, are the anti-self-dual connections
        when $\dim X=4$. For certain choices of the functional, we obtain the \textbf{Yang-Mills equations}.
    \item In the case where $X$ as above is complex, and $\dim_\C X=2$, we find that the anti-self-dual connections
        equip $E$ with a holomorphic vector bundle structure carrying a Hermitian-Einstein metric.
    \item There is a theorem (Donaldson, Uhlenbeck-Yau) that states that given $E\to (X,g_{\bar k,j})$ K\"ahler,
        there exists a Hermitian-Einstein metric if and only if $E$ is stable in the sense of Mumford-Takemoto.
\end{enumerate}

Let us start by studying the global topology of smooth vector bundles. Given an open cover $X=\cup_\alpha X_\alpha$,
we define a vector bundle $E$ of rank $r$ to be the data of $t_{\mu\nu\beta}^\alpha(x)$ smooth functions on $X\mu\cap X_\nu$
with $1\leq\alpha,\beta\geq r$ satisfying the standard compatibility conditions. As usual, a section $\phi\in\Gamma(X,E)$
is a vector-valued function $\phi^\alpha_\mu(x)$ on $X_\mu$ for $1\leq\alpha\leq r$ that transforms as dictated by $E$.
We now define a connection $\nabla_j$ as a rule of the of the form
\[\Gamma(X,E)\ni\phi^\alpha\mapsto\nabla_j\phi^\alpha_\mu=\partial_j\phi^\alpha_\mu+(A_\mu)_{j\beta}^\alpha(x)\phi^\beta_mu\]
on $X_\mu$ which satisfies the requirement that
\begin{align*}
    (\nabla_j\phi_\mu^\alpha)&=t_{\mu\nu\beta}^\alpha(\nabla_j\phi_\nu^\beta),
\end{align*}
which, as is easy to check, is equivalent to
\[(\partial_jt_{\mu\nu\beta}^\alpha+(A_\mu)^\alpha_{j\gamma}t_{\mu\nu\beta}^\gamma=t_{\mu\nu\gamma}^\alpha(A_\nu)^\gamma_{j\beta}.\]
In matrix notation, we might write this as $\partial_jt_{\mu\nu}+(A_\mu)_jt_{\mu\nu}=t_{\mu\nu}(A_\nu)_j$, i.e.
\[(A_\mu)_j=t_{\mu\nu}(A_\nu)_jt_{\mu\nu}^{-1}-(\partial_jt_{\mu\nu})t_{\mu\nu}^{-1}.\]
The curvature of the connection $A$ is then given
\[ [\nabla_j,\nabla_k]\phi^\alpha=F_{kj\beta}^\alpha\phi^\beta,\]
where we find that
\[F_{kj\beta}^\alpha=\partial_jA_{k\beta}^\alpha-\partial_kA_{j\beta}^\alpha+A_{j\gamma}^\alpha A_{k\beta}^\gamma-A_{k\gamma}^\alpha A_{j\beta}^\gamma.\]
It will be useful for purposes of cohomology, to introduce forms. We do this by introducing the curvature form
\[F\equiv \frac{1}{2}\sum F_{kj}dx^j\wedge dx^k.\]
We also define the connection form
\[A=A_{j\beta}^\alpha dx^j,\]
which is an $\End(E)$-valued one-form. It is easy to see that
\[F=dA+A\wedge A.\]
Finally, recall that connections satisfy the second Bianchi identity:
\[dF+A\wedge F-F\wedge A=0\]
for any connection $A$, which is also easy to see using the above expression for $F$ in terms of $A$.

Let us now turn to characteristic classes of vector bundles (Chern-Weil theory). We define, given a connection $A$ with curvature $F(A)$,
\begin{align*}
    c_m(A)&=\tr\left( F\wedge\cdots \wedge F \right)\in\Lambda^{2m}
\end{align*}
where the trace is taken $m$ times. As an explicit example, note that $c_1(A)=\tr(F)$, i.e.
\[c_1(A)=\tr(F)=\frac{1}{2}\sum F_{jk\alpha}^\alpha dx^k\wedge dx^j.\]
For $m=2$, we see that
\begin{align*}
    c_2(A)=\tr\left( F\wedge F \right)=\frac{1}{4}\sum\left( F_{jk\beta}^\alpha dx^k\wedge dx^j \right)\wedge \left( F_{lm\alpha}^\beta dx^m\wedge dx^l \right).
\end{align*}
Doing this, we find that $c_m(A)$ is a $2m$-form on the base manifold $X$.
\begin{thm}
    For any connection $A$, $c_m(A)$ is closed. Furthermore, if $[c_m(A)]$ be the de Rham cohomology class of $c_m(A)$,
    then $[c_m(A)]$ is independent of $A$. More precisely, we can write $c_m(A')-c_m(A)=dT_m$, where
    \[T_m=m\int^1_0\tr(B\wedge F_t^{m-1}) dt,\]
    where $A'=A+B$ and $A_t=A+tB$, and $F_t=F(A_t)$.
\end{thm}
\begin{proof}
    Let us compute
    \begin{align*}
        dc_m&=d(\tr(F\wedge\cdots\wedge F))\\
        &=\tr(dF\wedge\cdots\wedge F+\cdots+F\wedge\cdots\wedge dF)\\
        &=\tr( (-A\wedge F+F\wedge A)\wedge F\wedge\cdots\wedge F+\cdots+F\wedge\cdots\wedge(-A\wedge F+F\wedge A))\\
        &=\tr\left( -A\wedge F^{m-1} \right)+\tr\left( F^{m-1}\wedge A \right)\\
        &=0,
    \end{align*}
    where we have used Bianchi identity as well as the cyclic property of the trace. Let us now check that the cohomology
    class is independent of $A$ for $m=3$. It is straightforward to extend this for larger $m$. It is important
    to first note that differences of connections transform precisely as forms, and hence it makes sense to
    connect $A$ to $A'$ via a path that consists of adding forms, as adding forms does not change the transformation of a connection.
    Hence
    \begin{align*}
        c_m(A')-c_m(A)&=\int_0^1\frac{d}{dt}[c_m(A_t)]dt\\
        &=\int_0^1\frac{d}{dt}\left( \tr(F_t\wedge F_t\wedge F_t) \right)\\
        &=3\int_0^1\tr(\dot F_t\wedge F_t\wedge F_t) dt,
    \end{align*}
    using the cyclic property of the trace. Compare this with
    \begin{align*}
        dT_3&=3d\left( \int_0^1\tr(B\wedge F_t\wedge F_t) \right)dt\\
        &=3\left( \int_0^1\tr(dB\wedge F_t\wedge F_t-B\wedge dF_t\wedge F_t-B\wedge F_t\wedge dF_t) \right).
    \end{align*}
    Applying Bianchi's identity to each occurence to $dF_t$ and keeping in mind that
    \[\dot F_t=d\dot A_t+\dot A_t\wedge A_t+A_t\wedge \dot A_t=dB+B\wedge A_t+A\wedge B,\]
    yields precisely the expression above.
\end{proof}

Let us now turn to the Yang-Mills functional. Let $E\to X$ be a smooth vector bundle, with $g_{ij}$ a metric on $X$
and $H_{\alpha\beta}$ a metric on $E$. 
The metric allows us to define a pointwise inner product $\langle\phi,\psi\rangle=\phi^\alpha\psi^\beta H_{\beta\alpha}$
on the sections of $E$. Recall that we can differentiate our sections with respect to a connection to obtain another connection
$\nabla^A_j\phi^\alpha=\partial_j\phi^\alpha+A_{j\beta}^\alpha\phi^\beta$. We say that $A$ is a \textbf{unitary connection}
(with respect to $H$) if
\[\partial_j\langle\phi,\psi\rangle=\langle\nabla_j^A\phi,\psi\rangle+\langle\phi,\nabla^A_j\psi\rangle.\]
Recall that we have the formula
\[F_{jk\beta}^\alpha=\partial_k A^\alpha_{j\beta}-\partial A^\alpha_{k\beta}+A^\alpha_{j\gamma}A^\gamma_{k\beta}-A^\alpha_{k\gamma}A^\gamma_{j\beta},\]
which in matrix notation can be written
\[F_{jk}=\partial_kA_j-\partial_jA_k+A_jA_k-A_kA_j.\]
We then define the norm of the curvature to be
\begin{align*}
    |F(A)|^2&=F_{jk\beta}^\alpha F_{lm\delta}^\gamma g^{jl}g^{km}H_{\alpha\gamma}H^{\beta\delta}.
\end{align*}
The \textbf{Yang-Mills functional} of a given unitary connection is then defined to be
\begin{align*}
    I(A)=\int_X |F(A)|^2\sqrt{g}.
\end{align*}

The question of interest is as to whether there exist critical points for $I(A)$. Moreover, are
there saddle points, etc.? Formally, we say that $A$ is a critical point of $I(A)$ if $\delta I/\delta A=0$.
Observe that $\delta F_{jk\beta}^\alpha=\nabla_{k}\delta A^\alpha_{j\beta}-\nabla_j\delta A^{\alpha}_{k\beta}$
via the formula for curvature above.
Explicitly,
\begin{align*}
    \delta I&= \int\left( \delta F_{jk\beta}^\alpha F_{lm\delta}^{\gamma}+F_{jk\beta}^\alpha \delta F_{lm\delta}^\gamma \right)g^{jl}g^{km}H_{\alpha\gamma}H^{\beta\delta}\sqrt{g}\\
    &=\int (\nabla_k\delta A^\alpha_{j\beta}-\nabla_j\delta A^\alpha_{k\beta})F_{lm\delta}^\gamma+\cdots\\
    &=\int \delta A^\alpha_{j\beta}(-\nabla_k F_{lm\delta}^\gamma)g^{jl}g^{km}H_{\alpha\gamma}H^{\beta\delta}\sqrt{g}+\cdots
\end{align*}
which implies that
\[\nabla^m F_{lm\delta}^\gamma=0,\]
as (one can convince one's self that) the other terms do not matter.
Note that we have integrated the first covariant derivative term by parts and contracted indices of the derivative.
We call this the \textbf{Yang-Mills equation}, a second-order nonlinear PDE. Solving this equation is a hard problem
in general. However, there is an interesting class of solutions that exist in four dimensions: the anti-self-dual connections.

Let $E\to X$ be a bundle with metric $H_{\alpha\beta}$ with $\text{rk } E=r$. An
orthonormal frame of $E$ in $\left\{ e_a \right\}^r_{a=1}$, $e_a\in\Gamma(X,E)$ such that $\langle e_a,e_b\rangle=\delta_{ab}$.
Explicitly, $e_a$ is described by its components $e_a^\alpha$, and hence for any section $\phi\in\Gamma(X,E)$,
we can write $\phi=\sum_{a=1}^r\phi^ae_a$, i.e. $\phi^\alpha=\sum\phi^a e^\alpha_a$.
Note that we can relate $\phi^\alpha=e^\alpha_a\phi^a$ -- this is essentially a simple change of basis. However trivial this seems,
we do gain something in the case of unitary connections. Set $(\nabla_j\phi)^a\equiv\partial_j\phi^a+A^a_{jb}\phi^b$.
Note further that $\langle\phi,\psi\rangle=\sum\phi^a\psi^a$, as one can easily check.
Then the unitarity condition on the connection $A$ 
\begin{align*}
    \partial_j\langle\phi,\psi\rangle&=\langle\nabla_j\phi,\psi\rangle+\langle\phi,\nabla_j\psi\rangle\
\end{align*}
becomes
\begin{align*}
    \partial(\phi^a\psi^a)&=(\nabla_j\phi)^a\psi^a+\phi^a(\nabla_j\psi^a)\\
    &=(\partial_j\phi^a+A_{jb}^a\phi^b)\phi^a(\partial_j\psi^a+A_{jb}^a\psi^b).
\end{align*}
Hence the unitarity condition is equivalent to the fact that $A^a_{jb}=-A_{ja}^b$, i.e. the connection
matrices live in the tangent space to the orthogonal group of matrices (as they govern infinitesimal behavior).
Note that this also implies that $F_{jkb}^a=-F_{jka}^b$ by the formula for the curvature stated above.

Observe now that if we have an antisymmetric matrix $M=(M^a_b)$, then (a Hilbert-Schmidt kind of norm)
\[|M|^2=\sum_{ab}M^a_bM^a_b=-\sum M^a_bM^b_a=-\tr M^2.\]
However, the curvature is a matrix-valued form, so we can simplify the norm of the form part of $F(A)$ via the Hodge star.
Recall that the Hodge star operator is defined as (up to some factorials or something), for any $p$-forms $\phi,\psi$
such that $*\psi\in\lambda^{n-p}$ and 
\[\phi\wedge*\psi=\langle\phi,\psi\rangle\sqrt{g}dx^1\wedge\cdots\wedge dx^n.\]
With these observations in mind, we can rewrite the Yang-Mills functional in the following way:
\[I(A)=-\int_X\tr(F\wedge*F),\]
where $F$ is seen as a two-form valued in the space of anti-symmetric matrices. Note that so far,
this holds for all dimensions of $X$.

When $\dim X=4$, in particular, $*:\Lambda^2\to\Lambda^2$ such that $*^2=1$ with eigenvalues $\pm 1$
(we leave this as an exercise). We denote the eigenspace decomposition as $\Lambda\equiv\Lambda_+\oplus\Lambda_-$.
Write now $F=F_++F_-$, which yields
\begin{align*}
    I(A)&=-\int\tr(F\wedge *F)=-\int\tr\left( (F_++F_-)\wedge (F_+-F_-) \right)\\
    &=-\int\tr(F_+\wedge F_+)+\int\tr(F_-\wedge F_-)\\
    &=||F_+||^2_{L^2}+||F_-||^2_{L^2}.
\end{align*}
Compare this to the invariant Chern class
\begin{align*}
    \int_Xc_2(A)&=\int_X\tr(F\wedge F)
\end{align*}
that is independent of $A$. We can compute this
\begin{align*}
    \int_Xc_2(A)&=\int_X\tr(F\wedge F)\\
    &=\int_X\tr\left( (F_++F_-)\wedge(F_++F_-) \right)\\
    &=\int_X \tr(F_+\wedge F_+)+\tr(F_-\wedge F_-)+\tr(F_+\wedge F_-)+\tr(F_-\wedge F_+)\\
    &=\int_X \tr(F_+\wedge *F_+)+\tr(F_-\wedge -*F_-)\\
    &=||F_-||^2_{L^2}-||F_+||^2_{L^2}.
\end{align*}
where we have used orthogonality of the eigenspaces. Hence we find that
\[I(A)\geq |\int_Xc_2(E)|,\]
i.e. we have a non-trivial lower bound for the Yang-Mills functional. Moreover, suppose $\int_X c_2(E)<0$: then
\[I(A)=||F_+||^2_{L^2}+||F_-||^2_{L^2}\geq ||F_-||^2_{L^2}-||F_+||^2_{L^2},\]
which shows that $A$ achieves the minimum of the Yang-Mills functional if and only if $F_+=0$.
This is much easier than second-order PDE we obtained earlier, as we have ``integrated'' away
one of the derivatives.



\end{document}
