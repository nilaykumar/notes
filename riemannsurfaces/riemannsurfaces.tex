\documentclass{../mathnotes}

\usepackage{tikz-cd}
\usepackage{amsmath}
\usepackage{todonotes}


\title{Riemann Surfaces: Lecture Notes}
\author{Nilay Kumar}
\date{Last updated: \today}


\begin{document}

\maketitle

%\setcounter{section}{-1}

\section*{Class 10}

Recall that we are studying function theory on the torus $\C/\Lambda$, where $\Lambda=\left\{ m\omega_1+n\omega_2; m,n\in\Z \right\}$.
We had produced a candidate
\begin{align*}
    \sigma(z)=z\prod_{\omega\in\Lambda^\times}\left( 1+\frac{z}{w} \right)e^{-\frac{z}{\omega}+\frac{1}{2}\frac{z^2}{\omega^2}}.
\end{align*}
Let us check that this product converges by examining its logarithm:
\begin{align*}
    \log\left\{ \cdots \right\}&=\log(1+\frac{z}{\omega})-\frac{z}{\omega}+\frac{z^2}{2\omega^2}\\
    &=\left( \frac{z}{\omega}-\frac{1}{2}\frac{z^2}{\omega^2}+\ldots \right)-\frac{z}{\omega}+\frac{1}{2}\frac{z^2}{\omega^2},
\end{align*}
which clearly converges. Hence $\sigma(z)$ is holomorphic for $z\in\C$. Recall that $\sigma'(z)/\sigma(z)=\zeta(z)$ and so $\partial_z\log\sigma(z+\omega_a)=\zeta(z+\omega_a)$.
Thus we have (from before) that
\begin{align*}
    \eta_a=\zeta(z+\omega_a)-\zeta(z)=\partial_z\log\sigma(z+\omega_a)-\partial_z\log\sigma(z),
\end{align*}
which gives us periodicity information. Integrating and exponentiating, we see that
\begin{align*}
    \sigma(z+\omega_a)=\sigma(z)e^{\eta_az+c_a}
\end{align*}
where $c_a$ is the constant of integration, and taking $z=-\omega_a/2$, we find that
\begin{align*}
    \sigma(\omega_a/2)=\sigma(-\omega_a/2)e^{-\eta_a\frac{\omega_a}{2}+c_a}.
\end{align*}
It is easy to check, however, that $\sigma$ is odd, and hence we find that
\begin{align*}
    \sigma(z+\omega_a)=-\sigma(z)e^{\eta_a(z+\frac{\omega_a}{2})}.
\end{align*}

So we have found that $\sigma(z)$ is holomorphic on $\C$ and that $\sigma(z)=0$ if and only if $z=0\mod\Lambda$.
Now that we have constructed such a $\sigma$, let us give another proof of Abel's theorem. First recall our previous statement of
Abel's theorem.

\begin{thm}[Abel's theorem]
    Let $P_1,\ldots,P_M,Q_1,\ldots,Q_N$ be points in $\C$. Then there exists a meromorphic $f$ with zeroes at $P_i$ and poles at $Q_i$
    if and only if $M=N$ and $\sum_{i=1}^MA(P_i)=\sum_{i=1}^NA(Q_i)$.
\end{thm}
Recall that the Abel map takes $\C/\Lambda\ni p\mapsto A(p)=\int_{p_0}^p\omega$ where the value of the integral is taken modulo the lattice
generated by $\oint_A\omega,\oint_B\omega$. Take $p_0=0$ and $\omega=dz$, which is a well-defined form, and if we take $A$ to align with $\omega_2$
and $B$ to align with $\omega_1$, we see that $\oint_A\omega=\oint_A dz=\omega_1$ and similarly $\oint_B\omega=\omega_2$. Hence the map simply
takes $p$ to $\int_0^p dz \mod\Lambda=p$ where $p$ is viewed as a complex number.

Let us now restate Abel's theorem.
\begin{thm}[Abel's theorem, v.2]
    Let $P_1,\ldots,P_M,Q_1,\ldots,Q_N$ be points in $\C$. Then there exists a meromorphic $f$ with zeroes at $P_i$ and poles at $Q_i$
    if and only if $M=N$ and $\sum_{i=1}^M P_i=\sum_{i=1}^NQ_i\mod\Lambda$.
\end{thm}
\begin{proof}
    Consider the function
    \begin{align*}
        f(z)=\frac{\prod_{i=1}^M\sigma(z-P_i)}{\prod_{i=1}^N\sigma(z-Q_i)}.
    \end{align*}
    We should be a little careful to note that $\sigma$ is a function not on the torus $\C/\Lambda$, but a function on $\C$ (it transforms under a lattice
    translation!). Hence we must be cognizant of the fact that $P_i,Q_i$ here are some chosen representatives in $\C$ of the equivalence classes of the points $P_i,Q_i$.
    It should be clear that $f(z)$ is meromorphic with zeroes at every representative of each $P_i$s and poles at every representative of each $Q_i$.
    The natural question, now, is whether this function extends to a function on the torus. 
    To check this, let us see whether it is doubly periodic using what we know about $\sigma$:
    \begin{align*}
        f(z+\omega_a)&=f(z)\frac{\prod_{i=1}^Me^{\eta_a(z-P_i)}}{\prod_{i=1}^Ne^{\eta_a(z-Q_i)}}\\
        &=f(z)e^{-\eta_a\left( \sum_{i=1}^MP_i-\sum_{i=1}^NQ_i \right)}.
    \end{align*}
    Hence we wish to choose $P_i,Q_i$ representatives such that the exponential becomes unity. By hypothesis, this can be done (by shifting one, if necessary).
\end{proof}


Let us now return to Weierstrass theory. Given $\omega=dz$, we defined $\omega_0=\mathcal{P}(z)dz$ which has a double pole at 0 and
$\partial_z\log\sigma(z)=\zeta(z)$ and $\zeta'(z)=-\mathcal{P}(z)$. Now we can construct a form $\omega_{PQ}$ with residues $1,-1$ at $P,Q$ respectively,
by assigning $\omega_{PQ}(z)=\left( \zeta(z-P)-\zeta(z-Q) \right)\omega=\partial_z\log\frac{\sigma(z-P)}{\sigma(z-Q)}dz$. What Weierstrass theory
tells us that we can write everything in terms of $\sigma$, our analog of $z$.

\subsection*{Jacobi theory: $\theta$-functions}

Consider again the torus $\C/\Lambda$, where we now normalize the lattice as $\Lambda=\left\{ m+n\tau; m,n\in\Z\right\}$ with $\text{Im }\tau>0$ (by linear independence,
it cannot be real). This simply corresponds to picking $\omega_1=1,\omega_2/\omega_1=\tau$. Next define the \textbf{theta-function}
\begin{align*}
    \theta(z|\tau)=\sum_{n\in \Z}e^{\pi in^2\tau+2\pi inz},
\end{align*}
in which the structure of the lattice is explicitly clear (unlike in the Weierstrass theory). Let us examine its main properties.

First, note that $\theta(z|\tau)$ is holomorphic in $z\in\C$ because the series converges for all $z$; this is due to the term
\begin{align*}
    |e^{\pi in^2(\tau_1+i\tau_2)}|=|e^{\pi in^2\tau_1}e^{-\pi^2n^2\tau_2}|=e^{-\pi n^2\tau_2}
\end{align*}
for $\tau=\tau_1+i\tau_2$, whose decay dominates due to the $n^2$. Next, notice that
\begin{align*}
    \theta(z+1|\tau)&=\theta(z|\tau)\\
    \theta(z+\tau|\tau)&=e^{-\pi i\tau-2\pi iz}\theta(z|\tau),
\end{align*}
where the second is obtained by completing the square. Though $\theta$ is not invariant, its zeroes are.

Furthermore, we claim that $\theta(z|\tau)$ vanishes at exactly one point modulo lattice translates. It suffices to compute
the integral $\oint_C\frac{\theta'(z|\tau)}{\theta(z|\tau)}dz$, as it yields $2\pi i$ times the difference in the number of zeroes and poles in
a given region. We shall integrate over the curve $C$ where $C$ traverses the circumference of one lattice segment (i.e. the whole torus):
\begin{align*}
    \oint_C \frac{\theta'(z|\tau)}{\theta(z|\tau)}dz=\oint_B\left(-\frac{\theta'(z|\tau)}{\theta(z|\tau)}+\frac{\theta'(z+1|\tau)}{\theta(z+1|\tau}\right)
    +\oint_A\left(\frac{\theta'(z|\tau)}{\theta(z|\tau)}-\frac{\theta'(z+\tau|\tau)}{\theta(z+\tau|\tau)}\right).
\end{align*}
But these are just the shifts in the logarithmic derivative, and since $\partial_z\log\theta(z+\tau|\tau)=-2\pi i+\partial_z\log\theta(z|\tau)$ using
the transformation rules above, we see that our integral simplifies to
\begin{align*}
    \oint_C \frac{\theta'(z|\tau)}{\theta(z|\tau)}dz=2\pi i\oint_A dz=2\pi i.
\end{align*}
Of course, since $\theta$ is holomorphic, it has no poles, and hence we see that we have one zero. The zero, in fact, occurs in the center:
$\theta\left( (1+\tau)/2|\tau \right)=0$. To see this, consider the following function:
\begin{align*}
    \theta\left( z+\frac{1+\tau}{2}|\tau \right)&=\sum_{n\in\Z}\exp\left( \pi i n^2\tau+2\pi i n\left(z+\frac{1+\tau}{2}\right) \right)\\
    &=i\exp\left( -\pi i\frac{\tau}{4}-\pi iz \right)\sum_{n\in\Z}\exp\left( \pi i(n+\frac{1}{2})^2\tau+2\pi i(n+\frac{1}{2})(z+\frac{1}{2}) \right)\\
    &=i\exp\left( -\pi i\frac{\tau}{4}-\pi iz \right)\theta_1(z|\tau)
\end{align*}
where we have completed the square and defined the function $\theta_1$. We claim that $\theta_1$ is an odd function, which would imply that $\theta_1$ vanishes
at zero, which would prove the claim about the location of the zero. Hence let us verify that $\theta_1$ is odd; switching $z\mapsto-z$ yields in the exponent
\begin{align*}
    \log\theta_1(z|\tau)=\pi i\left( n+\frac{1}{2} \right)^2\tau+2\pi i\left( n+\frac{1}{2} \right)\left( -z+\frac{1}{2} \right).
\end{align*}
If we switch the indices $n\mapsto m$ such that $n+\frac{1}{2}=-\left( m+\frac{1}{2} \right)$, we find that the exponent is now
\begin{align*}
    \log\theta_1(z|\tau)=\pi i\left( m+\frac{1}{2} \right)^2\tau+2\pi i\left( m+\frac{1}{2} \right)\left(\left( z+\frac{1}{2} \right)-2\pi i\left( m+\frac{1}{2} \right)\right),
\end{align*}
and hence $\theta_1$ is odd. Now we see that the function we want is in fact $\theta_1(z|\tau)$ as it is odd, holomorphic, and has one zero.

We leave it as an exercise to show that
\begin{align*}
    \sigma(z)=\omega_1\exp\left( \eta_1\frac{z^2}{\omega_1}\right)\frac{\theta_1\left( \frac{z}{\omega_1}|\tau \right)}{\theta_1'(0|\tau)}
\end{align*}

\section*{Class 11}

We claim that the theta-function theory is more powerful than what we have been using so far - to see this, let us prove Abel's theorem.
Recall that the theorem states that there exists a meromorphic $f$ with zeroes at $P_i$ and poles and $Q_j$ if and only if $N=M$ and $\sum_i A(P_i)=\sum_j A(Q_j)$.
The idea is to express
\begin{align*}
    f(z)=\frac{\prod_{i=1}^N\theta_1(x-P_i)}{\prod_{i=1}^N\theta_1(z-Q_i)}
\end{align*}
and check double-periodicity. It is an exercise to check that
\begin{align*}
    \theta_1(z+1|\tau)&=-\theta_1(z|\tau)\\
    \theta_1(z+\tau|\tau)&=\exp\left( -\pi i\tau-2\pi i(z+1/2) \right)\theta_1(z|\tau),
\end{align*}
from which periodicity follows easily. Of course, we must be careful to note that the $P_i,Q_i$ used here are in fact chosen representatives.

Next let us define a meromorphic form
\begin{align*}
    \omega_{PQ}=\partial_z\log\frac{\theta_1(z-P)}{\theta_1(z-Q)}dz,
\end{align*}
which, it is easy to check, has poles at $P,Q$ with opposite residues. Additionally, one can check that this expression is well-defined on the lattice, 
i.e. invariant under a shift. We leave it as a simple exercise to show that
\[\omega_P(z)=\partial^2_z\log\frac{\theta_1(z-P|\tau)}{\theta_1'(0|\tau)}dz\]
is a meromorphic form with a double pole at $P$ and is well-defined on the lattice.

But in fact, we can go even farther with this theta-function. Indeed, one attractive feature is that there exists a product expansion for $\theta(z|\tau)$.
\begin{thm}
    We can expand
    \begin{align*}
        \theta(z|\tau)=\prod_{n=1}^\infty (1-q^{2n)}(1+q^{2n-1}e^{2\pi iz})(1+q^{2n-1}e^{-2\pi iz})
    \end{align*}
    where $q\equiv e^{\pi i\tau}$.
\end{thm}
\begin{proof}
    Define
    \begin{align*}
        T(z|\tau)=\prod_{n=1}^\infty (1-q^{2n)}(1+q^{2n-1}e^{2\pi iz})(1+q^{2n-1}e^{-2\pi iz}).
    \end{align*}
    We claim that $T(z|\tau)$ is equal to zero exactly when $z$ is $(1+\tau)/2\mod\Lambda$ and that the zeros are simple. This can be
    checked by some simple algebra. It's also easy to show that $T(z|\tau)$ is holomorphic in $\C$ and that $\tau(z+1|\tau)=T(z|\tau)$.
    Moreover
    \begin{align*}
        T(z+\tau|\tau)&=\prod_{n=1}^\infty (1-q^{2n})\prod\left( 1+q^{2n+1}e^{2\pi iz} \right)\left( 1+q^{2n-3}e^{-2\pi iz}q^{-2} \right)\\
        &=\prod_{n=1}^\infty (1-q^{2n})\frac{\prod_{n=1}^\infty\left( 1+q^{2n-1}e^{2\pi i z} \right)}{1+qe^{2\pi iz}}\prod_{n=1}^\infty\left( 1+q^{2n-1}e^{-2\pi iz} \right)
        &=\frac{1-q^{-1}e^{-2\pi iz}}{1-qe^{2\pi iz}}T(z|\tau)=q^{-1}e^{-2\pi iz}.
    \end{align*}
    Recall that $\theta$ followd a similar condition. This shows that $\theta(z|\tau)/T(z|\tau)=c$,
    where $c$ is a constant independent of $z$ that can depend on $\tau$. Next we claim that $c(\tau)=1$. For this we show that there exists a $c$ such that $c(\tau)=c(4\tau)=c(4^k\tau)$
    and $c(\tau)=\lim_{k\to\infty}c(4^k\tau)=1$, which shows the proof. Hence let us prove that $c(\tau)=c(4\tau)$ using $\theta(z|\tau)=C(\tau)T(z|\tau)$.

    Take $z=1/2$. Then $e^{2\pi iz=e^{\pi i}}\geq 1-$ and $\theta(1/2|\tau)=\sum_{n\in\Z}e^{\pi i n^2\tau}(-1)^n$ but $T(1/2|\tau)=\prod^{\infty}_{n=1}(1-q^{2n)(1-q^{2n-1}}(1-q^{2n-1})$.
    and hence $c(\tau)=\sum e^{\pi in\tau}(-1)^n/\prod(1-q^n)(1-q^{2n-1})$. Next take $z=1/4$
\end{proof}

\section{Semester 2}

This semester we will start by describing the $L^2$ estimates of Hormander. Later we will delve into its applications,
including the Kodaira embedding theorem, the lower bounds for the Bergman kernel, and the ideas of canonical metrics
and stability.

\begin{rem}
    Suppose we have a holomorphic line bundle $L\to X$. We may ask the following questions.
    \begin{enumerate}[(a)]
        \item $H^0(X,L)=0$?
        \item Take some $s\in H^0(X,L)$ with $||s||_{L^2}=1$. How big can $s(z_0)$ be at a given point $z_0$?
    \end{enumerate}
    We will be building machinery to address these questions, which depend sensitively on the geometry.
\end{rem}

\subsection{Review}

Let us recall some techniques from last semester. Let $X=\cup_{\mu}X_\mu$ be a complex $n$-manifold with $X_\mu$ coordinate charts.
Hence each $X_\mu$ is homeomorphic (and thus biholomorphic) to $\C^n$ and $\Phi_\mu\circ\Phi_\nu^{-1}$ is holomorphic with invertible differentials\todo{what?}.
Let $E\to X$ be a holomorphic vector bundle. Recall that a rank-$r$ vector bundle is completely characterized by its transition functions $t_{\mu\nu\beta}^\alpha(z)$ (matrix valued
in general) defined on $X_\mu\cap X_\nu$ with $1\leq\alpha,\beta\leq r$. Note that the transition functions satisfy the cocycle condition.
We denote by $\Gamma(X,E)$ the space of sections of $E$ (recall that this means that $\phi_\mu^\alpha(z_\mu)=t_{\mu\nu\beta}^\alpha(z)\phi_\nu^\beta(z_\nu)$.
For $E$ to be holomorphic, it must have holomorphic transition functions.

Given a section $\phi\in\Gamma(X,E)$, we obtain a section $\bar\partial\phi\in\Gamma(X,E\otimes\Lambda^{0,1})$ via \textbf{covariant differentiation}.
More explicitly, on $X_\mu$, we write naively
\[\bar\partial\phi^\alpha\equiv\left(\frac{\partial}{\partial \bar z^j_\mu}\phi_{\mu}^\alpha\right)(z_\mu).\]
Fortunately, this is indeed a section. To see this, we note that $\phi^\alpha_\mu(z_\mu)=t_{\mu\nu\beta}^\alpha(z)\phi_\nu^\beta(z_\nu)$ and hence
\begin{align*}
    \frac{\partial}{\partial \bar z^j_\mu}\phi^\alpha_\mu(z_\mu)&=t_{\mu\nu\beta}^\alpha(z)\frac{\partial}{\partial \bar z^j_\mu}\left( \phi_\nu^\beta(z_\nu) \right)\\
    &=t_{\mu\nu\beta}^\alpha(z)\overline{\frac{\partial z_\nu^k}{\partial z_\mu^j}}\frac{\partial \phi^\beta_\nu}{\partial\bar z_\nu^k}(z).
\end{align*}
Thus we define $\Lambda^{0,1}$ to be the (antiholomorphic) vector bundle with transition functions $\overline{\partial z_\nu^k/\partial z^j_\mu}$.

Next, recall the definition of a \textbf{Hermitian metric} $H=H_{\bar \alpha\beta}(z)$ on a vector bundle $E$: we have $(H_\mu)_{\bar\alpha\beta}(z_\mu)$ on $X_\mu$
such that it is a positive-definite matrix for each $z_\mu$ satisfying
\[|\phi|^2_H\equiv(H_\mu)_{\bar\alpha\beta}\overline{\phi_\mu^\alpha}\phi_\mu^\beta=(H_\nu)_{\bar\gamma\delta}\overline{\phi_\nu^\gamma}\phi_\nu^\delta\]
on $X_\mu\cap X_\nu$.
This quantity can be thought of as the length of the vector $\phi$ with respect to the metric $H$, which is by construction invariant of coordinate chart.
Using metrics, we can introduce covariant derivatives of sections on a holomorphic vector bundle $E$ with respect to a metric $H_{\bar\alpha\beta}$.
Take $\phi\in\Gamma(X,E)$ and define on $X_\mu$
\[(\nabla_j\phi)^\alpha\equiv H^{\alpha\bar\gamma}\partial_j(H_{\bar\gamma\beta}\phi_\mu^\beta),\]
where $H^{\alpha\bar\gamma}H_{\bar\gamma\beta}=\delta^{\alpha}_\beta$. It is easy to see that $\nabla\phi\in\Gamma(X,E\otimes\Lambda^{1,0})$,
whose transition functions are $\partial z_\nu^k/\partial z^j_\mu$.

In summary, we write
\begin{align*}
    \nabla_{\bar j}\phi^{\alpha}&=\partial_{\bar j}\phi^\alpha\\
    \nabla_j\phi^\alpha&=H^{\alpha\bar\gamma}\partial_j(H_{\bar\gamma\beta}\phi_\mu^\beta)\\
    &=\partial_j\phi^\alpha+(H^{\alpha\bar\gamma}\partial_jH_{\bar\gamma\beta})\phi^{\beta}\\
    &=\partial_j\phi^\alpha+A_{j\beta}^\alpha\phi^\beta,
\end{align*}
where, in matrix notation, we have the \textbf{connection} $A_j=H^{-1}\partial_j H$. It is a priori not obvious that these two derivatives
must commute. Indeed, we define the \textbf{curvature} $F$ of the metric $H_{\bar\alpha\beta}$ on $E\to X$ to be
\[ [\nabla_{\bar j},\nabla_k]\phi^\alpha=-F_{\bar jk\beta}^\alpha \phi^\beta.\]
We leave it as an exercise that $[\nabla_{\bar j},\nabla_{\bar k}]=0$ and $[\nabla_j,\nabla_k]=0$. More explicitly,
we can write
\begin{align*}
    [\nabla_{\bar j},\nabla_k]\phi^\alpha&=\nabla_{\bar j}\left( \nabla_k\phi^\alpha \right)-\nabla_k\left( \nabla_{\bar j}\phi^{\alpha} \right)\\
    &=(\partial_{\bar j}A_{k\beta}^\alpha)\phi^{\beta},
\end{align*}
and hence $F_{\bar jk\beta}^\alpha=-\partial_{\bar j}A_{k\beta}^\alpha$. In matrix notation, we can simply write 
\[F_{\bar jk}=-\partial_{\bar j}A_k=-\partial_{\bar j}\left( H^{-1}\partial_k H \right).\]
We define the corresponding \textbf{curvature form} to be
\[F=\frac{i}{2\pi}F_{\bar jk\beta}^\alpha(z)dz^k\wedge d\bar z^j\in\Gamma(X,E\otimes E^*\otimes\Lambda^{1,1})=\Gamma(X,\End(E)\otimes\Lambda^{1,1}),\]
i.e. a $\End(E)$-valued $(1,1)$-form.

\subsection{Bochner-Kodaira Formulas}

Let $X$ be a complex manifold, compact without boundary. Take $E\to X$ to be a holomorphic vector bundle of rank $r$ on $X$.
We consider the \textbf{$\bar\partial$ complex}:
\[\cdots\xrightarrow{\bar\partial}\Gamma(X,E\otimes\Lambda^{p,q})\xrightarrow{\bar\partial}\Gamma(X,E\otimes\Lambda^{p,q+1})\xrightarrow{\bar\partial}\cdots.\]

Let us be more precise. Consider some $\phi\in\Gamma(X,E\otimes\Lambda^{p,q})$. We can write explicitly:
\[\phi^\alpha=\frac{1}{p!q!}\sum\phi^\alpha_{\bar j_1,\ldots\bar j_q,i_1,\ldots,i_p}(z)dz^{i_p}\wedge\cdots\wedge dz^{i_1}\wedge d\bar z^{j_q}\wedge\cdots\wedge d\bar z^{j_i}.\]
Now what exactly do we mean by $\bar\partial$? We define
\begin{align*}
    \bar\partial\phi&\equiv\frac{1}{p!q!}\sum\left(\bar\partial\phi_{\bar j_1,\ldots\bar j_q,i_1,\ldots,i_p}\right)\wedge dz^{i_p}\wedge\cdots\wedge dz^{i_1}\wedge d\bar z^{j_q}\wedge\cdots\wedge d\bar z^{j_i}\\
    &=\frac{1}{p!q!}\sum\left(\partial_{\bar k}\phi_{\bar j_1,\ldots\bar j_q,i_1,\ldots,i_p}d\bar z^k\right)\wedge dz^{i_p}\wedge\cdots\wedge dz^{i_1}\wedge d\bar z^{j_q}\wedge\cdots\wedge     d\bar z^{j_i}.
\end{align*}
We leave it as an exercise for the reader to check that this is well-defined (follows as per the usual de Rham exterior derivative).

\begin{exmp}
    What is $\bar\partial$ on $\Gamma(X,E\otimes\Lambda^{0,0})=\Gamma(X,E)$? By definition, $\bar\partial\phi=\partial_{\bar k}\phi^\alpha d\bar z^k$.
\end{exmp}
\begin{exmp}
    What is $\bar\partial$ on $\Gamma(X,E\otimes\Lambda^{0,1})$? Given a section, we can write $\phi=\sum \phi^\alpha_{\bar j}d\bar z^j$.
    In this case,
    \begin{align*}
        \bar\partial\phi^\alpha&=\sum\left( \bar\partial \phi^\alpha_{\bar j} \right)\wedge d\bar z^j\\
        &=\sum\left( \partial_{\bar k}\phi^{\alpha}_{\bar j}d\bar z^k \right)\wedge d\bar z^j\\
        &=\frac{1}{2}\sum\left( \partial_{\bar k}\phi_{\bar j}^\alpha-\partial_{\bar j}\phi^{\alpha}_{\bar k} \right)d\bar z^k\wedge d\bar z^j.
    \end{align*}
    Hence one finds the coefficient $(\bar\partial \phi)_{\bar j\bar k}=\left(\partial_{\bar k}\partial_{\bar j}-\partial_{\bar j}\partial_{\bar k}\right)$
    with no factor of $1/2$ out front, because we now have a two-form.
\end{exmp}

Let us now introduce a metric $H_{\bar\alpha\beta}$ on $E$ and a metric $g_{\bar kj}$ on $T^{1,0}(X)$.
This allows us to compute scalar norms of sections as $|\phi|^2_H=H_{\bar\alpha\beta}\overline{\phi^\alpha}\phi^\beta$ for $\phi\in\Gamma(X,E)$.
We will work with the metric and hope that in the end, our results will be independent of the metric (where lengths are not involved).
There is an induced $L^2$ metric on $\Gamma(X,E\otimes \Lambda^{p,q})$: given $\phi,\psi\in\Gamma(X,E\otimes\Lambda^{p,q})$, we define, using
multi-index notation,
\begin{align*}
    \langle \phi,\psi\rangle = \frac{1}{p!q!}\sum\int \phi^\alpha_{\bar JI}\overline{\psi^\beta_{\bar KL}}H_{\bar\beta\alpha}g^{K\bar J}g^{I\bar L}\frac{\omega^n}{n!},
\end{align*}
where, by definition,
\[g^{K\bar J}=g^{k_1\bar j_1}\cdots g^{k_q\bar j_q}\]
if $K=(k_1,\ldots, k_q)$ and $J=(j_1,\ldots,j_q)$,
and
\[\omega\equiv \frac{i}{2}g_{\bar kj}dz^j\wedge d\bar z^k.\]
We now define the \textbf{formal adjoint of $\bar\partial$} by
\begin{align*}
    \langle \bar\partial \phi,\psi\rangle =\langle\phi,\bar\partial^\dagger\psi\rangle
\end{align*}
for all $\phi\in C^\infty(X,E\otimes\Gamma^{p,q})$ and $\psi\in C^\infty(X,E\otimes\Gamma^{p,q+1})$.
Hence we can draw:
\begin{equation*}
    \begin{tikzcd}
        \Gamma(X,E\otimes\Lambda^{p,q})\arrow[bend left]{r}{\bar\partial} & \Gamma(X,E\otimes\Lambda^{p,q+1})\arrow[bend left]{l}{\bar\partial^\dagger}
    \end{tikzcd}
\end{equation*}
Next define the \textbf{Laplacian} $\Box:\Gamma(X,E\otimes\Lambda^{p,q})\to\Gamma(X,E\otimes\Lambda^{p,q})$ by $\Box=\bar\partial^\dagger\bar\partial+\bar\partial\bar\partial^\dagger$.
This raises the question: when do we have that
\[\dim\ker\Box\bigg|_{\Gamma(X,E\otimes\Lambda^{p,q})}=0?\]
It will turn out that for $q=1$, if this is true, we will indeed be able to find sections on this bundle.

To approach this question, we use the Bochner-Kodaira formulas. We claim that
\begin{align*}
    (\Box\phi)^\alpha_{\bar JI}=-g^{k\bar l}\nabla_k\nabla_{\bar l}\phi^\alpha_{\bar JI}+\text{t.t} + \text{c.t.}
\end{align*}
where t.t. and c.t. stand for torsion and curvature forms respectively. This relates a kind of geometric laplacian (lhs) to a
more analytic laplacian (rhs) modulo certain correction terms. In practice, we will work with K\"ahler metrics, in which
the torsion terms disappear. In this sense, we can schematically write (after integrating by parts)
\[\langle \Box\phi,\phi\rangle=||\nabla_{\bar l}\phi_{\bar J I}^\alpha||^2+\langle\text{c.t.}\phi,\phi\rangle.\]
This immediately implies that if the curvature terms are positive (loosely speaking), then $\ker\Box=0$.

We begin by deriving the Bochner-Kodaira formula. We will simplify life by working with $(0,1)$-forms, but the results
will extend fairly easily. Let us first compute $\bar\partial^\dagger$ more explicitly. In this case, we look at
\begin{equation*}
    \begin{tikzcd}
        \Gamma(X,E\otimes\Lambda^{0,0})\arrow[bend left]{r}{\bar\partial}&\Gamma(X,E\otimes\Lambda^{0,1})\arrow[bend left]{r}{\bar\partial}\arrow[bend left]{l}{\bar\partial^\dagger} & \Gamma(X,E\otimes\Lambda^{0,2})\arrow[bend left]{l}{\bar\partial^\dagger}.
    \end{tikzcd}
\end{equation*}
Let us compute the second formal adjoint that appears in the diagram. Pick some $\phi\in\Gamma(X,E\otimes\Lambda^{0,1})$ such that $\phi=\sum\phi_{\bar j}^\alpha d\bar z^j$. Then $\bar\partial\phi^\alpha=\sum\partial_{\bar k}\phi_{\bar j}^\alpha d\bar z^k\wedge d\bar z^j$. Also, pick $\psi=\frac{1}{2}\sum\psi_{\bar j\bar k}d\bar z^k\wedge d\bar z^j$ in $\Gamma(X,E\otimes\Lambda^{0,2})$. We impose that
\begin{align*}
    \langle\bar\partial\phi,\psi\rangle&=\langle \phi,\bar\partial^\dagger\psi\rangle
\end{align*}
The left hand side appears to be
\[ \frac{1}{2}\left( \int(\partial_{\bar p}\phi_{\bar m}^\alpha-\partial_{\bar m}\phi^\alpha_{\bar p}) \overline{\psi_{\bar j\bar k}^\beta}H_{\bar \beta\alpha}g^{j\bar m}g^{k\bar p}\frac{\omega^n}{n!} \right)
    \]
But recall that
\begin{align*}
    \nabla_{\bar p}\phi^{\alpha}_{\bar m}&=\partial_{\bar p}\phi_{\bar m}^\alpha-\Gamma^{\bar l}_{\bar p\bar m}\phi_{\bar l}^\alpha
\end{align*}
where $\Gamma_{\bar p\bar m}^{\bar l}=g^{k\bar l}(\partial_{\bar p}g_{\bar mk}).$ Hence we may write
\[\partial_{\bar p}\phi^\alpha_{\bar m}-\partial_{\bar m}\phi_{\bar p}^\alpha=\nabla_{\bar p}\phi^\alpha_{\bar m}-\nabla_{\bar m}\phi_{\bar p}^{\alpha}+(\Gamma^{\bar l}_{\bar p\bar m}-\Gamma^{\bar l}_{\bar m\bar p})\phi_{\bar l}^\alpha.\]
We denote the term in the parentheses by $T^{\bar l}_{\bar p\bar m}$ and call it the \textbf{torsion of the covariant derivative $\nabla_{\bar p}$}. Let us now move to the K\"ahler case.
The metric $g_{\bar kj}$ is said to be \textbf{K\"ahler} if $\Gamma^{\bar l}_{\bar p\bar m}=\Gamma^{\bar l}_{\bar m\bar p}$. Note that this
condition is equivalent to $\partial_{\bar p}g_{\bar mk}=\partial_{\bar m}g_{\bar pk}$ or $\partial_pg_{\bar km}=\partial_m g_{\bar kp}$.
Henceforth we assume that $g_{\bar kj}$ is K\"ahler.

Now in the computation of the formal adjoint above, we can write
\begin{align*}
    \langle\bar\partial\phi,\psi\rangle&=\int (\nabla_{\bar p}\phi^\alpha_{\bar m})\overline{\psi^\beta_{\bar j\bar k}}H_{\bar\beta\alpha}g^{j\bar p}g^{k\bar m}\frac{\omega^n}{n!}\\
    &=\int\phi^\alpha_{\bar m}\overline{(-g^{p\bar j}\nabla_p\psi^{\beta}_{\bar j\bar k})}H_{\bar\beta\alpha}g^{k\bar m}\frac{\omega^n}{n!},
\end{align*}
where we have de-antisymmetrized the covariant derivatives and then integrated by parts. This yields immediately the expression for the formal adjoint:
\[(\bar\partial^\dagger\psi)_{\bar k}^\beta=-g^{p\bar j}\nabla_p\psi^{\beta}_{\bar j\bar k}.\]
To see this in more detail, see the next paragraph, where we will perform the computation explicitly for the other formal adjoint.
So much for the computation of the second formal adjoint in the diagram above.\todo{check this!}

Let us now compute the first formal adjoint. Of course,
\[\langle\bar\partial\phi,\psi\rangle=\langle\phi,\bar\partial^\dagger\psi\rangle\]
for $\phi\in C^\infty(X,E)$ and $\psi\in C^\infty(X,E\otimes\Lambda^{0,1})$. We can write $\bar\partial\phi=\partial_{\bar j}\phi^\alpha d\bar z^j$
and $\psi=\psi^{\alpha}_{\bar k}d\bar z^k$. The above equation requires
\begin{align*}
    \int \partial_{\bar j}\phi^\alpha\overline{\psi^\beta_{\bar k}}H_{\bar \beta\alpha}g^{k\bar j}\frac{\omega^n}{n!}=\int\phi^\alpha\overline{\left( \bar\partial^\dagger\psi \right)^\beta}H_{\bar\beta\alpha}\frac{\omega^n}{n!}.
\end{align*}
Let us for now define $W_\alpha^{\bar j}\equiv \overline{\psi^\beta_{\bar k}H_{\bar\alpha\beta}g^{j\bar k}}$. Observe now that
\begin{align*}
    \left( \partial_{\bar j}\phi^\alpha \right)W^{\bar j}_\alpha&\equiv\left( \nabla_{\bar j}\phi^\alpha \right)W^{\bar j}_\alpha\\
    &=\nabla_{\bar j}\left( \phi^\alpha W^{\bar j}_\alpha \right)-\phi^\alpha\left( \nabla_{\bar j} W^{\bar j}_\alpha\right).
\end{align*}
This will be useful when integrating by parts.
Now note that the $n$th wedge power of $\omega$ simplifies to yield
\begin{align*}
    \int\left( \partial_{\bar j}\phi^\alpha \right)W_\alpha^{\bar j}\left( \det g_{\bar qp} \right)&=\int\nabla_{\bar j}\left( \phi^\alpha W^{\bar j}_\alpha \right)\det g_{\bar qp}
    -\int \phi^\alpha\left( \nabla_{\bar j}W^{\bar j}_\alpha \right).
\end{align*}
We claim that if the metric $g_{\bar kj}$ is K\"ahler, then $\int \nabla_{\bar j}(\phi^\alpha W_\alpha^{\bar j})\det g_{\bar qp}=0$. To see this, first define
$V^{\bar j}=\phi^\alpha W_\alpha^{\bar j}$. Consider
\begin{align*}
    \left( \nabla_{\bar j}V^{\bar j} \right)\det g_{\bar qp}&=\left(\partial_{\bar j}V^{\bar j}+\Gamma^{\bar j}_{\bar j\bar k}V^{\bar k}\right)\det g_{\bar qp}\\
    &=\partial_{\bar j}\left( V^{\bar j}\det g_{\bar qp} \right)-V^{\bar j}\left( \partial_{\bar j}\det g_{\bar qp} \right)+\Gamma^{\bar j}_{\bar j\bar k}V^{\bar k}\det g_{\bar qp}.
\end{align*}
Now note that
\begin{align*}
    \partial_{\bar j}\det g_{\bar qp}=\left( \det g_{\bar qp}\right)g^{l\bar n}\partial_{\bar j}g_{\bar nl},
\end{align*}
which comes from the fact that
\begin{align*}
    \delta\log\left( \det A \right)&=\sum\frac{\delta \lambda_j}{\lambda_j}\\
    &=\tr\left( A^{-1}\delta A \right).
\end{align*}
But now recall that for $g_{\bar kj}$ is K\"ahler if and only if $\Gamma^{\bar j}_{\bar k\bar m}=\Gamma^{\bar j}_{\bar m\bar k}$,
and hence the we see the last two terms in  the expression above cancel. Hence the term picked up by integration by parts vanishes.
Hence we are left with the equality
\begin{align*}
    -\int\phi^\alpha\nabla_{\bar j}\overline{\left( \psi^{\beta}_{\bar k}H_{\bar\alpha\beta}g^{j\bar k} \right)}\frac{\omega^n}{n!}&=
    \int\phi^\alpha\overline{\left( -g^{j\bar k}\nabla_j\psi^\beta_{\bar k} \right)}H_{\bar \beta\alpha}\frac{\omega^n}{n!}.
\end{align*}
This yields the desired formula:
\begin{align*}
    ( \bar\partial^\dagger\psi )^{\beta}=-g^{j\bar k}\nabla_j\psi^\beta_{\bar k}.
\end{align*}

Now let us compute the Laplacian $\Box=\bar\partial\bar\partial^\dagger+\bar\partial^\dagger\bar\partial$. Set $\phi=\sum \phi^\alpha_{\bar j}d\bar z^j\in\Gamma(X,E\otimes\Lambda^{0,1})$.
First note that
\begin{align*}
    \left( \bar\partial\bar\partial^{\dagger}\phi \right)&=\bar\partial\left( -g^{j\bar k}\nabla_j\phi^{\alpha}_{\bar k} \right)\\
    &=\partial_{\bar l}\left( -g^{j\bar k}\nabla_j\phi_{\bar k}^\alpha \right)d\bar z^l.
\end{align*}
Hence, noting that the expression in parentheses is a section of a holomorphic bundle, the $\partial_j$ is simply a covariant derviative, which
commutes with the metric, and we can write
\[\left( \bar\partial\bar\partial^{\dagger}\phi \right)^\alpha_{\bar l}=-g^{j\bar k}\nabla_{\bar l}\nabla_{j}\phi^\alpha_{\bar k}.\]
Next note that
\begin{align*}
    \bar\partial\phi^\alpha=\frac{1}{2}\sum\left( \nabla_{\bar k}\phi^\alpha_{\bar j}-\nabla_{\bar j}\phi^\alpha_{\bar k} \right)d\bar z^k\wedge d\bar z^j
\end{align*}
for $g_{\bar kj}$ K\"ahler. Hence we can write
\begin{align*}
    \left( \bar\partial^\dagger\bar\partial\phi \right)^\beta_{\bar l}&=-g^{k\bar m}\nabla_{k}(\bar\partial\phi)^\beta_{\bar l\bar m}\\
    &=-g^{k\bar m}\nabla_k\left( \nabla_{\bar m}\phi^\alpha_{\bar l}-\nabla_{\bar l}\phi^\alpha_{\bar m} \right)\\
    &=-g^{k\bar m}\nabla_k\nabla_{\bar m}\phi^\alpha_{\bar l}+g^{k\bar m}\nabla_k\nabla_{\bar l}\phi^\alpha_{\bar m}.
\end{align*}
Summing the two terms of the Laplacian and switching appropriate dummy indices, we find that
\begin{align*}
    (\Box\phi)^\alpha_{\bar l}&=-g^{k\bar m}\nabla_k\nabla_{\bar m}\phi^\alpha_{\bar l}+g^{k\bar m}\nabla_k\nabla_{\bar l}\phi^\alpha_{\bar m}-g^{k\bar m}\nabla_{\bar l}\nabla_{\bar k}\phi^\alpha_{\bar m}\\
    &=-g^{k\bar m}\nabla_k\nabla_{\bar m}\phi^\alpha_{\bar l}+g^{k\bar m}[\nabla_k,\nabla_{\bar l}]\phi^\alpha_{\bar m}\\
    &=-g^{k\bar m}\nabla_k\nabla_{\bar m}\phi^\alpha_{\bar l}+g^{k\bar m}\left( F_{\bar lk\beta}^\alpha\phi^\beta_{\bar m}+R_{\bar lk\bar m}^{\bar p}\phi^\alpha_{\bar p} \right),
\end{align*}
where as usual $F_{\bar lk\beta}^\alpha=-\partial_{\bar l}\left( J^{\alpha\bar\gamma}\partial_k H_{\bar\gamma\beta} \right)$
and $R_{\bar lk\bar m}^{\bar p}=g^{\bar pq}R_{\bar lkq}^rg_{\bar mr}$ where $R_{\bar lkq}^r=-\partial_{\bar l}\left( g^{r\bar s}\partial_kg_{\bar sq} \right)$.
We can simplify this a little bit more, obtaining
\begin{align}
    (\Box\phi)^\alpha_{\bar l}=-g^{k\bar m}\nabla_k\nabla_{\bar m}\phi^\alpha_{\bar l}+F_{\bar l\beta}^{\bar m\alpha}\phi^\beta_{\bar m}+R_{\bar l}^{\bar p}\phi^\alpha_{\bar p},
\end{align}
where $R_{\bar l}^{\bar p}\equiv g^{k\bar m}R_{\bar lk\bar m}^{\bar p}$ is the \textbf{Ricci curvature}.

Consider now the inner product with $\phi$:
\begin{align*}
    \int (\Box\phi)^{\alpha}_{\bar l}\overline{\phi^\beta_{\bar m}}H_{\bar \beta\alpha}g^{m\bar l}\frac{\omega^n}{n!}&=-\int g^{j\bar k}\nabla_j\nabla_{\bar k}\phi^\alpha_{\bar l}\overline{\phi^\beta_{\bar m}}H_{\bar\beta\alpha}g^{m\bar l}\frac{\omega^n}{n!}+\int (F_{\bar l\gamma}^{\bar p\alpha}\phi^\gamma_{\bar p}+R_{\bar l}^{\bar p}\phi^\alpha_{\bar p})\overline{\phi^\beta_{\bar m}}H_{\bar\beta\alpha}g^{m\bar l}\frac{\omega^n}{n!}\\
    \langle\Box\phi,\phi\rangle&=\int \nabla_{\bar k}\phi^\alpha_{\bar l}\overline{\nabla_{\bar j}\phi^\beta_{\bar m}}g^{j\bar k}H_{\bar\beta\alpha}g^{m\bar l}\frac{\omega^n}{n!}+\int F^{m\bar p}_{\bar\beta\alpha}\phi_{\bar p}^\alpha\overline{\phi^\beta_{\bar m}}+(R^{m\bar p}H_{\bar\beta\alpha})\phi^\alpha_{\bar p}\overline{\phi^\beta_{\bar m}}\frac{\omega^n}{n!}\\
    \langle\Box\phi,\phi\rangle&=||\bar\nabla\phi||^2+\int (F^{m\bar p}_{\bar\beta\alpha}+R^{m\bar p}H_{\bar\beta\alpha})\phi^\alpha_{\bar p}\overline{\phi^\beta_{\bar m}}\frac{\omega^n}{n!}.
\end{align*}
This yields the following simple corollary.

\begin{cor}
If $F^{m\bar p}_{\bar\beta\alpha}+R^{m\bar p}_{\bar\beta\alpha}>0$, then $\ker\Box\bigg|_{\Gamma(X,E\otimes\Lambda^{0,1})}=0.$
\end{cor}

Further, we can prove the following result.

\begin{cor}
Let $L\to X$ be a \textbf{positive} holomorphic line bundle over a compact manifold $X$, i.e. there exists a metric $h$ on $L$
with $-\partial_{j}\partial_{\bar k}\log h>0$. Set $\omega=-\frac{i}{2}\partial\bar\partial\log h$ (as coefficients). Since $L$ is positive, $\omega$ is a metric, which
is automatically K\"ahler:
\[\partial_l g_{\bar kj}=-\partial_l\partial_j\partial_{\bar k}\log h=\partial_j g_{\bar kl}.\]
Equip $X$ with the K\"ahler metric $\omega$ and consider $\Box$ on $L^M\otimes\Lambda^{0,1}$.
Then for $M\geq 1$,
\[\ker\Box\bigg|_{\Gamma(X,L^m\otimes\Lambda^{0,1})}=0.\]
\end{cor}
\begin{proof}
    When $E=L$, we can write
    \begin{align*}
        F_{\bar kj}=-\partial_j\partial_{\bar k}\log h
    \end{align*}
    and hence in the case of line bundles, we have that
    \begin{align*}
        F^{m\bar p}\equiv F^{m\bar p}_{\bar\beta\alpha}=g^{m\bar k}g^{j\bar p}F_{\bar kj}h.
    \end{align*}
    Hence the Bochner-Kodaira formula simplifies to
    \begin{align*}
        \langle\Box\phi,\phi\rangle=||\bar\nabla\phi||^2+\int (F^{m\bar p}+R^{m\bar p})\phi_{\bar p}\overline{\phi_{\bar m}}h\frac{\omega^n}{n!}.
    \end{align*}
    Now if we take $L\mapsto L^M$, $R$ does not change, as it is the Ricci curvature of the metric on the base manifold. On the other hand,
    the curvature $F$ of $L$ is now multiplied by $M$, as the curvature is given by the logarithm above (and hence the power $M$ becomes
    multiplicative). For $L^M\to X$, then, the Bochner-Kodaira formula reads:
    \begin{align*}
        \langle\Box\phi,\phi\rangle=||\bar\nabla\phi||^2+\int (MF^{m\bar p}+R^{m\bar p})\phi_{\bar p}\overline{\phi_{\bar m}}h\frac{\omega^n}{n!}.
    \end{align*}
    In the case of positive line bundle $L$, we have a K\"ahler metric on $X$, which is precisely the curvature of the metric on $L$, $\omega$,
    which yields the fact that
    \[F^{m\bar p}=g^{m\bar p}.\]
    On $L^M\to X$, then, we can further simplify the Bochner-Kodaira formula to
    \begin{align*}
        \langle\Box\phi,\phi\rangle&=||\bar\nabla\phi||^2+\int (Mg^{m\bar p}+R^{m\bar p})\phi_{\bar p}\overline{\phi_{\bar m}}h\frac{\omega^n}{n!}.
    \end{align*}
    Since $g^{m\bar p}>0$, we can choose $M$ large enough such that
    \[Mg^{m\bar p}+R^{m\bar p}>0,\]
    which concludes the proof.
\end{proof}

\begin{exc}
    Derive the Bochner-Kodaira formula for the case of $\Lambda^{0,2}$. It will be good for your soul.
\end{exc}

\subsection{Other Bochner-Kodaira Formulae}

\begin{defn}
    Define the \textbf{Hodge} operator $\Lambda$ as follows.
    Let $\Phi\in\Gamma(X,E\otimes\Lambda^{p+1,q+1})$ for some
    bundle $E\to X$. Then $(\Lambda\Phi)_{\bar KJ}=g^{l\bar p}\Phi^\alpha_{p\bar l\bar KJ}\in\Gamma(X,E\otimes\Lambda^{p,q})$.
\end{defn}

\begin{exc}
    Show that $[\partial,\Lambda]=\bar\partial^\dagger$ and $[\bar\partial,\Lambda]=-\partial^\dagger$.
\end{exc}

Now consider $\bar\Box=\partial\partial^\dagger+\partial^\dagger\partial$ on $\Gamma(X,E\otimes\Lambda^{p,q})$.
\begin{thm}[Kodaira-Akizuki-Nakano]
    Let $L\to X$ be a line bundle. Then
    \[\Box=\bar\Box+[F,\Lambda].\]
\end{thm}
\begin{proof}
    We sketch the proof: 
    \begin{align*}
        \Box-\bar\Box&=[\partial,\Lambda]\bar\partial+\bar\partial[\partial,\Lambda]+[\bar\partial,\Lambda]\partial+\partial[\bar\partial,\Lambda]\\
        &=(\bar\partial\partial+\partial\bar\partial)\Lambda-\Lambda(\bar\partial\partial+\partial\bar\partial)\\
        &=[\bar\partial\partial+\partial\bar\partial,\Lambda]\\
        &=[F,\Lambda]
    \end{align*}
\end{proof}

In practice, we use the following lemma when we apply the KAN theorem, which gives us a handle on positivity.
\begin{lem}
    Let $\omega$ and $F$ be simultaneously diagonalized, i.e. if $\zeta^a=\zeta^a_j dz^j$ is a basis of forms, then
    $\omega=\frac{i}{2}\sum_a\zeta^a\wedge\overline{\zeta^a}$ and $F=\frac{i}{2}\sum_a\lambda_a\zeta^a\wedge\overline{\zeta^a}$.
    Then
    \begin{align*}
        \langle[F,\Lambda]\Psi,\Psi\rangle&=\frac{1}{2}\sum_{KJ}\left(\sum_{a\in J}\lambda_a+\sum_{b\in K}\lambda_b-\sum_{c=1}^n\lambda_c\right)|\Psi_{\bar KJ}|^2.
    \end{align*}
\end{lem}
Positivity can be tested directly from the term in parentheses. In particular, on positive line bundles, the sums are particularly simple: $p+q-n$.
Hence we find via this lemma that for a positive line bundle $L$,
\[\ker\Box\bigg|_{L\otimes\Lambda^{p,q}}=0\]
when $p+q-n>0$.

This concludes the easy part of this theory. Now we will turn to the cohomology of the $\bar\partial$ complexes.

\subsection*{Cohomology}

Let $E\to X$ be a holomorphic line bundle, with $H_{\bar\alpha\beta}$ a metric on $E$ and $g_{\bar kj}$ a (K\"ahler) metric on $X$.
Recall the $\bar\partial$ complex from above. We define the \textbf{cohomology} of the $\bar\partial$ complex as
\begin{align*}
    H^{p,q}_{\bar\partial}(X,E\otimes\Lambda^{p,q})=\ker\bar\partial|_{\Gamma(X,E\otimes\Lambda^{p,q})}/\text{Im }\bar\partial|_{\Gamma(X,E\otimes\Lambda^{p,q-1})}.
\end{align*}
This makes sense because the complex is exact by construction.
This is the analogue of de Rham cohomology, where the \textbf{de Rham cohomology}
\begin{align*}
    H^p_{dR}(X)=\ker d|_{\Gamma(x,\Lambda^p)}/\text{Im }d|_{\Gamma(X,\Lambda^{p-1})}
\end{align*}
is defined for smooth compact manifolds $X$, and carries some sort of topological data of $X$. In our case, we are dealing with complex manifolds
and hence the cohomology of the $\bar\partial$ complex depends sensitively on the complex structure of the manifold.
There are two approaches to computing $H^p(X)$ (or at least, to determine when it is 0), namely Hodge theory and the $L^2$ estimates
of H\"ormander. 
In Hodge theory, the key statment is that $H^{p}_{dR}(X)$ is isomorphic to $\ker\Delta$, where $\Delta$ is the Laplacian $\Delta=dd^\dagger+d^\dagger d$
(having introduced a metric). What is of course remarkable is that the left-hand side of the equality is metric-independent. Then one can use certain
vanishing theorems to show that this kernel, and hence the cohomology group, is zero.
On the other hand, the $L^2$ estimates approach gives conditions under which the equation $d=v$ for $v\in L^2(X,\Lambda^{p+1})$ and $dv=0$, admits solutions.
This approach has intrinsic interest also, however, because this equation involves only one derivative.

The key step in the Hodge theory approach is the construction of an operator $G$ such that
\[G\Delta=\Delta G=I-\Pi\]
where $\Pi:L^2(X,\Lambda^p)\to\ker\Delta$ is the orthogonal projection. Furthermore, $dG=Gd$ and $d^\dagger G=Gd^\dagger$.
Assume the existence of $G$. Then
\begin{align*}
    u=(\Delta g)u+\Pi u=d(d^\dagger Gu)+d^\dagger(dGu)+\Pi u
\end{align*}
for any smooth $u$. If $u\in\ker d$ then $du=0$ and we are left with
\begin{align*}
    u&=d(d^\dagger Gu)+\Pi u\\
    [u]&=[\Pi u].
\end{align*}
But by definition, $\Pi u\in\ker\Delta$ and we are done.
Moreover, suppose we want to solve the $d$ equation $du=v$. Just take $u=d^\dagger Gv$. Then
\[du=dd^\dagger Gv=(\Delta-d^\dagger d)Gv=\Delta Gv=v-\Pi v.\]
Hence if $v$ is of class zero, then $v$ is given by this equation. Otherwise, the equation cannot be solved.

Now how does one prove the existence of $G$? This follows (with more work, of course) from the following estimate
\begin{align*}
    ||u||_{W^{k+2,2}}\leq c(||\Delta u||^2_{W^{k,2}}+||u||^2_{W^{k,2}})
\end{align*}
for all $u\in C^\infty(X,\Lambda^p)$ where we have the Sobolev norm
\begin{align*}
    ||u||^2_{W^{k,p}}\equiv \sum_{|\alpha|\leq k}||D^\alpha u||^2_{L^p}.
\end{align*}
for $p>1$. These kinds of estimates hold for elliptic operators (in this case the Laplacian).

\subsection*{The $L^2$ estimates approach}

Let $L\to X$ be a holomorphic line bundle over $X$ a compact complex manifold and let $h$ be a metric on $L$
and $g_{\bar kj}$ to be K\"ahler. We will in fact work with distributions instead of functions. Consider
$L^1_{loc}(\R^n)$. Let $f\in L^1_{loc}(\R^n).$ The derivative $\partial f/\partial x_1$ in the sense of
distributions is the functional
\begin{align*}
    C^\infty_0(\R^n)\ni\phi\mapsto -\int f\frac{\partial\phi}{\partial x_1}
\end{align*}
If $f\in C^1_{loc}(\R^n)$ we can integrate by parts and the functional becomes
\[C^\infty_0(\R^n)\ni\phi\mapsto \int\frac{\partial f}{\partial x_1}\phi.\]
This gives us a generalized way of thinking about the derivative. Hence we define the domain of $\bar\partial$
to be \[\text{Dom }\bar\partial_{p,q}=\left\{\phi\in L^2(X,L\otimes\Lambda^{p,q})\mid\exists\psi\in L^2(X,L\otimes\Lambda^{p,q+1})\text{ with }\bar\partial\phi=\psi\right\},\]
where the equality is taken in the sense of distributions. Similarly, we must define the domain of $\bar\partial^\dagger$:
\[\text{Dom }\bar\partial^\dagger=\left\{ \psi\in L^2(X,L\otimes\Lambda^{p,q+1})\mid \exists \phi\in L^2(X,L\otimes\Lambda^{p,q})\text{ with }\bar\partial^\dagger\psi=\phi \right\}\]
again where the equality is taken in the sense of distributions as well as $\langle\bar\partial\lambda,\psi\rangle=\langle\lambda,\phi\rangle$ for all $\lambda\in\text{Dom }\bar\partial$.
Restricting this extra equation to $\lambda\in C^\infty_0$, we obtain the weaker condition that $\bar\partial^\dagger\psi=\phi$ in the sense of distributions.

Our main goal is to solve $\bar\partial u=f$, where $f\in L^2(X,L\otimes\Lambda^{0,1})$.
The key lemma is as follows.
\begin{lem}
 Let $g_{\bar kj}$ be a K\"ahler metric on $X$ and $h$ a metric on $L$. Assume that the following
inequality holds:
\[||\bar\partial u||^2_{L^2}+||\bar\partial^\dagger u||^2_{L^2}\geq\int_X \langle Au,u\rangle\]
for all $u\in\text{Dom }\bar\partial_1\cap\text{Dom }\bar\partial^\dagger_0$ and $A$ is a positive definite matrix.
Then, for all $f\in L^2(X,L\otimes \Lambda^{0,1})$, with $\bar\partial f=0$ in the sense of distributions, and there exists $v\in L^2(X,L)$
satisfying $\bar\partial v=f$ and $\int_x|v|^2\leq\int_X\langle A^{-1}f,f\rangle$. 
\end{lem}
Here we are distinguishing the two different $\bar\partial$ operators
that are present. More explicitly, we write $u=\sum u_{\bar j}d\bar z^j$ with $u_{\bar j}$ a section of $L$ and
\[\langle Au,u\rangle=A^{l\bar j}u_{\bar j}\overline{u_{\bar l}}h.\]
Then
\[\int_X \langle Au,u\rangle=\int_X A^{l\bar j}u_{\bar j}\overline{u_{\bar l}}h\frac{\omega^n}{n!}\]
Furthermore, $v\in L^2(X,L\otimes \Lambda^{0,0})$ and hence \[\int|v|^2=\int |v|^2h\frac{\omega^n}{n^!}\]
and
\[\int\langle A^{-1}f,f\rangle=\int (A^{-1})^{l\bar j}f_{\bar j}\overline{f_{\bar l}}h\frac{\omega^n}{n!}.\]
Hence we can write the restriction on $v$ more explicitly as
\[\int |v|^2h\frac{\omega^n}{n!}\leq \int (A^{-1})^{l\bar j}f_{\bar j}\overline{f_{\bar l}}h\frac{\omega^n}{n!}.\]

We use the following easy fact from functional analysis.
\begin{lem}
    Let $\phi\in\text{Dom }\bar\partial_0^\dagger$, and let $\phi=\phi_1+\phi_2$ where $\phi_1\in\ker\bar\partial_1$
    and $\phi_2\in(\ker\bar\partial_1)^\perp$. Then $\phi_1,\phi_2\in\text{Dom }\bar\partial_0^\dagger$.
\end{lem}
Note that the decomposition in the lemma makes sense because $\ker\bar\partial_1$ is a closed subspace of $L^2$.
\begin{proof}
    Since $\bar\partial_1\bar\partial_0=0$, it follows that the range of $\bar\partial_0$ is contained in $\ker\bar\partial_1$.
    This implies that if $\phi_2\perp\ker\bar\partial_1$ then $\phi_2\perp(\text{Range }\bar\partial_0)^\perp$, i.e. perpendicular
    to the space $0=\langle\bar\partial_0\psi,\phi\rangle$ for $\psi\in\text{Dom }\bar\partial_0$. This in turn means that
    $\langle\bar\partial_0\psi,\phi_2\rangle=\langle\psi,0\rangle$ for all $\psi\in\text{Dom }\bar\partial$, and thus that
    $\phi_2\in\text{Dom }\bar\partial_0^\dagger$ and $\bar\partial_0^\dagger\phi_2=0$.
\end{proof}

\begin{proof}[Proof of earlier lemma]
    Consider the functional $T$ defined by: $\bar\partial_0^\dagger\phi\mapsto\langle\phi,f\rangle$ for $\phi\in\text{Dom }\bar\partial_0^\dagger$.
    Note this is defined only on a subspace.
    It is not \textit{a priori} obvious that this is well defined - there might be multiple such $\phi$. Decompose $\phi=\phi_1+\phi_2$
    with $\phi_2\in\ker\bar\partial_1,\phi_2\perp\ker\bar\partial_1$. We can write 
    \begin{align*}
        \langle\phi,f\rangle&=\langle\phi_1,f\rangle+\langle\phi_2,f\rangle\\
        &=\langle\phi_1,f\rangle.
    \end{align*}
    By the lemma just proved, we see that $\phi_1\in\text{Dom }\bar\partial_0^\dagger\cap\text{Dom }\bar\partial_1$.
    Using Cauchy-Schwarz with respect to the $L^2$ norm determined by $A$ (it is positive definite), we find
    \begin{align*}
        |\langle\phi_1,f\rangle|^2&\leq\left( \int\langle A\phi_1,\phi_1\rangle \right)\left( \int\langle A^{-1}f,f\rangle\right)\\
        &\leq\left( ||\bar\partial\phi_1||^2+||\bar\partial^\dagger\phi_1||^2 \right)\left( \int\langle A^{-1}f,f\rangle\right)\\
        =||\bar\partial^\dagger\phi_1||^2\int\langle A^{-1}f,f\rangle.
    \end{align*}
    This shows that the functional defined above does indeed make sense, once in the context of the estimates of the hypothesis.
    Now, by the Hahn-Banach theorem stated below, applied to this functional, we obtain a functional $\tilde T:L^2\to \C$
    extending $T$ such that $||\tilde T||=||T||$. This implies that there exists a $u\in L^2$ such that $\tilde T(\psi)=\langle\psi,u\rangle$
    and $||u||=||\tilde T||$ for any $\psi\in L^2$. In particular, take $\psi=\bar\partial_0^\dagger\phi$ with $\phi\in C_0^\infty$.
    Then $\langle\bar\partial_0^\dagger\phi_1,u\rangle=T(\bar\partial_0^\dagger\phi)=\langle\phi,f\rangle$, which is equivalent
    to the fact that $\bar\partial u=f$ in the sense of distributions.
\end{proof}

\begin{thm}[Hahn-Banach]
    Let $V\subset\mathcal{B}$ a Banach space and $T$ be a linear functional $V\ni v\mapsto T(v)$
    with $|T(v)|\leq A||v||$. Then there exists an extension $\tilde T$ of $T$ to the whole of $\mathcal{B}$,
    such that $\tilde T$ satisfies $|T(v)|\leq A||v||$ for all $v\in\mathcal{B}$.
\end{thm}
\begin{proof}
    Omitted.
\end{proof}

When do the estimates in the hypothesis of the lemma hold? Recall the Bochner-Kodaira formula on $C^\infty(X,L\otimes\Lambda^{0,1})$,
\begin{align*}
    \langle\Box\phi,\phi\rangle=||\bar\partial \phi||^2+||\bar\partial^\dagger \phi||^2
\end{align*}
where $\Box=\bar\partial\bar\partial^\dagger+\bar\partial^\dagger\bar\partial$. Acting on smooth forms, we found
\begin{align*}
    \langle\Box\phi,\phi\rangle=||\nabla_{\bar k}\phi_{\bar j}||^2_{L^2}+\int(F_{\bar kj}+R_{\bar kj})\phi^j\overline{\phi^k}h\frac{\omega^n}{n!}.
\end{align*}
Recall that $F_{\bar kj}=-\partial_j\partial_{\bar k}\log h$ and $R_{\bar kj}=R_{\bar kj}P_p=R_{\bar kp}P_j$.
Note that the above lemma will hold if $F_{\bar kj}+R_{\bar kj}>0$ and the inequality required by the lemma extends
from $C_0^\infty$ to $\text{Dom }\bar\partial_0\cap\text{Dom }\bar\partial_1^\dagger$.

Note first that on a compact manifold $C^\infty_0$ is dense in $\text{Dom }\bar\partial_0\cap\text{Dom }\bar\partial_1^\dagger$ with respect
to the norm $||u||_{L^2}+||\bar\partial u||_{L^2}+||\bar\partial^\dagger u||_{L^2}$.

\begin{thm}
    Let $L\to X,h,g_{\bar kj}$ be as above. Writing $h=e^{-\phi}$, assume that
    \begin{align*}
        -\partial_j\partial_k\phi+ R_{\bar kj}\geq\epsilon g_{\bar kj}
    \end{align*}
    for some $\epsilon>0$. Then for any $f\in L^2(X,L\otimes\Lambda^{0,1})$ with $\bar\partial f=0$, there exists
    $u\in L^2(X,L)$ solving $\bar\partial u=f$ and
    \[\int |u|^2e^{-\phi}\frac{\omega^n}{n!}\leq\frac{1}{\epsilon}\int g^{k\bar j}f_{\bar j}\overline{f_k}e^{-\phi}\frac{\omega^n}{n!}.\]
\end{thm}

Note that a useful variation of this setup is to solve the equation $\bar\partial u=f$ for $f\in L^2(X,L\otimes\Lambda^{n,1})$. Why?
Observe that if $u\in L^2(X,L\otimes\Lambda^{n,0})$, then $\int|u|^2_h\equiv\int u\bar ue^{-\phi}$. Note that here there is no volume form
needed, as the integrand is an $n,n$ form. Hence there is no need for a metric, and the estimates tend to be better in this case.
Note also that $L\otimes\Lambda^{n,1}=(L\otimes\Lambda^{n,0})\otimes\Lambda^{0,1}$. But $L'=L\otimes\Lambda^{n,0}$ is simply
another holomorphic line bundle. Hence we can apply our previous theorem with $L$ replaced by $L'$. Thus we impose
$\varepsilon g_{\bar kj}\leq F'_{\bar k}j+R_{\bar kj}=(F_{\bar kj}-R_{\bar kj})+R_{\bar kj}=F_{\bar kj}$ (the curvature of $\Lambda^{n,0}$ is negative
the Ricci curvature). Hence we obtain the theorem above but for this case of $f\in L^2(X,L\otimes\Lambda^{n,1})$.

\begin{exc}
    Write this theorem down carefully and supply the missing steps.
\end{exc}

Observe that this in fact extends to $X$ not compact, but complete as a metric space, i.e.
the estimate in the hypothesis still extends from $C^\infty_0$ to $\text{Dom }\bar\partial_0^\dagger\cap\text{Dom }\bar\partial_1$.
Even further, if $(X,g_{\bar kj})$ is not necessarily complete, but instead there exists a metric $g'_{\bar kj}$ that is K\"ahler
and complete, then one applies the theorem to $(X,g^\delta_{\bar kj}=g_{\bar kj}+\delta g'_{\bar kj}).$ Applying the theorem
now, we obtain a sequence $u_\delta$, uniformly bounded, which weakly converges to a solution $u$ (c.f. J.P. Demailly's online
book).


\end{document}
