\documentclass{../mathnotes}

\usepackage{tikz-cd}
\usepackage{amsmath}
\usepackage{todonotes}


\title{Complex Analysis and Riemann Surfaces: Midterm}
\author{Nilay Kumar}
\date{Last updated: \today}


\begin{document}

\maketitle

\section*{Problem 1}

Consider the function $w=\sqrt{\prod_{j=1}^5(z-a_j)}$, where $a_j$ (for $j=1,\ldots,5$) are pairwise distinct complex numbers. Let us order $a_j$
increasing in magnitude.
Let us construct the Riemann surface of $w$. We start by considering the behavior of $w$ in the complex plane $\C$. Note first that $w$
is clearly well-defined and holomorphic in the disc $D_{|a_1|}(0)$ (of course, if $a_1=0$, this is not true), because we can write
\[w=\exp\left( \frac{1}{2}\sum_j \ln(z-a_j) \right),\]
where, as usual, we view the first logarithm with a branch cut along the ray incident with the origin and $a_1$. Note, however, that
in the annulus $D_{|a_2|}(0)-\overline{D_{|a_1|}(0)}$, $w$ is not holomorphic, as we obtain a discontinuity along a curve from $a_1$ to $a_2$
(the curve depends on the branch cut chosen for $\ln(z-a_1)$, so for our purposes we will keep it simple and pick such that the discontinuities are straight lines):
along this discontinuity it is easy to see that $w$ jumps to $-w$. One might expect to encounter a similar discontinuity in the annulus $D_{|a_3|}(0)-\overline{D_{|a_2|}(0)}$,
but in fact, because both $\sqrt{z-a_1}$ and $\sqrt{z-a_2}$ take $w\mapsto -w$, the signs cancel out and hence $w$ is in fact holomorphic in that domain.
Going out a little further, we again will see a sign discontinuity (as the signs do not cancel out this time) in the annulus $D_{|a_4|}(0)-\overline{D_{|a_3|}(0)}$,
none in $D_{|a_5|}(0)-\overline{D_{|a_4|}(0)}$, and a sign discontinuity starting from $a_5$ and going out to infinity.

Thus to maximize the domain of definition one which $w$ is holomorphic, we take two copies of the ``cut'' complex plane that we have been discussing - call them $\C_1$
and $\C_2$ on which we define $w=\sqrt{\prod_{j=1}^5(z-a_j)}$ and $w=-\sqrt{\prod_{j=1}^5(z-a_j)}$ respectively. We can now glue the two planes together in a way
such that $w$ will, in fact, be continuous, as the sign changes will now be compensated for. Before we do this, let us simplify the visual picture by
thinking in terms of stereographic projections - each copy of the plane can be seen as topologically just the Riemann sphere with some arcs and the north pole cut out.
Expanding the cuts to holes continuously we see that gluing together $\C_1$ and $\C_2$ in this sense will yield a surface with 2 handles. We claim that the resulting
topology $\Sigma$ is in fact a one-dimensional complex manifold in the sense that it is locally homeomorphic to $\C$. In fact, we can even compactify by adding
the north pole (as the point at $\infty$) as we did in class, resulting in the compact surface $\hat\Sigma$, which is also topologically a manifold. To see this,
let us examine the (structure of) open neighborhoods of varous points in $\hat\Sigma$. There are four cases: away from any cuts, at an $a_i$, on a cut, and at $\infty$.

It is clear that points away from discontinuities admit neighbhorhoods that can be holomorphically (and in a 1-to-1 fashion) mapped to a disc in the complex plane.
Next, consider the case where we have a point $z_0$ physically on the cut. Then we can simply take two half-discs (one from each copy) and glue them together - this clearly
yields a disc. Furthermore, if $z_0$ is one of the points $a_i$, we see that there is an almost-full disc about $z_0$ contained in each copy $\C_i$. To glue these together 
we have to be a little more careful, and we construct our chart to send a $z$ in the neighborhood of $z_0$ to $t=\sqrt{z-a_i}$ on $\C_1$ and $t=-\sqrt{z-a_i}$ on $\C_2$.
In other words, we are taking these almost-full discs to half discs at the origin in $\C$ and gluing them together to obtain a full disc. Finally we must deal with the
point at infinity. This is done almost identically as was done with the $a_i$, except we take a reciprocal for the coordinates to get a well-defined disc: $1/\sqrt{z}$ on $\C_1$
and $-1/\sqrt{z}$ on $\C_2$.

Note that we have two meromorphic functions defined on $\hat\Sigma$. The first is simply $w$, which flips sign under the involution that switches $\C_1$ and $\C_2$. 
Note that $w$ has a pole of order 5 at infinity:
\[w=\sqrt{\prod_{j=1}^5(z-a_j)}=\sqrt{\prod_{j=1}^5\left(\frac{1}{t^2}-a_j\right)}=\frac{1}{t^5}\sqrt{\prod_{j=1}^5\left( 1-t^2a_j \right)}.\]
Furthermore, $w$ has a simple zero at each $a_i$, because near $a_i$ we see that:
\[w=\sqrt{\prod_{j=1}^5(z-a_j)}=\sqrt{\prod_{j=1}^5\left(t^2+a_i-a_j\right)}=t\sqrt{\prod_{j\neq i}^5\left( t^2+a_i-a_j \right)},\]
which is an order one zero because what is left is holomorphic (in the local coordinate patch) and because the $a_i$ are distinct.
Similarly, the function $z$ at infinity is locally $1/t^2$ and hence has a pole of order 2, and near zero is $t^2$ and hence has a zero of order 2.

Let us now show that the forms given by $\omega_1=dz/w$ and $\omega_2=zdz/w$ are holomorphic. First, near an $a_i$, since $z=t^2+a_i$, we find that
\begin{align*}
    \omega_1=\frac{dz}{w}=\frac{2tdt}{t\sqrt{\prod_{j\neq i}\left( t^2+a_i-a_j \right)}}=\frac{2dt}{\sqrt{\prod_{j\neq i}\left( t^2+a_i-a_j \right)}},
\end{align*}
which is clearly holomorphic in this local chart. On the other hand, near infinity we have that $z=1/t^2$ and hence we get
\begin{align*}
    \omega_1=\frac{dz}{w}=\frac{-2dt/t^3}{\frac{1}{t^5}\sqrt{\prod_{j=1}^5\left( 1-t^2a_j \right)}}&=\frac{-2t^2dt}{\sqrt{\prod_{j=1}^5(1-t^2a_j)}},
\end{align*}
which is holomorphic in the chart near infinity. Similarly, for $\omega_2$, near $a_i$ and $\infty$:
\begin{align*}
    \omega_2&=\frac{zdz}{w}=\frac{t^2\cdot 2tdt}{t\sqrt{\prod_{j\neq i}\left( t^2+a_i-a_j \right)}}=\frac{2t^2dt}{\sqrt{\prod_{j\neq i}\left( t^2+a_i-a_j \right)}}\\
    \omega_2&=\frac{zdz}{w}=\frac{1/t^2\cdot-2dt/t^3}{\frac{1}{t^5}\sqrt{\prod_{j=1}^5\left( 1-t^2a_j \right)}}=\frac{-2dt}{\sqrt{\prod_{j=1}^5(1-t^2a_j)}},
\end{align*}
both of which are clearly holomorphic. Hence, since these forms are holomorphic in their local coordinate charts, they define holomorphic forms on $\hat\Sigma$.

We can, in fact, construct a meromorphic form $\omega_P(z)$ on $\hat\Sigma$ which has a double pole at the point $P=a_i$ but is holomorphic everywhere else.
To do so, note that near $a_i$, the function $z$ takes the form $t^2+a_i$; hence if we take a holomorphic form and multiply by $1/(z-z(a_i))$ the $a_i$'s will
cancel and yield a double pole at the coordinate $t=0$, i.e. $P=a_i$. In particular, consider
\[\omega_P(z)=\frac{1}{z-z(a_i)}\omega_1(z)=\frac{1}{z-z(a_i)}\frac{dz}{w}.\]
In coordinates near $a_i$, it's clear (by looking at the equation for $\omega_1$ in coordinates above) that we have a double pole. Near another $a_j$, we
have neither a zero or a pole as the $a_i$ are all distinct (and so the denominator out front does not blow up). Finally, at infinity, the factor out front
becomes $1/\left( 1/t^2-a_i \right)=t^2/\left( 1-a_it^2 \right)$, which has a zero of order 2 at infinity (while the factor of $\omega_1$ does not vanish as
one can check above). Hence $\omega_P(z)$ is indeed a meromorphic form with a double pole at $a_i$.

Finally, note that although we have constructed a meromorphic form with a double pole at one point, it is impossible for a meromorphic form on $\hat\Sigma$ to have a single pole.
This follows from the statement we proved in class that any meromorphic form $\omega$ on a compact Riemann surface must have residues that add up to zero; of course,
if a form has only one pole, the residue will be nonzero, which is a direct contradiction. We proved this by applying Stoke's theorem the integral of $\omega$
about the boundary of the complement of the union of little neighborhoods around the poles (since $\omega$ is holomorphic there).

\section*{Problem 2}

Let $\Lambda=\left\{ m\omega_1+n\omega_2;m,n\in\Z \right\}$ be the lattice generated by $\omega_1,\omega_2$, where $\omega_1,\omega_2$ are complex numbers
which are linearly independent over $\R$ and let $\Lambda^\times=\Lambda\setminus 0$. Define the function $\sigma(z)$ on $\C$ by
\[\sigma(z)=z\prod_{\omega\in\Lambda^\times}\left( 1- \frac{z}{w}\right)e^{\frac{z}{w}+\frac{z^2}{2w^2}}.\]
Let us first show that $\sigma$ is well-defined in that it converges. To check if this product converges, we take a logarithm (picking a branch cut on $\C$):
\begin{align*}
    \log\sigma(z)&=\log z+\sum_{w\in\Lambda^\times}\left( \log\left(1-\frac{z}{w}\right)+\frac{z}{w}+\frac{z^2}{2w^2} \right)\\
    &=\log z+\sum_{w\in\Lambda^\times}\left( \frac{z}{w}-\frac{1}{2}\frac{z^2}{w^2}+O(\frac{1}{|w|^3})+\frac{z}{w}+\frac{z^2}{2w^2} \right)\\
    &=\log z+\sum_{w\in\Lambda^\times}O(\frac{1}{|w|^3}),
\end{align*}
which clearly converges. Note that here we have used the Taylor expansion for the logarithm. One may inquire as to the zeroes of $\sigma(z)$ - if $z=0$ or
any term in the product is zero, the whole function $\sigma(z)$ goes to zero. This, of course, only happens when $z=w$, i.e. $z=0\mod\Lambda$.
Next let us determine the transformation law for $\sigma(z)$ when $z\mapsto z+\omega_a$ for $a=1,2$. Recall how we defined the function
\[\zeta(z)=\frac{1}{z}+\sum_{w\in\Lambda^\times}\left( \frac{1}{z+w}-\frac{1}{w}+\frac{z}{w^2} \right).\]
From this, it is easy to see that we can write $\zeta(z)=\sigma'(z)/\sigma(z)$. But note that the periodicity condition for $\zeta$ we have that
$\zeta(z+\omega_a)-\zeta(z)=\eta_a$, which becomes
\[\eta=\frac{\sigma'(z+\omega_a)}{\sigma(z+\omega_a)}-\frac{\sigma'(z)}{\sigma(z)}=\partial_z\log\sigma(z+\omega_a)-\partial_z\log\sigma(z).\]
Integrating both sides with respect to $z$ and exponentiating, we find that
\[\sigma(z+\omega_a)=\sigma(z)e^{\eta_az+c_a}\]
with $c_a$ a constant of integration, which we can find by taking $z=-\omega_a/2$ and using the fact that $\omega$ is odd,
\[\sigma(\omega_a/2)=\sigma(-\omega_a/2)e^{-\eta_a\omega_a/2+c_a},\]
which yields
\[\sigma(z+\omega_a)=-\sigma(z)e^{\eta_a(z+\omega_a/2)}.\]
Let us now give another proof of Abel's theorem (for $\hat\Sigma=\C/\Lambda)$. First recall our previous statement of
Abel's theorem.

\begin{thm}[Abel's theorem]
    Let $P_1,\ldots,P_M,Q_1,\ldots,Q_N$ be points in $\C$. Then there exists a meromorphic $f$ with zeroes at $P_i$ and poles at $Q_i$
    if and only if $M=N$ and $\sum_{i=1}^MA(P_i)=\sum_{i=1}^NA(Q_i)$.
\end{thm}
Recall that the Abel map takes $\C/\Lambda\ni p\mapsto A(p)=\int_{p_0}^p\omega$ where the value of the integral is taken modulo the lattice
generated by $\oint_A\omega,\oint_B\omega$. Take $p_0=0$ and $\omega=dz$, which is a well-defined form, and if we take $A$ to align with $\omega_2$
and $B$ to align with $\omega_1$, we see that $\oint_A\omega=\oint_A dz=\omega_1$ and similarly $\oint_B\omega=\omega_2$. Hence the map simply
takes $p$ to $\int_0^p dz \mod\Lambda=p$ where $p$ is viewed as a complex number.

Let us now restate Abel's theorem in the context of this fact.
\begin{thm}[Abel's theorem, v.2]
    Let $P_1,\ldots,P_M,Q_1,\ldots,Q_N$ be points in $\C$. Then there exists a meromorphic $f$ with zeroes at $P_i$ and poles at $Q_i$
    if and only if $M=N$ and $\sum_{i=1}^M P_i=\sum_{i=1}^NQ_i\mod\Lambda$.
\end{thm}
\begin{proof}
    Consider the function
    \begin{align*}
        f(z)=\frac{\prod_{i=1}^M\sigma(z-P_i)}{\prod_{i=1}^N\sigma(z-Q_i)}.
    \end{align*}
    We should be a little careful to note that $\sigma$ is a function not on the torus $\C/\Lambda$, but a function on $\C$ (it transforms under a lattice
    translation!). Hence we must be cognizant of the fact that $P_i,Q_i$ here are some chosen representatives in $\C$ of the equivalence classes of the points $P_i,Q_i$.
    It should be clear that $f(z)$ is meromorphic with zeroes at every representative of each $P_i$s and poles at every representative of each $Q_i$.
    The natural question, now, is whether this function extends to a function on the torus. 
    To check this, let us see whether it is doubly periodic using what we know about $\sigma$:
    \begin{align*}
        f(z+\omega_a)&=f(z)\frac{\prod_{i=1}^Me^{\eta_a(z-P_i)}}{\prod_{i=1}^Ne^{\eta_a(z-Q_i)}}\\
        &=f(z)e^{-\eta_a\left( \sum_{i=1}^MP_i-\sum_{i=1}^NQ_i \right)}.
    \end{align*}
    Hence we wish to choose $P_i,Q_i$ representatives such that the exponential becomes unity. By hypothesis, this can be done (by shifting one, if necessary).
\end{proof}

\section*{Problem 3}

Let now $\omega_1=1,\omega_2=\tau,$ with $\text{Im }\tau>0$. Define the function
\[\theta_1(z|\tau)=\sum_{n\in\Z}\exp\left( \pi i\left( n+\frac{1}{2} \right)^2\tau+2\pi i\left( n+\frac{1}{2} \right)\left( z+\frac{1}{2} \right) \right).\]
The presence of an $(n+1/2)^2$ in the first term of the exponential dominates, as for large $|n|$ it is the leading order term, and it decays rapidly. More explicitly,
we have the $n^2$ terms
\[|e^{\pi in^2(\tau_1+i\tau_2)}|=|e^{\pi in^2\tau_1}e^{-\pi n^2\tau_2}|=e^{-\pi n^2\tau_2},\]
where we have written $\tau=\tau_1+i\tau_2$. Due to this decay, it's clear that the sum converges (one might use the integral test) and hence $\theta_1(z|\tau)$ is holomorphic. Next let us show
that $\theta_1(z|\tau)=0$ only at $z=0\mod\Lambda$.
Hence let us verify that $\theta_1$ is odd; switching $z\mapsto-z$ yields in the exponent
\begin{align*}
    \log\theta_1(z|\tau)=\pi i\left( n+\frac{1}{2} \right)^2\tau+2\pi i\left( n+\frac{1}{2} \right)\left( -z+\frac{1}{2} \right).
\end{align*}
If we switch the indices $n\mapsto m$ such that $n+\frac{1}{2}=-\left( m+\frac{1}{2} \right)$, we find that the exponent is now
\begin{align*}
    \log\theta_1(z|\tau)=\pi i\left( m+\frac{1}{2} \right)^2\tau+2\pi i\left( m+\frac{1}{2} \right)\left(\left( z+\frac{1}{2} \right)-2\pi i\left( m+\frac{1}{2} \right)\right),
\end{align*}
and hence $\theta_1$ is odd. To see that this is indeed the only zero, we recall that we can compute an integral to count the number of zeroes
enclosed (as $\theta_1$ is holomorphic):
\begin{align*}
    \oint_C \frac{\theta_1'(z,\tau)}{\theta_1(z, \tau)}  dz  &=  \oint_B \left( - \frac{\theta_1'(z,\tau)}{\theta_1(z, \tau)} + \frac{\theta_1'(z+1,\tau)}{\theta_1(z+1, \tau)} \right) dz   +  \oint_A \left( \frac{\theta_1'(z,\tau)}{\theta_1(z, \tau)} - \frac{\theta_1'(z+ \tau,\tau)}{\theta_1(z+ \tau, \tau)} \right) dz\\
    &=2\pi i\oint_A dz=2\pi i,
\end{align*}
where we have used the transformation properties of $\theta_1$ that we will derive shortly. But recall that this integral gives us $2\pi i$ times the number of zeroes (for a holomorphic
function) and hence $z=0\mod\Lambda$ is in fact the only zero.


Let us compute two transformation properties of the $\theta_1$ function that we used above (and will use below). First note that we can write
\[\theta_1(z)=e^{\pi i/2}\sum_{n=-\infty}^\infty\exp\left( i\pi\tau(n+\frac{1}{2})^2+2\pi i(n+\frac{1}{2})z \right)(-1)^n.\]
Then we can compute what happens under translations:
\begin{align*}
    \theta_1(z+1)&=e^{\pi i/2}\sum_{n=-\infty}^\infty\exp\left( i\pi\tau(n+\frac{1}{2})^2+2\pi i(n+\frac{1}{2})z +2\pi i(n+\frac{1}{2})\right)(-1)^n\\
    &=-e^{\pi i/2}\sum_{n=-\infty}^\infty\exp\left( i\pi\tau(n+\frac{1}{2})^2+2\pi i(n+\frac{1}{2})z \right)(-1)^n\\
    &=-\theta_1(z)\\
    \theta_1(z+\tau)&=e^{\pi i/2}\sum_{n=-\infty}^\infty\exp\left( i\pi\tau(n+\frac{1}{2})^2+2\pi i(n+\frac{1}{2})(z+\tau) \right)(-1)^n\\
    &=-e^{\pi i/2}\sum_{n=-\infty}^\infty\exp\left( i\pi\tau(n+\frac{1}{2})^2+2\pi i\tau(n+\frac{1}{2})+2\pi i(n+\frac{1}{2}+1)z-2\pi iz \right)(-1)^{n+1}\\
    &=-e^{\pi i/2}e^{-\pi i\tau-2\pi iz}\sum_{n=-\infty}^\infty\exp\left( i\pi\tau(n+\frac{1}{2}+1)^2+2\pi i(n+\frac{1}{2}+1)z\right)(-1)^{n+1}\\
    &=-e^{-\pi i\tau-2\pi i z}\theta_1(z)
\end{align*}


Let us now prove Abel's theorem using the machinery of $\theta$-functions. In other words, $\sum_{i=1}^NA(P_i)=\sum_{j=1}^NA(Q_j)$ if and only if
there exists an $f$ meromorphic with zeros at $P_i$ and poles at $Q_j$. Note first that we can write $\sum_i P_i=\sum_j Q_j+n+m\tau$. To construct
$f$, let us try
\[f(z)=\frac{\prod_{i=1}^N\theta_1(z-P_i)}{\prod_{j=1}^N\theta_1(z-Q_j)}.\]
From the result about the zeros of $\theta_1$ above, it is clear that this function should have the appropriate zero and pole behavior. It remains to
check that $\theta_1$ is well-defined on $\C/\Lambda$, i.e. doubly-periodic. First note that, using the periodicity properties of the $\theta_1$ function,
\begin{align*}
    f(z+1)&=\frac{\prod_{i=1}^N\theta_1(z+1-P_i)}{\prod_{j=1}^N\theta_1(z+1-Q_j)}=\frac{(-1)^N\prod_{i=1}^N\theta_1(z-P_i)}{(-1)^N\prod_{j=1}^N\theta_1(z-Q_j)}\\
    &=f(z).
\end{align*}
Next,
\begin{align*}
    f(z+\tau)&=\frac{\prod_{i=1}^N\theta_1\left( z+\tau-P_i \right)}{\prod_{j=1}^N\theta_1(z+\tau-Q_j)}=\frac{(-1)^N\prod_{i=1}^N\theta_1\left( z-P_i \right)e^{-2\pi i(z-P_i)-\pi i\tau}}{(-1)^N\prod_{j=1}^N\theta_1\left( z-Q_j \right)e^{-2\pi i(z-Q_j)-\pi i\tau}}\\
    &=e^{2\pi i(\sum_j Q_j-\sum_i P_i)}f(z).
\end{align*}
Hence we may replace one of the points, say $Q_1$, by $\tilde Q_1+n+m\tau$, thus enforcing $\sum_i P_i=\sum_jQ_j$ and the result follows.

\section*{Problem 4}

Let $L\to X$ be a holomorphic line bundle over a compact Riemann surface $X$, defined by the transition functions $t_{\alpha\beta}(z)$ on $X_\alpha\cap X_\beta$,
where $X=\cup_{\alpha}X_\alpha$ is a covering of $X$ by holomorphic charts. A smooth metric $h$ on $L$ is a smooth section of $L^{-1}\otimes \overline{L}^{-1}$ that is strictly positive.
Given any two metrics, $h$ and $h'$, it is clear that we can relate them by $h'=e^{-\phi}h$ for some smooth $\phi$ a scalar function on $X$. This follows simply because we may choose
$\phi=-\log(h'/h)$. As $h,h'$ are strictly positive everywhere, this is well-defined and $\phi$ is smooth.

Next let $F_{\bar zz},F_{\bar zz}'$ be the curvatures of $L$ with respect to the metrics $h$ and $h'$ respectively. Recall that the curvature is given explicitly as
$F_{\bar zz}=-\partial_{\bar z}\Gamma=-\partial_{\bar z}\partial_{z}\log h$ in terms of the metric $h$. Writing it instead in terms of the curvature $h'=e^{-\phi}h$,
we find that
\begin{align*}
    F_{\bar zz}'&=-\partial_{\bar z}\partial_{z}\log\left( e^{-\phi}h \right)=\partial_{\bar z}\partial_z\phi+F_{\bar zz}
\end{align*}
If we now compute the Chern class $c_1(L)$ using each of these curvatures we find that
\begin{align*}
    c_1(L)&=\frac{i}{2\pi}\int_X F_{\bar zz}dz\wedge d\bar z\\
    c_1'(L)=\frac{i}{2\pi}\int_X F_{\bar zz}'dz\wedge d\bar z&=\frac{i}{2\pi}\int_XF_{\bar zz}+\partial_{\bar z}\partial_z\phi dz\wedge d\bar z\\
    &=c_1(L)+\frac{i}{2\pi}\int_X \partial_{\bar z}\partial_z\phi dz\wedge d\bar z\\
    &=c_1(L)+\frac{i}{2\pi}\int_X d\left( \partial_{\bar z}\phi d\bar z \right)\\
    &=c_1(L)
\end{align*}
and hence the curvature is independent of the choice of metric.

\section*{Problem 5}

\begin{thm}[Riemann-Roch]
    Let $X$ be a Riemann surface and $H^0(X,L)$ denote the space of holomorphic sections of a line bundle $L$ over $X$. Furthermore let $K_X$ be the canonical line bundle over $X$ and
    $c_1(L)$ be the first Chern class of a line bundle $L$. Then
    \[\dim H^0(X,L)-\dim H^0(X,K_X\otimes L^{-1})=c_1(L)+\frac{1}{2}c_1(K_X^{-1}).\]
\end{thm}

To deduce that a line bundle $L$ over $X$ always admits non-trivial meromorphic sections, recall the construction of point bundles. For some $p\in X$
pick a coordinate system in a neighborhood $X_0$ of $p$, and set $X_\infty=X\setminus\{p\}$. Let $L$ be $\{t_{0\infty}(z)=z\text{ on }X_0\cap X_\infty\}$. This defines
a holomorphic line bundle which admits a holomorphic section $1_p=1$ on $X_\infty$; $z$ on $X_0$, and $1_p|_{X_0}=z=z\cdot 1=t_{0\infty}\cdot 1_p|_{x_\infty}$.
Hence we see that $1_p$ is a holomorphic section of $L\equiv [p]$ and has exactly one zero at $p$. In particular, $c_1(L)=1$ here. Similarly, we may define, for any
integer $n$, the bundle $[np]$ by the transition function $\left\{ z^n \right\}$. It is easy to see that $c_1([np])=n$. Let us now proceed to prove the existence of
a non-trivial section. Pick a point $p$ and consider the bundle $L\otimes [np]$. It's clear that $c_1(L\otimes [np])=c_1(L)+n$. Then, by the Riemann-Roch theorem,
we find that
\[\dim H^0(X,L\otimes [np])-\dim H^0(X,L^{-1}\otimes [-np]\otimes K)=c_1(L)+n-\frac{1}{2}\chi(X)\]
where $\chi$ is the Euler characteristic as usual. But now we can take $n\gg 0$ arbitrarily large, and since dimensions are positive and the Chern class
and Euler characteristic are independent of the point bundle chosen, we see that
\[\dim H^0(X,L\otimes[np])>0,\]
which implies that there exists a non-trivial holomorphic section of $L\otimes [np]$. Finally we note that multiplying this section by $1_{-np}$ yields a non-zero
meromorphic section of $L$, and we are done.

%\setcounter{section}{-1}
\end{document}
