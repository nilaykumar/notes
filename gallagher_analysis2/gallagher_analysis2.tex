\documentclass{mathnotes}

\usepackage{tikz-cd}
\usepackage{amsmath}
\usepackage{todonotes}


\title{Introduction to Modern Analysis II}
\author{Patrick X. Gallagher\footnote{Notes typeset by Nilay Kumar}}
\date{Spring 2012}


\begin{document}

\maketitle

\section{Irrationality of $e$ and $\pi$, transcendence of $e$}

We begin by repeating the proof, from last term, that $e$ is irrational.
We define $e$ by the series \[e=\sum_{k=0}^\infty \frac{1}{k!}.\]
Assuming $e$ is rational, i.e. $e=m/n$ for some $m,n\in\N$, we can write
\[\frac{m}{n}=\sum_{k=0}^n\frac{1}{k!}+\sum_{k=n+1}^\infty\frac{1}{k!}.\]
Multiplying this by $n!$ yields
\[m(n-1)!=\sum_{k=0}^n\frac{n!}{k!}+\sum^\infty_{k=n+1}\frac{n!}{k!}.\]
The terms in the first sum on the right are integers, so moving it to the left
hand side, we find that
\begin{align*}
    m(n-1)!-\sum_{k=0}^n\frac{n!}{k!} &= \frac{1}{n+1}+\frac{1}{(n+1)(n+2)}+\cdots\\
    &<\frac{1}{n+1}+\frac{1}{(n+1)^2}+\cdots\\
    &=\frac{1}{n+1}\left( 1+\frac{1}{n+1}+\frac{1}{(n+1)^2}+\cdots \right)\\
    &=\frac{1}{n+1}\cdot \frac{1}{1-\frac{1}{n+1}}=\frac{1}{n}\\
    &\leq 1.
\end{align*}
The sum on the right is positive, so the left hand side must be a positive integer
less than 1, which is a contradiction. Hence $e$ is irrational.

\newpage
\begin{thm}[Lambert, 1761]
   $\pi$ is irrational.
\end{thm}
\begin{proof}[Proof (Niven, 1947).]
    We show that $\pi^2$ is irrational, which is a slightly stronger statement,
    since if $\pi=u/v$ with $u,v\in\N$ then $\pi^2=u^2/v^2$. For $n\in\N$, set
    \[I_n=\int_0^1p_n(x)\sin\pi x dx\]
    where 
    \[p_n(x)=\frac{(x(1-x))^n}{n!}.\]
    Since $0\leq p_n(x)\leq1/n!$ and $0\leq\sin\pi x\leq1$ for $0\leq x\leq 1$, we see that
    \begin{equation}
        0\leq I_n\leq\frac{1}{n!}.
        \label{eq:niven_est}
    \end{equation}

    Next we claim that for $k=0,1,2,\ldots$, both $p_n^{(k)}(0)$ and $p_n^{(k)}(1)$ are integral:
    clearly $p_n^{(k)}(0)=0$ for $k=0,1,\cdots,n-1$ due to the presence of the overall $x^n$. By
    the Binomial theorem,
    \[(u+v)^n=\sum_{k=0}^n \binom{n}{k}u^{n-k}v^k,\]
    we get
    \[p_n(x)=\frac{1}{n!}\sum_{k=0}^n\binom{n}{k}(-1)^kx^{n+k}\]
    so
    \begin{align*}
        p_n^{(n)}(x)&=\frac{1}{n!}\sum_{k=0}^n\binom{n}{k}(-1)^k(n+k)\cdots(1+k)x^k\\
        &=\sum_{k=0}^n\binom{n}{k}(-1)^k\binom{n+k}{n}x^k,
    \end{align*}
    a polynomial with integer coefficients. It follows that each $p_n^{(k)}$ with $k\geq n$
    is a polynomial with integer coefficients and therefore each $p_n^{(k)}(0)$ is an integer.
    Since $p_n(x)=p_n(1-x)$ we have $p_n^{(k)}(x)=(-1)^kp_n^{(k)}(1-x)$, so $p_n^{(k)}(1)=(-1)^kp_n^{(k)}(0)$.
    Thus each $p_n^{(k)}(1)$ is an integer as well.

    Furthermore, we claim that for each $n\in\N$ there are integers $c_0,c_1,\ldots,c_n$ such
    that
    \[I_n=\frac{c_0}{\pi}+\frac{c_1}{\pi^3}+\cdots+\frac{c_n}{\pi^{2n+1}}.\]
    This follows by integration by parts:
    \begin{align*}
        \int p_n(x)\sin\pi x dx &= -p_n(x)\frac{\cos\pi x}{\pi}+\int p_n'(x)\frac{\cos\pi x}{\pi} dx\\
        &=-p_n(x)\frac{\cos\pi x}{\pi} + p_n'(x)\frac{\sin\pi x}{\pi^2}-\int p_n''(x)\frac{\sin\pi x}{\pi^2}dx\\
        &= -p_n(x)\frac{\cos\pi x}{\pi}+p_n'(x)\frac{\sin \pi x}{\pi^2}+p_n''(x)\frac{\cos\pi x}{\pi^3}-p_n'''(x)\frac{\sin \pi x}{\pi^4}+\cdots
    \end{align*}
    following the same pattern, the last term involving $p_n^{(2n+1)}(x)$. Inserting $x=1$ and $x=0$ evaluates $I_n$,
    by the fundamental theorem of calculus. The terms with $\sin\pi x$ drop out as $\sin0=\sin\pi=0$ and the terms with
    $\cos\pi x$ yield integers following the previous paragraph.

    Finally, we suppose that $\pi^2=a/b$ for $a,b\in\N$. It follows from above that
    \[I_n=\frac{c_0+c_1a^{2n-1}b+\cdots+c_nb^{2n}}{\pi a^{2n}}.\]
    Since $I_n$ is positive, the numerator is a positive integer, and we can write $I_n\geq 1/(\pi a^{2n})$.
    Combining this inequality with Eq.~(\ref{eq:niven_est}), we find that
    \[\frac{a^{2n}}{n!}\geq \frac{1}{\pi}\]
    for all $n\in\N$, which is false as $\sum_{n=0}^\infty a^{2n}/n!$ converges.
\end{proof}

\begin{defn}
    A real number $\alpha$ is \textbf{algebraic} if $\alpha$ is a root of some non-zero polynomial
    with integer coefficients. If $\alpha\in\R$ but $\alpha$ is not algebraic, then $\alpha$ is
    \textbf{transcendental}.
\end{defn}

\begin{thm}
    The set of all algebraic numbers is countable.
\end{thm}
\begin{proof}
    The set of polynomials of degree less than or equal to $n$ with integer coefficients is countable as
    there is a bijection between this set and $\Z^{n+1}$. Therefore the set of all polynomials with integer
    coefficients is countable since a countable union of countable sets is countable. Each non-zero
    polynomial of degree $n$ has at most $n$ roots. Therefore the set of all algebraic numbers, as a
    countable union of finite sets, is countable.
\end{proof}

\begin{cor}
    Transcendental numbers exist.
\end{cor}
\begin{proof}
    If not, every real number would be algebraic. But $\R$ is uncountable by Cantor's theorem.
\end{proof}

It is much harder to exhibit a transcendental number than it is to prove that they exist. Here is a famous example.

\begin{thm}[Hermite, 1873]
   $e$ is transcendental.
\end{thm}

\begin{lem}
    For $m=0,1,2,\ldots$,
    \[\int_0^\infty x^me^{-x}dx = m!.\]
\end{lem}
\begin{proof}
    For $m=0$ we find that
    \[\int_0^\infty e^{-x}=1\]
    and for $m>0$, we integrate by parts to obtain
    \begin{align*}
        \int_0^\infty x^m e^{-x}dx &= m\int_0^\infty x^{m-1}e^{-x} dx.
    \end{align*}
    The result follows by induction.
\end{proof}

\begin{cor}
    If $p\in\N$ and $c_{p-1},c_p,\ldots,c_N\in\Z$, then
    \[\frac{1}{(p-1)!}\int_0^\infty \left( c_{p-1}x^{p-1}+c_px^p+\cdots+c_Nx^N \right)e^{-x} dx\]
    is an integer equivalent to $c_{p-1}\mod p$.
\end{cor}

\begin{lem}[Hermite]
    For $n,p\in\N$ and $k=1,\ldots, n$ set
    \[H_{n,p}=\frac{1}{(p-1)!}\int_0^\infty x^{p-1}\left( (x-1)\cdots(x-n) \right)^pe^{-x}dx\]
    and 
    \[H_{n,k,p}=\frac{1}{(p-1)!}\int_0^\infty (x+k)^{p-1}\left( (x+k-1)\cdots(x+k-n) \right)^pe^{-x}dx.\]
    Then $H_{n,p}$ and $H_{n,k,p}$ are integers, with
    \[H_{n,p}\equiv \left( (-1)^n n! \right)^p\mod p\]
    and
    \[H_{n,k,p}\equiv 0\mod p\]
    for $k=1,\ldots,n$.
\end{lem}
\begin{proof}
    We use the corollary above.
    The polynomial $x^{p-1}\left( (x-1)\cdots(x-n) \right)^p$ has integer coefficients and its
    lowest degree term is $\left( (-1)^nn! \right)^px^{p-1}$. However, for $k=1,\ldots,n$,
    the polynomial $x^{p-1}\left( (x+k-1)\cdots(x+k-n) \right)^p$ has integer
    coefficients and starts later than the $x^{p-1}$ term so the coefficient of $x^{p-1}$ is 0.
\end{proof}

\begin{proof}[Hilbert's proof of Hermite's theorem (1893)]
    If $e$ is algebraic then
    \[a_0+a_1e+\cdots+a_ne^n=0\]
    for some integers $a_0,a_1,\ldots,a_n$ not all 0. We may suppose $a_0\neq 0$. It follows that for
    each positive integer $p$,
    \[0=\left( \sum_{k=0}^na_ke^k \right)H_{n,p}=\sum_{k=0}^na_k\frac{1}{(p-1)!}\int_0^\infty x^{p-1}\left( (x-1)\cdots(x-n) \right)^pe^{-x} dx.\]
    For $k=1,\ldots,n$ we write the integral as $\int_0^k+\int_k^\infty$ and in $\int_k^\infty$ replace $x$ by $x+k$ giving
    \begin{equation}
        a_0H_{n,p}+\sum_{k=1}^n a_kH_{n,k,p}=-\sum_{k=1}^na_kR_{n,k,p}
        \label{eq:hilb_e}
    \end{equation}
    with
    \[R_{n,k,p}=\frac{1}{(p-1)!}\int_0^k x^{p-1}\left( (x-1)\cdots(x-n) \right)^pe^{k-x}dx.\]
    By Hermite's lemma, the left side of Eq.~(\ref{eq:hilb_e}) is an integer equivalent to $a_0\left( (-1)^nn! \right)^p\mod p$.
    Therefore, if $p$ is a prime greater than both $n$ and $|a_0|$, then the left side of Eq.~(\ref{eq:hilb_e}) is not
    equivalent to $0\mod p$. In particular, the left side is a non-zero integer.
    The right side is small for large $p$: take
    \[A=\sum_{u=1}^n|a_u|\]
    and
    \[B=\max_{[0,n]}|(x-1)\cdots(x-n)|.\]
    Then
    \begin{align*}
        -\sum_{k=1}^na_kR_{n,k,p}&\leq \frac{A}{(p-1)!}\int_0^nx^{p-1}|(x-1)\cdots(x-n)|^pe^{n-x}dx\\
        &\leq \frac{Ae^n(nB)^p}{(p-1)!},
    \end{align*}
    since $\int_0^ne^{-x}dx<1$, which goes to zero as $p$ grows large. Since there are infinitely many primes, this yields
    a contradiction.
\end{proof}

\section{Approximating roots}

Omitted.

\section{Laplace's asymptotic method}

Let $a<b$ and $f:[a,b]\to\R$ be continuous and non-negative with maximum $M$. We show first that
\[\left( \int^b_af^p \right)^{1/p}\to M\text{ for }p\to\infty\]
and then, under stronger assumptions about $f$, we obtain Laplace's asymptotic formula for $\int^b_af^p$ for $p\to\infty$.

\begin{lem}
    For each $d>0$, $d^{1/p}\to 1$ for $p\to\infty$.
\end{lem}
\begin{proof}
    For $d>0$ and $p>0$, as $p\to \infty$,
    \[\log d^{1/p}=\frac{1}{p}\log d\to 0.\]
    Therefore
    \[d^{1/p}=e^{\log(d^{1/p})}\to e^0=1,\]
    where we have used the continuity of the exponential function at zero.
\end{proof}

\begin{thm}
For $f:[a,b]\to\R$ continuous and non-negative with $a<b$, 
\begin{equation}
    \left( \int^b_af^p \right)^{1/p}\to M\text{ for }p\to\infty
    \label{eq:lap1}
\end{equation}
with $M=\max f$.
\end{thm}
\begin{proof}
    First we obtain an upper bound:
    \[\int_a^b f^p\leq \int_a^bM^p=(b-a)M^p,\]
    so
    \[\left( \int_a^bf^p \right)^{1/p}\leq (b-a)^{1/p}M.\]
    By the lemma above with $d=b-a$ we see that the first factor on the right approaches 1 for $p\to\infty$.
    Therefore, for each $\mu>1$, we have
    \[\left( \int_a^bf^p \right)^{1/p}\leq\mu M\]
    for all sufficiently large $p$.

    Next we obtain a lower bound: we have $f(c)=M$ for some $c\in[a,b]$. If $a<c<b$ then by continuity
    of $f$ there must exist a $\delta>0$ such that $f(x)>\sqrt{\lambda}M$ for $x\in[c-\delta,c+\delta]\subset[a,b]$
    and $\lambda<1$.
    It follows that
    \[\int_a^bf^p\geq \int_{c-\delta}^{c+\delta}f^p\geq 2\delta(\sqrt{\lambda}M)^p,\]
    so
    \[\left( \int_a^bf^p \right)^{1/p}\geq (2\delta)^{1/p}\sqrt{\lambda}M.\]
    Using the lemma above with $d=2\delta$, the first factor on the right goes to 1 for $p\to\infty$.
    Therefore it is greater than or equal to $\sqrt{\lambda}$ for all sufficiently large $p$ giving
    \[\left( \int_a^b f^p \right)^{1/p}\geq \lambda M\]
    for all sufficiently large $p$. If $c=a$ or $c=b$, we consider just $[a,a+\delta]$ or $[b-\delta,b]$
    instead of $[c-\delta,c+\delta]$ and obtain the bound above as well.

    Combining the upper and lower bounds, yields the desired formula, as $\lambda<1$ and $\mu>1$ can
    be chosen arbitrarily close to 1.
\end{proof}

\begin{thm}[Laplace, 1774]
    For $a<b$ and $\phi:[a,b]\to\R$ continuous with continuous and negative second derivative on $(a,b)$
    and assuming $\phi$ has a maximum at some point $c\in(a,b)$,
    \begin{equation}
        \int_a^be^{p\phi(x)}dx\sim\sqrt{\frac{2\pi}{-p\phi''(c)}}e^{p\phi(c)}
        \label{eq:lap2}
    \end{equation}
    for $p\to \infty$.\footnote{Here for positive functions $g,h$ defined on $(0,\infty)$, $g(p)\sim h(p)$ for
    $p\to\infty$ means that $g(p)/h(p)\to 1$ for $p\to\infty$.}
    \label{thm:lap1}
\end{thm}

\begin{lem}
    \begin{equation}
        \int_{-\infty}^\infty e^{-\pi x^2}dx = 1
        \label{eq:erf}
    \end{equation}
\end{lem}
\begin{proof}
    The square of the integral is readily evaluated:
    \begin{align*}
        \left( \int_{-\infty}^\infty e^{-\pi x^2}dx \right)^2 &= \int_{-\infty}^{\infty} e^{-\pi x^2}\int_{-\infty}^{\infty} e^{-\pi y^2}dy\\
        &=\int_{-\infty}^\infty\int_{-\infty}^\infty e^{-\pi(x^2+y^2)} dxdy\\
        &=\int_0^\infty\int_0^{2\pi}e^{-\pi r^2}rd\theta dr\\
        &=\int_0^\infty e^{-\pi r^2}2\pi rdr\\
        &=\int_0^\infty e^{-u}du=1.
    \end{align*}
    Since the integral is clearly positive and has square 1, it must be 1.
\end{proof}

We prove Laplace's theorem for the special case where $c=0$, along with the additional assumptions that
$\phi(0)=0,\phi'(0)=0,\phi''(0)=-1$. The general case is left for the reader in Exercise \ref{exc:1}.

\begin{proof}[Proof of (special case of) Laplace's theorem.]
    We are now supposing that $\phi$ is continuous on $[a,b]$ with $a<0<b$, has a maximum at zero, and has $\phi''<0$ on $[a,b]$,
    along with the special assumptions above. Of course, $\phi'(0)=0$ follows from the maximum assumption.
    We wish to show that
    \[\int_a^b e^{p\phi(x)} dx\sim\sqrt{\frac{2\pi}{p}}\]
    for $p\to\infty$. Let's look first at an even more special case: $\phi(x)=-x^2/2$. In this case, the assumptions
    above are satisfied, and for $a<0<b$,
    \begin{align*}
        \int_a^b e^{p\phi(x)} dx &= \int_a^be^{-px^2/2} dx\\
        &=\sqrt{\frac{2\pi}{p}}\int_{a_1}^{b_1}e^{-\pi u^2}du
    \end{align*}
    with $a_1=\sqrt{p/2\pi}a$ and $b_1=\sqrt{p/2\pi}b$. Thus for $p\to\infty$ we have $a_1\to-\infty$ and $b_1\to\infty$ so the last
    integral approaches 1 by the previous lemma. This proves the theorem in this case.
    
    We now return to the slightly more general case. Since $\phi(0)=0$ and $\phi'(0)=0$, Taylor's formula gives
    \begin{align*}
        \phi(x)&=\int_0^x\phi''(t)(x-t)dt\\
        &=\int_0^x\phi''(0)(x-t)dt+\int_0^x\left( \phi''(t)-\phi''(0) \right)(x-t)dt\\
        &=\frac{1}{2}\phi''(0)x^2+\int_0^x\left( \phi''(t)-\phi''(0) \right)(x-t)dt\\
    \end{align*}
    The norm of the second term on the right is clearly bounded by $\max_{0\leq t\leq x}|\phi''(t)-\phi''(0)|\cdot x^2/2$.
    Since $\phi''$ is continuous at zero, the maximum goes to zero for $x\to 0$. Thus $\phi(x)\sim\phi''(0)x^2/2=-x^2/2$ for
    $x\to 0$. It follows that for each pair of real numbers $\lambda,\mu$ with $\lambda<1<\mu$, there is a $\delta>0$
    with $[-\delta,\delta]\subset[a,b]$ so that $-\mu x^2/2\leq\phi(x)\leq -\lambda x^2/2$ for $|x|\leq\delta$.
    Therefore, for all $p>0$,
    \[\int_{-\delta}^\delta e^{-\mu px^2/2}dx\leq \int_{-\delta}^\delta e^{p\phi(x)}dx \leq \int_{-\delta}^\delta e^{-\lambda px^2/2}dx,\]
    which simplifies to
    \[\sqrt{\frac{2\pi}{\mu p}}\int_{-\delta_\mu}^{\delta_\mu} e^{-\pi u^2}du\leq \int_{-\delta}^\delta e^{p\phi(x)}dx \leq \sqrt{\frac{2\pi}{\lambda p}}\int_{-\delta_\lambda}^{\delta_\lambda} e^{-\pi u^2}du,\]
    with $\delta_\mu=\delta\sqrt{\mu p/2\pi}$ and $\delta_\lambda=\delta\sqrt{\lambda p/2\pi}$, both going to infinity as $p\to\infty$.
    It follows that the left and right integrals approach $\sqrt{2\pi/\mu p}$ and $\sqrt{2\pi/\lambda p}$ as $p\to\infty$ and hence
    $\int_{-\delta}^\delta e^{p\phi(x)}dx/\sqrt{2\pi/p}$ can be made arbitrarily close to 1 by first choosing $\delta$ sufficiently
    small, then choosing $p$ sufficiently large. What about the rest of the integral?

    Since $\phi$ decreases on $[0,b]$, we have $\int_\delta^b e^{p\phi(x)}dx\leq be^{p\phi(\delta)}$. For each $\delta>0$
    we have $\phi(\delta)<0$, so 
    \[\frac{e^{p\phi(\delta)}}{\sqrt{2\pi/p}}=\frac{\sqrt{p}e^{p\phi(\delta)}}{\sqrt{\pi}}\to 0\text{ for }p\to\infty.\]
    Thus for each $\delta>0$,
    \[\frac{1}{\sqrt{2\pi/p}}\int_\delta^b e^{p\phi(x)}dx\to 0\text{ for }p\to\infty.\]
    Similarly for $[a,\delta]$. Putting it all together yields the desired result.
\end{proof}

\begin{exc}\hfill
    \label{exc:1}
    \begin{enumerate}
        \item Show that the general case of Thm. \ref{thm:lap1} can be reduced to the case in which $c=0$ by making
            the change of variables $x=c+u$, $\phi(x)=\psi(u)$.
        \item Show that the case in (1) can be further reduced to the case in which $c=0$ and $\phi(0)=0$ by
            writing $\phi(x)=\phi(0)+\psi(x)$.
        \item Show that the case in (2) can be further reduced to the case in which $c=0$, $\phi(0)=0,\phi'(0)=0$,
            and $\phi''(0)=-1$ by making a change of variables $x=\xi u$ for some positive constant $\xi$.
    \end{enumerate}
\end{exc}

\begin{thm}[Generalizing Stirling's formula]
    As $p\to\infty$ for $p$ real,
    \[\int_0^\infty x^p e^{-x}dx\sim \sqrt{2\pi p}\left( \frac{p}{e} \right)^p.\]
\end{thm}
\begin{proof}
    In the integral let $x=p(u+1)$, giving
    \[\int_0^\infty x^p e^{-x}dx=p^{p+1}e^{-p}\int_{-1}^\infty\left( (u+1)e^{-u} \right)^p du.\]
    To obtain the result, it suffices to show that
    \[\int_{-1}^\infty e^{p\phi(u)} du\sim\sqrt{\frac{2\pi}{p}}\]
    for $p\to\infty$, with
    \[\phi(u)=\log(u+1)-u\]
    for $u>-1$. This is left as the following exercise.
\end{proof}

\begin{exc}
    Show how Laplace's theorem can be used to finish the proof above. Hint:
    \begin{align*}
        \phi'(u) &= \frac{1}{u+1}-1\\
        \phi''(u) &= -\frac{1}{(u+1)^2}.
    \end{align*}
    There are two slight difficulties: $\phi(u)\to-\infty$ as $u\to 0,u>0$, and $b=\infty$.
    Show how to overcome these difficulties.
\end{exc}


\end{document}
