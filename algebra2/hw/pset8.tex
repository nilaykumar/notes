\documentclass{../../mathnotes}

\usepackage{enumerate}

\title{Modern Algebra II: Problem Set 8}
\author{Nilay Kumar}
\date{Last updated: \today}


\begin{document}

\maketitle

\subsection*{Problem 1}

Let $F$ be a field of characteristic $p$ and $p$ be a positive integer that divides $n$, i.e. $n=pq$ for some positive integer $q$.
Take the polynomial $f(x)=x^n-1$. The derivative is $nx^{n-1}=0$, i.e. every root will be a multiple root in some extension field of $F$.
One, for example, is a multiple root of $f$.

If, on the other hand, $F$ has characteristic 0, the derivative of $f$, $Df=nx^{n-1}\neq 0$ in general, and thus only zero can possibly
be a multiple root of $f$. Zero, however, is not a root of $f$, so $f$ has no multiple roots in any extension field $E$. Precisely
the same reasoning holds if $F$ has characteristic $p$ but $p$ does not divide $n$.

\subsection*{Problem 2}

Let $F$ be a field.
\begin{enumerate}[(i)]
    \item By polynomial long division, we see that $(y^n-x^n)/(y-x)$ is given by $y^{n-1}+xy^{n-2}+\ldots+x^{n-2}y+x^{n-1}$,
        which is an element of $F[x,y]$.
    \item If we now let $f(x)=\sum_i a_ix^i\in F[x]$, we can write
        \begin{align*}
            \frac{f(y)-f(x)}{y-x}=\frac{\sum_ia_i(y^n-x^n)}{y-x}=\sum_ia_i\left( y^{n-1}+xy^{n-2}+\ldots+x^{n-2}y+x^{n-1} \right),
        \end{align*}
        which is again a polynomial, call it $Q_f(x,y)$ in $x$ and $y$.
    \item Given any $c\in F$, and $f,g\in F[x]$,
        \begin{align*}
            Q_{cf}(x,y)=\frac{cf(y)-cf(x)}{y-x}=c\frac{f(y)-f(x)}{y-x}=cQ_f.
        \end{align*}
        Furthermore, we can write
        \begin{align*}
            Q_{f+g}(x,y)&=\frac{f(y)+g(y)-f(x)-g(x)}{y-x}\\
            &=\frac{f(y)-f(x)}{y-x}+\frac{g(y)-g(x)}{y-x}\\
            &=Q_f(x,y)+Q_g(x,y)
        \end{align*}
        Additionally, the product behaves as
        \begin{align*}
            Q_{fg}(x,y)&=\frac{f(y)g(y)-f(x)g(x)}{y-x}\\
            &=\frac{f(y)g(y)-f(x)g(x)-f(y)g(x)+f(y)g(x)}{y-x}\\
            &=\frac{f(y)g(y)-f(y)g(x)}{y-x}+\frac{f(y)g(x)-f(x)g(x)}{y-x}\\
            &=f(y)Q_g(x,y)+Q_f(x,y)g(x)
        \end{align*}
    \item If we let $f_n(x)=x^n,$ we can compute
        \begin{align*}
            Q_{f_n}(x,y)=\frac{y^n-x^n}{y-n}=y^{n-1}+xy^{n-2}+\ldots+x^{n-2}y+x^{n-1}.
        \end{align*}
        If we now evaluate $Q_{f_n}(x,x)$, we simply insert $x$ for $y$ into the above expression. Since each of the $n$ terms has products
        of $x$'s and $y$'s totalling to powers of $n-1$, when we perform the substitution we are simply left with $nx^{n-1}$.
        Using this result, given any polynomial $f(x)=\sum_ia_ix^i$, and its formal derivative $Df(x)=\sum_iia_ix^{i-1}$,
        we can write
        \begin{align*}
            Df(x)=\sum_{i=1}^nia_ix^{i-1}=\sum_{i=1}^n a_iQ_{x^i}(x,x)=Q_{f}(x,x),
        \end{align*}
        where in the last equality we have used the definition of $f(x)$ and the linear property of the difference quotient.
\end{enumerate}

\subsection*{Problem 3}

Let $F$ be a field, and suppose that $F$ is a subring of a ring $R$. Let $r\in R$ and let $M_r:R\to R$ be multiplication
by $r$, i.e. $M_r(s)=rs$. Note that if $s,t\in R$ and $a,b\in F$,
\begin{align*}
    M_r(as+bt)=(as+bt)r=asr+btr=M_r(as)+M_r(bt)
\end{align*}
and so $M_r$ is an $F$-linear map.

Furthermore, $\ker M_r$ is the set of all elements of $R$ that do not vanish when multiplied by $r$. Consequently, $M_r$ is injective
if $r$ is not a zero divisor. Conversely, if $r$ is not a zero divisor, $M_r(s)\neq0$ for all $s\neq0$, i.e. $\ker M_r=0$ so
$M_r$ is injective.

If $r$ happens to be a unit, one can always find an $s'\in R$ such that $M_r(s')=s$ for any $s\in R$:
\begin{align*}
    M_r(s')&=s'r=s\\
    s'&=s/r
\end{align*}
and so $M_r$ will be surjective. Conversely, if we know that $M_r$ is surjective, it means that for each $s$, we can find an $s'$
such that $s=s'r=M_r(s)$. Since this holds for every non-zero $s\in R$, and the associated $s'$ obviously cannot be zero, this implies that
$r$ cannot a zero divisor (if it were, $s'r$ would be zero, not $s$). Consequently, by the above observations, $M_r$ is injective as well,
and thus an isomorphism from $R$ to $R$ (as $F$-vector spaces). Conversely, if we know that $M_r$ is an isomorphism, it must be a surjection,
and as above, $r$ must be a unit. 

\subsection*{Problem 4}

Consider the field $\Q(\sqrt{2})$, viewed as a vector space of dimension 2 over $\Q$. Let $r+s\sqrt{2}\in\Q(\sqrt{2})$.
We now define the multiplication map $M_{r+s\sqrt{2}}:\Q(\sqrt{2})\to\Q(\sqrt{2})$ as above: $M_{r+s\sqrt{2}}(\alpha)=(r+s\sqrt{2})\alpha$.

\begin{enumerate}[(i)]
    \item We can write a basis for $\Q(\sqrt{2})$ is $\left\{ 1,\sqrt{2} \right\}$. Take any $a+b\sqrt{2}$ in $\Q(\sqrt{2})$.
        Then
        \[M_{r+s\sqrt{2}}(a+b\sqrt{2})=ra+2sb+(rb+sa)\sqrt{2}.\]
        In matrix notation, we can write
        \begin{align*}
           M\left(
           \begin{array}{c}
               a\\
               b
           \end{array}
           \right)=
           \left(
           \begin{array}[]{c}
               ra+2sb\\
               rb+sa
           \end{array}
           \right)
        \end{align*}
        and so we can represent
        \begin{align*}
            M=\left(
            \begin{array}[]{cc}
                r & 2s\\
                s & r
            \end{array}
            \right).
        \end{align*}
    \item Given any matrix $A\in M_2(\Q)$, in order for $A=M_{r+s\sqrt{2}}$ for some $r,s$, it must be of the form above
        for some $r,s$.
    \item It should be clear the determinant of $M_{r+s\sqrt{2}}$ is given by $r^2-2s^2$. Furthermore, if the determinant is
        zero and the number we are multiplying is non-zero, the map is not full-rank, i.e. the kernel forms a linear subspace
        of non-zero dimension (by the rank-nullity theorem). However, this contradicts that $\Q(\sqrt{2})$ is a field and that
        it has no zero divisors, and thus the number we are multiplying by must be zero. Conversely, if the number we are
        multiplying by is zero, the matrix is the zero matrix, and thus the determinant must be zero.
    \item We may compute the inverse matrix using the usual technique for 2-by-2 matrices,
        \begin{align*}
            M^{-1}_{r+s\sqrt{2}}=\frac{1}{r^2-2s^2}
            \left( 
            \begin{array}[]{cc}
                r & -2s\\
                -s & r
            \end{array}
            \right).
        \end{align*}
        This shows that the inverse map is of the form $M_{t+u\sqrt{2}}$, where $t=r/(r^2-2s^2)$ and $u=-s/(r^2-2s^2)$.
        We can thus explicitly construct a multiplicative inverse for $r+s\sqrt{2}$ as
        \[(r+s\sqrt{2})^{-1}=\frac{r}{r^2-2s^2}-\frac{s\sqrt{2}}{r^2-2s^2}\]
\end{enumerate}

\subsection*{Problem 5}

Consider the field $\Q(\sqrt{2})$ as a $\mathbb{Q}$-vector space of dimension 3. Take $a+b\sqrt[3]{2}+c\sqrt[3]{2}^2\in\Q(\sqrt{2})$.
Similar to above, we define the multiplication map (with subscripts supressed),
\[M(\alpha)=\left( a+b\sqrt[3]{2}+c\sqrt[3]{2}^2 \right)\alpha\]

\begin{enumerate}[(i)]
    \item Let us choose the basis $\{ 1,\sqrt[3]{2},\sqrt[3]{2}^2 \}$ for $\Q(\sqrt[3]{2})$. Then we have
        \begin{align*}
            M(r+s\sqrt[3]{2}+t\sqrt[3]{2}^2)&=ra+rb\sqrt[3]{2}+rc\sqrt[3]{2}^2\\
            &+sa\sqrt[3]{2}+sb\sqrt[3]{2}^2+2sc\\
            &+ta\sqrt[3]{2}^2+2tb+2tc\sqrt[3]{2}
        \end{align*}
        This is written in matrix form as
        \begin{align*}
            M\left( 
            \begin{array}[]{c}
                r\\
                s\\
                t
            \end{array}
            \right)
            =
            \left( 
            \begin{array}[]{c}
                ra+2sc+2tb\\
                rb+sa+2tc\\
                rc+sb+ta
            \end{array}
            \right)
        \end{align*}
        and so we can represent
        \begin{align*}
            M=\left( 
            \begin{array}[]{ccc}
                a & 2c & 2b\\
                b & a & 2c\\
                c & b & a
            \end{array}
            \right)
        \end{align*}
    \item Given any matrix $A\in M_3(\Q)$, in order for $A=M_{a+b\sqrt[3]{2}+c\sqrt[3]{2}^2}$, it must be of the form above
        for some $a,b,c$.
    \item We can compute
        \begin{align*}
            \det M_{a+b\sqrt[3]{2}+c\sqrt[3]{2}^2}&=a^3+4c^3+2b^3-6abc.
        \end{align*}
        Just as above, the determinant must be non-zero as long as $a,b,c$ are all non-zero.
        If the determinant is zero and the number we are multiplying is non-zero, the map is not full-rank, i.e. the kernel forms a linear subspace
        of non-zero dimension (by the rank-nullity theorem). However, this contradicts that $\Q(\sqrt[3]{2})$ is a field and that
        it has no zero divisors, and thus the number we are multiplying by must be zero. Conversely, if the number we are
        multiplying by is zero, the matrix is the zero matrix, and thus the determinant must be zero.
    \item The inverse of $M$ can be written
        \begin{align*}
            M^{-1}_{a+b\sqrt[3]{2}+c\sqrt[3]{2}^2}=\frac{1}{a^3+4c^3+2b^3}
            \left( 
            \begin{array}[]{ccc}
                a^2-2bc & 2b^2-2ac & 4c^2-2ab\\
                2c^2-ab & a^2-2bc & 2b^2-2ac\\
                b^2-ac & 2c^2-ab & a^2-2bc
            \end{array}
            \right)
        \end{align*}
        Note that if we take $d=(a^2-2bc)/(a^3+4c^3+2b^3),e=(2c^2-ab)/(a^3+4c^3+2b^3),f=(b^2-ac)/(a^3+4c^3+2b^3)$,
        this matrix is of the form of a multiplication map $M_{d+e\sqrt[3]{2}+f\sqrt[3]{2}^2}$. In fact, we can use this 
        to explicitly construct a multiplicative inverse for $a+b\sqrt[3]{2}+c\sqrt[3]{2}^2$ as
        \[(a+b\sqrt[3]{2}+c\sqrt[3]{2}^2)^{-1}=\frac{a^2-2bc}{a^3+4c^3+2b^3}+\frac{2c^2-ab}{a^3+4c^3+2b^3}\sqrt[3]{2}+\frac{b^2-ac}{a^3+4c^3+2b^3}\sqrt[3]{2}^2.\]
\end{enumerate}



\end{document}
