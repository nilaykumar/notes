\documentclass{../../mathnotes}

\usepackage{enumerate}

\title{Modern Algebra II: Problem Set 9}
\author{Nilay Kumar}
\date{Last updated: \today}


\begin{document}

\maketitle

\subsection*{Problem 1}

Let $F$ be a field of a characteristic $p\geq 0$.
\begin{enumerate}[(i)]
    \item Suppose that every element of $F$ is a $p$th power, i.e. for all $a\in F$, there exists an element $b\in F$
        such that $b^p=a$. Equivalently, the Frobenius homomorphism $\sigma_p:F\to F$ is surjective. Such a field
        is called \textbf{perfect}. We wish to show that if $f(x)\in F[x]$ is irreducible, then $f(x)$ does not
        have multiple roots. Let us assume for the sake of contradiction that $f(x)=\sum_{i=0}^na_ix^i$ irreducible has multiple roots.
        Then $Df(x)=0$ by a corollary proved in class. Since $f(x)$ is not a constant, in order for the derivative to be
        identically zero we must have that $f$ is of the form $f(x)=\sum_{i=0}^na_ix^{ip}$, i.e. bringing down the exponents
        annihilates each term. Using the fact that $F$ is perfect, we can now rewrite, for some $b_i\in F$:
        \begin{align*}
            f(x)&=\sum_{i=0}^na_ix^{ip}=\sum_{i=0}^nb_i^px^{ip}=\sum_{i=0}^n(b_ix)^p\\
            &=\sum_{i=0}^n\sigma_p(b_ix)=\sigma_p\left(\sum_{i=0}^nb_ix\right)=\left( \sum_{i=0}^nb_ix \right)^p,
        \end{align*}
        and thus $f$ is a $p$th power. This contradicts that $f$ is irreducible, and thus $f$ cannot have multiple roots.
    \item Given $F$ finite, we wish to show that $F$ is perfect. Consider the Frobenius homomorphism
        $\sigma_p:F\to F$. Suppose $\sigma_p(a)=\sigma_p(b)=c$ for some $a,b,c\in F$. Then,
        \[\sigma_p(a)-\sigma_p(b)=\sigma_p(a-b)= (a-b)^p=0,\]
        and thus $a=b$, as there are no zero divisors in a field. Consequently, $\sigma_p$ is injective. Furthermore,
        since $\sigma_p$ is a map from a finite $F$ to itself, injectivity implies surjectivity. Consequently, every
        element of $F$ can be written as a $p$th power, and thus $F$ is perfect.
    \item Let $F$ be a finite field and let $k$ be a positive integer. In general, then, $a\mapsto a^k$ is not
        a homomorphism, as the binomial coefficients do not disappear as usual when taking $(p+q)^k$.
        It is thus not necessarily true that for all $a\in F$ there exists an element $b\in F$ such that $b^k=a$.
\end{enumerate}

\subsection*{Problem 2}

Throughout this problem, $\F_2$ denotes the finite field with 2 elements.
\begin{enumerate}[(i)]
    \item Let $\F_2(\alpha)$ be a simple extension of $\F_2$, generated by an element $\alpha$ such that $\alpha^2+\alpha+1=0$,
        i.e. $\alpha$ is a root of the polynomial $x^2+x+1$. Note that since $\alpha$ satisfies a polynomial of degree 2, $\alpha^2$
        and higher powers can be written using $\F_2$ and $\alpha$. Thus $[\F_2(\alpha):\F_2]=2$, as we can write a basis
        $\left\{ 1,\alpha \right\}$. In addition, note that
        \begin{align*}
            (\alpha+1)^2+(\alpha+1)+1=(\alpha+1+1)+(\alpha+1)+1=0.
        \end{align*}
        Consequently, we can write
        \[x^2+x+1=(x+\alpha)(x+\alpha+1).\]
    \item Let $\F_2(\beta)$ be a simple extension of $\F_2$, generated by an element $\beta$ such that $\beta^3+\beta+1=0$, i.e.
        $\beta$ is a root of the polynomial $x^3+x+1$. In this case, we can write $\beta^3$ and all higher powers in terms of
        elements of $\F_2$ and $\beta$. Then, $[\F_2(\beta):\F_2]=3$, as we can write a basis $\left\{ 1,\beta,\beta^2 \right\}$.
        $\F_2(\beta)$ then has $2^3=8$ elements. Note that we can compute
        \begin{align*}
            \sigma_2(\beta^3+\beta+1)=\beta^6+\beta^2+1=(\beta^2)^3+\beta^2+1=0
        \end{align*}
        and thus $\beta^2$ is also a root of $x^3+x+1$. The same can be shown for $\beta^4$,
        \begin{align*}
            \sigma_2(\beta^6+\beta^2+1)=\beta^{12}+\beta^4+1=(\beta^4)^3+\beta^4+1=0.
        \end{align*}
        In fact, we can express
        \[\beta^4=\beta\beta^3=\beta(\beta+1)=0+\beta+\beta^2.\]
        Consequently, we can write
        \begin{align*}
            x^3+x+1&=(x+\beta)(x+\beta^2)(x+\beta^4)\\
            &=(x+\beta)(x+\beta^2)(x+\beta^2+\beta).
        \end{align*}
    \item Since $x^3+x^2+1$ is irreducible, take $\gamma$ to be a root in some extension field. Then we can construct
        an extension $\F_2(\gamma)$ of degree three over $\F_2$ that has 8 elements. But we know that two finite fields
        with the same number of elements are isomorphic to each other, i.e. there exists an isomorphism
        $\phi:\F_2(\beta)\to\F_2(\gamma)$. Thus, if we take $\alpha=\phi(\gamma)$, we have
        \[0=\phi(\gamma^3+\gamma^2+1)=\alpha^3+\alpha^2+1\]
        and $x^3+x^2+1$ thus has a root in $\F_2(\beta)$. Now note that the roots of $x^3+x+1$ are all different that the roots of
        $x^3+x^2+1$ (this can be shown by explicit computation or by noticing that since $\beta$ is not a root, the Frobenius
        homomorphism used as above will show that $\beta^2,\beta^4$ are not roots). Consequently, since we have 8 elements,
        6 are roots of either $x^3+x^2+1$ or $x^3+x+1$, and the other 2 (since these polynomials are irreducible in $\F_2$)
        must be the elements of $\F_2$. In other words, every element's irreducible polynomial is of degree either 1 or 3.
        Indeed, we can see explicitly that no element of $\F_2(\beta)$ could possibly have an irreducible polynomial of degree
        two, because if there were such an element, $\alpha$, the extension $\F_2(\alpha)\subset\F_2(\beta)$ and
        $\F_2(\alpha)(\beta)=\F_2(\beta)(\alpha)=\F_2(\beta)$. But $[\F_2(\alpha,\beta)=\F_2(\beta):\F_2]=[\F_2(\alpha,\beta):\F_2(\alpha)][\F_2(\alpha):\F_2]$
        and the left hand side is equal to the three while the right hand side is 2 times some natural number. This is impossible,
        and thus, such an $\alpha$ cannot exist.

        Recall that we know that any finite field with $q$ elements is defined by the roots of the polynomial $x^q-x$, and thus,
        in our case, the polynomial $x^8-x$ has every element of $\F_2(\beta)$ as its roots. Consequently, in $\F_2(\beta)[x]$,
        $x^8-x=\prod_i^8 (x-a_i)$ where $a_i$ are the elements. In $F_2[x]$, then, we can write (using above computations)
        $x^8-x=x(x+1)(x^3+x+1)(x^3+x^2+1)$ because these irreducible polynomials must all divide $x^8-x$.
\end{enumerate}

\subsection*{Problem 3}

Let $R$ be a PID. We assume that $R$ is not a field and so $(0)$ is not a maximal ideal. Let $I$ be an ideal in $R$.
Let us prove that the following three statements are equivalent:
\begin{enumerate}[(i)]
    \item $I$ is a maximal ideal
    \item $I$ is a prime ideal and $I\neq\left\{ 0 \right\}$
    \item $I=(r)$, where $r$ is irreducible.
\end{enumerate}

(i)$\implies$(ii): Suppose $I$ is maximal. Then $I$ is non-zero and must be prime, as every maximal ideal is prime.

(ii)$\implies$(iii): Now suppose that $I\neq (0)$ is a prime ideal. Since every ideal in $R$ is principal, $I=(r)$ for
some $r\in R$. We wish to show that $r$ is irreducible. First of all, $r$ cannot be a unit, because otherwise the ideal would
contain 1, and thus all of $R$, which would contradict that $I$ is prime. Furthermore $r\neq 0$, because this would contradict
$I\neq(0)$. To show that $r$ is irreducible, we must show that if $r=st$, then one of $s,t$ is a unit. Since $(r)$ is prime,
one of $s,t$ must be in $(r)$, and thus $r=qrt$ (we've chosen $s$, the other case follows similarly). Cancelling, we find that $qt=1$,
i.e. that $s$ is a unit, and thus $r$ is irreducible.

(iii)$\implies$(i): Assuming that $r$ is irreducible, we wish to show that $I=(r)$ is a maximal ideal, i.e. $(r)\neq R$ and if
$(r)\subset J$ some ideal then either $J=(r)$ or $J=R$. First note that $(r)\neq R$ because this would imply that $1\in(r)$,
but this is not possible as $r$ is not a unit. Since $J$ is a principal ideal (working in a PID), $J=(s)$ for some $s\in R$.
If $(r)\subset(s)$, then $r\in(s)$ so $r=st$ for some $t\in R$. But $r$ is irreducible so either $t$ is a unit, in which case
$J=(t)=R$, or $t=cr$ for $c$ a unit, in which case $J=(t)=(r).$ Consequently, $I$ is maximal.

\subsection*{Problem 4}

Let $R$ be an integral domain and let $N$ be a submultiplicative Euclidean norm on $R$.

\begin{enumerate}[(a)]
    \item Since $N$ is submultiplicative, it should be clear that $N(1)\leq N(r)=N(r\cdot 1)$.
    \item If $r$ is a unit, then by the lemma proven in class, $N(r)=N(rs)$ for any $s\in R$. If we choose $s=r^{-1}$, we find that
        $N(r)=N(rr^{-1})=N(1)$. Conversely, we know that $N(r)=N(1)$ is only true if $r$ is unit, also by the lemma proven in class.
    \item Let $r\in R$, with $r\neq0$, and suppose that $N(r)>N(1)$ and that $N(r)$ is minimal with respect to this property.
        Assume for the sake of contradiction that $r$ is not irreducible, i.e. that it can be written as a product $p_1\cdots p_n$.
        We also assume that $r$ is not a unit, because if it were, we'd reach a contradiction immediately (as $N(r)=N(1)$).
        We have $N(r)=N(p_1\cdots p_n)>N(p_1)$, i.e. that $N(p_1)<N(r)$, and we reach a contradiction. Thus, $r$ must be irreducible.
\end{enumerate}

\end{document}
