\documentclass{../../mathnotes}

\usepackage{enumerate}

\title{Modern Algebra II: Problem Set 13}
\author{Nilay Kumar}
\date{Last updated: \today}


\begin{document}

\maketitle

\subsection*{Problem 1}
Let $F$ be a field of characteristic zero, let $f(x)\in F[x]$ be an irreducible polynomial
of degree $n$, and let $E$ be a splitting field of $f(x)$, with roots $\alpha_1,\ldots,\alpha_n\in E$.
\begin{enumerate}[(i)]
    \item By virtue of being a splitting field, $E=F(\alpha_1,\ldots,\alpha_n)$, and $E$ is a Galois
        extension of $F$. Then, the order of $\Gal(E/F)$ is simply the degree $[E:F]$. Consider
        the sequence of extensions:
        \[F\leq F(\alpha_1)\leq E.\]
        Since the irreducible polynomial for $\alpha_1$ over $F$ is $f(x)$, which has degree $n$, we
        can compute
        \[ [E:F]=[E:F(\alpha_1)][F(\alpha_1):F]=[E:F(\alpha_1)]\cdot n. \]
        Hence, $n$ must divide the order of $\Gal(E/F)$. 
    \item Consider $F=\Q$ and $f(x)=x^4-10x^2+1$. We saw in class that $\Gal(\Q(\sqrt{2},\sqrt{3}))=
        \left\{ 1, \sigma_1,\sigma_2,\sigma_3\right\}$, where 
        \begin{align*}
            \sigma_1(\sqrt{2})=-\sqrt{2},\sigma_1(\sqrt{3})=\sqrt{3}\\
            \sigma_2(\sqrt{2})=\sqrt{2},\sigma_2(\sqrt{3})=-\sqrt{3}\\
            \sigma_3(\sqrt{2})=-\sqrt{2},\sigma_3(\sqrt{3})=-\sqrt{3}
        \end{align*}
        Note that every element of the Galois group is of order 2, and thus there does not necessarily
        have to be an element of order 4.
\end{enumerate}

\subsection*{Problem 2}
Let $A_2$ be the element $a+b\sqrt[3]{2}+c(\sqrt[3]{2})^2\in\Q(\sqrt[3]{2})$. Note that $\Q(\sqrt[3]{2})$
is a subfield of the splitting field $\Q(\sqrt[3]{2},\omega)$. Take some $\sigma\in\Gal(\Q(\sqrt[3]{2},\omega)/\Q)$.
If we let 
\begin{align*}
    A_1&=a+b\sqrt[3]{2}+c(\sqrt[3]{2})^2\\
    A_2&=a+b\omega\sqrt[3]{2}+c\omega^2(\sqrt[3]{2})^2\\
    A_3&=a+b\omega^2\sqrt[3]{2}+c\omega(\sqrt[3]{2})^2
\end{align*}
then $\sigma(A_1)=a+b\sigma(\sqrt[3]{2})+c\sigma(\sqrt[3]{2})^2$. Clearly, all we need to know is what $\sigma(\sqrt[3]{2})$
is -- but we know that it can only take on the values $\sqrt[3]{2},\omega\sqrt[3]{2},\omega^2\sqrt[3]{2}$ because the
elements of the Galois group act on the roots. Hence, inserting each possibility into $\sigma(A_1)$ above we find that
$\sigma(A_1)$ can only be one of $A_1,A_2,A_3$. Note that this implies $A_1A_2A_3$ must be fixed by every $\sigma\in\Gal(\Q(\sqrt[3]{2},\omega)/Q)$
because $\sigma$ can be identified by its action on the indices and because $\sigma$ is bijective. Furthermore, this means
that $A_1A_2A_3\in\Q(\sqrt[3]{2},\omega)^{\Gal(\Q(\sqrt[3]{2},\omega)/\Q)}$, but by the main theorem, this is simply $\Q$.
Note that $D=A_1A_2A_3=0$ would imply that one of the $A_i$'s must be zero. But because the expressions for $A_i$'s can be seen
as linear combinations in a $\Q$-vector space, we see that in this case $a=b=c=0$ by linear independence.

We can compute $A_1A_2A_3$ now -- it is a straightforward but tedious computation, the details of which I will omit in order
to spare the grader the enormous burden of grading. Indeed, simply using $\omega^2+\omega+1=0$ (and the fact that the final answer
has to be in $\Q$) results in:
\[D=A_1A_2A_3=a^3+2b^2+4c^3-6abc.\]
We have seen this expression before in problem 6 of homework 2 
Using this, we see that we must have
\[A_1^{-1}=\frac{A_2A_3}{a^3+2b^3+4c^2-6abc}.\]
To be more explicit, one could can multiply out $A_2A_3$:
\[A_1^{-1}=\frac{(a^2-2bc)+(-ab+2c^2)\sqrt[3]{2}+(b^2-ac)\sqrt[3]{2}^2}{a^3+2b^3+4c^2-6abc}.\]


\subsection*{Problem 3}

Let $f(x)\in \Q[x]$ be an irreducible cubic polynomial with exactly one real root. Let $E$ be the splitting field of $f(x)$.

\begin{enumerate}[(i)]
    \item By the fundamental theorem of algebra we know that $f(x)$ must have 3 complex roots. Thus, since it has
        one real root, it must have 2 complex roots. We know that complex roots always occur in conjugates; indeed,
        it is easy to check that permuting these two conjugates is an automorphism, and thus $\sigma$, the conjugation
        automorphism, is an element of $\Gal(E/\Q)$. Clearly $\sigma$ is an element of order 2, and hence it is impossible for
        $\Gal(E/\Q)$ to be equal to $A_3$, as all elements of $A_3$ have order one or three (by Lagrange's theorem, since the
        order of $A_3$ is 3).
    \item Since $E$ is the splitting field for $f(x)$ over $\Q$, we know that the order of the Galois group $\Gal(E/\Q)$
        is equal to $[E:\Q]$. We also know that this is divisible by 3 and that this must divide $3!=6$ (first problem).
        We can write:
        \[ [E:\Q]=[E:\Q(\alpha)][\Q(\alpha):\Q]=3[E:\Q(\alpha)]. \]
        Hence, $[E:\Q(\alpha)]$ is either 1 or 2. It cannot be 1, as that would imply that $\Q(\alpha)$ is a splitting field
        for $f(x)$ over $\Q$. This is a contradiction, as $f(x)$ is irreducible in $\Q[x]$ and cannot be factored into linear factors.
        Thus we have that $[E:\Q(\alpha)]=2$. Consequently $[E:\Q]=6$, i.e. $E$ has degree 6 over $\Q$.
\end{enumerate}

\subsection*{Problem 4}

Let $F$ be a field of characteristic zero and let $E$ be a normal extension of $F$ with Galois group isomorphic
to $S_3$. Since $E$ is a Galois extension of $F$ we may invoke the main theorem of Galois theory: $[E:F]=6$.
Note that there are no order 2 normal subgroups of $S_3$, and thus, by the main theorem, there exists an intermediate
field $K$, not normal over $F$, such that $[K:F]=3$. $K$ must be a simple extension of $F$ (by the usual divisibility
arguments since 3 is prime). Hence, $K=F(\alpha)$ where $\alpha\in E$ is the root of some polynomial $f(x)\in F[x]$
irreducible in $F[x]$. Note that $K=F(\alpha)$ is not normal, and therefore cannot be a splitting field for $F$.
$E$, however, is a normal extension of $F$, and since we know that $f(x)$ is an irreducible polynomial in $F[x]$
with a root in $E$, $f(x)$ must factor into a product of linear factor in $E[x]$. Thus, there must be a splitting field
for $f(x)$, call it $L$, such that $F(\alpha)\leq L\leq E$. We know that:
\begin{align*}
    [E:F]&=[E:F(\alpha)][F(\alpha):F]=6\\
    [E:F(\alpha)]&=2
\end{align*}
and hence,
\begin{align*}
    [E:F(\alpha)]=[E:L][L:F(\alpha)]=2
\end{align*}
but since $[L:F(\alpha)]>1$, we must have that $[L:F(\alpha)]=2$ and thus $[E:L]=1$, i.e. $E=L$.
By construction, $L$ is a splitting field for $f(x)$, and thus $E$ must be as well.

\subsection*{Problem 5}
Let $F$ be a field of characteristic zero containing all the cube roots of unity and let $\omega$ be a generator
of this group. Suppose that $E$ is a normal extension of $F$ whose Galois group is cyclic of order 3, and let
$\sigma$ be a generator for $\Gal(E/F)$. Suppose that $\beta\in E$ is nonzero and that $\sigma(\beta)=\omega\beta$.
First note that $\beta\notin F$, obviously, as otherwise it would be fixed by $\beta$. Furthermore,
\[ \sigma(\beta^3)=\sigma(\beta)^3=\omega^3\beta^3=\beta^3,  \]
i.e. $\sigma$ fixes $\beta^3$. Since this is true of the generator $\sigma$ of $\Gal(E/F)$, 
every element in the Galois group must fix $\beta^3$.  Hence, $\beta^3$ must actually be in $F$, by
the main theorem (as in problem 2). Finally, note that $x^3-\beta^3\in F[x]$ is irreducible, because
its roots are $\beta,\omega\beta,\omega^2\beta$, none of which are in $F$ (as they are not fixed by $\sigma$:
$\sigma(\beta)=\omega\beta,\sigma(\omega\beta)=\omega^2\beta, \sigma(\omega^2\beta)=\beta$ using
the fact that $\omega\in F$ is fixed).
Consequently, we have that
\begin{align*}
    [E:F]=[E:F(\beta)][F(\beta):F]=3[E:F(\beta)]
\end{align*}
since $x^3-\beta^3=\irr(\beta,F,x)$ has degree 3 but since $[E:F]=3$ we must have that $[E:F(\beta)]=1$,
and hence, $E=F(\beta)$. Thus, assuming that we can find a $\beta\neq 0$ such that $\sigma(\beta)=\omega\beta$,
$E$ is obtained from $F$ by adding a cube root.

\subsection*{Problem 6}

Let $\zeta=\zeta_5$ be the fifth root of unity $e^{2\pi i/5}$, and consider the field $\Q(\zeta)$.
\begin{enumerate}[(i)]
    \item We know that $\Phi_5(x)=x^4+x^3+x^2+x+1$ is an irreducible polynomial in $\Q[x]$ of degree four of which
        $\zeta_5$ is a root. It follows that $[\Q(\zeta):\Q]=4$.
    \item Take $\alpha=\zeta+\zeta^{-1}$. Then we have that:
        \begin{align*}
            (\zeta+\zeta^{-1})^2+\zeta+\zeta^{-1}-1&=\zeta^2+\zeta^{-2}+2+\zeta+\zeta^{-1}-1\\
            &=\zeta^4+\zeta^3+\zeta^2+\zeta+1=0
        \end{align*}
        By the quadratic formula, then, we must have that
        \[\alpha=\left( -1\pm\sqrt{5} \right)/2.\]
        Let us take
        \[\zeta=e^{2\pi i/5}=\cos\left( 2\pi/5 \right)+\sin\left( 2\pi/5 \right).\]
        Then,
        \[\zeta^{-1}=e^{-2\pi i/5}=\cos\left( 2\pi/5 \right)-\sin\left( 2\pi/5 \right)\]
        and thus $\alpha=2\cos\left( 2\pi/5 \right)$. It's clear, since $2\pi/5<\pi/2$ that
        $\alpha>0$, and hence we must choose:
        \[\alpha=\left( -1+\sqrt{5} \right)/2.\]
    \item We now have
        \[\zeta^2-\alpha\zeta+1=\zeta^2-\left( \zeta+\zeta^{-1} \right)\zeta+1=\zeta^2-\zeta^2-\zeta^5+1=0.\]
        Hence, by the quadratic formula, we have that:
        \begin{align*}
            \zeta&=\frac{\alpha\pm\sqrt{\alpha^2-4}}{2}\\
            &=\frac{\left( -1+\sqrt{5} \right)/2\pm\sqrt{\frac{-5-\sqrt{5}}{2}}}{2}\\
            &=\frac{-1+\sqrt{5}}{4}+\left( \frac{1}{2}\sqrt{\frac{5+\sqrt{5}}{2}} \right)i
        \end{align*}
        Note that the sign must be positive, as we know that both the real and imaginary parts must be positive.
    \item Now let $\zeta=\zeta_7$ be the seventh root of unity, $e^{2\pi i/7}$, and consider the field $\Q(\zeta)$.
        Let $\alpha=\zeta+\zeta^2+\zeta^4$. Then,
        \begin{align*}
            \alpha^2+\alpha+2&=(\zeta+\zeta^2+\zeta^4)^2+\zeta+\zeta^2+\zeta^4+2\\
            &=2\zeta^2+2\zeta^3+2\zeta^5+2\zeta^4+2\zeta^6+\zeta^8+\zeta+2\\
            &=2\Phi_7(\zeta)=0
        \end{align*}
        By the quadratic formula, then, we see that
        \[\alpha=\frac{-1\pm\sqrt{-7}}{2}.\]

        Next let $\beta=\zeta+\zeta^{-1}$:
        \begin{align*}
            \beta^2&=\zeta^2+\zeta^{-2}=\zeta^2+\zeta^5+2\\
            \beta^2-2&=\zeta^2+\zeta^5\\
            (\beta^2-2)\beta&=\left( \zeta^2+\zeta^5 \right)\left( \zeta+\zeta^6 \right)\\
            &=\zeta^3+\zeta+\zeta^6+\zeta^4
        \end{align*}
        Hence, we see that
        \[(\beta^2-2)\beta+\left( \beta^2-2 \right)+1=\Phi_7(\zeta)=0\]
        and hence $\beta$ is the root of $(x^2-2)x+(x^2-2)+1=(x^2-2)(x+1)+1=x^3+x^2-2x-1$.
\end{enumerate}


\end{document}
