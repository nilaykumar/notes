\documentclass{../../mathnotes}

\usepackage{enumerate}

\title{Modern Algebra II: Problem Set 10}
\author{Nilay Kumar}
\date{Last updated: \today}


\begin{document}

\maketitle

\subsection*{Problem 1}

Take the polynomial $f(x)=x^2+x+1$. Let us find a root of this polynomial in $\Q(\sqrt{-3})[x]$:
\begin{align*}
    0&=(p+q\sqrt{-3})^2+p+q\sqrt{-3}+1\\
    &=p^2-3q^2+p+1+(2p+1)b\sqrt{-3},
\end{align*}
and thus there are two roots $-\frac{1}{2}\pm\frac{1}{2}\sqrt{-3}$. These coefficients are clearly not
integers and thus we have solutions that are not in $\Z(\sqrt{-3})$. Consequently, $f(x)$ is not irreducible
in $\Q(\sqrt{-3})[x]$. Now suppose that we can factor $x^2+x+1=(ax+b)(cx+d)$ with $a,b,c,d\in\Z[\sqrt{-3}]$.
Expanding, we find that $ac=1$, i.e. $a,c$ must be units. The units are $\pm 1$, and thus $b,d$ (up to
a minus sign) are roots of $f$, and we reach a contradiction. Finally, we cannot factor $f(x)=rg(x)$ where
$r\in\Z[\sqrt{-3}]$ is not a unit, as we would have to have $a_ir=1$ for $i=0,1,2$, i.e. $r$ would be a unit.
Consequently, we cannot factor anything out of $x^2+x+1$ in $\Z[\sqrt{-3}][x]$, and thus $f$ is irreducible
in this ring.

\subsection*{Problem 2}

\begin{enumerate}[(a)]
    \item Let $f(x)=2x^4-50x^3+100x^2-750x+60$. Note that $f$ is Eisenstein at 5, as $(5)$ is a prime ideal,
        2 does not divide 5 and 25 does not divide 60. Thus $f$ is irreducible as an element of $\Q[x]$. Note,
        additionally, that we can factor out a 2 from $f(x)$, and thus $f$ is not irreducible in $\Z[x]$, and thus
        factors as $f(x)=2(x^4-25x^3+50x^2-375x+30)$.
    \item Let $f(x)=x^3-2x^2+x+1$. If we examine $f$ in $\F_2[x]$, we find that $\bar f(x)=x^3+x+1$, which is
        irreducible as it has no root in $\F_2$. By the theorem proved in class, then $f(x)$ is irreducible
        in $\Q[x]$. By exactly the same reasoning as the first problem, we see that in $\Z[x]$ the polynomial does
        not factor as $(ax+b)(cx+d)$ or as $rg(x)$. Consequently, $f$ is also irreducible in $\Z[x]$. 
    \item Let $f(x)=2x^3+3x^2+3x+1$. By the rational roots test, we are motivated to test $-1/2$ as a root, as a positive number will clearly not yield zero,
        and because the numerator must divide the constant term and the denominator must divide the leading coefficient.
        It turns out that $x=-1/2$ is indeed a root of $f$, and thus $f(x)$ is reducible in $\Q[x]$. Furthermore, we can long divide
        and show that $f(x)/(2x+1)=x^2+x+1$ and thus $f(x)=(2x+1)(x^2+x+1)$ and thus is not irreducible in $\Z[x]$.
    \item Let $f(x)=x^4+5x^2+6=(x^2+2)(x^2+3)$. Clearly, then, $f$ is reducible in both $\Z[x]$ and $\Q[x]$. This is, in fact,
        the complete factorization in $\Z[x]$ and in $\Q[x]$ because neither $x^2+2$ or $x^2+3$ has a root in $\Q$.
    \item  Let $f(x)=3x^{27}-84$. $f$ is not irreducible in $\Z[x]$ because we can factor out a 3. However, $f$ is irreducible
        in $\Q[x]$, as it is Eisenstein at $4$.
\end{enumerate}

\subsection*{Problem 3}

We prove the contrapositive. Suppose $f(x)=g(x)h(x)$ is reducible with the degrees of $g,h$ greater than 0.
Clearly, then $f(ax+b)=g(ax+b)h(ax+b)$ is reducible. The degrees of $g(ax+b)$ and $h(ax+b)$ are equal to the
degrees of $g(x)$ and $h(x)$, just by term-by-term inspection (and because $a^n\neq 0$).

Conversely, suppose $f(ax+b)=g(x)h(x)$ is reducible. Now we simply change variables:
\[f(x)=g(a^{-1}x-a^{-1}b)h(a^{-1}x-a^{-1}b)\]
and find that $f(x)$ is reducible, again because the polynomials on the right hand side will have the same degree
as they did before.

\subsection*{Problem 4}

\begin{enumerate}[(i)]
    \item Let $f(x)=x^4+c$. Suppose $g(x)=x^2+ax+b$ is a factor of $f(x)$. Then, as in the above problem,
        we can claim that $g(-x)$ factors $f(-x)$. But note that $f(-x)=f(x)$, and thus $g(-x)=x^2-ax+b$
        must factor $f(x)$. Note that then $f(x)=x^4+(b-a^2)x^2+b^2$ and thus $2b=a^2$.
        The converse argument follows almost identically. Now suppose that $x^2+b$ is a factor
        of $f(x)$. Then we can write $f(x)=(x^2+b)(x^2+ex+f)=x^4+ex^3+(b+f)x^2+bex+bf$. Clearly we must have $e=0$,
        $b+f=0$, and $bf=c$, i.e. $-b^2=c$. Consequently, $x^2-b$ must also factor $f(x)$. 
        Thus if $c=b^2$ and $2b=a^2$, $f(x)$ cannot be irreducible as it is divisible by $x^2\pm ax+b$. Similarly,
        if $c=-b^2$, $f(x)$ is not irreducible as it is divisible by $x^2\pm b$. Conversely, note that
        if $f(x)$ is reducible into two quadratic polynomials. Then, if the linear term in one of these
        polynomials is zero, we must have by above, that $c=b^2$ for some $b\in F$, but if the linear term
        does not vanish, then $c=-b^2$ for some $b\in F$ and $2b=a^2$ for some $a\in F$.
    \item Now suppose that $f(x)=x^4+c_1x^2+c_2$. Just as before, since $f(x)=f(-x)$, and if $g(x)=x^2+ax+b$
        factors $f(x)$ then $g(-x)=x^2-ax+b$ must factor it as well. The algebra from before carries over identically
        and thus if $f$ factors in this way, $c_2=b^2$ and $c_1=2b-a^2$. Thus if $c_2$ is a square and there exists
        a square root $b$ of $c_2$ such that $2b-c_1=a^2$ is a square, $f(x)$ factors non-trivially as above. This shows
        that $f(x)=(x^2+ax+b)(x^2-ax+b)$ if and only if these conditions hold.
    \item Let $f(x)=x^4+c_1x^2+c_2$ as above. By the quadratic formula on $x^2$ we can find two terms linear in $x^2$,
        $x^2-a$ and $x^2-b$, as long as the discriminant in the formula is a square in $F$.
\end{enumerate}


\subsection*{Problem 5}

\begin{enumerate}[(i)]
    \item Let $f(x)\in\Z[x]$ be the polynomial $x^4-10x^2+1$. Using the notation from the previous problem, we have
        $c_1=-10$ and $c_2=1$. Note that $c_2$ is a square of $\pm 1$ and $2b-c_1=2(\pm1)+10$ is either 12 or 8.
        Since $12=2^2\cdot 3$ and $8=2^{3}$, if either 2 or 3 are squares in $\Z/p\Z$, 12 and 8 can be written as squares,
        i.e. the previous result holds in this case: $\bar f(x)=(x^2+ax+b)(x^2-ax+b)$.
    \item Using the third part of the previous problem we see that if $c_1^2-4c_2$ is a square, we can factor as desired.
        In our case we have $100-4=96=2^5\cdot 3=2^4\cdot 6$, and thus we can write $\bar f(x)=(x^2+c)(x^2+d)$ if and only if
        6 is a square in $\Z/p\Z$.
    \item By the given group theory hint we know that if two elements in $(\Z/p\Z)^*$ are not squares, then their product is always a square.
        In our case, if 2 and 3 are not squares, then $2\cdot 3=6$ must be a square and thus $\bar f(x)=(x^2+c)(x^2+d)$. On the other hand,
        if either 2 or 3 is a square, then $\bar f(x)=(x^2+ax+b)(x^2-ax+b)$ using the reasoning above.
    \item By the rational root test, rational roots of $f(x)$ must be $\pm 1$. However, neither of these actually is.
        Thus, $f(x)$ must factor as the product of two quadratics. But by the previous part, since 12 and 8 are
        note squares, and 96 is not a square, $f(x)$ cannot factor as the product of quadratics, either. Hence,
        $f(x)$ must be irreducible in $\Q[x]$.
\end{enumerate}


\end{document}
