\documentclass{../../mathnotes}

\usepackage{enumerate}

\title{Modern Algebra II: Problem Set 5}
\author{Nilay Kumar}
\date{Last updated: \today}


\begin{document}

\maketitle

\subsection*{Problem 1}

Let $f(x)=x^2+3x+2=(x+1)(x+2)\in(\mathbb{Z}/6\mathbb{Z})[x]$. Note that $f(1)=6\equiv0$, and so we can long divide $f(x)$
by $x-1$ to get $x+4$ with a remainder of $6\equiv0$. Thus, $-4\equiv 2$ is another root of $f(x)$, and we can write
\begin{align*}
    f(x)=(x-1)(x-2).
\end{align*}

\subsection*{Problem 2}

Let $R$ be the subring $\mathbb{Z}[2]=\left\{ a+b\sqrt{2}:a,b\in\mathbb{Z} \right\}$ of $\mathbb{R}$. Let $I=(6+\sqrt{2})$
be the principal ideal generated by $6+\sqrt{2}$. Similar to the last problem set, let us show that $\mathbb{Z}\cap I=34\mathbb{Z}$.
First we can show that $34\mathbb{Z}\subset \mathbb{Z}\cap I$. For some $n\in\mathbb{Z}$,
\begin{align*}
    \frac{34n}{6+\sqrt{2}}\cdot\frac{6-\sqrt{2}}{6-\sqrt{2}}=n(6-\sqrt{2}),
\end{align*}
which is in $(6+\sqrt{2})$, as $34n$ can be written as a $\mathbb{Z}\sqrt{2}$-multiple of $6+\sqrt{2}$. To show that
$\mathbb{Z}\cap I\subset 34\mathbb{Z}$, we take some $n\in \mathbb{Z}\cap I$,
\begin{align*}
    \frac{n}{6+\sqrt{2}}\cdot\frac{6-\sqrt{2}}{6-\sqrt{2}}=\frac{6n-n\sqrt{2}}{34},
\end{align*}
and so $6n$ and $n$ must be divisble by 34. Thus, $34|n$ and so $n\in 34\mathbb{Z}$ and $\mathbb{Z}\cap I=34\mathbb{Z}$.

Now, by problem 6 on the last problem set, we know that there exists an injective homomorphism $f:\mathbb{Z}/34\mathbb{Z}\to\mathbb{Z}[\sqrt{2}]/(6+\sqrt{2})$
defined by $f(n+34\mathbb{Z})=n+(6+\sqrt{2})$. Note additionally, that since $\sqrt{2}\equiv -6 \mod I$, we have
\begin{align*}
    a+b\sqrt{2}=a-6b\mod I.
\end{align*}
Thus, $f$ is surjective (again by problem 6 of the last set), because for some $a+b\sqrt{2}\mod I=a-6b$, we can take $n=a-6b\in\mathbb{Z}$, and its (quotient) projection down
to $\mathbb{Z}/34\mathbb{Z}$ will get sent to $a-6b\in\mathbb{Z}[\sqrt{2}]/(6+\sqrt{2})$. Consequently, $f$ is an isomorphism, and
thus $\mathbb{Z}[\sqrt{2}]/(6+\sqrt{2})\cong \mathbb{Z}/34\mathbb{Z}.$ As $\mathbb{Z}/34\mathbb{Z}$ and $\mathbb{Z}[\sqrt{2}]/(6+\sqrt{2})$ are fields, it follows that $(6+\sqrt{2})$ is a
maximal (and prime) ideal.

\subsection*{Problem 3}

Let $F$ be a field, and consider the ring (integral domain) $F[x]$.

\begin{enumerate}[(i)]
    \item Let $I$ be the principal ideal $(x-r)$ in $F[x]$. Every coset $p(x)+I\in F[x]/I$ contains a unique constant polynomial representative $a\in F$
        (by what we showed about polynomial long division in class, since $x-r$ has degree 1), so $F[x]/I\subset F$. Of course, since $F\subset F[x]$,
        $F\subset F[x]/I$, so it is clear
        that $F[x]/I\cong F$. In fact, take the homomorphism $\phi:F\to F[x]/I$ defined by $\phi(a)=a+I$. It is injective, as only multiples of
        $x-r$ are sent to zero, but the only multiple of $x-r$ in $F$ is 0. It is surjective as well, since for any $b+I\in F[x]/I$, $\phi(b)=b+I$. 
        Consequently, $\phi$ is an isomorphism. This agrees with the fact that if ${\rm ev}_r:F[x]\to F$ is the evaluation homomorphism, then
        $I=\ker {\rm ev}_{r}$, so that $F[x]/I\cong {\rm Im}\; {\rm ev}_r=F$. Since $F$ is a field, $F[x]/I$ is as well, and $I$ must be a maximal ideal.

    \item Let $I$ be the principal ideal $(x^2)$ in $F[x]$. Take any non-zero polynomial $p(x)\in F[x]$ (the zero case is trivial). By the long division algorithm, we 
        we know that there exist unique $q(x), r(x)$ such that $p(x)=x^2q(x)+r(x)$, where $\deg r(x)<2$ (or $r(x)=0$). In terms of cosets,
        we can write $p(x)=r(x)+I=a_0+a_1x+I$, where $a_0,a_1$ are unique. Consequently, every coset $p(x)+I$ contains a unique
        representative of the form $a_0+a_1x$, which we can write as $a_0+a_1\alpha$, if we let $\alpha=x+I$. In this notation,
        \begin{align*}
            (a_0+a_1\alpha)+(b_0+b_1\alpha)=a_0+b_0+(a_1+b_1)\alpha
        \end{align*}
        by the distributive property, and for multiplication:
        \begin{align*}
            (a_0+a_1\alpha)\cdot(b_0+b_1\alpha)&=a_0b_0+(a_0b_1+a_1b_0)\alpha+a_1b_1\alpha^2\\
            &=a_0b_0+(a_0b_1+a_1b_0)\alpha+a_1b_1(x^2 + I)\\
            &\equiv a_0b_0+(a_0b_1+a_1b_0)\alpha\mod I
        \end{align*}
        Note that we have used the fact that $\alpha^2=(x+I)^2$, which, by coset multiplication, is simply $x^2+I\equiv 0\mod I$.
        Thus, $I$ is not a prime (maximal) ideal, as there exist elements not in $I$ (namely, $\alpha$) that
        when multiplied together, yield a member of $I$.
    \item Let $I$ be the principal ideal $(x^2-1)$ in $F[x]$. Similar to above, take any non-zero polynomial $p(x)\in F[x]$. By the
        long division algorithm, we know that there exist unique $q(x),r(x)$ such that $p(x)=(x^2-1)q(x)+r(x)$, where $\deg r(x)<2$ (or $r(x)=0$).
        Thus we can write uniquely $p(x)=r(x)+I=a_0+a_1x+I$, or in terms of $\alpha=x+I$, $p(x)=a_0+a_1\alpha$. In this notation,
        \begin{align*}
            (a_0+a_1\alpha)+(b_0+b_1\alpha)=a_0+b_0+(a_1+b_1)\alpha
        \end{align*}
        by the distributive property, and
        \begin{align*}
            (a_0+a_1\alpha)\cdot(b_0+b_1\alpha)&=a_0b_0+(a_0b_1+a_1b_0)\alpha+a_1b_1\alpha^2\\
            &=a_0b_0+(a_0b_1+a_1b_0)\alpha+a_1b_1x^2\\
            &=(a_0b_0+a_1b_1)+(a_0b_1+a_1b_0)\alpha,
        \end{align*}
        where we have used the fact that, in $F[x]/I$, $x^2-1=0$, so $\alpha^2=x^2=1$. Still, $I$ is not a prime (maximal) ideal,
        as we can take $x-1$ and $x+1$ not in $I$ and multiply them to get and element of $I$:
        \begin{align*}
            (x-1)(x+1)=x^2-1\equiv0\mod I
        \end{align*}
    \item Continuing with $I=(x^2-1)$, and now assuming that $F$ is not of characteristic 2, we consider the ring homomorphism
        $F[x]\to F\times F$ defined by $p(x)\mapsto (p(1),p(-1))$. In other words, we consider the homomorphism $({\rm ev}_1,{\rm ev}_{-1})$.
        First note that any element $f(x)=(x^2-1)g(x)\in I$, with $g(x)\in F[x]$, is sent to $(0,0)$ in the product ring:
        \begin{align*}
            (x^2-1)g(x)\mapsto (0\cdot g(1), 0\cdot g(-1))=(0,0),
        \end{align*}
        so $I\subset\ker {\rm ev}_1$ and $I\subset\ker {\rm ev}_{-1}$. Now, if we define $\phi:F[x]/I\to F\times F$, such that
        $\phi(p(x)+I)=(p(1),p(-1))$, then $\phi$ is a homomorphism (considering $F\times F$ as a product ring):
        \begin{align*}
            \phi(p+I+q+I)=&( (p+I+q+I)(1),(p+I+q+I)(-1))\\
            =&(p(1)+q(1),p(-1)+q(-1))=\phi(p+I)+\phi(q+I)\\
            \phi( (p+I)(q+I))=&( (p+I)(q+I)(1),(p+I)(q+I)(-1) )\\
            =&( p(1)q(1),p(-1)q(-1))=\phi(p+I)\phi(q+I)\\
            \phi(0+I)=&(0,0)\\
            \phi(1+I)=&(1,1).
        \end{align*}
        Note also that
        \begin{align*}
            \phi(\alpha)=\phi(x+I)=(1,-1).
        \end{align*}
        To find elements that get mapped to $(0,1)$ and $(1,0)$, let us take
        \begin{align*}
            \phi(a_0+a_1\alpha+I)&=(a_0+a_1,a_0-a_1)=(1,0)\\
            \phi(b_0+b_1\alpha+I)&=(b_0+b_1,b_0-b_1)=(0,1).
        \end{align*}
        One solution to this system is $1/2+\alpha/2$ and $1/2-\alpha/2$ respectively. Here, in using $1/2$, we have used the fact that the field
        is not of characteristic. It follows immediately that $\phi$ is surjective, for we have, in some sense
        created a basis for $F\times F$: $(a,b)=a(1,0)+b(0,1)=\phi(a(1/2+\alpha/2)+b(1/2-\alpha/2)+I)$.

        We already know that $I\subset\ker\phi$. It is clear that $\ker\phi\subset I$ using the formula for $(a,b)$ that we just derived:
        \begin{align*}
            (0,0)=\phi(0+I).
        \end{align*}
        Hence, since the homomorphism $\phi$ is both injective and surjective, it is an isomorphism from $F[x]/I$ to $F\times F$.
\end{enumerate}


\subsection*{Problem 4}

Let $F$ be an infinite field, and let $E:F[x]\to F^F$ be the homomorphism from polynomials with coefficients in $F$ to the ring of all functions from
$F$ to $F$. Take any polynomial $p\in\ker E$, i.e. polynomials for which $E(p)=0$. However, since a non-zero polynomial can never be zero on
infinitely many elements of $F$, $p\equiv 0$, the zero polynomial. Consequently, $\ker E=\left\{ 0 \right\}$, and $E$ is injective. It follows
similarly that $E$ is not surjective -- take the function $f\in F^F$ that takes all but one element of $F$ to zero. By the above logic,
there is no polynomial that corresponds to this function, and thus $E$ cannot be surjective.

Now let $F=\left\{ a_1 \cdots a_n \right\}$ be a finite field, with $E$ defined identically as above.
The ring of functions $F^F$, must now be finite, since $F$ is finite. Then, since $E$ is mapping elements of
an infinite field $F[x]$ to elements of a finite one, $E$ cannot be injective.

To see that $E$ must be surjective, let
\begin{align*}
    q_i(x)&=\prod_{j\neq i}(x-a_j)\\
    p_i(x)&=\frac{q_i(x)}{q_i(a_i)}.
\end{align*}
$p_i(x)\in F[x]$ has the property that it evaluates to zero for all members of $F$, except for $a_i$, at which it evaluates to unity.
Any function $f\in F^F$ is specified by its action on the elements of $F$: let $f(a_i)=c_i$. Then we can write
\begin{align*}
    f(x)=\sum_{i=1}^n c_ip_i(x),
\end{align*}
because for some $x\in F$, every term will vanish except for the $p_i$ that corresponds to $x$, which will yields $c_i\cdot 1=c_i$,
as desired. Hence, $E$ is surjective.

\end{document}
