\documentclass{../../mathnotes}

\usepackage{enumerate}

\title{Modern Algebra II: Problem Set 6}
\author{Nilay Kumar}
\date{Last updated: \today}


\begin{document}

\maketitle

\subsection*{Problem 1}

Let $r\in\Q$ and $r=\delta^2$ for some $\delta\in\Q(\sqrt{2})$. We can write $\delta=a+b\sqrt{2}$ with $a,b\in\Q$,
and thus, $\delta^2=a^2+2b^2+2ab\sqrt{2}$. Since we know $r\in\Q$, it must be that $ab=0$, i.e. $a=0$ or $b=0$.
Then, $r=a^2$ or $r=2b^2$, as desired. Applying this to the case of $r=3$, we see that $3=a^2$ or $3=2b^2$.
Noting that $a$ is rational, i.e. can be written as $p/q$ for $p,q$ relatively prime integers, one can carry out the standard high
school argument that there are no such $p,q$ that the two above equations can hold. Let me detail the argument for
$3=a^2$: this means that $3=p^2/q^2$, which means that $p$ is divisble by 3, which implies that $q$ must also be divisible
by 3, which contradicts that $p,q$ are relatively prime. Consequently, no such $a$ exists in $\Q$, and it follows very
similarly that no such $b$ exists in $\Q$ either.
Such a $\delta$ cannot exist, then, and $\sqrt{3}\notin\Q(\sqrt{2})$. Thus, $x^2-3$ is irreducible in
$\Q(\sqrt{2})[x]$, as it has no roots in $\Q(\sqrt{2})[x]$. In other words, $x^2-3=\irr(\sqrt{3},\Q(\sqrt{2}),x)$.

\subsection*{Problem 2}

Let $\alpha=\sqrt{2}+\sqrt{3}$; $\alpha$ is a root of $x^4-10x^2+1$:
\begin{align*}
    (\sqrt{2}+\sqrt{3})^4&-10(\sqrt{2}-\sqrt{3})^2+1\\
    &=(5+2\sqrt{6})^2-10(5+2\sqrt{6})+1\\
    &=25+24+20\sqrt{6}-50-20\sqrt{6}+1\\
    &=0
\end{align*}
By the remark that we proved in class, then, $\irr(\alpha,\Q,x)$ divides $x^4-10x^2+1$. Note also that any subfield
$S$ of $\R$ contains a subfield isomorphic to $\Q$. Thus, if $S$ contains $\sqrt{2}$ and $\sqrt{3}$, it contains 
every number of the form $a+b(\sqrt{2}+\sqrt{3})$, where $a,b\in\Q$, i.e. $S$ contains $\Q(\alpha)$. Additionally,
\begin{align*}
    \alpha^2=(\sqrt{2}+\sqrt{3})^2=5+2\sqrt{6},
\end{align*}
so $\sqrt{6}\in\Q(\alpha)$. Then, if we multiply, 
\begin{align*}
    \alpha\sqrt{6}=\sqrt{12}+\sqrt{18}=2\sqrt{3}+3\sqrt{2},
\end{align*}
which is in $\Q(\alpha)$. But note that we can now isolate $\sqrt{2}$ and $\sqrt{3}$ as:
\begin{align*}
    \sqrt{2}=(2\sqrt{3}+3\sqrt{2})-2(\sqrt{2}+\sqrt{3})\\
    \sqrt{3}=3(\sqrt{2}+\sqrt{3})-(2\sqrt{3}+3\sqrt{2}),
\end{align*}
and thus $\sqrt{2}$ and $\sqrt{3}$ are in $\Q(\alpha)$, and $\Q(\alpha)$ is the smallest field that contains both.

\subsection*{Problem 3}

Let $\alpha=\sqrt{3+2\sqrt{2}}$; $\alpha$ is a root of $x^4-6x^2+1$:
\begin{align*}
    (3+2\sqrt{2})^2&-6(3+2\sqrt{2})+1\\
    &=17+12\sqrt{2}-18-12\sqrt{2}+1\\
    &=0.
\end{align*}
This polynomial is, in fact, reducible. Take the product
\begin{align*}
    (x^2+ax+b)(x^2-ax+b)=x^4+(2b-a^2)x^2+b^2.
\end{align*}
If we choose $b=-1$ and $a=2$ we obtain our polynomial:
\begin{align*}
    x^4-6x^2+1=(x^2+2x-1)(x^2-2x-1).
\end{align*}
Let us try to write the nested radical as $r+s\sqrt{2}$:
\begin{align*}
    \sqrt{3+2\sqrt{2}}&=r+s\sqrt{2}\\
    3+2\sqrt{2}&=r^2+2s^2+2rs\sqrt{2},
\end{align*}
which implies that $3=r^2+2s^2$ and $rs=1\iff r=1/s$. Inserting the second equation into the first yields
solutions for $s=\pm1,\pm1/\sqrt{2}$. Of course, the second pair are not rational, so we neglect them.
The first pair yields $\sqrt{3+2\sqrt{2}}=1+\sqrt{2}$. Thus, we see that our quartic polynomial has no
root in $\Q$, though it is reducible. Instead, there is a solution in $\Q(\sqrt{2})$.

\subsection*{Problem 4}

Working in the ring $\F_2[x]$, it is clear that the only linear polynomials are $x$ and $x+1$. Since these are
linear, they are irreducible. Recall from class that any linear or cubic polynomials over a field are
irreducible if and only if they have no roots. Note that any polynomial without a constant term is reducible,
as an $x$ can always be factored out. In addition, any polynomial with a constant term and an even number of
terms will have 1 as a root, and will thus be reducible. This leaves us with only $x^2+x+1$ as an irreducible
quadratic and $x^3+x+1$ and $x^3+x^2+1$ as irreducible cubics. Now, let us check if $x^4+x^3+x^2+x+1$ is
irreducible in $\F_2[x]$. First note that it has no roots, as $x=0,1$ both yield 1. Thus, this polynomial
cannot have a linear factor, and thus we only check whether our quadratic irreducible squares to it:
\begin{align*}
    (x^2+x+1)^2=x^4+(x+1)^2=x^4+x^2+1,
\end{align*}
and our polynomial must be irreducible.

\subsection*{Problem 5}

Let $F$ be a field and let $f(x)\in F[x]$ and $g(x)\in F[x]$ be relatively prime. We define a homormorphism
$\rho:F[x]\to (F[x]/(f(x)))\times (F[x]/(g(x)))$ by
\begin{align*}
    \rho(h(x))=(h(x)+(f(x)), h(x)+(g(x))).
\end{align*}
\begin{enumerate}[(i)]
    \item $\rho(h(x))=0$ if and only if $h(x)+(f(x))=0$ and $h(x)+(g(x))=0$, i.e. $h(x)\equiv 0\mod f(x)$
        and $h(x)\equiv 0\mod g(x)$. Of course, this is true if and only if $f(x)$ and $g(x)$ and both divide $h(x)$.
        But since $f,g$ are relatively prime, by what we showed in class, this is true if and only if $f(x)g(x)$
        divides $h(x)$. Consequently, $h(x)\in(f(x)g(x))$.
    \item By definition of relatively prime, there must exist $rs\in F[x]$ such $1=rf+sg$. We can use this to show
        that $\rho$ is surjective. If we take any $a(x),b(x)\in F[x]$ and set $h(x)=r(x)b(x)f(x)+s(x)a(x)g(x)$, we
        have
        \begin{align*}
            r(x)f(x)&=1-s(x)g(x)\\
            s(x)g(x)&=1-r(x)f(x),
        \end{align*}
        which allows us to simplify
        \begin{align*}
            \rho(h(x))=&(r(x)b(x)f(x)+s(x)a(x)g(x)+(f(x)),\\
            &\;r(x)b(x)f(x)+s(x)a(x)g(x)+(g(x)))\\
            =&(a(x)-r(x)a(x)f(x)+r(x)b(x)f(x)+(f(x)),\\
            &\;b(x)-s(x)b(x)g(x)+s(x)a(x)g(x)+(g(x)))\\
            =&(a(x)+(f(x)),b(x)+(g(x))),
        \end{align*}
        and hence $\rho$ is surjective.
\end{enumerate}
In particular, if $a,b\in F$ and $a\neq b$, then $x-a$ and $x-b$ are relatively prime, we have
\begin{align*}
    \frac{F[x]}{\left( (x-a)(x-b)\right)}\cong \frac{F[x]}{(x-a)}\times \frac{F[x]}{(x-b)}\cong F\times F.
\end{align*}
where we have used the fact that elements of $F[x]/(x-a)$ and $F[x]/(x-b)$ can uniquely be written
as constants (as we know from long division), and are thus each isomorphic to $F$.

\subsection*{Problem 6}

Let $E=\F_2(\alpha)$ where $\alpha$ is the root of $f(x)=x^2+x+1\in \F_2[x]$. Thus $f(x)$ must factor into
a product of linear polynomials $f(x)=(x+\alpha)(x+\beta)$, i.e.
\begin{align*}
    x^2+x+1=x^2+(\alpha+\beta)x+\alpha\beta.
\end{align*}
The only $\beta$ that satisfies $\alpha+\beta=1$ and $\alpha\beta=1$ in $E$ is $\alpha+1$, as
$\alpha(\alpha+1)=\alpha+\alpha^2=-1=1$:
\begin{align*}
    x^2+x+1=(x+\alpha)(x+\alpha+1)
\end{align*}
$\alpha$ cannot be a repeated root, because $(x+\alpha)^2=x^2+\alpha^2$, which is not, in general, equal to $x^2+x+1$.

\subsection*{Problem 7}

Let $f(x)=x^2+1\in\F_3[x]$. Note that $f$ has no roots, as $f(0)=1, f(1)=f(2)=2$. Consequently, since $f$ is
degree 2, it is irreducible in $\F_3[x]$. Then, $E=\F_3(\alpha)=\F_3[x]/(f(x))$ is a field (where
$\alpha=x+(f(x))$). Since $f(\alpha)=0$, we can find a linear factor (other than $x-\alpha$) of $x^2+1$ by long division, to be
$x+\alpha$ with remainder $1+\alpha^2=0$:
\begin{align*}
    (x+\alpha)(x-\alpha)=x^2-\alpha^2=x^2+1,
\end{align*}
using the fact that $\alpha^2+1=0$. Note that every element of $E$ can be written uniquely as $a_0+a_1\alpha$
where $a_0,a_1\in\F_3$, so $E$ has $3\times3=9$ elements. Indeed, as a group, $(E,+)$ is isomorphic
to $\Z/3\Z\times \Z/3\Z$, as one can add elements of $E$ ``componentwise'', and that addition is simply
the addition of $\Z/3\Z$ (more rigorously, $f(a_0+a_1\alpha)=(a_0,a_1)$ is an isomorphism). Since $E$ is a
field, every element but zero is a unit, i.e. $(E^*,\cdot)$ has
8 elements. Note that $\varphi(8)=4$, and so we expect $E^*$ to have 4 generators. It should be clear that
1 and 2 cannot be generators ($2^2=1$), as well as $\alpha,2\alpha$ ($\alpha^4=1$). This leaves
$1+\alpha,2+\alpha,1+2\alpha,2+2\alpha$. Let us check these:
\begin{align*}
    \begin{array}[]{cc}
        (1+\alpha)^2=2\alpha & (1+2\alpha)^2=\alpha\\
        (1+\alpha)^3=1+2\alpha & (1+2\alpha)^3=\alpha-2\\
        (1+\alpha)^4=2 & (1+2\alpha)^4=2\\
        (1+\alpha)^5=2+2\alpha & (1+2\alpha)^5=\alpha+2\\
        (1+\alpha)^6=\alpha & (1+2\alpha)^6=2\alpha\\
        (1+\alpha)^7=\alpha+2 & (1+2\alpha)^7=2+2\alpha\\
        (1+\alpha)^8=1 & (1+2\alpha)^8=1,
    \end{array}
\end{align*}
and the others follow just as powers of negatives of these elements. Consequently, there are four
generators.

\end{document}
