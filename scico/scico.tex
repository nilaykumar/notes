\documentclass{mathnotes}

\title{Notes on Scientific Computing}
\author{Nilay Kumar}
\date{Last updated: \today}


\begin{document}

\maketitle

\setcounter{section}{-1}

\section{Administrativia}

A standard reference on numerics is \textit{Numerical Recipies}. For a standard programming book on C, Kernigham and Ritchie is an excellent
place start to start. 

There will be 6 to 8 homework sets and a final project. There will be a review session every Friday at 3pm.

\subsection{Tentative topics}
\begin{itemize}
    \item Introduction to basic numerics and numerical stability
    \item Simple differential equation solvers
    \item Statistical mechanics of liquid argon; simulate $10^3-10^4$ interacting argon atoms; compute equation of state $f(P,V,T)=0$
    \item Monte Carlo techniques for a canonical ensemble; MC used to simulate thermal fluctuations; random number generation
    \item General discussion of topics in statistics; applications to the above two topics
    \item Linear algebra algorithms; large systems (sparse matrices of order $n=10^9$)
    \item Linear algebra on parallel computers
    \item \textbf{Time permitting:} General relativity; quantum Monte Carlo
\end{itemize}

\end{document}
