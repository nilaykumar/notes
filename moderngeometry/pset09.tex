\documentclass{../mathnotes}

\usepackage{tikz-cd}
\usepackage{todonotes}

\newgeometry{margin=1.75in}

\title{Modern Geometry I: PSET 9}
\author{Nilay Kumar\footnote{Collaborated with Matei Ionita.}}
\date{Last updated: \today}


\begin{document}

\maketitle

\subsection*{Problem 1}
Consider the curve $\gamma:\R\to S^n$ given by
\[t\mapsto \cos\left( |b|t \right)a + \sin\left( |b|t \right)\frac{b}{|b|}\]
where $a,b\in\R^{n+1}$. We claim that this is the unique geodesic on $(S^n,g_{\rm can})$
such that $\gamma(0)=a$ and $\gamma'(0)=b$. Uniqueness follows from the uniqueness
theorem proven in class, so it suffices to show that $\gamma$ satisfies the
geodesic equations,
%\[\frac{d^2x_k}{dt^2}+\sum_{ij}\Gamma_{ij}^k\frac{dx_i}{dt}\frac{dx_j}{dt}=0,\]
\[\frac{D}{dt}\frac{d\gamma}{dt}=\pi_{\gamma(t)}\left( \frac{d^2\gamma}{dt^2} \right)=0\]
where $\pi_{\gamma}(t)$ is the orthogonal projection $T_{\gamma(t)}\R^{n+1}$ to
$T_{\gamma(t)}S^n$. By definition,
\[\frac{d^2\gamma}{dt^2}=\sum_{k=1}^{n+1}\frac{d^2x_k}{dt^2}\frac{\partial}{\partial x_k},\]
which, after taking second derivatives, yields that
\[\frac{d^2\gamma}{dt^2}=-|b|^2\sum_{k=1}^{n+1}x_k\frac{\partial}{\partial x_k}.\]
But this is clearly orthogonal to the sphere, and hence 
\[\frac{D}{dt}\frac{d\gamma}{dt}=0.\]

\subsection*{Problem 2}
We first show that $\phi$ is an immersion. We compute the differential,
\begin{equation*}
    d\phi =
    \begin{pmatrix}
        -f(v)\sin u & f(v)\cos u & 0\\
        f(v)\cos u & f(v)\sin u & g'(v)
    \end{pmatrix},
\end{equation*}
and if no $2\times 2$ minor is full rank, then we must have
\begin{align*}
    0=f(v)f'(v)=f(v)g'(v)\sin u=f(v)g'(v)\cos u.
\end{align*}
Since $f(v)\neq 0$, $f'$ and $g'$ are never simultaneously zero,
and $\cos$ and $\sin$ are never simultaneously zero, this is not possible.
Hence $\phi$ is an immersion.
\begin{enumerate}[(a)]
    \item We compute the induced metric as follows:
        \begin{align*}
            \phi^*g_0 &= \left( f'(v)\cos u dv-f(v)\sin u du \right)^2\\
            &+(f'(v)\sin u dv + f(v)\cos u du)^2 + g'(v)^2 dv^2\\
            &= f'(v)^2dv^2+f(v)^2du^2+g'(v)^2dv^2\\
            &= f(v)^2 du^2 + \left( f'(v)^2+g'(v)^2 \right)dv^2.
        \end{align*}
    \item We compute the Christoffel symbols
        \begin{align*}
            \Gamma_{11}^1&=\Gamma_{22}^1 = 0\\
            \Gamma_{12}^2&=\Gamma_{21}^2 = 0\\
            \Gamma_{12}^1&=\Gamma_{21}^1 =\frac{f'}{f}\\
            \Gamma_{11}^2 &= -\frac{ff'}{f'^2+g'^2}\\
            \Gamma_{22}^2 &= \frac{f'f''+g'g''}{f'^2+g'^2}.
        \end{align*}
        which immediately yield the local equations for a geodesic $\gamma$:
        \begin{align*}
            \frac{d^2u}{dt^2}&+2\frac{f'}{f}\frac{du}{dt}\frac{dv}{dt} = 0\\
            \frac{d^2v}{dt^2}-\frac{ff'}{f'^2+g'^2}\left(\frac{du}{dt}\right)^2&+\frac{f'f''+g'g''}{f'^2+g'^2}\left( \frac{dv}{dt} \right)^2=0.
        \end{align*}
    \item The energy is given
        \begin{align*}
            |\gamma'(t)|^2=f^2\dot u^2+(f'^2+g'^2)\dot v^2.
        \end{align*}
        Differentiating with respect to time, assuming that the energy is constant, we find that
        \begin{align*}
            ff'\dot u^2\dot v+(f'f''+g'g'')\dot v^3+(f'^2+g'^2)\dot v\ddot v+f^2\dot u\ddot u=0,
        \end{align*}
        and using the first geodesic equation the last term becomes $-2f'\dot u\dot v/f$,
        and we find that
        \[\left(-ff'\dot u^2+(f'f''+g'g'') \dot v^2\right)\dot v+(f'^2+g'^2)\dot v\ddot v=0,\]
        as desired (since the geodesic is neither a meridian or a parallel).

        Now suppose $r\cos\beta=f\cos\beta$ is constant. Then, in $\R^{n+1}$, we find that
        \[\langle u',v'\rangle\cdot_g\langle 1,0\rangle = |\langle u',v'\rangle|_g|\langle 0,1\rangle|_g\cos\beta.\]
        This gives us the relation
        \begin{align*}
            f^2\dot u &= f\sqrt{f^2\dot u^2+(f'^2+g'^2)\dot v^2}\cos\beta\\
            f\cos\beta &= \frac{f^2\dot u}{\sqrt{f^2\dot u^2+(f'^2+g'^2)\dot v^2}}.
        \end{align*}
        Taking a time derivative using the quotient rule, the first term in the numerator is
        \begin{align*}
            (2ff'\dot v\dot u + f^2\ddot u)\sqrt{f^2\dot u^2+(f'^2+g'^2)\dot v^2},
        \end{align*}
        which by the first geodesic equation goes to zero. The second term is more complicated,
        \begin{align*}
            \frac{f^2\dot u\left(2ff'\dot v\dot u^2+2f^2\dot u\ddot u+(2f'f''+2g'g'')\dot v^3+2(f'^2+g'^2)\dot v\ddot v\right)}{\sqrt{f^2\dot u^2+(f'^2+g'^2)\dot v^2}}
        \end{align*}
        %The first geodesic equation yields
        %\begin{align*}
        %    2ff'\dot v\dot u^2+2f^2\dot u\ddot u = \dot u\ddot u/f^2=-\frac{2ff'}{f^4}\dot u^2\dot v,
        %\end{align*}
        %while the second yields
        %\begin{align*}
        %    2(f'f''+2g'g'')\dot v^3+2(f'^2+g'^2)\dot v\ddot v &= -2ff'\dot v\dot u^2%-2(f'f''+g'g'')\dot v^2
        %\end{align*}
        Invoking the first geodesic equation on the first two terms and the second on the last two,
        and noting that the leftover terms cancel, we find that the derivative vanishes and hence
        \[r\cos\beta=\text{const}.\]
    \item The geodesic equations for the paraboloid yield
        \[\frac{d^2u}{dt^2}+\frac{2}{v}\frac{du}{dt}\frac{dv}{dt}=0\]
        We solve this by separating,
        \[\left( \frac{du}{dt} \right)^{-1}\frac{d^2u}{dt^2}=\frac{2}{v}\frac{dv}{dt},\]
        which yields
        \[\frac{du}{dt}=\frac{c}{v^2},\]
        for some constant $c$. Now, it is clear that $r(t)=v$ must have a minimum by Clairaut's relation,
        otherwise $\cos\beta\to\infty$, which is absurd. Suppose the minimum occurs at $t=0$ with value $v_0$.
        The geodesic is clearly tangent to the parallel at this point, and $\beta(0)=0$. For
        both $t<0$ and $t>0$, we must have that $v$ is increasing, and hence, as long as
        $u\to\infty$ as $t\to\infty$, the geodesic will keep circling the paraboloid and hence intersect
        itself infinitely many times. Then, since the geodesic is parameterized by the arc length,
        $v(t)^2-v_0^2\leq t$, and so
        \[\frac{du}{dt}=\frac{c}{v^2}\geq\frac{c}{t+v_0^2}\]
        and \[u(t)\geq c\log|t+v_0^2|+\text{const},\]
        as desired.
\end{enumerate}

\subsection*{Problem 3}

Suppose, without loss of generality, that the integral curve $\gamma$ starts at $e$.
By the uniqueness of geodesics for a given point $e\in G$ and velocity $X_e\in T_eG=\fr g$,
it suffices to show that the one-parameter subgroup $\eta:t\mapsto \exp(tX_e)$ is a geodesic,
i.e. $\nabla_XX=0$. But applying the Koszul formula for $Y$ any left-invariant vector field
(with $Z=X$), we find that 
\[\langle Y,\nabla_XX\rangle = \langle X,[Y,X]\rangle,\]
as the first three terms vanish due to the (left-invariant) inner products of left-invariant vector fields being
constant, and the last two terms obviously cancel. Moreover, we claim that the bi-invariance of the inner
product yields the identity
\[\langle [U,V],W\rangle = -\langle U,[V,W]\rangle.\]
Before we prove this, note that the identity yields
\begin{align*}
    \langle Y,\nabla_XX\rangle &= \langle X,[Y,X]\rangle \\
    &= -\langle [Y,X],X\rangle\\
    &=-\langle Y,[X,X]\rangle\\
    &=0
\end{align*}
for all $X$. Hence one-parameter subgroups are geodesics, and by the uniqueness
theorem, geodesics must be one-parameter subgroups.

Let us now prove the above identity for bi-invariant metrics. Recall first that
if $A,B\in T_eG$ and $\phi^A_t$ is the flow of $A$, then
\[ [B,A] = \lim_{t\to 0}\frac{d\phi^A_tB-B}{t}.\]
The left-invariance of the vector field generated by $X$ yields $L_g\circ \phi^A_t=\phi^A_t\circ L_g$,
and hence
\[\phi^A_tg=\phi_t^A(L_ge)=L_g(\phi_t^Ae)=g\phi_t^Ae=R_{\phi_t^Ae}g.\]
Taking the differential, we find that $d\phi_t^A=dR_{\phi_t^Ae}$ and so we can write
\[ [B,A] = \lim_{t\to 0}\frac{dR_{\phi_t^Ae}B-B}{t}=\lim_{t\to 0}\frac{Ad(\phi_{-t}^Ae)B-B}{t}\]
since $Ad(g)B=dR_{g^{-1}}dL_gB=dR_{g^{-1}}B$ by left-invariance of the vector field generated by $Y$.
Now, returning to the metric, we write for $U,V\in T_eG$,
\begin{align*}
    \langle U,V\rangle &= \langle dR_{\phi^A_te}\circ dL_{\phi^A_{-t}e}U, dR_{\phi^A_te}\circ dL_{\phi^A_{-t}e}V\rangle\\
    &= \langle dR_{\phi^A_te} U,dR_{\phi_t^Ae} V\rangle,
\end{align*}
via bi-invariance.
Denoting by $\tilde U=dR_{\phi^A_te}U$ and $\tilde V=dR_{\phi^A_te}V$ and the metric by $g$,
and taking a $t$-derivative, we obtain
\begin{align*}
    0 &= \frac{d}{dt}\bigg|_{t=0} \langle \tilde U,\tilde V\rangle
    = \frac{d}{dt}\bigg|_{t=0}\sum_{ij}g_{ij}\tilde U_i\tilde V_j\\
    &= \sum_{ij}g_{ij}\left( \frac{d\tilde U_i}{dt}\bigg|_{t=0}\tilde V_j|_{t=0}+\tilde U_i|_{t=0}\frac{d\tilde V_j}{dt}\bigg|_{t=0} \right)\\
    &= \sum_{ij}g_{ij}\left( \lim_{t\to 0}\frac{1}{t}(dR_{\phi^A_te}U_i-U_i)\tilde V_j|_{t=0}+\tilde U_i|_{t=0}\lim_{t\to 0}\frac{1}{t}(dR_{\phi_t^Ae}V_j-V_j) \right)\\
    &= \sum_{ij}g_{ij}\left( [U_i,A]V_j+U_i[V_j,A] \right)\\
    &= \langle [U,A],V\rangle + \langle U, [V,A]\rangle,
\end{align*}
as desired.

\subsection*{Problem 4}
\begin{enumerate}[(a)]
    \item For all $X,Y\in C^\infty(M,TM)$, the pushforwards $F_*X,F_*Y$ are sections of
        the pullback bundle $C^\infty(M,F^*TN)$. By construction of the pullback bundle,
        we can find $\tilde X,\tilde Y\in C^\infty(N,TN)$ such that
        \begin{align*}
            F_*X &= F^*\tilde X\\
            F_*Y &= F^*\tilde Y.
        \end{align*}
        Then, since the pullback connection satisfies $D_{F_*X}(F^*\tilde Y)=F^*(\nabla_{\tilde X}\tilde Y)$,
        we find that
        \begin{align*}
            D_{F_*X}(F_*Y)-D_{F_*Y}(F_*X) &= D_{F_*X}(F^*\tilde Y)-D_{F_*Y}(F^*\tilde X)\\
            &= F^*(\nabla_{\tilde X}\tilde Y-\nabla_{\tilde Y}\tilde X)\\
            &= F^*([\tilde X,\tilde Y]).
        \end{align*}
        Now, since $X$ is $F$-related to $\tilde X$ and $Y$ is $F$-related to $\tilde Y$,
        $[X,Y]$ must be $F$-related to $[\tilde X,\tilde Y]$ and hence
        \[F^*([\tilde X,\tilde Y])=F_*([X,Y]),\]
        as desired.
    \item Similar to above, we write
        \begin{align*}
            \langle D_XV,W\rangle + \langle V,D_XW\rangle &= \langle D_XF^*\tilde V,F^*\tilde W\rangle + \langle F^*\tilde V,D_XF^*\tilde W\rangle\\
            &= F^*\left(\langle \nabla_{\tilde X}\tilde V,\tilde W\rangle + \langle \tilde V, \nabla_{\tilde X}\tilde W\rangle\right)\\
            &= F^*\left( \tilde X\langle\tilde V,\tilde W\rangle \right)\\
            &= X\langle V, W\rangle,
        \end{align*}
        where we have used the fact that $V,W$ are $F$-related to $\tilde V,\tilde W$.
\end{enumerate}

\end{document}
