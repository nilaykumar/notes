\documentclass{../mathnotes}

\usepackage{tikz-cd}
\usepackage{todonotes}

\newgeometry{margin=1.75in}

\title{Algebraic Topology I: Final Exam}
\author{Nilay Kumar}
\date{Last updated: \today}


\begin{document}

\maketitle

\begin{enumerate}
    \item[Q1.] Construct a non-normal covering space of the Klein bottle
        by a Klein bottle. Compute the map induced by this covering on the fundamental
        groups and on cohomology with coefficients in $\Z_2$.
    \item[A1.] Denote by $K$ the Klein bottle and consider its triple cover by $\tilde K$
        another Klein bottle as drawn. Choose basepoints $x_0,\tilde x_0$ to be the bottom
        left corners in $K$ and $\tilde K$. Note that the loop $a$ based at the basepoint of $K$ has
        three different lifts in $\tilde K$. Only the lift at $\tilde x_0$, however, 
        remains a loop, as the endpoints of the other lifts of $a$ are not identified in
        the structure of the Klein bottle. Hence this covering is non-normal, as there does not
        exist an isomorphism of covering spaces that swaps $\tilde x_0$ with another lift.
        The Klein bottle is constructed here as a wedge of two circles (labelled by $a$ and $b$)
        with a 2-cell attached by the word $aba^{-1}b$. Hatcher's Proposition 1.26 allows
        us then to write $\pi_1(K,x_0)=\langle a,b\mid aba^{-1}b\rangle$, i.e. the group
        generated by homotopy classes of loops $a$ and $b$ with relations $aba^{-1}b=1$.
        The covering map $p:\tilde K\to K$ drawn induced a morphism
        $p_*:\pi_1(\tilde K,\tilde x_0)\to\pi_1(K, x_0)$ whose image, by Hatcher Proposition
        1.31, consists of the loops in $K$ (based at $x_0$) whose lifts to $\tilde K$
        (starting at $\tilde x_0$) are also loops. The map $p_*$, then, maps the generators
        $a,b\mapsto a,b^3$, as evident in the drawing. The corresponding map on homologies,
        via the abelianization functor, yields a map $\Z\oplus\Z_2\to\Z\oplus\Z_2$ (since
        $1=aba^{-1}b=b^2$ by commutativity) taking $a,b\mapsto a,3b=b$. Hence the covering
        map induces an isomorphism $p_*:H_1(\tilde K;\Z)\to H_1(K;\Z)$. The naturality
        of the universal coefficient theorem yields a commutative diagram
        \begin{equation*}
            \begin{tikzcd}
                0\ar{r}& \Ext(\Z,\Z_2)\ar{r}\ar{d}& H^1(\tilde K;\Z_2)\ar{r}\ar{d}& \Hom(\Z\oplus\Z_2,\Z_2)\ar{r}\ar{d}{p_*}& 0\\
                0\ar{r}& \Ext(\Z,\Z_2)\ar{r}& H^1(K;\Z_2)\ar{r}& \Hom(\Z\oplus\Z_2,\Z_2)\ar{r}& 0
            \end{tikzcd}
        \end{equation*}
        but since $\Ext(\Z,\Z_2)=0$ and the rightmost vertical map is an isomorphism by
        the argument above, with $\Hom(\Z\oplus\Z_2,\Z_2)\cong\Z_2\oplus\Z_2$, we find
        that the map $p^*$ induced on $\Z_2$-cohomology is the identity, $p^*:\Z_2^2\cong\Z_2^2$.
        To compute the map on second cohomology, we apply the naturality again: this time,
        since the second homology of the Klein bottle is zero (and the third term in the exact
        sequence vanishes), the cohomology is given purely by $\Ext(\Z\oplus\Z_2,\Z_2)=\Z_2$
        and the morphism takes $b\mapsto b$, which is an isomorphism.
        \newpage
    \item[Q2.] Take two copies $M_1,M_2$ of the M\"obius band. Identify the boundary
        circle of $M_1$ with the core circle of $M_2$, and in the same way, identify the boundary
        circle of $M_2$ with the core circle of $M_1$. Let $M$ be the resulting space.
        \begin{enumerate}[(a)]
            \item Describe a cell decomposition of $M$.
            \item Describe the fundamental group of $M$ via generators and relations.
            \item Compute integral homology groups of $M$.
        \end{enumerate}
    \item[A2.]
        \begin{enumerate}[(a)]
            \item Recall the usual cell decomposition for the M\"obius band, as drawn to the left,
                with two 0-cells, three 1-cells, and one 2-cell (attached along $abac$).
                If we wish to think about
                identifying the core and boundary circles, however, we will need to refine
                our complex. Hence we take a decomposition of the M\"obius band $M_1$, as drawn to the
                right, with four 0-cells, six 1-cells, and two 2-cells (attached along $d_1d_2^{-1}a_2ba_1$
                and $d_2d_1^{-1}a_2ca_1$). Now let us introduce an identical copy of this
                M\"obius band, $M_2$, where we identify the core circle of $M_2$ with the boundary
                circle of $M_1$ and the boundary circle of $M_2$ with the core circle of $M_1$ in
                order to obtain $M$.
                This yields the two graphs on the left (with 2-cells left implicit), which can be
                drawn as one graph, on the right (again, with the four 2-cells left implicit),
                where two of the 2-cells are attached via 
                \begin{align*}
                    r_1&=d_1d_2^{-1}a_2ba_1\\
                    r_2&=d_2d_1^{-1}a_2ca_1,\\
                    r_3&=d_1\tilde a_1bc^{-1}\tilde a_2,\\
                    r_4&=d_2\tilde a_1cb^{-1}\tilde a_2.
                \end{align*}
                This yields the desired cell decomposition of $M$.
            \item Computing the fundamental group of $M$ in terms of generators and relations
                is rather computational. Unfortunately, we cannot use the Seifert-van Kampen theorem
                with $M_1$ and $M_2$ as our open subsets, as $M_1\cap M_2$ is not path-connected.
                Instead, we can use a brute force method, where we invoke Hatcher's Proposition 1.26,
                i.e. compute the fundamental group of the one skeleton $M^1$, which is a graph,
                and then impose relations according to how the four 2-cells are attached to $M^1$.
                Computing the fundamental group of the graph $M^1$ is straightforward (c.f. Hatcher's
                Example 1.22): we choose the maximal tree given by the word $ba_1d_1$, and conclude
                that since there are five leftover edges, $\pi_1(M^1,v_0)=\Z^{*5}$. The generators are
                given explicitly by the loops
                \begin{align*}
                    g_1 &= d_2d_1^{-1}\\
                    g_2 &= a_1^{-1}b^{-1}ca_1\\
                    g_3 &= a_2ba_1\\
                    g_4 &= d_1\tilde a_1ba_1\\
                    g_5 &= a_1^{-1}b^{-1}\tilde a_2.
                \end{align*}
                Some computation reveals that the words by which the 2-cells are attached can
                be rewritten in terms of these generators as:
                \begin{align*}
                    r_1 &= d_1d_2^{-1}a_2ba_1 = g_1g_3,\\
                    r_2 &=d_2d_1^{-1}a_2ca_1 = g_1g_3g_2,\\
                    r_3 &= d_1\tilde a_1bc^{-1}\tilde a_2 = g_4g_2^{-1}g_5,\\
                    r_4 &= d_2\tilde a_1cb^{-1}\tilde a_2 = g_1g_4g_2g_5.
                \end{align*}
                Let us quickly check that these do indeed hold:
                \begin{align*}
                    g_1g_3 &= (d_2d_1^{-1})(a_2ba_1)\\
                    &= r_1,\\
                    g_1g_3g_2 &= (d_2d_1^{-1})(a_2ba_1)(a_1^{-1}b^{-1}ca_1)\\
                    &=d_2d_1^{-1}a_2ca_1\\
                    &= r_2,\\
                    g_4g_2^{-1}g_5 &= (d_1\tilde a_1ba_1)(a_1^{-1}c^{-1}ba_1)(a_1^{-1}b^{-1}\tilde a_2)\\
                    &=d_1\tilde a_1bc^{-1}\tilde a_2\\
                    &= r_3,\\
                    g_1g_4g_2g_5 &= (d_2d_1^{-1})(d_1\tilde a_1ba_1)(a_1^{-1}b^{-1}ca_1)(a_1^{-1}b^{-1}\tilde a_2)\\
                    &=d_2\tilde a_1cb^{-1}\tilde a_2\\
                    &= r_4.
                \end{align*}
                Phew. We conclude that
                \[\pi_1(M,v_0) = \langle g_1,g_2,g_3,g_4,g_5\mid g_1g_3,g_1g_3g_2,g_4g_2^{-1}g_5,g_1g_4g_2g_5\rangle.\]
            \item Let us now compute the integral homology groups of $M$. It is immediate from the cell decomposition
                of $M$ that $H_0(M,\Z)=\Z$ and $H_n(M;\Z)=0$ for $n>2$. It remains to compute the first and second
                homologies. The first homology is computed by abelianization of the fundamental group above:
                the relations yield $g_1=-g_3,g_2=0,g_4=-g_5,g_1=0,$ and so we obtain $\Z=\langle g_4\rangle$, i.e.
                $H_1(M;\Z)=\Z$. To compute the second homology, we write the the following portion of the Mayer-Vietoris
                long exact sequence for the subspaces $M_1$ and $M_2$:
                \begin{equation*}
                    \begin{tikzcd}
                        \cdots\ar{r}& H_2(M)\ar{r}& H_1(M_1\cap M_2)\ar{r}& H_1(M_1)\oplus H_1(M_2)\ar{r}& \cdots,
                    \end{tikzcd}
                \end{equation*}
                which becomes
                \begin{equation*}
                    \begin{tikzcd}
                        0\ar{r}& H_2(M)\ar{r}& \Z\oplus\Z\ar{r}& \Z\oplus \Z\ar{r}& \cdots,
                    \end{tikzcd}
                \end{equation*}

                where we have used the fact that $H_2(M_i)=0,H_1(M_i)=\Z,$ and $M_1\cap M_2=S^1\sqcup S^1$
                (a disjoint union of the boundary and core circles of $M_1$). We claim that the map
                $\Z\oplus\Z\to\Z\oplus\Z$ is injective, from which it follows that $H_2(M)=0$. This
                map is the induced map on homology of the map $\phi:C_1(M_1\cap M_2)\to C_1(M_1)\oplus C_1(M_2)$
                taking $x\mapsto (x,-x)$. In our case, since $M_1\cap M_2=S^1\sqcup S^1$, any chain is
                written $na+mb$ where $a,b$ are generators for the homologies of the core and boundary circles.
                The inclusion to $C_1(M_1)$ takes $na+mb$ to a coboundary if and only if $m=2n$ (since
                the boundary of the 2-cell yields $2a+b$) and the inclusion to $C_1(M_2)$ takes $na+mb$
                to a coboundary if and only if $n=2m$ (because the core and boundary circles are switched
                for the second copy of the M\"obius band); thus we obtain a coboundary if and only if $n=m=0$,
                which implies that the map is injective. To conclude, we have computed
                \begin{align*}
                    H_0(M;\Z) &= \Z,\\
                    H_1(M;\Z) &= \Z,
                \end{align*}
                with all higher homology groups zero.
        \end{enumerate}
        \newpage
    \item[Q3.]
        \begin{enumerate}[(a)]
            \item Show that the space $M$ in problem 2 is a $K(\pi, 1)$-space.
            \item Prove that a connected 3-dimensional CW-complex $X$ such that
                $\pi_2(X)=\pi_3(X)=0$ is a $K(\pi,1)$-space.
        \end{enumerate}
    \item[A3.]
        \begin{enumerate}[(a)]
            \item Note that is suffices to show that the universal cover $\tilde M$ of $M$
                is contractible, as the long exact sequence of fibrations yields isomorphisms
                $\pi_n(\tilde M)\cong\pi_n(M)$ for $n>1$. It is easy to see that the universal
                cover $\tilde M_1$ of the M\"obius strip $M_1$ is the infinite strip (say, $\R\times I$).
            \item Let $\tilde X$ be the universal covering space of $X$. We deduce, from
                the properties of covering maps, that $\tilde X$ must have the same dimension
                as $X$, i.e. must have no $n$-cells for $n>4$. Cellular homology immediately
                implies that $H_n(\tilde X;\Z)=0$ for $n>4$. Now, since
                $\pi_1(\tilde X)=\pi_2(\tilde X)=\pi_3(\tilde X)=0$, $\tilde X$ is 3-connected.
                The Hurewicz theorem thus implies that $\pi_4(\tilde X)=H_4(\tilde X;\Z)=0$.
                Iterating, we find that $\pi_n(\tilde X)=0$ for $n>0$. For $n>1$, the
                long exact sequence of homotopy groups yields isomorphism $\pi_n(\tilde X)\cong\pi_n(X)$,
                so $\pi_n(X)=0$ for $n>1$, whence $X$ is a $K(\pi,1)$-space.
        \end{enumerate}
        \newpage
    \item[Q4.] Modify the construction of the Alexander horned sphere to produce an embedding
        $S^2\hookrightarrow\R^3$ for which neither component of $\R^3\setminus S^2$ is simply connected.
        The Alexander horned sphere is discussed in Example 2B.2 on page 170 of Hatcher.
    \item[A4.] Hatcher's construction of the Alexander horned sphere consists of a space $X$
        homeomorphic (via an embedding $f$ constructed in the text) to ball in $\R^3$ whose
        boundary is, of course, a sphere $S$. Hatcher shows that the unbounded component of the
        complement $\R^3-S$ has nontrivial $\pi_1$. The bounded component, on the other hand,
        by virtue of being homeomorphic to an open ball in $\R^3$, has trivial fundamental group
        and is thus simply connected. To make the bounded component non-simply connected
        as well, we can simply delete a circle $S^1$ in the main body of $X$; the construction
        presented in Hatcher proceeds exactly as before, just instead of using balls at each step,
        one now uses the complement in each ball of $S^1$. The image of the new embedding $\tilde X$
        has boundary the usual sphere but now also the (disjoint) circle that was deleted. By Hatcher's
        arguments, the unbounded portion of the complement $\R^3- S^2$ still has nontrivial
        $\pi_1$, but now the bounded portion clearly does as well, since any loop winding around
        the deleted circle cannot be contracted to a point.

        \newpage
    \item[Q5.] Suppose that $X$ has integral homology groups
        \[H_0(X)=\Z, H_2(X)=\Z_9\oplus\Z_3, H_3(X)=\Z_{72}\oplus\Z,\]
        and all other groups are zero. Determine homology groups of $X$ with
        coefficients in $\Z_4,\Z_9,$ and $\Q$. Determine cohomology groups
        of $X$ with coefficients in $\Z,\Z_2,\Z_9,\Q/\Z$.
    \item[A5.] Recall that the universal coefficient theorem for homology
        yields a split short exact sequence
        \begin{equation*}
            \begin{tikzcd}
                0\ar{r}& H_n(X)\otimes G\ar{r}& H_n(X;G)\ar{r}& \Tor(H_{n-1}(X),G)\ar{r}& 0.
            \end{tikzcd}
        \end{equation*}
        Taking $G=\Q$, we find that 
        \begin{align*}
            H_0(X;\Q)&\cong\Z\otimes\Q\oplus 0=\Q,\\
            H_1(X;\Q)&\cong0\oplus\Tor(\Z,\Q)=0,\\
            H_2(X;\Q)&\cong (\Z_9\oplus\Z_3)\otimes\Q\oplus0=0,\\
            H_3(X;\Q)&\cong (\Z_{72}\oplus\Z)\otimes\Q\oplus\Tor(\Z_{72}\oplus\Z,\Q)=\Q.
        \end{align*}
        with all other rational homologies zero.
        In the following, we use the fact that $\Tor(\Z_n,\Z_m)=\Z_n\otimes\Z_m=\Z_{\text{gcd}(m,n)}$.
        Taking $G=\Z_4$, we find that
        \begin{align*}
            H_0(X;\Z_4) &\cong\Z\otimes\Z_4\oplus0=\Z_4,\\
            H_1(X;\Z_4) &\cong0\otimes\Z_4\oplus \Tor(\Z,\Z_4)=0,\\
            H_2(X;\Z_4) &\cong\Z_9\otimes\Z_4\oplus\Z_3\otimes\Z_4=0\\
            H_3(X;\Z_4) &\cong(\Z_{72}\oplus\Z)\otimes\Z_4\oplus\Tor(\Z_9\oplus\Z_3,\Z_4)\\
                        &= \Z_4^2\oplus\Tor(\Z_9,\Z_4)\oplus\Tor(\Z_3,\Z_4)\\
                        &= \Z_4^2\\
            H_4(X;\Z_4) &\cong\Tor(\Z_{72}\oplus\Z,\Z_4)=\Z_4.
        \end{align*}
        Finally, taking $G=\Z_9$, we find that
        \begin{align*}
            H_0(X;\Z_9) &\cong \Z\otimes\Z_9=\Z_9,\\
            H_1(X;\Z_9) &\cong \Tor(\Z,\Z_9)=0,\\
            H_2(X;\Z_9) &\cong \Z_9\otimes\Z_9\oplus\Z_3\otimes\Z_9=\Z_9\oplus\Z_3,\\
            H_3(X;\Z_9) &\cong \Z_{72}\otimes\Z_9\oplus\Z\otimes\Z_9\oplus\Tor(\Z_9\oplus\Z_3,\Z_9)\\
                        &= \Z_9^3\oplus\Z_3,\\
            H_4(X;\Z_9) &\cong \Tor(\Z_{72}\oplus\Z,\Z_9)=\Z_9.
        \end{align*}
        Next recall that the universal coefficient theorem for cohomology yields 
        a split short exact sequence
        \begin{equation*}
            \begin{tikzcd}
                0\ar{r}& \Ext(H_{n-1}(X),G)\ar{r}& H^n(X;G)\ar{r}& \Hom(H_n(X),G)\ar{r}& 0.
            \end{tikzcd}
        \end{equation*}
        Taking $G=\Z$, we find that
        \begin{align*}
            H^0(X;\Z) &\cong \Hom(\Z,\Z)=\Z,\\
            H^1(X;\Z) &\cong \Ext(\Z,\Z)=0,\\
            H^2(X;\Z) &\cong \Hom(\Z_9\oplus\Z_3,\Z)=0,\\
            H^3(X;\Z) &\cong \Ext(\Z_9\oplus\Z_3,\Z)\oplus\Hom(\Z_{72}\oplus\Z,\Z)\\
            &= \Ext(\Z_9,\Z)\oplus\Ext(\Z_3,\Z)\oplus\Z=\Z\oplus\Z_3\oplus\Z_9,\\
            H^4(X;\Z) &\cong\Ext(\Z_{72}\oplus\Z,\Z)=\Z_{72}.
        \end{align*}
        Taking $G=\Z_2$, we find that
        \begin{align*}
            H^0(X;\Z_2) &\cong\Hom(\Z,\Z_2)=\Z_2,\\
            H^1(X;\Z_2) &\cong\Ext(\Z,\Z_2)=0,\\
            H^2(X;\Z_2) &\cong\Hom(\Z_9\oplus\Z_3,\Z_2)=0,\\
            H^3(X;\Z_2) &\cong\Ext(\Z_9\oplus\Z_3,\Z_2)\oplus\Hom(\Z_{72}\oplus\Z,\Z_2)\\
            &=\Z_2^2,\\
            H^4(X;\Z_2) &\cong\Ext(\Z_{72}\oplus\Z,\Z_2)=\Z_2^2.
        \end{align*}
        Taking $G=\Z_9$, we find that
        \begin{align*}
            H^0(X;\Z_9) &\cong\Hom(\Z,\Z_9)=\Z_9,\\
            H^1(X;\Z_9) &\cong\Ext(\Z,\Z_9)=0,\\
            H^2(X;\Z_9) &\cong\Hom(\Z_9\oplus\Z_3,\Z_9)=\Z_9\oplus\Z_3,\\
            H^3(X;\Z_9) &\cong\Ext(\Z_9\oplus\Z_3,\Z_9)\oplus\Hom(\Z_{72}\oplus\Z,\Z_9)\\
            &=\Z_9^3\oplus\Z_3,\\
            H^4(X;\Z_9) &\cong\Ext(\Z_{72}\oplus\Z,\Z_9)=\Z_9^2.
        \end{align*}
        Finally, taking $G=\Q/\Z$, we find that (since $\Q/\Z$ is injective)
        \begin{align*}
            H^0(X;\Q/\Z) &\cong\Hom(\Z,\Q/\Z)=\Q/\Z,\\
            H^1(X;\Q/\Z) &\cong\Ext(\Z,\Q/\Z)=0,\\
            H^2(X;\Q/\Z) &\cong\Hom(\Z_9\oplus\Z_3,\Q/\Z)=\Z_9\oplus\Z_3,\\
            H^3(X;\Q/\Z) &\cong\Ext(\Z_9\oplus\Z_3,\Q/\Z)\oplus\Hom(\Z_{72}\oplus\Z,\Q/\Z)\\
            &=\Z_{72}\oplus\Q/\Z,\\
            H^4(X;\Q/\Z) &\cong\Ext(\Z_{72}\oplus\Z,\Q/\Z)=0,
        \end{align*}
        as it is easy to see that $\Hom(\Z_n,\Q/\Z)=\Z_n$.
        \newpage
    \item[Q6.] 
        \begin{enumerate}[(a)]
            \item Show that the spaces $(S^1\times\CP^\infty)\setminus(S^1\times\{x_0\})$ and
                $S^3\times\CP^\infty$ have isomorphic cohomology rings with coefficients in
                any commutative ring $R$.
            \item Explain how Steenrod squares act on cohomology of $S^1\times\CP^\infty$
                and $S^3\times\CP^\infty$ with coefficients in $\Z_2$.
            \item Using Steenrod squares show that the two spaces in (a) are not
                homotopy equivalent.
        \end{enumerate}
    \item[A6.]
        \begin{enumerate}[(a)]
            \item Denote $X=(S^1\times \CP^\infty)\setminus(S^1\times\{x_0\})$ and
                $Y=S^3\times\CP^\infty$. Note first that, by the universal coefficient theorem
                and the triviality of the cup product (for degree reasons),
                \[H^\bullet(S^3;R)=R[\gamma]/(\gamma^2),\]
                where $|\gamma|=3$ and $R$ is an arbitrary ring. For $\CP^\infty$, on the
                other hand, the cup product is nontrivial when computed with $\Z$ coefficients,
                so we must be a little more careful. From the unique morphism of rings $\Z\to R$
                for arbitrary $R$, we obtain a cochain map
                \begin{equation*}
                    \begin{tikzcd}
                        0\ar{r}& \Z\ar{r}\ar{d}& 0\ar{r}\ar{d}& \Z\ar{r}\ar{d}& 0\ar{r}\ar{d}& \cdots\\
                        0\ar{r}& R\ar{r}& 0\ar{r}& R\ar{r}& 0\ar{r}& \cdots
                    \end{tikzcd}
                \end{equation*}
                which induces a map on cohomologies. Now, the fact that cup products commute with
                induced maps yields
                \[H^\bullet(\CP^\infty;R)=R[\beta],\]
                where $|\beta|=2$. Now the K\"unneth formula given in Hatcher's Theorem 3.16
                implies that
                \[H^\bullet(Y;R)=H^\bullet(S^3\times\CP^\infty;R)\cong H^\bullet(S^3;R)\otimes H^\bullet(\CP^\infty;R),\]
                whence
                \[H^\bullet(Y;R)=R[\beta,\gamma]/(\gamma^2),\]
                with $|\beta|=2,|\gamma|=3$.

                The cohomology ring of $X$ is trickier. Writing $A=S^1\times\CP^\infty$ and $B=S^1\times\{x_0\}$,
                we first apply the long exact sequence for cohomology to obtain the groups $\tilde H^n(X;R)\cong H^n(A,B;R)$.
                It is easy to see that $H^0(A,B;R)=0$ and $H^n(A,B;R)\cong H^n(A;R)$ for $n>2$, as higher
                cohomology of the circle is zero. Since it is easy to compute
                \[H^\bullet(A;R)=H^\bullet(S^1\times\CP^\infty;R)=R[\alpha,\beta]/(\alpha^2),\]
                with $|a|=1,|\beta|=2$, we find that $H^n(A;R)=R$ for all $n$. It
                remains to compute $H^1(A,B;R)$ and $H^2(A,B;R)$. The first few terms of the
                long exact sequence (invoked above) look like
                \begin{equation*}
                    \begin{tikzcd}
                        0\ar{r}& 0\ar{r}{j^*}& R\ar{r}{i^*}& R& \\
                        \ar{r}& H^1(A,B;R)\ar{r}{j^*}& R\ar{r}{i^*}& R& \\
                        \ar{r}& H^2(A,B;R)\ar{r}{j^*}& R\ar{r}{i^*}& 0&
                    \end{tikzcd}
                \end{equation*}
                from which we conclude that $H^1(A,B;R)=0$ and $H^2(A,B;R)=R$.
                We conclude, then, that $H^n(X;R)=R$ for all $n\neq 1$ and
                $H^1(X;R)=0$. This yields the desired ring isomorphism
                \[H^\bullet(X;R)\cong H^\bullet(Y;R).\]
            \item
            \item
        \end{enumerate}
\end{enumerate}

\end{document}
