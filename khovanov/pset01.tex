\documentclass{../mathnotes}

\usepackage{tikz-cd}
\usepackage{todonotes}

\newgeometry{margin=1.75in}

\title{Algebraic Topology I: PSET 1}
\author{Nilay Kumar\footnote{Collaborated with Matei Ionita.}}
\date{Last updated: \today}


\begin{document}

\maketitle

\subsection*{Problem 1}
\begin{enumerate}[(a)]
    \item We construct a CW complex for $\R^2$ as follows. Let $X^0$ contain a point for
        every pair of integers. Then, for every pair of horizontally or vertically neighboring
        0-cells, attach a line segment with endpoints identified to said 0-cells. Finally,
        for every square of side length 1 line segment, attach a 2-cell such that the boundary
        of the 2-cell is identified with the square.
    \item The CW complex for the punctured plane, $\R^2-\{(0,0)\}$ is similar to that of $\R^2$.
        Start with the CW complex from $(a)$ except remove all cells in the square of side length
        one in the first quadrant with bottom-left corner $(0,0)$. In this square, line the
        borders with a square annulus of some size (adding 0,1, and 2-cells as necessary). Repeat
        this process with smaller and smaller square annuli such that we obtain a countable number
        of annuli (each with radius, say, $1/n$). Note that this complex does not ``contain the
        origin.''
\end{enumerate}

\subsection*{Hatcher 0.5}
The space $X$ deformation retracts to $x\in X$. Let $U\subset X$ is an open containing $x$.
We wish to find an open $V\subset U$ containing $x$ such that the inclusion $V\hookrightarrow U$
is nullhomotopic. We are provided with a homotopy $F=f_t:X\times I\to X$ taking $\id_X$ to $\{x\}$;
hence it suffices to find a neighborhood $V$ which, under $f_t$, is contained in $U$ for all time $t$.
To do so, consider the open set $F^{-1}(U)\subset X\times I$ containing the slice $\{x\}\times I$. 
We can find an open neighborhood $V\ni x$ such that the cylinder $V\times I$ is contained $F^{-1}(U)$:
take an open cover of $\{x\}\times I$ (by basis opens of the product topology). Compactness of $I$ yields
a finite subcover, from which the intersection of the opens in the second components yields an open
$V\subset U$ containing $x$ such that $V\times I\subset F^{-1}(U)$. This implies that, for all $t$,
$f_t(V)$ is contained in $U$, and hence $V$ is the desired open.

\subsection*{Hatcher 0.6a}
Suppose $x_0$ is a point in the segment $[0,1]\times\{0\}$; then there exists a deformation retract
taking $X$ to $x_0$ as follows. For $t\in[0,1/2]$ we collapse the teeth of the comb via
$(x,y)\mapsto (x,y(1-2t))$. Then, for $t\in[1/2,1]$, we collapse the remaining line segment onto
the chosen point $x_0$ via $(x,0)\mapsto(x+2(x-x_0)(t-1/2),0)$. Note that $x_0$ is fixed throughout,
and hence this provides a deformation retraction.

There is no deformation retraction taking $X$ to a point not on the segment $\{0,1\}\times\{0\}$.
We show this by contradiction. If there is, and $U$ is an open about said point, there must
exist an open $V\subset U$ such that the inclusion $V\hookrightarrow U$ is nullhomotopic,
i.e. contractible in $U$. In this case, $U$ is the disjoint union of countably many line segments;
any open subset will be the disjoint union of countably many intervals, which is clearly
not contractible (this can be shown by noting that no map $X\times I\to X$ could continuously
move disconnected components to a single component).

\subsection*{Hatcher 0.8}
Suppose $n=2m$ for $m>1$. Then the construction of the house with $n$ rooms is simply $m$ copies of
the house with 2 rooms glued in the obvious way. The case of $n=2m+1$ is a little subtler.
Take the house with $2m$ rooms and add another ``roof'' as shown in my diagram and add
a tunnel that traverses the top 2 rooms (with vertical support walls as necessary), connecting
the outside to the third room from the top. In both cases, these structures are clearly
contractible, as one can imagine ``unscooping'' the space created in the rooms backwards
along the tunnels to obtain a space homeomorphic to a ball.

\subsection*{Hatcher 0.9}
Let $X$ be a contractible space and $r:X\to A$ be a retract of $X$ to the subspace $A$.
There exists a homotopy $F=f_t:X\times I\to X$ from the identity map on $X$ to the map that
takes $X$ to a point $x\in X$. The composition $r\circ f_t$, when restricted to $A$ provides
a homotopy from $\id_A$ to the map that takes $A$ to $r(x)\in A$, and hence $A$ is
contractible.

\subsection*{Hatcher 0.17}
\begin{enumerate}[(a)]
    \item Let $f:S^1\to S^1$ be a continuous map and consider the associated mapping cylinder $M_f$.
        We can give $M_f$ a CW structure as follows. The zero-skeleton consists of two points, $x$
        and $f(x)$. The one-skeleton consists of three lines, one forming a circle at $x$ and another
        forming a circle at $f(x)$, and the last, snaking down to connect $x$ to $f(x)$ as the ``graph''
        of $x$. Finally, we attach a single two cell to this one-skeleton, completing the cylinder.
    \item Denote the M\"obius strip by $M$ and the annulus by $A$. Consider the retractions
        $f:M\to S^1$ and $g:A\to S^1$. We can construct the mapping cylinders $X_1=S^1\coprod_fM$
        and $X_2=S^1\coprod_gA$. These spaces clearly deformation retract onto $M$ and $A$ respectively.
        Gluing these spaces together via the identity map on $S^1$ in each of these spaces, we
        obtain a space that deformation retracts onto $M$ and $A$. Note that this is a CW complex
        as we can give the M\"obius strip and the annulus CW structures.
\end{enumerate}

\subsection*{Hatcher 0.20}
Consider the disk at which the Klein bottle intersects itself. As it is a subcomplex, we can contract
it to a point without changing the homotopy type of the Klein bottle. We can deform the Klein bottle
into a sphere-like shape and replace both the inward and outward flutes with the north and south pole
of this sphere and extend the pinched point into inward and outward line segments. Shifting the endpoints
of these line segments yields $S^1\vee S^1\vee S^2$ (see diagram).


\subsection*{Hatcher 0.23}
Let $X=A\cup B$ be a union of two contractible subcomplexes with contractible intersection
$A\cap B$. We use Hatcher Proposition 0.17 repeatedly to show that $X$ is contractible:
\begin{align*}
    A\cup B &\approx (A\cup B)/(A\cap B)\\
    &=A/(A\cap B)\vee B/(A\cap B)\\
    &\approx A\vee B\\
    &\approx A\\
    &\approx\{\rm pt\},
\end{align*}
as every CW pair satisfies the HEP.

\end{document}
