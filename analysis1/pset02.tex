\documentclass{../mathnotes}

\usepackage{tikz-cd}
\usepackage{todonotes}

\newgeometry{margin=1.75in}

\title{Analysis I: Solutions to PSET 2}
\author{}
\date{}

\begin{document}

\maketitle

\subsection*{Rudin 1.17}
Take two vectors $x,y\in\R^k$. We claim that
\[|x+y|^2+|x-y|^2=2|x|^2+2|y|^2,\]
which follows by the commutativity and distributivity of the dot product,
\begin{align*}
    |x+y|^2+|x-y|^2 &= (x+y)\cdot(x+y) + (x-y)\cdot (x-y)\\
    &= x\cdot x+2x\cdot y+y\cdot y+x\cdot x-2x\cdot y+y\cdot y\\
    &= 2x\cdot x+2y\cdot y\\
    &=|x|^2+|y|^2.
\end{align*}
Interpreted geometrically, we find that the sum of the length-squareds
of the diagonals of a parallelogram is equal to the sum of the squares of the side lengths.

\subsection*{Rudin 2.2}
We claim that the set of all algebraic numbers $A$ is countable. Consider the first the set
of all polynomials $P$ with integer coefficients. The set $P$ admits a natural decomposition
$P=\cup_n P_n$ where $P_n$ is the subset of polynomials with degree $n$.
Any polynomial of degree $n$ is uniquely determined by its $n+1$ coefficients and hence is
equivalent to the set of $(n+1)$-tuples of integers.
This set is countable (c.f. Rudin Theorem 2.13) and hence $P_n$ is countable.

Given any polynomial $p\in P_n$, the set of its roots $R_p$ is finite (by the fundamental theorem of algebra)
and hence we can write
\[A=\bigcup_{p\in P}R_p=\bigcup_{n\in\N}\left(\bigcup_{p\in P_n}R_p\right),\]
i.e. a countable union of a countable union of finite sets, which is countable (c.f. Rudin Theorem 2.12).

\subsection*{Problem 3}
We claim that the sets $(0,1)\subset\R,[0,1]\subset\R,$ and $\R$ are equivalent. We show only two equivalences,
using the fact that equivalence is an equivalence relation to obtain the third.

We know from precalculus that the function $\tan x:\R\to\R$, when restricted to $(-\pi/2,\pi/2)$ is bijective,
with inverse $\tan^{-1} x$. This gives an equivalence between $(-\pi/2,\pi/2)\subset\R$ and $\R$.
Moreover, $(-\pi/2,\pi/2)$ is equivalent to $(0,1)$ by the map $x\mapsto (x+\pi/2)/\pi$, which is
obviously bijective. This establishes the equivalence of $(0,1)$ and $\R$.

Consider now the function $g:[0,1]\to(0,1)$ given piecewise as
\begin{align*}
    g(x) = \left\{
        \begin{array}{lr}
            1/2, &  x=0\\
            1/4, &  x=1\\
            1/2^{n+2}, &  x=1/2^n\text{ for }n\in\N\\
            x, & \text{otherwise}.
        \end{array}
        \right.
\end{align*}
Intuitively, this function takes advantage of the ``space'' available in the infinite sequence $1/2^n$ by sending
the ``extra'' points $0$ and $1$ to $1/2$ and $1/4$, respectively, and then pushing down what
normally would have been sent to powers of $1/2$ by a factor of $1/4$.
Note that this trick can be done with any sequence tending to zero, not just $1/2^n$.
Let us show that $g$ is bijective; it will take some case work, due to the piecewise definition.

First surjectivity; suppose $y\in(0,1)$. If $y$ is $1/2$ or $1/4$ we note that $y$ must be $g(0)$ or $g(1)$. If
$y$ is more generally $1/2^n$ (for $n\in\N$ greater than 2), we find that $y=g(1/2^{n-2})$. Otherwise, $g(y)=y$.

Injectivity is clear, but a little tedious; suppose $x,y\in[0,1]$ with $g(x)=g(y)$.
If either of $x$ or $y$ is zero, it is clear that $g(x)=g(y)$ forces $x=y=0$, as only $0$ is sent to $1/2$.
Now suppose neither $x$ nor $y$ is zero. Note that $x$ and $y$ must both either be a power of $1/2$
or not, as otherwise $g(x)$ could not possibly be equal to $g(y)$.
If neither $x$ nor $y$ are powers of $1/2$, it follows from the definition of $g$ that $x=y$.
If both $x$ and $y$ are powers of $1/2$, say $x=1/2^n$ and $y=1/2^m$, then $g(x)=g(y)$ implies that $n=m$,
i.e. $x=y$.

This proves the equivalence of $[0,1]$ and $(0,1)$. As $(0,1)$ is equivalent to $\R$ as shown above,
all three sets are equivalent.

\subsection*{Problem 4}
Let $\{E_n\}$ be a sequence of countable sets, and $S=\prod_n E_n$ be their Cartesian product. 
Suppose $S$ is countable; then there exists a bijection $f:\N\to S$. Note that each $i\in\N$ is taken
to an infinite sequence of elements $\{e_{ij}\}$, i.e.
\begin{align*}
    f(1) &= (e_{11}, e_{12}, \ldots)\\
    f(2) &= (e_{21}, e_{22}, \ldots)\\
    &\vdots\\
    f(i) &= (e_{i1}, e_{i2}, \ldots)\\
    &\vdots
\end{align*}
Now choose $a=(a_1,a_2,\ldots)\in S$ such that for each $k\in\N$, $a_k\neq e_{kk}$. Surjectivity
of $f$ implies that there exists some $\ell\in\N$ such that $f(\ell)=a$. This is, of course, a contradiction,
as $a_k\neq e_{kk}$ for each $k$. Hence $S$ must be uncountable (as $S$ is clearly not finite).

Similarly, if each $E_n=\{0,1\}$, we assume that there exists a bijection $f:\N\to S=\prod_nE_n$. Surjectivity of
$f$ implies the existence of some $\ell\in\N$ mapped to $a=(a_1,a_2,\ldots)\in S$ where $a_k=0$ if $e_{kk}=1$
and $a_k=1$ if $e_{kk}=0$. This is again a contradiction, and hence $S$ must be uncountable (as $S$ is
clearly not finite).

\end{document}
