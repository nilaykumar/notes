\documentclass{../mathnotes}

\usepackage{tikz-cd}
\usepackage{todonotes}

\newgeometry{margin=1.75in}

\title{Analysis I: Solutions to PSET 10}
\author{}
\date{}

\begin{document}

\maketitle

\subsection*{Rudin 6.4}
For $a<b\in\R$ let $f:[a,b]\to\R$ be 0 for all irrationals and 1 for all rationals.
The density of $\Q$ and $\R\setminus\Q$ in $\R$ force
\begin{align*}
    U(P,f) &= \sum_{i=1}^n \left(\sup_{x_{i-1}\leq x\leq x_i} f(x)\right)\Delta x_i=\sum_{i=1}^n \Delta x_i=b-a\\
    L(P,f) &= \sum_{i=1}^n \left( \inf_{x_{i-1}\leq x\leq x_i} f(x) \right)\Delta x_i=\sum_{i=1}^n 0\cdot \Delta x_i=0
\end{align*}
for any partition $P$ of $[a,b]$. This implies that
\[\overline{\int_a^b}fdx=\inf_P U(P,f)\neq\sup_P L(P,f)=\underline{\int_a^b}fdx,\]
and hence $f\notin\mathcal{R}$.

\subsection*{Rudin 6.7\footnote{Solution due to Theodore Coyne}}
\begin{enumerate}[(a)]
    \item If $f$ is Riemann integrable on $[0,1]$, we can find an upper bound $M$
        of $|f|$ on $[0,1]$. Then
        \begin{align*}
            \left|\int_0^c f(x)dx\right| \leq \int_0^c|f(x)|dx\leq cM.
        \end{align*}
        Now fix $\varepsilon>0$. Choosing $\delta=\varepsilon/M$, we find that for
        all $0<c<\delta$
        \[\left|\int_0^1f(x)dx-\int_c^1f(x)dx\right|=\left|\int_0^cf(x)dx\right|\leq cM<\varepsilon,\]
        which implies that
        \[\lim_{c\to 0}\int_c^1f(x)dx=\int_0^1f(x)dx,\]
        as desired.
    \item Define the function $f(x)$ on the interval $(1/(n+1), 1/n]$ for any $n\in \N$
        to be given by $(n+1)(-1)^n$. We choose this function for the following reason:
        for any $N\in\N$ with $N>1$, we have
        \begin{align*}
            \int_{1/N}^1 f(x)dx &= \sum_{n=1}^{N-1}\int_{1/(n+1)}^{1/n} f(x) dx\\
            &= \sum_{n=1}^{N-1}\left( \frac{1}{n}-\frac{1}{n+1} \right)(-1)^n(n+1)\\
            &=\sum_{n=1}^{N-1}\frac{(-1)^n}{n}.
        \end{align*}
        Then it is clear that as a sequence (i.e. $N$ natural),
        \[\lim_{N\to\infty}\int_{1/N}^1 f(x)dx=\lim_{N\to\infty}\sum_{n=1}^{N-1}\frac{(-1)^n}{n}\]
        exists. Now, for $0<c<1$ and $N$ the unique positive integer satisfying $1/(N+1)<c\leq 1/N$,
        we compute
        \begin{align*}
            \left|\int_c^1f(x)dx - \int_{1/N}^1f(x)dx\right| &= \left|\int_c^{1/N}f(x)dx\right|\\
            &=\frac{1}{N+1}\left( \frac{1}{N}-c \right)\\
            &\leq \frac{1}{N+1}\left( \frac{1}{N}-\frac{1}{N+1} \right)\\
            &=\frac{1}{N(N+1)^2},
        \end{align*}
        which approaches zero as $c$ approaches zero. Hence $\lim_{c\to 0}\int_c^1f(x)dx$ exists
        and is equal to $\lim_{n\to\infty}\int_{1/n}^1f(x)dx$ above.

        On the other hand,
        \[\int_{1/N}^1|f(x)|dx=\sum_{n=1}^{N-1}\frac{1}{n}\]
        diverges as $N$ increases without bound. Now, for any $c\in(0,1]$, let $M$ be the
        unique positive integer such that $1/(M+1)<c\leq 1/M$. Then, since
        \[\int_c^1|f(x)|dx\geq\int_{1/M}^1|f(x)|dx,\]
        we find that $\lim_{c\to0}\int_c^1f(x)dx$ does not exist.
\end{enumerate}

\subsection*{Rudin 6.8\footnote{Solution due to Theodore Coyne}}
Suppose $\int_1^\infty f(x)dx$ converges. Then there must exist an $M\in\R$ such
that $\int_1^Nf(x)dx\leq M$ for all $N\in\N$. Moreover, since $f$ is monotonically
decreasing,
\begin{align*}
    \int_1^N f(x) dx - \sum_{n=1}^{N-1}f(n+1)=\sum_{n=1}^{N-1}\left( \int_n^{n+1}f(x)dx-f(n+1) \right)\geq 0.
\end{align*}
This implies that
\begin{align*}
    \sum_{n=1}^Nf(n+1)\leq \int_1^Nf(x)dx\leq M
\end{align*}
for all $N\in\N$. Since $f(x)\geq 0$ for all $x$, the partial sums of of the sum on the
left hand side are monotonically increasing, and hence since monotonic and bounded
sequences converge, we find that $\sum_{n=1}^\infty f(n)$ exists.

Conversely, suppose that $\sum_{n=1}^\infty f(n)$ exists. By the argument above, we
find that
\begin{align*}
    \sum_{n=1}^Nf(n+1)\leq \int_1^Nf(x)dx&\\
    \int_1^Nf(x)dx&\leq \sum_{n=1}^{N-1} f(n).
\end{align*}
Thus, for any natural $N>1$, since $f$ is decreasing, we find that
\begin{align*}
    \sum_{n=1}^{N-1}f(n+1)\leq \int_1^n f(x)dx\leq \sum_{n=1}^{N-1}f(n)
\end{align*}
and hence the sequence $\lim_{N\to\infty}\int_1^N f(x)dx$ exists. We extend
this to real $b$ by taking $m\leq b\leq m+1$ and noting that the non-negativity
of $f$ yields
\begin{align*}
    \int_1^m f(x)dx\leq \int_1^b f(x)dx\leq \int_1^{m+1} f(x)dx,
\end{align*}
and so $\lim_{N\to\infty}\int_{1}^bf(x)dx$ exists.


\subsection*{Rudin 6.10abc\footnote{Solution due to Ellen Vitercik}}
\begin{enumerate}[(a)]
    \item Note that $\log x$ is strictly concave, as its second derivative is
        everywhere negative. Thus
        \[\log\left( \frac{u^p}{p} + \frac{v^q}{q} \right)\geq \frac{\log u^p}{p}+\frac{\log v^q}{q}=\log (uv),\]
        with equality acheived only if $u^p=v^q$. Hence
        \[uv\leq \frac{u^p}{p}+\frac{v^q}{q}.\]
    \item Using part (a) we see that
        \begin{align*}
            \int_a^b fg d\alpha\leq\int_a^b\left(\frac{f^p}{p}+\frac{g^q}{q}  \right)d\alpha=\frac{1}{p}\int_a^b f^pd\alpha+\frac{1}{q}\int_a^bg^pd\alpha=1.
        \end{align*}
    \item Suppose that
        \[\left\{ \int_a^b|f|^pd\alpha \right\}^{1/p}>0\text{ and }\left\{ \int_a^b|g|^qd\alpha \right\}>0,\]
        for if not, the result follows trivially. Now we use part (b): since
        \begin{align*}
            \int_a^b\left|\frac{f}{ \left\{ \int_a^b|f|^pd\alpha \right\}^{1/p} }\right|^p&=1\\
            \int_a^b\left|\frac{g}{ \left\{ \int_a^b|g|^qd\alpha \right\}^{1/q} }\right|^q&=1,
        \end{align*}
        we find by (b) that
        \[\int_a^b|f||g|d\alpha\leq \left\{ \int_a^b|f|^pd\alpha \right\}^{1/p}\left\{ \int_a^b|g|^qd\alpha \right\}^{1/q}.\]
        But recall that Rudin 6.13 yields
        \[\left|\int_a^b fgd\alpha\right|\leq\int_a^b|f||g|d\alpha.\]
        Combining this with the inequality above yields the desired result.
\end{enumerate}

\end{document}
